\documentclass[a4paper,10pt]{report}
\usepackage[T1]{fontenc}
%\usepackage[standard-baselineskips]{cmbright}
\usepackage{cmbright}
\def\usedfonts{CM-Bright}
\usepackage{graphicx} % includegraphics
\begin{document}
\title{{\Huge Molecule Benchmark Environment G2}}
\author{Peter Bl\"ochl}
\date{\today}
\maketitle
\tableofcontents
\chapter{Description}
%=========================================================================
\section{Overview}
%=========================================================================
The normal workflow is as follows:
\begin{enumerate}
\item change to the directory with the G2 database, containing a
  directory ``src'', the directory ``doc'' and the file sample.cntl,
\item execute \verb|src/g2_makedo|. A file \verb|g2_do_sample| is
  constructed, the directory \verb|Cases| holding the directories of
  the individual projects is constructed, and the stucture files in
  all these directories will be constructed.
\item adjust the file sample.cntl. This file will be used for all paw
  calculations.
\item make a copy \verb|g2_do| of \verb|g2_do_sample|, delete or
  comment out the lines with the projects that shall not be
  considered.
\item execute \verb|g2_do|. 
\item execute \verb|src/g2_analyse|
\item inspect file readpaw.out
\end{enumerate}


In order to adjust the speciesfiles, change the files with extension
\verb|.species| in the directory \verb|src/Speciesfiles|.
 

%=====================================================
\section{Description of Individual scripts}
%=====================================================


%=====================================================
\section{Problems}
%=====================================================
\begin{itemize}
\item Some of the atoms cannot be converged with Safeortho=F because of 
its energy-level structure
\item The hydrogen atom has problems probably because there is no spin
density in the minority spin direction. NO! it seems to be also
safeortho. Again YES: I get problems also with SAFEORTHO=T. It helps
to make a calculation with spin[hbar]=0.499 and then multiply the
energy level difference with 0.001. 
\begin{center}
\includegraphics[width=0.5\linewidth,clip=t]{Figs/hydrogenetot.eps}
\begin{tabular}{|l|r|r|r|}
\hline
$\frac{1}{2}\hbar-S$ & $E_{tot}[H]$ & $\epsilon_\uparrow$[eV]& $\epsilon_\downarrow$[eV] \\
\hline
0.001 &  -0.4984656 &  -7.621&   -0.388 \\
0.005 &  -0.4974234 &  -7.646&   -0.627\\ 
0.010 &  -0.4961265 &  -7.686&   -0.831 \\
0.020 &  -0.4935614 &  -7.766&   -1.091 \\
\hline
\end{tabular}
\end{center}
\begin{eqnarray*}
E_{tot}[H]=-0.49872+0.258(\frac{1}{2}-S/hbar)
\end{eqnarray*}
If we perform calculations with a spin of $0.4990 \hbar$, the total
energy is too high by $0.2544\times 10^3$~H.

The unit cell for hydrogen has been fixed to a lattice constant of
10~\AA to avoid instabilitis.

\item Be2 may need a larger cell!
\item take care of zero point vibration energy
\item in DFT soe of the atoms are non-psherical
\item The G2 database contains some open shell systems besides atoms:
BeH (spin=0.5$\hbar$), CH,\ldots
\begin{itemize}
\item C$_2$ is has a band crossing of a sigma state and a doublet of
$\pi$ states right at the energy minimum
\end{itemize}
\end{itemize}

\end{document}
