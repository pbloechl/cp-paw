\documentclass[11pt,a4paper]{report}
%%%%%%%%%%%%%%%%%%%%%%%%%%%%%%%%%%%%%%%%%%%%%%%%%%%%%%%%%%%%%%%%%%%%%
%%                                                                 %%
%%    Header file for the Phi-S-X Series                           %%
%%                                                                 %%
%%    german version header_gm.tex is derived from header.tex      %%
%%    by uncommenting the line ``\setboolean{german}{true}'' below %%
%%                                                                 %%
%%    Never edit the german version! all changes must be done      %%
%%    in the english version header.tex                            %%
%%                                                                 %%
%%%%%%%%%%%%%%%%%%%%%%%%%%%%%%%%%%%%%%%%%%%%%%%%%%%%%%%%%%%%%%%%%%%%%
%%%%%%%%%%%%%%%%%%%%%%%%%%%%%%%%%%%%%%%%%%%%%%%%%%%%%%%%%%%%%%%%%%%%%
%%                                                                 %%
%%    Header file for the Phi-S-X Series                           %%
%%                                                                 %%
%%    german version header_gm.tex is derived from header.tex      %%
%%    by uncommenting the line ``\setboolean{german}{true}'' below %%
%%                                                                 %%
%%    Never edit the german version! all changes must be done      %%
%%    in the english version header.tex                            %%
%%                                                                 %%
%%%%%%%%%%%%%%%%%%%%%%%%%%%%%%%%%%%%%%%%%%%%%%%%%%%%%%%%%%%%%%%%%%%%%
%====================================================================
%-- define flag for language adaptations
\usepackage{ifthen}   % allows to select only certain text
\provideboolean{german}
\setboolean{german}{false}
%\setboolean{german}{true}  % uncomment this line for german editions
%====================================================================
%
% Textschriftart: Computer modern Bright
% body:            CM-Bright 10pt
% section titles:  CM-Bright Bold
% formulas:        CM-Bright Math Oblique
%
\usepackage[standard-baselineskips]{cmbright}
\usepackage{cmbright}
\usepackage[T1]{fontenc}
\def\usedfonts{CM-Bright}
\usepackage{typearea}
%\typearea[current]{calc} % benutzt die aktuelle 
       % bindekorrektur (BCOR angabe als parameter in koma usepackage)
       % und berechnet satzspiegel neu
\typearea[current]{11} %fixed div value

\usepackage{textcomp} % special symbols
\usepackage{amsfonts} % special symols
                      % see ftp://ftp.ams.org/pub/tex/doc/amsfonts/amsfndoc.pdf
\usepackage{amssymb}  % CM-Bright provides the AMS symbols
\usepackage{exscale}  % allows to scale math expressions to big fonts, 
                      % e.g. \Huge
\usepackage{curves}
\usepackage{braket}
\usepackage{miller}     % miller indices
\usepackage{chemmacros} % http://www.mychemistry.eu/mychemistry/
\usepackage[numbers]{natbib}     % bibliography style
\usepackage{url}\urlstyle{tt}
\usepackage{float}
\usepackage{bm}       % provides the command \bm{} that makes bold math symbols
\usepackage{amsmath}
\usepackage{amsbsy}   % allows bold mathematical symbols
\usepackage{amscd}
 \usepackage{a4wide}  % it is better to use the ``geometry'' package
\usepackage{array}    % 
\usepackage{fancyhdr} %  defines pagestyle fancy
\usepackage{epsfig}   % include graphics with epsfig
\usepackage{graphicx} % includegraphics
\usepackage{epstopdf}
\usepackage{wrapfig}
\usepackage{fancybox} % allows shadow-boxes
\usepackage{color}    % allows to use color in the text
%\usepackage{eepic}
\usepackage{flafter}  % places picture next to its reference
\usepackage{makeidx}  % make an index
%\usepackage{MnSymbol}  % 
%\usepackage{marvosym}  % 
\usepackage{textcase}
\usepackage{ulem} % defines strikeout \sout{}; underline \uline{}
                  % double underline \uuline{}; wave underline \uwave{}
                  % cross out \xout{}
%
%==========================================================================
%==  page layout  =========================================================
%==========================================================================
% eqnarray environment: reduce with of space in place of each ``&''
\setlength\arraycolsep{1.4pt}
\pagestyle{fancy}
%\renewcommand{\chaptermark}[1]{\markboth{\thechapter\ #1}{}}
\renewcommand{\chaptermark}[1]{\markboth{\MakeUppercase{\thechapter\ #1}}{}}
\fancyhf{} 
\fancyhead[LE]{\textsc{\thepage}\qquad\textsc{\leftmark}}
\fancyhead[RO]{\textsc{\leftmark}\qquad\textsc{\thepage}}
\renewcommand{\headrulewidth}{0.5pt}
\renewcommand{\footrulewidth}{0pt} 
\addtolength{\headheight}{2.5pt}
\fancypagestyle{plain}{\fancyhead{}
   \renewcommand{\headrulewidth}{0pt}
   \fancyfoot[CO]{\bfseries\thepage}}

% Line spacing -----------------------------------------------------------
\newlength{\defbaselineskip}
\setlength{\defbaselineskip}{\baselineskip}
\newcommand{\setlinespacing}[1]%
           {\setlength{\baselineskip}{#1 \defbaselineskip}}
\newcommand{\doublespacing}{\setlength{\baselineskip}%
                           {2.0 \defbaselineskip}}
\newcommand{\singlespacing}{\setlength{\baselineskip}{\defbaselineskip}}

% Absatz einr\"ucken ------------------------------------------------------
%\setlength{\parindent}{0pt}
\setlength{\parskip}{2pt}
% -------------------------------------------------------------------------
\ifthenelse{\boolean{german}}
  {\def\figurename{Abb.}}
  {\def\figurename{Fig.}}
%--------------------------------------------------------------------------
\renewcommand{\arraystretch}{1.15}  % skaliert den Zeilen abstand in der 
    % tabular und array umgebung
%
%==========================================================================
%==  boxes etc ============================================================
%==========================================================================
%== minipage in a shadowbox ===============================================
\newenvironment{myshadowminipage}[1]%
  {\par\noindent\begin{Sbox}\begin{minipage}{\linewidth}\vspace{0.1cm}\begin{center}\uppercase{#1}\end{center}}%
  {\vspace{0.1cm}\end{minipage}\end{Sbox}\shadowbox{\TheSbox}}
%
%== minipage in a framedbox ===============================================
\newenvironment{myframedminipage}%
  {\par\noindent\begin{Sbox}\begin{minipage}\linewidth\vspace{0.1cm}}%
  {\vspace{0.1cm}\end{minipage}\end{Sbox}\fbox{\TheSbox}}
%
\newcommand{\myshadowbox}[1]{\noindent\shadowbox{\parbox{\linewidth}{\smallskip #1\smallskip}}}
\newcommand{\myfbox}[1]{\noindent\fbox{\parbox{\linewidth}{\smallskip #1\smallskip}}\medskip}
%== minipage in a framedbox ===============================================
\newtheorem{defi}{Definition}[chapter]
\newenvironment{definition}[1]%
  {\par\noindent\begin{Sbox}\begin{minipage}{\linewidth}\vspace{0.1cm}\begin{defi}\uppercase{#1}\\\vspace{0.1cm}}%
  {\vspace{0.1cm}\end{defi}\end{minipage}\end{Sbox}\shadowbox{\TheSbox}}
%
%=========================================================================
% color used to point out information to the teacher
\definecolor{highlight}{rgb}{1.0,0.7,0.}
\newcommand{\Special}[1]{\textbf{\textcolor{highlight}{#1}}}
%=========================================================================
%  switch certain parts on and off. uses ifthen package
\newboolean{teacher}\setboolean{teacher}{false}
% this parameter can be changed in the manuscript again
\setboolean{teacher}{true} %private version if true!
\newcommand{\teacheronly}[1]{\ifthenelse{\boolean{teacher}}{#1\hfill\\ }}
\newcommand{\editor}[1]{\textcolor{blue}{\texttt{Editor: #1}}}
\newcommand{\MARK}[1]{\textcolor{blue}{#1}} 
\newcommand{\RED}[1]{\textcolor{red}{#1}} 
%
%==========================================================================
%==  define new symbols                                                 ===
%==========================================================================
% define \stat (stationary state) as an operator like \min
\DeclareMathOperator*{\stat}{stat}
\let\Vec=\mathbold   % cmbright.sty provides a bold/italic math alphabet
\let\Dot=\mathbold   % cmbright.sty provides a bold/italic math alphabet
\let\Ddot=\mathbold   % cmbright.sty provides a bold/italic math alphabet
%
\newcommand{\e}[1]{\mathrm{e}^{#1}}% exponential function
\renewcommand{\Re}{\mathrm{Re}}    % real part
\renewcommand{\Im}{\mathrm{Im}}    % imaginary part
\newcommand{\lagr}{\ell}           % Lagrange dichte
\newcommand{\Lagr}{\mathcal{L}}    % Lagrange Funktion
\newcommand{\erf}{{\rm erf}}       %
\newcommand{\atan}{{\rm atan}}     % arcus tangens
\newcommand{\mat}[1]{\bm{#1}}  % Matrix
\newcommand{\gmat}[1]{{\boldsymbol #1}}  % Matrix(symbol)
\newcommand{\defas}{\stackrel{\text{def}}{=}}  %  is defined as
\ifthenelse{\boolean{german}}
  {\newcommand{\rot}{{\rm\bf rot}}}    % curl
  {\newcommand{\rot}{{\rm\bf curl}}}   % curl
\newcommand{\sgn}{{\rm sgn}}       % sign
\ifthenelse{\boolean{german}}
   {\newcommand{\Tr}{\mathrm{Sp}}}      % trace
   {\newcommand{\Tr}{\mathrm{Tr}}}      % trace
\ifthenelse{\boolean{german}}
   {\newcommand{\grmn}[2]{\footnote{``#2'' hei{\ss}t in englisch ``#1''}}}
   {\newcommand{\grmn}[2]{\footnote{``#1'' translates as ``#2'' into German}}}
% define the equation reference
\ifthenelse{\boolean{german}}
   {\newcommand{\eq}[1]{\text{Gl.}~\ref{#1}}}
   {\newcommand{\eq}[1]{\text{Eq.}~\ref{#1}}}
% define a relation with an equation number ontop
\newcommand{\eqrel}[2]{\stackrel{\eq{#1}}{#2}}
\newcommand{\zero}{\varnothing}
%\newcommand{\ket}[1]{|#1\rangle} % contained in package braket
\newcommand{\sumint}{\int\hspace{-15pt}\sum}
\newcommand{\marker}[1]{\textcolor{blue}{\emph{#1}}}
\renewcommand*{\dot}[1]{\overset{\mbox{\large\bfseries .}}{#1}}
\renewcommand*{\ddot}[1]{\overset{\mbox{\large\bfseries\hspace{+0.1ex}.\hspace{-0.1ex}.}}{#1}}
%
%==========================================================================
%==                                                                     ===
%==========================================================================
% Prevent figures from appearing on a page by themselves
% from http://dcwww.camd.dtu.dk/~schiotz/comp/LatexTips/LatexTips.html
\renewcommand{\topfraction}{0.85}
\renewcommand{\textfraction}{0.1}
\renewcommand{\floatpagefraction}{0.75}
%
%==========================================================================
%==                                                                     ===
%==========================================================================
\makeindex    % make index. uses makeidx package.

%== allow links between documents ============================================
\usepackage{xr}
\usepackage{xr-hyper}
%==  hyperref package (must be last package)
\usepackage[colorlinks=true]{hyperref} %specify this as last package
\hypersetup{citecolor=blue}
\hypersetup{menucolor=magenta}
\hypersetup{urlcolor=blue}      % 
\hypersetup{filecolor=green}    % file links
\hypersetup{linkcolor=magenta}  %table of contents
\hypersetup{pdfauthor={Peter E. Bl\"ochl}}
\hypersetup{pdfdisplaydoctitle=true}
\externaldocument[phisx1-]{/Users/ptpb/Tree/PhiSX/ClassicalMechanics/Book/cm-gm}
\externaldocument[phisx2-]{/Users/ptpb/Tree/PhiSX/Electrodynamics/Book/el-gm}
\externaldocument[phisx3-]{/Users/ptpb/Tree/PhiSX/QuantumMechanics/Book/qm}
\externaldocument[phisx4-]{/Users/ptpb/Tree/PhiSX/StatisticalMechanics/Book/sm}
\externaldocument[phisxqm2-]{/Users/ptpb/Tree/PhiSX/QuantumMechanicsII/Book/qm2}
\externaldocument[phisxsm2-]{/Users/ptpb/Tree/PhiSX/StatisticalMechanicsII/Book/sm2}
\externaldocument[phisxcb-]{/Users/ptpb/Tree/PhiSX/Chemicalbond/Book/cb}
% Example: Figure~PhiSX:Quantum
% Mechanics-\ref{phisx3-fig:doubleslitwave} on page
% \pageref{phisx3-fig:doubleslitwave}


\hypersetup{pdftitle=paw_brillouin}
\begin{document}
\begin{titlepage}
\begin{center}
\vspace*{3.5cm}
{\huge \textbf{The LMTO object of the CP-PAW code}}\\
\vspace{0.5cm}
{\large Peter E. Bl\"ochl}
\vspace{0.5cm} 
\end{center}

\vfill
\begin{center}
Copyright Peter E. Bl\"ochl; Sept.2, 2013-\today\\
{\small
Institute of Theoretical Physics;
Clausthal University of Technology;\\ 
D-38678 Clausthal Zellerfeld; Germany;\\
http://www.pt.tu-clausthal.de/atp/}
\end{center}
\end{titlepage}
\noindent            
\tableofcontents
%====================================================================
\chapter{Todo}
%====================================================================
\begin{itemize}
\item lmto\_overlapphi calculates the onsite overlap matrix of partial
  waves in a sphere.
%
\item the core-valence exchange contribution differs from the old version,
  because it also includes the projection on the phidot functions.
%
\item there has been a bug in \verb|lmto$screen|, which has been fixed
  with version 3. It may be better to rewrite all structure constants
  routines with teh transposed structure constants.
%
\item using only sp like tight-binding orbitals and local exchange
  lead to an increase of the band gap of silicon avove 1.3 eV. After
  adding the d-orbitals to the HF term collapsed the band gap
  dramatically below the dft value. Core-valence echange seems to have
  an important effect on the band gap too.
\end{itemize}

%====================================================================
\chapter{Purpose and theoretical background of the LMTO Object}
%====================================================================
The LMTO object maps the wave functions expressed in augmented plane
waves into a basiset of natural tight-binding orbitals. The natural
tight-binding orbitals are a kind of LMTO's, screened such that the
tails exhibit only scattering character in the context of nodeless
wave functions\cite{bloechl12_arxiv1210_5937}.

%====================================================================
\section{Augmentation}
%====================================================================
The concept of linear augmented waves is as follows: 
\begin{enumerate}
\item At first a so-called \textbf{envelope function}\index{envelope
  function} $|K^\infty_\alpha\rangle$ is defined.
%
\item In a second step, this envelope function is expanded about each
  atomic site into spherical harmonics.  More generally, they are
  expanded into \textbf{head functions}\index{head function}
  $|K^\Omega_\alpha\rangle$ and \textbf{tail functions}\index{tail
    function} $|J^\Omega_\alpha\rangle$. The head function is the
  dominant contribution and carries the quantum number of the final
  orbital, while the tail functions are the minor contributions with
  different quantum numbers. In practice, the head functions are solid
  Hankel functions and the tail functions are solid Bessel functions.

   The coefficients of the tail functions are called \textbf{structure
     constants}\index{structure constants}.

  The difference between the full envelope function and its expansion
  into head and tail functions is the interstitial contribution
  $|K^I_\alpha\rangle$.
%
\item In the third step the head and tail functions are replaced
  differentially at some sphere radius by partial waves of the atomic
  potential. For that purpose, we use a solution of the Schr\"odinger
  equation for some energy, denoted as $|\phi_{\alpha}\rangle$ and its
  energy derivative function $|\dot{\phi}_\alpha\rangle$.

  The matching parameters are called \textbf{potential
    parameters}\index{potential parameters}.
\end{enumerate}

%====================================================================
\section{Structure constants}
%====================================================================
%====================================================================
\subsection{Hankel functions as envelope function}
%====================================================================
In practice, we will use solid Hankel functions $H_L(\vec{r})$ as
envelope functions, so that 
\begin{eqnarray}
\langle\vec{r}|K^\infty_{R,L}\rangle= H_L(\vec{r}-\vec{R})
\end{eqnarray}

Solid Hankel functions are irregular solutions of the the
inhomogeneous Helmholtz equation\footnote{I am not sure whether also
  the three dimensional differential equation or only the
  one-dimensional differential equation for the radial part is called
  Helmholtz equation.}
\begin{eqnarray}
\Bigl[\vec{\nabla^2}+k^2\Bigr]H_{L}(\vec{r})
=-4\pi(-1)^\ell\mathcal{Y}(\vec{\nabla})\delta(\vec{r})
\label{eq:solidhelmholtzequation}
\end{eqnarray}
Here $\mathcal{Y}_\ell(\vec{r})=r^\ell Y_\ell(\vec{r})$ is a
polynomial. With a gradient as argument, it becomes a differential
operator.

Further detail about the Hankel and Bessel functions can be found in
appendix~\ref{app:solidhankel}.

%====================================================================
\subsection{Hankel and Bessel functions as head and tail functions}
%====================================================================
By defining the envelope function via a isotropic and translationally
invariant differential equation of second order, has the advantage
that the solution can be expanded about different centers into regular
solutions of the same differential equation with specific angular
momenta. The regular solutions of the Helmholtz equation are the
Bessel functions.


Hankel and Bessel functions are defined so that they behave at the
origin as
\begin{eqnarray}
K^\Omega_{R,L}(\vec{r})&=&
\Bigl[(2\ell-1)!! \frac{1}{|\vec{r}-\vec{R}|^{\ell+1}} 
+...\Bigr]
Y_L(\vec{r}-\vec{R})\theta_{\Omega_R}(\vec{r})
\\
J^\Omega_{R,L}(\vec{r})&=&
\biggl[\frac{1}{(2\ell+1)!!} |\vec{r}-\vec{R}|^{\ell+1} 
+...\Bigr]Y_L(\vec{r}-\vec{R})\theta_{\Omega_R}(\vec{r})
\end{eqnarray}
$\theta_{\Omega_R}(\vec{r})$ is a step function that is one within the
augmentation region $\Omega_R$ centered at site $R$, while it vanishes
outside. The terms neglected are higher orders in $|\vec{r}-\vec{R}|$.

%====================================================================
\subsection{Bare structure constants}
%====================================================================
The \textbf{bare structure constants}\index{structure constants !bare}
$S^\dagger_{\beta,\alpha}$ are the expansion constants for an
off-center expansion of solid spherical Hankel functions\index{Hankel
  function} $|K_{\alpha}^\infty\rangle$ into \textbf{solid Bessel
  functions}\index{Bessel function} $|J^\Omega_{\beta}\rangle$.
\begin{eqnarray}
|K_{\alpha}^\infty\rangle=|K^\Omega_{\alpha}\rangle
-\sum_{\beta}|J^\Omega_{\beta}\rangle S^\dagger_{\beta,\alpha}
+|K^I_{\alpha}\rangle
\end{eqnarray}
The index $\alpha$ denotes here an atomic site $R$ and a set of
angular momenta $L=(\ell,m)$.

The superscript $\infty$ denotes that the function extends over all
space, a superscript $\Omega$ denotes that the function is truncated
(set to zero) outside the augmentation sphere $\Omega_{R}$ centerd at
the site denoted by the index. The superscript $I$ denotes that the
function is limited to the interstitial region, that is outside all
augmentation spheres. If the augmentation spheres overlap, the
function in the interstitial region is defined by subtraction of all
sphere contributions.



\begin{myshadowminipage}{Bare structure constants}
The bare structure constants have the form
\begin{eqnarray}
S_{RL,R'L'}=(-1)^{\ell'+1} 4\pi \sum_{L''} C_{L,L',L''} 
H_{L''}(\vec{R}'-\vec{R})
\begin{cases}
(-ik)^{\ell+\ell'-\ell''}&\text{for $k^2>0$}\\
\delta_{\ell+\ell'-\ell''}&\text{for $k^2=0$}\\
\kappa^{\ell+\ell'-\ell''}&\text{for $k^2=-\kappa^2<0$}\\
\end{cases}
\label{eq:fortmulaforbarestructureconstants}
\end{eqnarray}
\end{myshadowminipage}


The bare structure constants are hermitean\footnote{We use that
  $H_L(\vec{r})=(-1)^\ell H_L(-\vec{r})$ and that the Gaunt
  coefficients $C_{L,L',L''}$ vanish unless $\ell+\ell'+\l''$ is
  even.}, i.e.
\begin{eqnarray}
S_{RL,R'L'}=S_{R'L',RL}
\end{eqnarray}
This is however not true for each angular-momentum block individually,
i.e. in general we have $S_{RL,R'L'}\neq S_{R,L',R'L}$.

%====================================================================
\subsection{Screened structure constants}
%====================================================================
The node-less scattering partial wave
$|\dot{\bar{\phi}}_\alpha\rangle$ define the screening constants
$\bar{Q}_{\alpha}$ such that the screened tail functions
$|\bar{J}_\alpha\rangle$ match with value and derivative to the
scattering partial wave
\begin{eqnarray}
|\dot{\bar{\phi}}_\alpha\rangle \rightarrow 
|\bar{J}^{\Omega}_{\alpha}\rangle
\defas
|J^{\Omega}_{\alpha}\rangle
-|K^{\Omega}_{\alpha}\rangle \bar{Q}_{\alpha}
\end{eqnarray}

A screened solid Hankel function $|\bar{K}_{\alpha}^\infty\rangle$ is
a superposition of bare solid Hankel functions on a set of atomic
positions
\begin{eqnarray}
|\bar{K}^\infty_\alpha\rangle=\sum_\beta|K^\infty_\beta\rangle c_{\beta,\alpha}
\label{eq:kbarassuperposofkbare}
\end{eqnarray}
with the property that the tail functions are made entirely
from screened Bessel functions $|\bar{J}^\Omega_{\beta}\rangle$, i.e.
\begin{eqnarray}
|\bar{K}_{\alpha}^\infty\rangle=|K^\Omega_{\alpha}\rangle
-\sum_{\beta}|\bar{J}^\Omega_{\beta}\rangle \bar{S}^\dagger_{\beta,\alpha}
+|\bar{K}^I_{\alpha}\rangle
\label{eq:kbarwithsbar}
\end{eqnarray}
The expansion coefficients $\bar{S}$ are the screened structure
constants\index{Structure constants !screened}.


By equating the two expressiones for the screened Hankel functions,
namely \eq{eq:kbarassuperposofkbare} and \eq{eq:kbarwithsbar}, we can
extract the screened structure constants and the superposition
coeffcients.
\begin{eqnarray}
\sum_\beta\Bigl[|K^\Omega_{\beta}\rangle
-\sum_{\gamma}|J^\Omega_{\gamma}\rangle S^\dagger_{\gamma,\beta}
+|K^I_{\beta}\rangle \Bigr]c_{\beta,\alpha}
=
|K^\Omega_{\alpha}\rangle
-\sum_{\beta}
\underbrace{\Bigl[|J^\Omega_{\beta}\rangle-|K^\Omega_{\beta}\rangle\bar{Q}_\beta\Bigr]}
_{|\bar{J}^\Omega_{\beta}\rangle} \bar{S}^\dagger_{\beta,\alpha}
+|\bar{K}^I_{\alpha}\rangle
\nonumber\\
\sum_\beta
|K^\Omega_{\beta}\rangle c_{\beta,\alpha}
-\sum_{\beta,\gamma}|J^\Omega_{\gamma}\rangle S^\dagger_{\gamma,\beta} c_{\beta,\alpha}
%+\sum_\beta|K^I_{\beta}\rangle c_{\beta,\alpha}
=
\sum_{\beta}|K^\Omega_{\beta}\rangle
\Bigl[\delta_{\beta,\alpha}+
\bar{Q}_\beta \bar{S}^\dagger_{\beta,\alpha}\Bigr]
-\sum_{\beta}|J^\Omega_{\beta}\rangle \bar{S}^\dagger_{\beta,\alpha}
%+|\bar{K}^I_{\alpha}\rangle
\end{eqnarray}
By comparing the coefficients, we obtain
\begin{eqnarray}
c_{\beta,\alpha}&=&\delta_{\beta,\alpha}+
\bar{Q}_\beta \bar{S}^\dagger_{\beta,\alpha}
\label{eq:definingeqsystemforsbara}
\\
\bar{S}^\dagger_{\gamma,\alpha}&=&\sum_\beta S^\dagger_{\gamma,\beta} c_{\beta,\alpha}
\label{eq:definingeqsystemforsbar}
\label{eq:definingeqsystemforsbarb}
\end{eqnarray}
which can be resolved to
\footnote{
\begin{eqnarray}
\mat{c}&=&\mat{1}
+\mat{\bar{Q}}\mat{\bar{S}}^\dagger
=\mat{1}+\mat{\bar{Q}}\mat{S}^\dagger\mat{c}
\qquad
\Rightarrow\qquad\sum_\gamma \Bigl[\delta_{\beta,\gamma}
-\bar{Q}_\beta S^\dagger_{\beta,\gamma}\Bigr] c_{\gamma,\alpha}=\delta_{\beta,\alpha}
\qquad\Rightarrow\qquad
\mat{c}=[\mat{1}-\mat{\bar{Q}}\mat{S}^\dagger]^{-1}
\nonumber\\
\mat{\bar{S}}^\dagger&=&\mat{S}^\dagger\mat{c}
=\mat{S}^\dagger[\mat{1}-\mat{\bar{Q}}\mat{S}^\dagger]^{-1}
\qquad\Leftrightarrow\qquad
[\mat{1}-\mat{S}\mat{\bar{Q}}]\mat{\bar{S}}=\mat{S}
\end{eqnarray}
  } the defining
equation of the screened structure constants
\begin{myshadowminipage}{screened structure constants}
\begin{eqnarray}
\mat{\bar{S}}^\dagger=\mat{S}^\dagger
\Bigl[\mat{1}-\mat{S}^\dagger\mat{\bar{Q}}\Bigr]^{-1}
\label{eq:defscreenedstructureconstants}
\end{eqnarray}
and the expression of the screened Hankel functions 
\begin{eqnarray}
|\bar{K}^\infty_\alpha\rangle=\sum_\beta |K^\infty_\beta\rangle
\Bigl[\delta_{\beta,\alpha}+
\bar{Q}_\beta \bar{S}^\dagger_{\beta,\alpha}\Bigr]
\;.
\end{eqnarray}

Because, we calculate the screened structure constants on finite
clusters, \eq{eq:defscreenedstructureconstants} should be considered
of only formal value and should not be used in the actual
calculations. Rather, the defining equations
\eq{eq:definingeqsystemforsbarb} shall be used as shown in the
following section.
\end{myshadowminipage}{}

%====================================================================
\section{Screening on finite clusters}
%====================================================================
The screened structure constants are calculated on a cluster of atomic
sites. The calculation can in principle be done for each single
screened Hankel function independently. In practice we do the
calculations for all atoms centered on a given site in one step.

We go back to the defining equation system
\eq{eq:definingeqsystemforsbar} and rewrite it in terms of vectors,
which are defined on the cluster $B$. The index $\alpha$ labeling the
vectors correspond to the envelope functions centered at the central
site.

The equations attain the form
\begin{eqnarray}
\vec{c}_\alpha&\eqrel{eq:definingeqsystemforsbara}{=}&
\vec{e}_\alpha+\mat{\bar{Q}}\vec{s}_\alpha
\label{eq:definingeqsystemforsbarveca}
\\
\vec{s}_\alpha&\eqrel{eq:definingeqsystemforsbarb}{=}&
\mat{S}^\dagger\vec{c}_\alpha
\label{eq:definingeqsystemforsbarvecb}
\label{eq:definingeqsystemforsbarvec}
\end{eqnarray}
where the vectors $\vec{c}$, $\vec{s}_\alpha$ and $\vec{e}_\alpha$ are
defined by its components
\begin{eqnarray}
\Bigl(\vec{c}_\alpha\Bigr)_\beta&=&c_{\beta,\alpha}
\nonumber\\
\Bigl(\vec{s}_\alpha\Bigr)_\beta&=&\bar{S}^\dagger_{\beta,\alpha}
\nonumber\\
\Bigl(\vec{e}_\alpha\Bigr)_\beta&=&\delta_{\beta,\alpha}
\end{eqnarray}


\begin{eqnarray}
\vec{c}_\alpha
&\eqrel{eq:definingeqsystemforsbarveca}{=}&
\vec{e}_\alpha+\mat{\bar{Q}}\vec{s}_\alpha
\eqrel{eq:definingeqsystemforsbarvecb}{=}
\vec{e}_\alpha+\mat{\bar{Q}}\mat{S}^\dagger\vec{c}_\alpha
\nonumber\\
\Rightarrow\qquad
\Bigl[\mat{1}-\mat{\bar{Q}}\mat{S}^\dagger\Bigr]\vec{c}_\alpha&=&\vec{e}_\alpha
\nonumber\\
\Rightarrow\qquad
\vec{c}_\alpha&=&[\mat{1}-\mat{\bar{Q}}\mat{S}^\dagger]^{-1}\vec{e}_\alpha
\nonumber\\
\vec{s}_\alpha
&\eqrel{eq:definingeqsystemforsbarvecb}{=}&
\mat{S}^\dagger\vec{c}_\alpha
=\mat{S}^\dagger[\mat{1}-\mat{\bar{Q}}\mat{S}^\dagger]^{-1}\vec{e}_\alpha
\end{eqnarray}
Interestingly the vector on the right-hand side $\vec{e}_\alpha$ can
not be simply ignored as the matrix form suggests. This is specific to
the calculation on the cluster. Because of this we cannot identify the
contribution of these vectors with a unit matrix.



\begin{myshadowminipage}{Calculation of screened structure constants}
Thus, we first evaluate the bare structure constants $\mat{S}^\dagger$
on the cluster, and from that
$[\mat{1}-\mat{\bar{Q}}\mat{S}^\dagger]$. Then we solve the equation
\begin{eqnarray}
[\mat{1}-\mat{\bar{Q}}\mat{S}^\dagger]\vec{c}_\alpha&=&\vec{e}_\alpha
\label{eq:forscreenedsa}
\nonumber\\
\vec{s}_\alpha&=&\mat{S}^\dagger\vec{c}_\alpha
\label{eq:forscreenedsb}
\end{eqnarray}
for $\vec{c}_\alpha$ first using a standard routine for linear
equation systems. From the result $\vec{c}_\alpha$, we obtain the
screened structure constants $\vec{s}_\alpha$ by multiplication with
the screened structure constants.

Finally we obtain the screened structure constants as
\begin{eqnarray}
\bar{S}^\dagger_{\gamma,\alpha}=\Bigl(\vec{s}_\alpha\Bigr)_\gamma
\end{eqnarray}
\end{myshadowminipage}

Note that the screened structure constants calculated on finite
clusters are no more exactly hermitean.



%====================================================================
\section{Augmentation and Potential parameters}
%====================================================================
%===============================================================================
\subsection{Local orbitals}
%===============================================================================
The local orbitals have the form
\begin{eqnarray}
|\chi_\alpha\rangle&=&|\phi^K_\alpha\rangle
- |\phi^{\bar{J}}_{R,L}\rangle \bar{S}^\dagger_{R,L,R_\alpha,L_\alpha}
\nonumber\\
&+&|K^I_{R',L'}\rangle\Bigl[ \delta_{R',L',R_\alpha,\L_\alpha}-\bar{Q}_{R',L'}
\bar{S}^\dagger_{R',L',R_\alpha,L_\alpha}\Bigr]
\end{eqnarray}
where, according to \eq{eq:tborbinlmto},
\begin{eqnarray}
|\phi^K_\alpha\rangle&=&
\overbrace{
|\phi_\alpha\rangle 
\underbrace{
\frac{W_\alpha[K,\dot{\bar{\phi}}]}{W_\alpha[\phi,\dot{\bar{\phi}}]}}_{Ktophi}
-\,|\dot{\bar{\phi}}_\alpha\rangle 
\underbrace{\frac{W_\alpha[K,\phi]}{W_\alpha[\phi,\dot{\bar{\phi}}]}}_{-Ktophidot}
}^{\rightarrow |K^\Omega_\alpha\rangle}
\nonumber\\
|\phi^{\bar{J}}_{R,L}\rangle
&=&\overbrace{
|\dot{\bar{\phi}}_\beta\rangle 
\underbrace{\biggl(-\frac{W_\beta[\bar{J},\phi]}{W_\beta[\phi,\dot{\bar{\phi}}]}\biggr)}_{JBARtophidot}
}^{\rightarrow |\bar{J}\;^\Omega_\beta\rangle}
\end{eqnarray}
Note, that in the factor $JBARTOPHIDOT$ does not depend on the choice
of $|\phi\rangle$.

Thus, the matrix elements
$\langle\tilde{p}_\gamma|\tilde{\chi}_\alpha\rangle$ has the form
\begin{eqnarray}
\langle\tilde{p}_\gamma|\tilde{\chi}_\alpha\rangle
=\langle\tilde{p}_\gamma|\tilde{\phi}^K_\alpha\rangle
-\sum_{R',L'}\langle\tilde{p}_\gamma|\tilde{\phi}^{\bar{J}}_{R',L'}\rangle
\bar{S}^\dagger_{R,L,R_\alpha,L_\alpha}
\nonumber\\
=\langle\tilde{p}_\gamma|\tilde{\phi}^K_\alpha\rangle
-\langle\tilde{p}_\gamma|\tilde{\phi}^{\bar{J}}_{R_\gamma,L_\gamma}\rangle
\bar{S}^\dagger_{R_\gamma,L_\gamma,R_\alpha,L_\alpha}
\end{eqnarray}
%====================================================================
\section{Coefficients of the tight-binding orbital}
%====================================================================
%===============================================================================
\subsection{Introduction}
%===============================================================================
In this section we describe how to determine the wave functions in
terms of local orbitals, if the projections onto the pseudo wave
functions are known.

The basic idea is to find a representation of the wave function in
local orbitals
\begin{eqnarray}
|\psi'_n\rangle=\sum_\alpha |\chi_\alpha\rangle q_\alpha\;,
\end{eqnarray}
such that the deviation from the true wave function $|\psi_n\rangle$
is as small as possible.

Ideally, this would amount to minimizing the mean square deviation of
the orbital expansion from the wave function.
\begin{eqnarray*}
Q'[\vec{q}]:=\Bigl(\langle\psi_n|-\sum_\alpha q^*_\alpha\langle\chi_\alpha|\Bigr)
\Bigl(|\psi_n\rangle-\sum_\beta |\chi_\beta\rangle q_\beta\Bigr)
\end{eqnarray*}

Because evaluating the mean square deviation as integral over all
space is time consuming, we limit the integral to the augmentation
spheres.
\begin{eqnarray}
Q[\vec{q}]&:=&
\Bigl(\langle\tilde{\psi}_n|
-\sum_\alpha q^*_\alpha\langle\tilde{\chi}_\alpha|\Bigr)
\biggl[\sum_{\delta,\gamma}|\tilde{p}_\delta\rangle\langle\phi_\delta|
\theta_{\Omega_{R_\delta}}|\phi_\gamma\rangle\langle\tilde{p}_\gamma|\biggr]
\Bigl(|\tilde{\psi}_n\rangle
-\sum_\beta |\tilde{\chi}_\beta\rangle q_\beta\Bigr)
\nonumber\\
&=&
\sum_{\gamma}
\biggl[
\sum_{\delta}
\Bigl(\langle\tilde{\psi}_n|\tilde{p}_\delta\rangle
-\sum_\alpha q^*_\alpha\langle\tilde{\chi}_\alpha|\tilde{p}_\delta\rangle\Bigr)
\langle\phi_\delta|\theta_{\Omega_{R_\delta}}|\phi_\gamma\rangle
\biggr]
\Bigl(\langle\tilde{p}_\gamma|\tilde{\psi}_n\rangle
-\sum_\beta \langle\tilde{p}_\gamma|\tilde{\chi}_\beta\rangle q_\beta\Bigr)
\end{eqnarray}
where $\theta_{\Omega_{R_\delta}}$ is a step function that vanishes
outside the augmentation sphere at $R_\delta$.

Minimization yields
\begin{eqnarray}
\frac{\partial Q}{\partial q^*_\alpha}
&=&
-\sum_{\gamma}
\biggl[
\sum_{\delta}
\langle\tilde{\chi}_\alpha|\tilde{p}_\delta\rangle
\langle\phi_\delta|\theta_{\Omega_{R_\delta}}|\phi_\gamma\rangle
\biggr]
\Bigl(\langle\tilde{p}_\gamma|\tilde{\psi}_n\rangle
-\sum_\beta \langle\tilde{p}_\gamma|\tilde{\chi}_\beta\rangle q_\beta\Bigr)
\stackrel{!}{=}0
\nonumber
\end{eqnarray}


\begin{eqnarray}
\Rightarrow\qquad\sum_{\gamma}
\biggl[
\sum_{\delta}
\langle\tilde{\chi}_\alpha|\tilde{p}_\delta\rangle
\langle\phi_\delta|\theta_{\Omega_{R_\delta}}|\phi_\gamma\rangle
\biggr]
\langle\tilde{p}_\gamma|\tilde{\psi}_n\rangle
=
\sum_{\gamma,\beta}
\biggl[
\sum_{\delta}
\langle\tilde{\chi}_\alpha|\tilde{p}_\delta\rangle
\langle\phi_\delta|\theta_{\Omega_{R_\delta}}|\phi_\gamma\rangle
\biggr]
\langle\tilde{p}_\gamma|\tilde{\chi}_\beta\rangle q_\beta\Bigr)
\nonumber
\end{eqnarray}

\begin{eqnarray}
\Rightarrow\qquad
q_\beta=
\sum_{\beta}
\biggl[
\sum_{\gamma',\delta'}
\langle\tilde{\chi}_\alpha|\tilde{p}_{\delta'}\rangle
\langle\phi_{\delta'}|\theta_{\Omega_{R_\delta}}|\phi_{\gamma'}\rangle
\langle\tilde{p}_{\gamma'}|\tilde{\chi}_\beta\rangle 
\biggr]^{-1}
\biggl[
\sum_{\gamma\delta}
\langle\tilde{\chi}_\alpha|\tilde{p}_\delta\rangle
\langle\phi_\delta|\theta_{\Omega_{R_\delta}}|\phi_\gamma\rangle
\biggr]
\langle\tilde{p}_\gamma|\tilde{\psi}_n\rangle
\nonumber\\
\end{eqnarray}


This allows one to write the wave function in the form
\begin{eqnarray}
|\psi_n\rangle\approx
\sum_\alpha|\chi_\alpha\rangle\langle\tilde{\pi}_\alpha|\tilde{\psi}_n\rangle
\end{eqnarray}
with
\begin{eqnarray}
\langle\tilde{\pi}_\alpha|=
\sum_\gamma \biggl[
\sum_{\gamma',\delta'}
\langle\tilde{\chi}_\alpha|\tilde{p}_{\delta'}\rangle
\langle\phi_{\delta'}|\theta_{\Omega_{R_\delta}}|\phi_{\gamma'}\rangle
\langle\tilde{p}_{\gamma'}|\tilde{\chi}_\beta\rangle 
\biggr]^{-1}
\biggl[
\sum_{\delta}
\langle\tilde{\chi}_\alpha|\tilde{p}_\delta\rangle
\langle\phi_\delta|\theta_{\Omega_{R_\delta}}|\phi_\gamma\rangle
\biggr]
\langle\tilde{p}_\gamma|
\label{eq:pitilde}
\end{eqnarray}

This expression works also if the number of local orbitals
$|\chi_\alpha\rangle$ is smaller than the number of projector
functions $\langle{p}_\gamma|$. Because of the inversion, this
expression needs to be evaluated in reciprocal space.

%===============================================================================
\subsection{Transformation between local-orbital and partial-wave projections}
%===============================================================================
In the previous section we derived in \eq{eq:pitilde} a relation
between orbital and partial wave projector functions.
\begin{eqnarray}
\langle\tilde{\pi}_\alpha|\tilde{\psi}_n\rangle
&=&\sum_\beta M_{\alpha,\beta}\langle\tilde{p}_\alpha|\tilde{\psi}_n\rangle
\end{eqnarray}
This operation is performed in \verb|lmto$projtontbo| with \verb|ID='FWRD'|




The derivatives are correspondingly derived as 
\begin{eqnarray}
dE&=&\sum_{\alpha,\beta}\underbrace{\frac{dE}{d\rho_{\alpha,\beta}}}
_{=:h_{\beta,\alpha}}d\rho_{\alpha,\beta}
\nonumber\\
&=&\sum_{\alpha,\beta}h_{\beta,\alpha}
\Bigl[
\sum_n\langle\pi_\alpha|d\psi_n\rangle f_n\langle\psi_n|\pi_\beta\rangle
+\sum_n\langle\pi_\alpha|\psi_n\rangle f_n\langle{d}\psi_n|\pi_\beta\rangle
\Bigr]
\nonumber\\
&=&\sum_n\sum_{\alpha}
f_n
\underbrace{\sum_{\beta}
\langle\psi_n|\pi_\beta\rangle h_{\beta,\alpha}}_{HTBC^\dagger_{n,\alpha}}
\langle\pi_\alpha|d\psi_n\rangle 
+
\sum_n\sum_{\beta}
\langle{d}\psi_n|\pi_\beta\rangle 
\underbrace{\sum_{\alpha}h_{\beta,\alpha} \langle\pi_\alpha|\psi_n\rangle}_{HTBC_{\beta,n}} f_n
\nonumber\\
&=&\sum_n\sum_{\gamma}
f_n
\underbrace{\sum_{\alpha}
\underbrace{\sum_{\beta}
\langle\psi_n|\pi_\beta\rangle h_{\beta,\alpha}}_{HTBC^\dagger_{n,\alpha}}
M_{\alpha,\gamma}}_{HPROJ^\dagger_{n,\gamma}}
\langle\tilde{p}_\gamma|d\tilde{\psi}_n\rangle 
\nonumber\\
&+&
\sum_n\sum_{\gamma}
\langle{d}\tilde{\psi}_n|\tilde{p}_\gamma\rangle 
\underbrace{
\sum_{\beta}M^\dagger_{\gamma,\beta}
\underbrace{\sum_{\alpha}h_{\beta,\alpha} \langle\pi_\alpha|\psi_n\rangle}_{HTBC_{\beta,n}}}_{HPROJ_{\gamma,n}} f_n
\end{eqnarray}

Thus, we first define the Hamiltonian $\mat{h}$ (HAMIL) 
\begin{eqnarray}
\underbrace{h_{\alpha,\beta}}_{HAMIL}&=&\frac{dE}{\rho_{\beta,\alpha}}
\nonumber\\
HTBC_{\beta,n}&=&
\sum_\alpha \underbrace{h_{\beta,\alpha}}_{HAMIL}
\underbrace{\langle\tilde{\pi}_\alpha|\tilde{\psi}_n\rangle}_{TBC_{\alpha,n}}
\nonumber\\
HPROJ_{\gamma,n}&=&\sum_\beta M^\dagger_{\gamma,\beta}\cdot HTBC_{\beta,n}
\end{eqnarray}


This operation is performed in
\verb|lmto$projtontbo| with \verb|ID='BACK'|.




%====================================================================
\section{Core-valence exchange}
%====================================================================
The exchange term between core and valence electrons acts like a
fixed, nonlocal potential acting on the electrons, of the form
\begin{eqnarray}
\hat{\tilde{v}}_{x,cv}=\sum_{\alpha,\beta}|\tilde{p}_\alpha\rangle 
M_{\alpha,\beta}\langle\tilde{p}_\beta|
\end{eqnarray}
The core-valence exchange is furthermore diagonal in the site indices.

\begin{eqnarray}
\langle\chi_\alpha|\hat{v}_{x,cv}|\chi_\beta\rangle
&=&
\sum_{\gamma,\delta}
\langle\chi_\alpha|p_\gamma\rangle 
M_{\gamma,\delta}\langle p_\delta|\chi_\beta\rangle
\nonumber\\
&=&
\sum_{\gamma,\delta}
\langle\tilde{\phi}^{K}_\alpha|\tilde{p}_\gamma\rangle 
M_{\gamma,\delta}
\langle\tilde{p}_\delta|\tilde{\phi}^{K}_\beta\rangle
\nonumber\\
&-&
\sum_{\gamma,\delta,\beta'}
\langle\tilde{\phi}^{K}_\alpha|\tilde{p}_\gamma\rangle 
M_{\gamma,\delta}
\langle\tilde{p}_\delta|\tilde{\phi}^{\bar{J}}_{\beta'}\rangle 
\bar{S}^\dagger_{\beta',\beta}
\nonumber\\
&-&
\sum_{\gamma,\delta,\alpha',\alpha}
\bar{S}_{\alpha,\alpha'}
\langle\tilde{\phi}^{\bar{J}}_{\alpha'}|\tilde{p}_\gamma\rangle 
M_{\gamma,\delta}
\langle\tilde{p}_\delta|\tilde{\phi}^{K}_{\beta}\rangle 
\nonumber\\
&+&
\sum_{\gamma,\delta,\alpha',\alpha}
\bar{S}_{\alpha,\alpha'}
\langle\tilde{\phi}^{\bar{J}}_{\alpha'}|\tilde{p}_\gamma\rangle 
M_{\gamma,\delta}
\langle\tilde{p}_\delta|\tilde{\phi}^{\bar{J}}_{\beta'}\rangle 
\bar{S}^\dagger_{\beta',\beta}
\end{eqnarray}
Here we used the augmented Hankel and screened Bessel fucntions,
respectively their pseudo versions.

As usual we build the expanded density matrix
\begin{eqnarray}
\left(\begin{array}{cc}
  \mat{\rho} \qquad& 
-\mat{\rho}\mat{\bar{S}}^\dagger\\
-\mat{\bar{S}}\mat{\rho} \qquad& 
\mat{\bar{S}}\mat{\rho}\mat{\bar{S}}^\dagger\\
\end{array}\right)
\end{eqnarray}

The matrix 
\begin{eqnarray}
\left(\begin{array}{cc}
\langle\tilde{\phi}^{K}|\tilde{p}\rangle 
\mat{M}\langle\tilde{p}|\tilde{\phi}^{K}\rangle &
\langle\tilde{\phi}^{K}|\tilde{p}\rangle 
\mat{M}\langle\tilde{p}|\tilde{\phi}^{\bar{J}}\rangle \\
\langle\tilde{\phi}^{\bar{J}}|\tilde{p}\rangle 
\mat{M}\langle\tilde{p}|\tilde{\phi}^{K}\rangle &
\langle\tilde{\phi}^{\bar{J}}|\tilde{p}\rangle 
\mat{M}\langle\tilde{p}|\tilde{\phi}^{\bar{J}}\rangle \\
\end{array}\right)
\end{eqnarray}
is calculated first using \verb|potpar1(isp)%prok| and
\verb|potpar1(isp)%projbar|.
\footnote{ In the earlier version the contribution from the
  $\dot{\bar{\phi}}$ has been ignored!!! It has been verified by
  temporarily switching off the jbar contributiuon to
  potpar1(isp)\%prok and potpar1(isp)\%projbar. In this old version
  only potpar(isp)\%ktophi is used to extract the $\phi$
  contribution.}







%====================================================================
\chapter{Description of Subroutines}
%====================================================================
%====================================================================
\section{Workflow}
%====================================================================
\begin{verbatim}
---initialization-----
POTPAR = potential parameters
SBAR = screened structureconstants
<ptilde|chitilde>
tailed partial waves
  overlap (Onsite) 
  utensor (Onsite)
  utensor (offsite)
...
----cycle-----------
TBC=<pi-tilde|psi> from PROJ=<ptilde|psitide>
DENMAT density matrix in local orbitals
...
total energy and derivatives
HAMIL hamiltonian matrix in tight-binding orbitals
...
HTBC = de/dtbc * 1/f
HPROJ = de/dproj * 1/f
\end{verbatim}

%====================================================================
\section{LMTO\$CLUSTERSTRUCTURECONSTANTS}
%====================================================================
\verb|LMTO\$CLUSTERSTRUCTURECONSTANTS| calculates the screened
structure constants \verb|SBAR| ($\mat{\bar{S}}$) for a cluster of
\verb|NAT| atomic sites \verb|RPOS|, of which the first site is called
the central site of the cluster. The number of angular momenta on each
site is defined by \verb|LX|. The screening is defined by the vector
\verb|QBAR| ($\bar{Q}$). \verb|K2| ($\vec{k}^2=-\kappa^2$) is the
squared wave vector. (For envelope functions that fall off
exponentially, this parameter is negative.)

\begin{verbatim}
SUBROUTINE LMTO$CLUSTERSTRUCTURECONSTANTS(K2,NAT,RPOS,LX,QBAR,NORB,N,SBAR)
REAL(8)   ,INTENT(IN) :: K2          
INTEGER(4),INTENT(IN) :: NAT         ! NUMBER OF ATOMS ON THE CLUSTER
REAL(8)   ,INTENT(IN) :: RPOS(3,NAT) ! ATOMIC POSITIONS ON THE CLUSTER
INTEGER(4),INTENT(IN) :: LX(NAT)     ! X(ANGULAR MOMENTUM ON EACH CLUSTER)
INTEGER(4),INTENT(IN) :: N
REAL(8)   ,INTENT(IN) :: QBAR(N)
INTEGER(4),INTENT(IN) :: NORB
REAL(8)   ,INTENT(INOUT):: SBAR(NORB,N)
\end{verbatim}




First, the bare structure constants are evaluates on the cluster using
\verb|LMTO\$STRUCTURECONSTANTS| and then the structure constants are
screened using \verb|LMTO\$SCREEN|.

%====================================================================
\subsection{LMTO\$STRUCTURECONSTANTS}
%====================================================================
\verb|LMTO\$STRUCTURECONSTANTS| calculates the bare structure
constants for a pair of sites. The first site is at the origin, where
the Hankel function is centered, and the second site at $\vec{R}$
specified by \verb|R21|, is the center of the expansion into solid
Bessel functions.

\begin{verbatim}
subroutine lmto$structureconstants(r21,K2,L1x,L2x,S)
REAL(8)   ,INTENT(IN) :: R21(3) ! EXPANSION CENTER
INTEGER(4),INTENT(IN) :: L1X
INTEGER(4),INTENT(IN) :: L2X
REAL(8)   ,INTENT(IN) :: K2 ! 2ME/HBAR**2
REAL(8)   ,INTENT(OUT):: S((L1X+1)**2,(L2X+1)**2)
\end{verbatim}

The bare structure constants are evaluated in 
\verb|LMTO$STRUCTURECONSTANTS| as
\begin{eqnarray}
S_{RL,R'L'}\eqrel{eq:fortmulaforbarestructureconstants}{=}
(-1)^{\ell'+1} 4\pi \sum_{L''} C_{L,L',L''} 
H_{L''}(\vec{R}'-\vec{R})
\begin{cases}
(-ik)^{\ell+\ell'-\ell''}&\text{for $k^2>0$}\\
\delta_{\ell+\ell'-\ell''}&\text{for $k^2=0$}\\
\kappa^{\ell+\ell'-\ell''}&\text{for $k^2=-\kappa^2<0$}\\
\end{cases}
\label{eq:fortmulaforbarestructureconstantscopy1}
\end{eqnarray}
where $H_L(k^2,\vec{R})$ is the solid Hankel function calculated in
\verb|LMTO$SOLIDHANKEL|. The solid Hankel function is the solution of
the Helmholtz equation, \eq{eq:solidhelmholtzequation}.\footnote{The
  factors and signs of the inhomogeneity need to be confirmed. The
  equation has been taken from the methods book, chapter
  \textit{``Working with spherical Hankel and Bessel functions.}}

More information on the solid Hankel function can be found in
appendix~\ref{app:solidhankel}.

Remark: Because the Gaunt coefficients vanish for odd
$\ell+\ell'-\ell''$, the structure constants are real even for
$k^2>0$.


%====================================================================
\subsection{LMTO\$SCREEN}
%====================================================================
\textit{I describe here what has been implemented as ``version 3''.}

\verb|LMTO$SCREEN| takes the bare structure constants $S_{RL,R'L'}$
connecting all orbitals on a specific cluster with each other and the
screening constants $\bar{Q}$ for all orbitals on the cluster. It
returns the screened structure constants connecting the orbitals on
the central (first) site (1st index) with all orbitals (2nd index).

The structure constants are defined so that
\begin{eqnarray}
\langle K_{RL}|=-\sum_{L'} S_{RL,R'L'}\langle J_{R'L'}|
\qquad\text{for $R'\neq R$}
\end{eqnarray}

First we evaluate 
\begin{eqnarray}
\mat{A}=\mat{1}-\mat{\bar{Q}}\mat{S}^\dagger
\end{eqnarray}
and the vectors $\vec{e}_\alpha$ defined by
$(\vec{e}_\alpha)_\beta=\delta_{\beta,\alpha}$. Note that the number
of vectors corresponds to the number of orbitals on the central site
only. Therefore, these vectors do not build up a complete unit matrix.

Then we solve the equation system
\begin{eqnarray}
\mat{A}\vec{c}_\alpha
&\eqrel{eq:forscreenedsa}{=}&
\vec{e}_\alpha
\end{eqnarray}
for $\vec{c}_\alpha$ and
\begin{eqnarray}
\vec{s}_\alpha&\eqrel{eq:forscreenedsb}{=}&
\mat{S}^\dagger\vec{c}_\alpha
\end{eqnarray}
$\Bigl(\vec{s}_\alpha\Bigr)_\beta=\bar{S}^\dagger_{\beta,\alpha}$
contains the transposed screened structure constants. After
transposition, $\bar{S}$ is returned.

%====================================================================
\section{Waves object}
%====================================================================
The data exchange betweeen the waves object and the lmto object is
determined by the local-orbital projections
$\langle\tilde{\pi}_\alpha|\tilde{\psi}_n\rangle$ specified by the
array \verb|THIS%TBC|, which in turn is obtained from the partial-wave
projections $\langle\tilde{p}|\tilde{\psi}_n\rangle$.



In \verb|waves$etot|
\begin{verbatim}
CALL WAVES$TONTBO
-> CALL LMTO$PROJTONTBO('FWRD'...)
..
..
CALL LMTO$ETOT(LMNXX,NDIMD,NAT,DENMAT)
..
..
CALL WAVES$FROMNTBO()
-> CALL LMTO$PROJTONTBO('BACK'...)
..
..
CALL WAVES$FORCE
-> CALL WAVES_FORCE_ADDHTBC
...
CALL WAVES$HPSI
\end{verbatim}


\begin{eqnarray*}
\vec{F}&=&
-\sum_\alpha
\frac{dE}{d\langle\tilde{p}_\alpha|\psi_n\rangle}
\langle\vec{\nabla}_R\tilde{p}_\alpha|\psi_n\rangle
+\mathrm{c.c.}
\nonumber\\
&=&-\sum_{\alpha,\beta}
\frac{dE}{d\langle\tilde{\pi}_\beta|\psi_n\rangle}
\frac{d\langle\tilde{\pi}_\beta|\psi_n\rangle}
{d\langle\tilde{p}_\alpha|\psi_n\rangle}
\langle\vec{\nabla}_R\tilde{p}_\alpha|\psi_n\rangle
+\mathrm{c.c.}
\nonumber\\
&=&-\sum_{\alpha,\beta}
\frac{dE}{d\langle\tilde{\pi}_\beta|\psi_n\rangle}
\frac{d\langle\tilde{\pi}_\beta|\psi_n\rangle}
{d\langle\tilde{p}_\alpha|\psi_n\rangle}
\Bigl[-\langle\vec{\nabla}_r\tilde{p}_\alpha|\psi_n\rangle\Bigr]
+\mathrm{c.c.}
\end{eqnarray*}





\appendix
%=======================================================================
\chapter{Definition of solid Hankel functions}
\label{app:solidhankel}
%=======================================================================
The solid Hankel function has the form
\begin{eqnarray}
H_L(\vec{R})=Y_L(\vec{R})
\begin{cases}
n_\ell(\sqrt{k^2}\cdot|\vec{R}|) \cdot \sqrt{k^2}^{\ell+1}
&\text{for $k^2>0$  (Abramovitz 10.1.26)}\\
m_\ell(\sqrt{-k^2}\cdot|\vec{R}|) \cdot \sqrt{\frac{2}{\pi}} \sqrt{-k^2}^{\ell+1}
&\text{for $k^2<0$  (Abramovitz 10.2.4)}\\
(2\ell-1)!! |\vec{R}|^{-\ell-1} 
&\text{for $k^2=0$  (Abramovitz 10.2.5)}\\
\end{cases}
\end{eqnarray}
The solid Hankel function is defined such that the boundary conditions
at the origin are independent of $k^2$.

\begin{itemize}
\item the function
\begin{eqnarray}
n_\ell(r)=r^\ell\Bigl(-\frac{1}{r}\partial_r\Bigr)^\ell\frac{1}{r}\cos(r)
\label{eq:defneumann}
\end{eqnarray}
is the spherical Neumann function (see Eq. 8.175 of Cohen Tannoudhi
Band 2), which is also called the spherical Bessel function of the
second kind. Abramowitz defines $n_\ell(r)=-y_\ell(r)$ (compare
Abramowitz Eq. 10.1.26)

The spherical Neumann function obeys the radial Helmholtz equation
(Abramowitz Eq. 10.1.1) for positive kinetic energy
\begin{eqnarray}
r^2\partial^2_r n_\ell+2r\partial_r n_\ell+\Bigl(r^2-\ell(\ell+1)\Bigr) n_\ell=0
\nonumber\\
\Rightarrow\quad
\Bigl[-\frac{1}{r}\partial_r r +\frac{\ell(\ell+1)}{r^2}\Bigr]
n_\ell(r)=+n_\ell(r)
\end{eqnarray}

\textbf{Note that the subroutine SPFUNCTION\$NEUMANN returns the
  Neumann function with the opposite sign, namely what Abramowitz
  defines as Bessel function of the second kind. The minus sign is
  added in the calling routine.}
%
\item The function 
\begin{eqnarray}
m_\ell(r)=r^\ell\Bigl(-\frac{1}{r}\partial_r\Bigr)^\ell\frac{1}{r}\e{-r}
\label{eq:defml}
\end{eqnarray}
used for $k^2<0$ is obeys the radial Helmholtz equation (Abramowitz
Eq. 10.2.1) for negative kinetic energy
\begin{eqnarray}
r^2\partial^2_r m_\ell
+2r\partial_r m_\ell
-\Bigl(r^2+\ell(\ell+1)\Bigr) m_\ell=0
\nonumber\\
\Rightarrow\quad
\Bigl[-\frac{1}{r}\partial_r r +\frac{\ell(\ell+1)}{r^2}\Bigr]
m_\ell(r)=-m_\ell(r)
\end{eqnarray}
They are solutions for negative energy and therefore they fall off
exponentially.  The solution $m_\ell(r)$ is proportional to the
modified spherical Bessel functions of the third kind as defined by
Abramowitz\cite{abramowitz64_book} in their Eq. 10.2.4.
\begin{eqnarray}
m_\ell(r)=\frac{2}{\pi} \Bigl[\sqrt{\frac{\pi}{2r}} K_{\ell+1}(r)\Bigr]
\end{eqnarray}
which can be verified by comparing the defining equation \eq{eq:defml}
with equations 10.2.24-25 and the definition Eq. 10.2.4 of Abramowitz.
\end{itemize}


%=======================================================================
\section{Bare structure constants}
%=======================================================================
\textit{ This section is copied from Methods-book, Section ``Working
  with spherical Hankel and Bessel functions'', Peter Bl\"ochl, private
  communication.}

The bare structure constants have been determined first by
Segall\cite{segall57_pr105_108}. He uses the
theorem\cite{korringa47_physica13_392} that supposedy goes back to
Kasterin (N. Kasterin, Proc. Acad. Sci Amsterdam 6, 460 (1897/98));
see Seegall\cite{segall57_pr105_108}, Eq. B4)
\begin{eqnarray}
h_\ell^{(1)}(\kappa r)Y_{\ell,m}(\vec{r})=i^{-\ell}
\mathcal{Y}_{\ell,m}(\vec{\nabla}_r)h_0^{(1)}(\kappa r)
\label{eq:hlfromh0}
\end{eqnarray}
where $h^{(1)}_\ell(x)$ i the spherical Hankel function of the first
kind (see \eq{eq:defsphericalhankelfirstkind} below)
and where (Eq. B5 of Segall\cite{segall57_pr105_108})
\begin{eqnarray}
\mathcal{Y}_{\ell,m}(\vec{\nabla})
&=&\sqrt{\frac{2\ell+1}{4\pi}\frac{(\ell-m)!}{(\ell+m)!}}
\biggl(\frac{1}{ik}\biggr)^{|m|}\biggl(\partial_x\pm i\partial_y\biggr)
\mathcal{P}_\ell^{|m|}\left(\frac{1}{ik}\partial_z\right)
\end{eqnarray}
where the positive sign applies for nonzero $m$ and the negative sign
for negative $m$. Furthermore (see Segall\cite{segall57_pr105_108}
Eq.~B5)
\begin{eqnarray*}
\mathcal{P}_\ell^{|m|}(z)=\frac{d^{|m|}P_\ell(z)}{dz^{|m|}}
\end{eqnarray*}
where $P_\ell(z)$ is the conventional Legendre polynomial.

In addition Segall\cite{segall57_pr105_108} refers in his Eq. B7 to
Morse and Feshbach\cite{morse53_book} (part II, p. 1574) for
\begin{eqnarray}
h_0^{(1)}(\kappa|\vec{r}-\vec{r'}|)&=&
4\pi\sum_L \biggl( h^{(1)}_\ell(\kappa|\vec{r'}|)Y_L(\vec{r'})\biggr)
j_\ell(\kappa|\vec{r}|)Y_L^*(\vec{r})
\label{eq:structureconstant0}
\end{eqnarray}
which is valid for $|\vec{r'}|>|\vec{r}|$.

The two equations, \eq{eq:hlfromh0} and \eq{eq:structureconstant0},
can be combined into
\begin{eqnarray*}
h_\ell^{(1)}(\kappa |\vec{r}|)Y_{\ell,m}(\vec{r})
&\eqrel{eq:hlfromh0}{=}&
i^{-\ell}\mathcal{Y}_{\ell,m}(\vec{\nabla}_r)h_0^{(1)}(\kappa |\vec{r}|)
\\
&=&
i^{-\ell}\mathcal{Y}_{\ell,m}(\vec{\nabla}_r)
h_0^{(1)}(\kappa |(\vec{r}-\vec{R})+\vec{R}|)
\\
&\eqrel{eq:structureconstant0}{=}&
i^{-\ell}\mathcal{Y}_{\ell,m}(\vec{\nabla}_r)
\biggl[
4\pi\sum_{L'}\biggl( h^{(1)}_{\ell'}(\kappa|\vec{R}|)Y_{L'}(-\vec{R})\biggr)
j_{\ell'}(\kappa|\vec{r}-\vec{R}|)Y_{L'}^*(\vec{r}-\vec{R})
\biggr]
\\
&\eqrel{eq:helper1forprestructureconstants}{=}&
4\pi\sum_{L'}\biggl(i^{-\ell}
\mathcal{Y}_{\ell,m}(\vec{\nabla}_R) 
h^{(1)}_{\ell'}(\kappa|\vec{R}|)Y_{L'}(-\vec{R})\biggr)
j_{\ell'}(\kappa|\vec{r}-\vec{R}|)Y_{L'}^*(\vec{r}-\vec{R})
\end{eqnarray*}
Here we used that 
\begin{eqnarray}
\vec{\nabla}_r \bigl[f(\vec{R})g(\vec{r}-\vec{R})\bigr]
&=&f(\vec{R})\vec{\nabla}_rg(\vec{r}-\vec{R})
=-f(\vec{R})\vec{\nabla}_Rg(\vec{r}-\vec{R})
\\
&=&-\underbrace{\vec{\nabla}_R\overbrace{\bigl[f(\vec{R})g(\vec{r}-\vec{R})\bigr]
}^{h(\vec{r})}}_{=0}
+\Bigl[\vec{\nabla}_Rf(\vec{R})\Bigr]g(\vec{r}-\vec{R})\bigr]
\label{eq:helper1forprestructureconstants}
\end{eqnarray}

We summarize the final result
\begin{myshadowminipage}{Condition for structure constants (positive energies)}
\begin{eqnarray}
h_\ell^{(1)}(\kappa |\vec{r}|)Y_{\ell,m}(\vec{r})
&=&
4\pi\sum_{L'}\biggl(i^{-\ell}
\mathcal{Y}_{\ell,m}(\vec{\nabla}_R) 
h^{(1)}_{\ell'}(\kappa|\vec{R}|)Y_{L'}(-\vec{R})\biggr)
j_{\ell'}(\kappa|\vec{r}-\vec{R}|)Y_{L'}^*(\vec{r}-\vec{R})
\nonumber\\
\label{eq:prestructureconstants}
\end{eqnarray}
where $h^{(1)}(x)$ is the spherical Hankel function of the first kind
defined in Abramowitz and Stegun (AS)\cite{abramowitz}
\begin{eqnarray}
h_\ell^{(1)}(x)\stackrel{AS 10.1.1}{=}j_\ell(x)+iy_\ell(x)
\stackrel{AS10.1.26}{=}
x^\ell\biggl(-\frac{1}{x}\partial_x\biggr)^\ell\frac{\sin(x)-i\cos(x)}{x}
\label{eq:defsphericalhankelfirstkind}
\end{eqnarray}
\end{myshadowminipage}



%
%====================================================================
\subsubsection{Expression for the structure constants}
%====================================================================
By comparing our notation to that of Daniel Grieger and using his
expression for the Structure constants, we arrive at the following
expression for the structure constants in our notation.

There was a misunderstanding with the sign of the structure
constants. Here I follow the signconvention $K=-\sum JS$, which is
opposite to the one I and Daniel had earlier.

\begin{eqnarray}
S_{R',L',R,L}
&=&
-4\pi \sum_{L''} H^B_{L''}(\vec{R'}-\vec{R}) C_{L,L'',L} 
\left\lbrace
\begin{array}{c}
(-1)^{\ell'}
(-ik)^{\ell+\ell'-\ell''}
\\
(-1)^{\ell'}\delta_{\ell+\ell',\ell''}
\\
(-1)^{\ell'}\kappa^{\ell+\ell'-\ell''}
\end{array}\right\rbrace 
\label{eq:expressionstructureconstants}
\end{eqnarray}




%=======================================================================
\section{Consistency checks}
%=======================================================================
We consider the case with $\kappa=0$, for which the solid Bessel and Hankel functions are
\begin{eqnarray}
K_{\vec{0},L}^\infty(\vec{r})&=&(2\ell-1)!! \frac{1}{|\vec{r}|^{\ell+1}}Y_L(\vec{r})
\\
J_{\vec{0},L}(\vec{r})&=&\frac{1}{(2\ell+1)!!}|\vec{r}|^\ell Y_L(\vec{r})
\end{eqnarray}

The explicit form of the first few is
\begin{eqnarray}
K_{\vec{0},s}^\infty(\vec{r})&=&\frac{1}{\sqrt{4\pi}}\frac{1}{|\vec{r}|}
\\
K_{\vec{0},p_x}^\infty(\vec{r})&=&\sqrt{\frac{3}{4\pi}}\frac{x}{|\vec{r}|^3}
\\
J_{\vec{0},s}(\vec{r})&=&\frac{1}{\sqrt{4\pi}}
\\
J_{\vec{0},p_x}(\vec{r})&=&\frac{1}{3}\sqrt{\frac{3}{4\pi}} x
\end{eqnarray}

Now we extract the structure constants from the off-site expansion
\begin{eqnarray}
K_{\vec{0},s}^\infty(\vec{r})&=&
-S_{\vec{0},s;\vec{R},s} J_s(\vec{r}-\vec{R}) 
\nonumber\\
&&-S_{\vec{0},s;\vec{R},p_x} J_{p_x}(\vec{r}-\vec{R}) 
-S_{\vec{0},s;\vec{R},p_y} J_{p_y}(\vec{r}-\vec{R}) 
-S_{\vec{0},s;\vec{R},p_z} J_{p_z}(\vec{r}-\vec{R}) 
\end{eqnarray}
which allows us to evaluate the structure constants directly
calculating value and derivatives at the second center and by
exploiting selection rules\footnote{Only an s-function has a finite
  value at the origin, only a p-function has a finite first derivative
  at the center, etc.}
\begin{eqnarray}
K_{\vec{0},s}^\infty(\vec{R})&=&\frac{1}{\sqrt{4\pi}}\frac{1}{|\vec{R}|}
=-
\underbrace{\Bigl(-\frac{1}{|\vec{R}|}\Bigr)}_{S_{\vec{0},s,\vec{R},s}}
\underbrace{\frac{1}{\sqrt{4\pi}}}_{J_{\vec{R},s}(\vec{R})} 
\\
\left.\partial_x\right|_{\vec{R}}K_{\vec{0},s}^\infty&=&
-\frac{1}{\sqrt{4\pi}}\frac{X}{|\vec{R}|^3}
=-
\underbrace{\sqrt{3}\frac{X}{|\vec{R}|^3}}_{S_{\vec{0},s;\vec{R},p_x}}
\underbrace{\frac{1}{3}\sqrt{\frac{3}{4\pi}}}_{\partial_x J_{\vec{R},p_x}(\vec{R})}
\\
K_{\vec{0},p_x}(\vec{R})&=&\sqrt{\frac{3}{4\pi}}\frac{X}{|\vec{R}|^3}
=-
\underbrace{\Bigl(-\sqrt{3}\frac{X}{|\vec{R}|^3}\Bigr)
}_{S_{\vec{0},p_x;\vec{R},s}}
\underbrace{\frac{1}{\sqrt{4\pi}}}_{J_{\vec{R},s}(\vec{R})}
\nonumber\\
\left.\partial_x\right|_{\vec{R}}K_{\vec{0},p_x}^\infty&=&
\sqrt{\frac{3}{4\pi}}
\left(\frac{1}{|\vec{R}|^3}-3\frac{X^2}{|\vec{R}|^5}\right)
=
-\underbrace{3\frac{3X^2-\vec{R}^2}{|\vec{R}|^2}}_{S_{\vec{0},p_x,\vec{R},p_x}}
\underbrace{\frac{1}{3}\sqrt{\frac{3}{4\pi}}|\vec{R}|^{-3}
}_{\left.\partial_x\right|_{\vec{R}} J_{\vec{R},p_x}}
\end{eqnarray}

Thus, the matrix of structure constants in the (s,p$_x$) subspace is
\begin{eqnarray}
\mat{S}_{\vec{0},\vec{R}}=\left(\begin{array}{cc}
-|\vec{R}|^{-1} & \sqrt{3}X/|\vec{R}|^3\\
-\sqrt{3}X/|\vec{R}|^3 & 
3[3X^2/R^2-1]\\
\end{array}\right)
\end{eqnarray}

We compare this result now for the one obtained from the direct
formula for $\kappa=0$. These structure constants have the form
\begin{eqnarray}
S_{RL,R'L'}=(-1)^{\ell'+1} 4\pi \sum_{L''} C_{L,L',L''} 
H_{L''}(\vec{R}'-\vec{R})
\delta^{\ell+\ell'-\ell''}
\end{eqnarray}
The structure constants obtained from this equation are
\begin{eqnarray}
S_{\vec{0},s;\vec{R},s}&=&(-1) 4\pi \underbrace{\frac{1}{\sqrt{4\pi}}}_{C_{sss}}
\cdot\underbrace{\frac{1}{\sqrt{4\pi}}\frac{1}{|\vec{R}|}}_{H_s(\vec{R})}
=-\frac{1}{|\vec{R}|}
\nonumber\\
S_{\vec{0},s;\vec{R},p_x}&=& 4\pi
\underbrace{\frac{1}{\sqrt{4\pi}}}_{C_{p_x,s,p_x}}
\underbrace{
\sqrt{\frac{3}{4\pi}}\frac{X}{|\vec{R}|^3}}
_{H_{p_x}(\vec{R})}
=\sqrt{3}\frac{X}{|\vec{R}|^3}
\nonumber\\
S_{\vec{0},p_x;\vec{R},s}&=& (-1)4\pi
\underbrace{\frac{1}{\sqrt{4\pi}}}_{C_{p_x,s,s}}\sqrt{\frac{3}{4\pi}}
\frac{X}{|\vec{R}|^3}=-\sqrt{3}\frac{X}{|\vec{R}|^3}
\nonumber\\
S_{\vec{0},p_x;\vec{R},p_x}&=& 4\pi
\underbrace{\frac{1}{\sqrt{5\pi}}}_{C_{p_x,p_x,d_{3x^2-r^2}}}
\underbrace{
\overbrace{\sqrt{\frac{5}{16\pi}} \frac{3X^2-R^2}{|\vec{R}|^2}}^{Y_{3x^2-r^2}}
\frac{3}{|\vec{R}|^{3}}
}_{H_{3x^2-r^2}(\vec{R})}
=3\frac{3X^2-R^2}{|R^5|}
\end{eqnarray}
For Gaunt coefficients see footnote.\footnote{
\begin{eqnarray}
Y_{p_x}Y_{p_x}=\frac{3}{4\pi}\frac{x^2}{r^2}
=\frac{1}{4\pi}\frac{x^2}{r^2} +\frac{1}{4\pi}\frac{3x^2-r^2}{r^2}
=\frac{1}{\sqrt{4\pi}}Y_s +\frac{1}{4\pi}\sqrt{\frac{16\pi}{5}}
Y_{3x^2-r^2}
=\frac{1}{\sqrt{4\pi}}Y_s +\sqrt{\frac{1}{5\pi}}Y_{3x^2-r^2}
\nonumber\\
\Rightarrow 
C_{p_x,p_x,s}=\frac{1}{\sqrt{4\pi}}\qquad\text{and}\qquad
C_{p_x,p_x,d_{3x^2-r^2}}=\frac{1}{\sqrt{5\pi}}
\end{eqnarray}}




%====================================================================
\chapter{Bloch theorem revisited}
%====================================================================
The Bloch states are eigenstates of the discrete lattice translation 
\begin{eqnarray}
\hat{S}(\vec{t})=\int d^3r\;|\vec{r}+\vec{t}\rangle\langle\vec{r}|
\end{eqnarray}
for the discrete lattice vectors $\vec{t}$. The eigenvalue equation has the form
\begin{eqnarray}
\hat{S}(\vec{t})|\psi_{\vec{k}}\rangle=|\psi_{\vec{k}}\rangle
\e{i\vec{k}\vec{r}}
\end{eqnarray}
This eigenvalue equation can be recast into the form
\begin{eqnarray}
\langle\vec{r}-\vec{t}|\psi_{\vec{k}}\rangle=\langle\vec{r}|\psi_{\vec{k}}\rangle
\e{i\vec{k}\vec{t}}
\end{eqnarray}
This implies that the states can be written  as product of a periodic function and a phase factor
\begin{eqnarray}
\langle\vec{r}|\psi_{\vec{k}}\rangle=u_{\vec{k}}(\vec{r})\e{i\vec{k}\vec{r}}
\end{eqnarray}
with
\begin{eqnarray}
u_{\vec{k}}(\vec{r})=u_{\vec{k}}(\vec{r}+\vec{t})
\end{eqnarray}

%====================================================================
\subsubsection{Bloch theorem in a local orbital basis}
%====================================================================
With $q_\alpha\defas \langle\pi_\alpha|\psi\rangle$, we obtain
\begin{eqnarray}
\hat{S}(\vec{t})\sum_\alpha|\chi_\alpha\rangle q_{\alpha,n}
&=&
\sum_\alpha|\chi_\alpha\rangle q_{\alpha,n}\e{i\vec{k}_n\vec{t}}
\nonumber\\
\int d^3r\;|\vec{r}+\vec{t}\rangle\langle\vec{r}|
\sum_\alpha|\chi_\alpha\rangle q_{\alpha,n}
&=&
\int d^3r\;|\vec{r}\rangle\langle\vec{r}|
\sum_\alpha|\chi_\alpha\rangle q_{\alpha,n}\e{i\vec{k}_n\vec{t}}
\nonumber\\
\sum_\alpha\langle\vec{r}-\vec{t}|\chi_\alpha\rangle q_{\alpha,n}
&=&
\sum_\alpha
\langle\vec{r}|\chi_\alpha\rangle q_{\alpha,n}\e{i\vec{k}_n\vec{t}}
\nonumber\\
\sum_\alpha\langle\vec{r}|\chi_{\alpha+\vec{t}}\rangle q_{\alpha,n}
&=&
\sum_\alpha
\langle\vec{r}|\chi_\alpha\rangle q_{\alpha,n}\e{i\vec{k}_n\vec{t}}
\nonumber\\
\sum_{\alpha'}\langle\vec{r}|\chi_{\alpha'}\rangle q_{\alpha'-\vec{t},n}
&=&
\sum_\alpha
\langle\vec{r}|\chi_\alpha\rangle q_{\alpha,n}\e{i\vec{k}_n\vec{t}}
\nonumber\\
q_{\alpha+\vec{t},n}&=&q_{\alpha,n}\e{-i\vec{k}_n\vec{t}}
\end{eqnarray}

%====================================================================
\subsubsection{Density matrix}
%====================================================================
\begin{eqnarray}
\rho_{\alpha,\beta+\vec{t}}
&=&
\sum_n \langle\pi_\alpha|\psi_n\rangle f_n
\langle\psi_n|\pi_\beta\rangle
\e{+i\vec{k}_n\vec{t}}
\end{eqnarray}


\clearpage
\bibliographystyle{unsrtnat} \bibliography{../all}
\end{document}  
