\documentclass[11pt,a4paper]{report}
%%%%%%%%%%%%%%%%%%%%%%%%%%%%%%%%%%%%%%%%%%%%%%%%%%%%%%%%%%%%%%%%%%%%%
%%                                                                 %%
%%    Header file for the Phi-S-X Series                           %%
%%                                                                 %%
%%    german version header_gm.tex is derived from header.tex      %%
%%    by uncommenting the line ``\setboolean{german}{true}'' below %%
%%                                                                 %%
%%    Never edit the german version! all changes must be done      %%
%%    in the english version header.tex                            %%
%%                                                                 %%
%%%%%%%%%%%%%%%%%%%%%%%%%%%%%%%%%%%%%%%%%%%%%%%%%%%%%%%%%%%%%%%%%%%%%
%%%%%%%%%%%%%%%%%%%%%%%%%%%%%%%%%%%%%%%%%%%%%%%%%%%%%%%%%%%%%%%%%%%%%
%%                                                                 %%
%%    Header file for the Phi-S-X Series                           %%
%%                                                                 %%
%%    german version header_gm.tex is derived from header.tex      %%
%%    by uncommenting the line ``\setboolean{german}{true}'' below %%
%%                                                                 %%
%%    Never edit the german version! all changes must be done      %%
%%    in the english version header.tex                            %%
%%                                                                 %%
%%%%%%%%%%%%%%%%%%%%%%%%%%%%%%%%%%%%%%%%%%%%%%%%%%%%%%%%%%%%%%%%%%%%%
%====================================================================
%-- define flag for language adaptations
\usepackage{ifthen}   % allows to select only certain text
\provideboolean{german}
\setboolean{german}{false}
%\setboolean{german}{true}  % uncomment this line for german editions
%====================================================================
%
% Textschriftart: Computer modern Bright
% body:            CM-Bright 10pt
% section titles:  CM-Bright Bold
% formulas:        CM-Bright Math Oblique
%
\usepackage[standard-baselineskips]{cmbright}
\usepackage{cmbright}
\usepackage[T1]{fontenc}
\def\usedfonts{CM-Bright}
\usepackage{typearea}
%\typearea[current]{calc} % benutzt die aktuelle 
       % bindekorrektur (BCOR angabe als parameter in koma usepackage)
       % und berechnet satzspiegel neu
\typearea[current]{11} %fixed div value

\usepackage{textcomp} % special symbols
\usepackage{amsfonts} % special symols
                      % see ftp://ftp.ams.org/pub/tex/doc/amsfonts/amsfndoc.pdf
\usepackage{amssymb}  % CM-Bright provides the AMS symbols
\usepackage{exscale}  % allows to scale math expressions to big fonts, 
                      % e.g. \Huge
\usepackage{curves}
\usepackage{braket}
\usepackage{miller}     % miller indices
\usepackage{chemmacros} % http://www.mychemistry.eu/mychemistry/
\usepackage[numbers]{natbib}     % bibliography style
\usepackage{url}\urlstyle{tt}
\usepackage{float}
\usepackage{bm}       % provides the command \bm{} that makes bold math symbols
\usepackage{amsmath}
\usepackage{amsbsy}   % allows bold mathematical symbols
\usepackage{amscd}
 \usepackage{a4wide}  % it is better to use the ``geometry'' package
\usepackage{array}    % 
\usepackage{fancyhdr} %  defines pagestyle fancy
\usepackage{epsfig}   % include graphics with epsfig
\usepackage{graphicx} % includegraphics
\usepackage{epstopdf}
\usepackage{wrapfig}
\usepackage{fancybox} % allows shadow-boxes
\usepackage{color}    % allows to use color in the text
%\usepackage{eepic}
\usepackage{flafter}  % places picture next to its reference
\usepackage{makeidx}  % make an index
%\usepackage{MnSymbol}  % 
%\usepackage{marvosym}  % 
\usepackage{textcase}
\usepackage{ulem} % defines strikeout \sout{}; underline \uline{}
                  % double underline \uuline{}; wave underline \uwave{}
                  % cross out \xout{}
%
%==========================================================================
%==  page layout  =========================================================
%==========================================================================
% eqnarray environment: reduce with of space in place of each ``&''
\setlength\arraycolsep{1.4pt}
\pagestyle{fancy}
%\renewcommand{\chaptermark}[1]{\markboth{\thechapter\ #1}{}}
\renewcommand{\chaptermark}[1]{\markboth{\MakeUppercase{\thechapter\ #1}}{}}
\fancyhf{} 
\fancyhead[LE]{\textsc{\thepage}\qquad\textsc{\leftmark}}
\fancyhead[RO]{\textsc{\leftmark}\qquad\textsc{\thepage}}
\renewcommand{\headrulewidth}{0.5pt}
\renewcommand{\footrulewidth}{0pt} 
\addtolength{\headheight}{2.5pt}
\fancypagestyle{plain}{\fancyhead{}
   \renewcommand{\headrulewidth}{0pt}
   \fancyfoot[CO]{\bfseries\thepage}}

% Line spacing -----------------------------------------------------------
\newlength{\defbaselineskip}
\setlength{\defbaselineskip}{\baselineskip}
\newcommand{\setlinespacing}[1]%
           {\setlength{\baselineskip}{#1 \defbaselineskip}}
\newcommand{\doublespacing}{\setlength{\baselineskip}%
                           {2.0 \defbaselineskip}}
\newcommand{\singlespacing}{\setlength{\baselineskip}{\defbaselineskip}}

% Absatz einr\"ucken ------------------------------------------------------
%\setlength{\parindent}{0pt}
\setlength{\parskip}{2pt}
% -------------------------------------------------------------------------
\ifthenelse{\boolean{german}}
  {\def\figurename{Abb.}}
  {\def\figurename{Fig.}}
%--------------------------------------------------------------------------
\renewcommand{\arraystretch}{1.15}  % skaliert den Zeilen abstand in der 
    % tabular und array umgebung
%
%==========================================================================
%==  boxes etc ============================================================
%==========================================================================
%== minipage in a shadowbox ===============================================
\newenvironment{myshadowminipage}[1]%
  {\par\noindent\begin{Sbox}\begin{minipage}{\linewidth}\vspace{0.1cm}\begin{center}\uppercase{#1}\end{center}}%
  {\vspace{0.1cm}\end{minipage}\end{Sbox}\shadowbox{\TheSbox}}
%
%== minipage in a framedbox ===============================================
\newenvironment{myframedminipage}%
  {\par\noindent\begin{Sbox}\begin{minipage}\linewidth\vspace{0.1cm}}%
  {\vspace{0.1cm}\end{minipage}\end{Sbox}\fbox{\TheSbox}}
%
\newcommand{\myshadowbox}[1]{\noindent\shadowbox{\parbox{\linewidth}{\smallskip #1\smallskip}}}
\newcommand{\myfbox}[1]{\noindent\fbox{\parbox{\linewidth}{\smallskip #1\smallskip}}\medskip}
%== minipage in a framedbox ===============================================
\newtheorem{defi}{Definition}[chapter]
\newenvironment{definition}[1]%
  {\par\noindent\begin{Sbox}\begin{minipage}{\linewidth}\vspace{0.1cm}\begin{defi}\uppercase{#1}\\\vspace{0.1cm}}%
  {\vspace{0.1cm}\end{defi}\end{minipage}\end{Sbox}\shadowbox{\TheSbox}}
%
%=========================================================================
% color used to point out information to the teacher
\definecolor{highlight}{rgb}{1.0,0.7,0.}
\newcommand{\Special}[1]{\textbf{\textcolor{highlight}{#1}}}
%=========================================================================
%  switch certain parts on and off. uses ifthen package
\newboolean{teacher}\setboolean{teacher}{false}
% this parameter can be changed in the manuscript again
\setboolean{teacher}{true} %private version if true!
\newcommand{\teacheronly}[1]{\ifthenelse{\boolean{teacher}}{#1\hfill\\ }}
\newcommand{\editor}[1]{\textcolor{blue}{\texttt{Editor: #1}}}
\newcommand{\MARK}[1]{\textcolor{blue}{#1}} 
\newcommand{\RED}[1]{\textcolor{red}{#1}} 
%
%==========================================================================
%==  define new symbols                                                 ===
%==========================================================================
% define \stat (stationary state) as an operator like \min
\DeclareMathOperator*{\stat}{stat}
\let\Vec=\mathbold   % cmbright.sty provides a bold/italic math alphabet
\let\Dot=\mathbold   % cmbright.sty provides a bold/italic math alphabet
\let\Ddot=\mathbold   % cmbright.sty provides a bold/italic math alphabet
%
\newcommand{\e}[1]{\mathrm{e}^{#1}}% exponential function
\renewcommand{\Re}{\mathrm{Re}}    % real part
\renewcommand{\Im}{\mathrm{Im}}    % imaginary part
\newcommand{\lagr}{\ell}           % Lagrange dichte
\newcommand{\Lagr}{\mathcal{L}}    % Lagrange Funktion
\newcommand{\erf}{{\rm erf}}       %
\newcommand{\atan}{{\rm atan}}     % arcus tangens
\newcommand{\mat}[1]{\bm{#1}}  % Matrix
\newcommand{\gmat}[1]{{\boldsymbol #1}}  % Matrix(symbol)
\newcommand{\defas}{\stackrel{\text{def}}{=}}  %  is defined as
\ifthenelse{\boolean{german}}
  {\newcommand{\rot}{{\rm\bf rot}}}    % curl
  {\newcommand{\rot}{{\rm\bf curl}}}   % curl
\newcommand{\sgn}{{\rm sgn}}       % sign
\ifthenelse{\boolean{german}}
   {\newcommand{\Tr}{\mathrm{Sp}}}      % trace
   {\newcommand{\Tr}{\mathrm{Tr}}}      % trace
\ifthenelse{\boolean{german}}
   {\newcommand{\grmn}[2]{\footnote{``#2'' hei{\ss}t in englisch ``#1''}}}
   {\newcommand{\grmn}[2]{\footnote{``#1'' translates as ``#2'' into German}}}
% define the equation reference
\ifthenelse{\boolean{german}}
   {\newcommand{\eq}[1]{\text{Gl.}~\ref{#1}}}
   {\newcommand{\eq}[1]{\text{Eq.}~\ref{#1}}}
% define a relation with an equation number ontop
\newcommand{\eqrel}[2]{\stackrel{\eq{#1}}{#2}}
\newcommand{\zero}{\varnothing}
%\newcommand{\ket}[1]{|#1\rangle} % contained in package braket
\newcommand{\sumint}{\int\hspace{-15pt}\sum}
\newcommand{\marker}[1]{\textcolor{blue}{\emph{#1}}}
\renewcommand*{\dot}[1]{\overset{\mbox{\large\bfseries .}}{#1}}
\renewcommand*{\ddot}[1]{\overset{\mbox{\large\bfseries\hspace{+0.1ex}.\hspace{-0.1ex}.}}{#1}}
%
%==========================================================================
%==                                                                     ===
%==========================================================================
% Prevent figures from appearing on a page by themselves
% from http://dcwww.camd.dtu.dk/~schiotz/comp/LatexTips/LatexTips.html
\renewcommand{\topfraction}{0.85}
\renewcommand{\textfraction}{0.1}
\renewcommand{\floatpagefraction}{0.75}
%
%==========================================================================
%==                                                                     ===
%==========================================================================
\makeindex    % make index. uses makeidx package.

%== allow links between documents ============================================
\usepackage{xr}
\usepackage{xr-hyper}
%==  hyperref package (must be last package)
\usepackage[colorlinks=true]{hyperref} %specify this as last package
\hypersetup{citecolor=blue}
\hypersetup{menucolor=magenta}
\hypersetup{urlcolor=blue}      % 
\hypersetup{filecolor=green}    % file links
\hypersetup{linkcolor=magenta}  %table of contents
\hypersetup{pdfauthor={Peter E. Bl\"ochl}}
\hypersetup{pdfdisplaydoctitle=true}
\externaldocument[phisx1-]{/Users/ptpb/Tree/PhiSX/ClassicalMechanics/Book/cm-gm}
\externaldocument[phisx2-]{/Users/ptpb/Tree/PhiSX/Electrodynamics/Book/el-gm}
\externaldocument[phisx3-]{/Users/ptpb/Tree/PhiSX/QuantumMechanics/Book/qm}
\externaldocument[phisx4-]{/Users/ptpb/Tree/PhiSX/StatisticalMechanics/Book/sm}
\externaldocument[phisxqm2-]{/Users/ptpb/Tree/PhiSX/QuantumMechanicsII/Book/qm2}
\externaldocument[phisxsm2-]{/Users/ptpb/Tree/PhiSX/StatisticalMechanicsII/Book/sm2}
\externaldocument[phisxcb-]{/Users/ptpb/Tree/PhiSX/Chemicalbond/Book/cb}
% Example: Figure~PhiSX:Quantum
% Mechanics-\ref{phisx3-fig:doubleslitwave} on page
% \pageref{phisx3-fig:doubleslitwave}


\hypersetup{pdftitle=paw_brillouin}
\begin{document}
\begin{titlepage}
\begin{center}
\vspace*{3.5cm}
{\huge \textbf{The LMTO object of the CP-PAW code}}\\
\vspace{0.5cm}
{\large Peter E. Bl\"ochl}
\vspace{0.5cm} 
\end{center}

\vfill
\begin{center}
Copyright Peter E. Bl\"ochl; Sept.2, 2013-\today\\
{\small
Institute of Theoretical Physics;
Clausthal University of Technology;\\ 
D-38678 Clausthal Zellerfeld; Germany;\\
http://www.pt.tu-clausthal.de/atp/}
\end{center}
\end{titlepage}
\noindent            
\tableofcontents
%====================================================================
\chapter{Todo}
%====================================================================
%====================================================================
\section{Fixes}
%====================================================================
\begin{itemize}
\item lmto\_overlapphi calculates the onsite overlap matrix of partial
  waves in a sphere.
%
\item using only sp like tight-binding orbitals and local exchange
  lead to an increase of the band gap of silicon above 1.3 eV. After
  adding the d-orbitals to the HF term collapsed the band gap
  dramatically below the dft value. Core-valence echange seems to have
  an important effect on the band gap too.
%
%% \item double counting in \verb|paw_lmto| and \verb|paw_dmft| is in
%%   error. It takes the partial wave density matrix do do it
%%   properly. currently only the density of AEF is used.

%%   Double counting fixed before Nov. 13, 2014 (Commit 106f8a0). Both
%%   versions of double counting can be addressed by a hard-wired switch.
%
\item The charge sumrule is not correct! The calculation for an H-atom
  yields $Tr[\rho O]\approx 0.2$. The problem is not the difference
  between tailed orbitals and the multicenter expansion, because
  $(r*\chi)^2$ agrees quire well.
%
\end{itemize}

%====================================================================
\section{Ideas}
%====================================================================
\begin{itemize}
\item With the introduction of the tailed representation of the NTBO's
  we \textbf{switch from a multi-center expansion to a one-center
    expansion}.  This, however, requires to increase the number of
  angular momenta for the tail functions beyond that used for the
  augmentation: Each (screening) Hankel function centered on the
  neighboring site contributes arbitrary many angular momenta at the
  central site.

  The tail functions for the higher angular momenta behave, at the
  central site, like a bare Bessel function. Beyond the central site,
  we can add the same pairs of exponential tails as for the lower
  tailed partial waves.

  (The following is probably no more true. please check!) Currently we
  use the following rule: Rule: (1) Each head function has exactly one
  tail function attributed to it.  Thus we can identify the phidot
  function uniquely by looking for the tail function with the correct
  angular momentum.  From (1) follows, that there is at most one tail
  function per site and angular momentum.
%
\item Suggestion: the exponential functions for the tailed expansion
  are determined by trial and error. A more systematic approach for
  the tail function would be to (1) build a Hankel function on a
  neighboring site, which is augmented by (a) a bare Bessel function
  and (b) by a screened scattering partial wave
  $|\dot{\bar{\phi}}\rangle$. (1a) In order to make the formulation
  less dependent on the neighbor distance, we could also include the
  gradients of these functions. (2) These functions are expanded about
  the central site into an angular-momentum expansion.  (3) a
  reasonably weighted least-square fit of the exponential tails to
  these functions will provide a ``optimum shape'' of the tails for
  each angular momentum.
%
\item The orbitals in the tailed one-center expansion and as
  multicenter expansions can be compared using \verb|LMTO_PLOTORBS|.
  Calculations for hydrogen suggest that a good choice is $K2=-1.$ and
  $(\lambda_1,\lambda_2)=(4,2)$. The rationale for $K2$ is that the
  energy of the hydrogen orbital is at
  $-1Ry=-\frac{1}{2}H=\frac{1}{2}K2$. (see
  \url{Dummy/Sitest/Test/Occtest/H2test})
\end{itemize}

%====================================================================
\chapter{Purpose and theoretical background of the LMTO Object}
%====================================================================
The LMTO object maps the wave functions expressed in augmented plane
waves onto a basiset of \textbf{natural tight-binding
  orbitals}\index{natural tight-binding orbitals}. The natural
tight-binding orbitals are a kind of LMTO's, screened such that the
tails exhibit only scattering character in the context of nodeless
wave functions\cite{bloechl12_arxiv1210_5937}.

%====================================================================
\section{Augmentation}
%====================================================================
The concept of linear augmented waves\cite{andersen75_prb12_3060} is
as follows:
\begin{enumerate}
\item At first, a so-called \textbf{envelope function}\index{envelope
  function} $|K^\infty_\alpha\rangle$ is defined.\footnote{The
  superscript $\infty$ denotes that the function extends over all
  space, a superscript $\Omega$ denotes that the function is truncated
  (set to zero) outside the augmentation sphere $\Omega_{R}$ centered
  at the site $R_\alpha$ denoted by the index $\alpha$. The
  superscript $I$ denotes that the function is non-zero only in the
  interstitial region, that is outside all augmentation spheres. If
  the augmentation spheres overlap, the function in the interstitial
  region is defined by subtraction of all sphere
  contributions. Similarly, the interstitial function constains terms
  from the higher angular momenta within the augmentation regions.}
%
\item In a second step, this envelope function is expanded about each
  atomic site $R_\alpha$ into spherical harmonics.  More generally,
  they are expanded into \textbf{head functions}\index{head function}
  $|K^\Omega_\alpha\rangle$ and \textbf{tail functions}\index{tail
    function} $|J^\Omega_\alpha\rangle$. The head function is the
  dominant contribution and carries the quantum number of the final
  orbital, while the tail functions are the minor contributions with
  different quantum numbers. In practice, the head functions are solid
  Hankel functions and the tail functions are solid Bessel functions.
\begin{eqnarray}
|K_{\alpha}^\infty\rangle=|K^\Omega_{\alpha}\rangle
-\sum_{\beta}|J^\Omega_{\beta}\rangle S^\dagger_{\beta,\alpha}
+|K^I_{\alpha}\rangle
\end{eqnarray}
   The coefficients $S_{\alpha,\beta}$ of the tail functions are
   called \textbf{structure constants}\index{structure constants}.

  The difference between the full envelope function and its expansion
  into head and tail functions is the \textbf{interstitial envelope
    function}\index{interstitial envelope function}\footnote{The
    interstitial envelope function is confined mostly in the region in
    between the atoms, but it also accounts for the overlap of the
    atomic regions and so-called higher partial waves not taken care of
    in the regular partial-wave expansion.}  $|K^I_\alpha\rangle$.
%
\item In the third step, the head and tail functions are replaced
  differentiably at some sphere radius by \textbf{partial
    waves}\index{partial waves} of the atomic potential. For that
  purpose, we use as partial waves a solution of the Schr\"odinger
  equation for some energy, denoted as $|\phi_{\alpha}\rangle$ and its
  energy derivative function $|\dot{\phi}_\alpha\rangle$.

  The matching parameters are called \textbf{potential
    parameters}\index{potential parameters}.
\end{enumerate}

%====================================================================
\section{Structure constants}
%====================================================================
%====================================================================
\subsection{Hankel functions as envelope function}
%====================================================================
In practice, we will use solid Hankel functions $H_L(\vec{r})$ as
envelope functions, so that 
\begin{eqnarray}
\langle\vec{r}|K^\infty_\alpha\rangle= H_{L_\alpha}(\vec{r}-\vec{R}_\alpha)
\end{eqnarray}


Solid Hankel functions are irregular solutions of the the
inhomogeneous Helmholtz equation\footnote{I am not sure whether also
  the three dimensional differential equation or only the
  one-dimensional differential equation for the radial part is called
  Helmholtz equation.}
\begin{eqnarray}
\Bigl[\vec{\nabla^2}+k^2\Bigr]H_{L}(\vec{r})
=-4\pi(-1)^\ell\mathcal{Y}(\vec{\nabla})\delta(\vec{r})
\label{eq:solidhelmholtzequation}
\end{eqnarray}
Here $\mathcal{Y}_L(\vec{r})\defas r^\ell Y_L(\vec{r})$ is a
polynomial. With a gradient as argument, it becomes a differential
operator. With $Y_L(\vec{r})$ we denote a spherical or real harmonic
function and $L\defas(\ell,m)$ is a composite index of angular
momentum quantum number $\ell$ and magnetic quantum number $m$.

Further details about the Hankel and Bessel functions can be found in
appendix~\ref{app:solidhankel}.

%====================================================================
\subsection{Hankel and Bessel functions as head and tail functions}
%====================================================================
Defining the envelope function via an isotropic and translationally
invariant differential equation of second order has the advantage
that the solution can be expanded about different centers into regular
solutions of the same differential equation with specific angular
momenta. The regular solutions of the Helmholtz equation are the
Bessel functions.

Hankel and Bessel functions are defined\footnote{This is our
  definition, not a generally accepted convention.} so that they
behave at the origin as
\begin{eqnarray}
K^\Omega_{R,L}(\vec{r})&=&
\Bigl[(2\ell-1)!! \frac{1}{|\vec{r}-\vec{R}|^{\ell+1}} 
+...\Bigr]
Y_L(\vec{r}-\vec{R})\theta_{\Omega_R}(\vec{r})
\\
J^\Omega_{R,L}(\vec{r})&=&
\biggl[\frac{1}{(2\ell+1)!!} |\vec{r}-\vec{R}|^{\ell+1} 
+...\Bigr]Y_L(\vec{r}-\vec{R})\theta_{\Omega_R}(\vec{r})
\end{eqnarray}
$\theta_{\Omega_R}(\vec{r})$ is a step function, which equals unity within the
augmentation region $\Omega_R$ centered at site $R$, while it vanishes
outside. The terms neglected are higher orders in $|\vec{r}-\vec{R}|$.

%====================================================================
\subsection{Bare structure constants}
%====================================================================
The \textbf{bare structure constants}\index{structure constants !bare}
$S^\dagger_{\beta,\alpha}$ are the expansion constants for an
off-center expansion of solid spherical Hankel functions\index{Hankel
  function} $|K_{\alpha}^\infty\rangle$ into \textbf{solid Bessel
  functions}\index{Bessel function} $|J^\Omega_{\beta}\rangle$.
\begin{eqnarray}
|K_{\alpha}^\infty\rangle=|K^\Omega_{\alpha}\rangle
-\sum_{\beta}|J^\Omega_{\beta}\rangle S^\dagger_{\beta,\alpha}
+|K^I_{\alpha}\rangle
\end{eqnarray}
The index $\alpha$ denotes here an atomic site $R$ and a set of
angular momenta $L=(\ell,m)$.

The superscript $\infty$ denotes that the function extends over all
space, a superscript $\Omega$ denotes that the function is truncated
(set to zero) outside the augmentation sphere $\Omega_{R}$ centered at
the site denoted by the index. The superscript $I$ denotes that the
function is limited to the interstitial region, that is, outside all
augmentation spheres. If the augmentation spheres overlap, the
function in the interstitial region is defined by subtraction of all
sphere contributions.



\begin{myshadowminipage}{Bare structure constants}
The bare structure constants have the form
\begin{eqnarray}
S_{RL,R'L'}=(-1)^{\ell'+1} 4\pi \sum_{L''} C_{L,L',L''} 
H_{L''}(\vec{R}'-\vec{R})
\begin{cases}
(-ik)^{\ell+\ell'-\ell''}&\text{for $k^2>0$}\\
\delta_{\ell+\ell'-\ell''}&\text{for $k^2=0$}\\
\kappa^{\ell+\ell'-\ell''}&\text{for $k^2=-\kappa^2<0$}\\
\end{cases}
\label{eq:fortmulaforbarestructureconstants}
\end{eqnarray}
\end{myshadowminipage}

With $C_{L,L',L''}$, we denote the \textbf{Gaunt
  coefficients}\index{Gaunt coeffocients} defined by
\begin{eqnarray}
Y_{L'}(\vec{r})Y_{L''}(\vec{r})=\sum_L Y_{L}(\vec{r})C_{L,L',L''}
\end{eqnarray}
Note, that the Gaunt coefficients for spherical and real spherical
harmonics differ.\footnote{In practice, we use real spherical
  harmonics and the corresponding Gaunt coefficients.}

The bare structure constants are hermitean\footnote{We use that
  $H_L(\vec{r})=(-1)^\ell H_L(-\vec{r})$ and that the Gaunt
  coefficients $C_{L,L',L''}$ vanish unless $\ell+\ell'+\l''$ is
  even.}, i.e.
\begin{eqnarray}
S_{RL,R'L'}=S_{R'L',RL}
\end{eqnarray}
This is, however, not true for each angular-momentum block individually,
i.e. in general we have $S_{RL,R'L'}\neq S_{R,L',R'L}$.

%====================================================================
\subsection{Screened structure constants}
%====================================================================
A so-called \textbf{screened LMTO representation}\index{screened
  LMTO's} is defined by a set of screened scattering partial waves
$|\dot{\bar{\phi}}_\alpha\rangle$.  In what we call the
nodeless-representation, the scattering partial waves are defined as
nodeless scattering wave functions.

The node-less scattering partial wave
$|\dot{\bar{\phi}}_\alpha\rangle$ define the screening constants
$\bar{Q}_{\alpha}$ such that the screened tail functions
$|\bar{J}_\alpha\rangle$ match with value and derivative to the
scattering partial wave
\begin{eqnarray}
|\dot{\bar{\phi}}_\alpha\rangle \rightarrow 
|\bar{J}^{\Omega}_{\alpha}\rangle
\defas
|J^{\Omega}_{\alpha}\rangle
-|K^{\Omega}_{\alpha}\rangle \bar{Q}_{\alpha}
\end{eqnarray}

A screened solid Hankel function $|\bar{K}_{\alpha}^\infty\rangle$ is
a superposition of bare solid Hankel functions on a set of atomic
positions
\begin{eqnarray}
|\bar{K}^\infty_\alpha\rangle=\sum_\beta|K^\infty_\beta\rangle c_{\beta,\alpha}
\label{eq:kbarassuperposofkbare}
\end{eqnarray}
with the property that the tail functions are made entirely
from screened Bessel functions $|\bar{J}^\Omega_{\beta}\rangle$, i.e.
\begin{eqnarray}
|\bar{K}_{\alpha}^\infty\rangle=|K^\Omega_{\alpha}\rangle
-\sum_{\beta}|\bar{J}^\Omega_{\beta}\rangle \bar{S}^\dagger_{\beta,\alpha}
+|\bar{K}^I_{\alpha}\rangle
\label{eq:kbarwithsbar}
\end{eqnarray}
The expansion coefficients $\bar{S}$ are the \textbf{screened structure
constants}\index{Structure constants !screened}.


By equating the two expressions for the screened Hankel functions,
namely \eq{eq:kbarassuperposofkbare} and \eq{eq:kbarwithsbar}, we can
extract the screened structure constants
$\bar{S}^\dagger_{\beta,\alpha}$ and the superposition coefficients
$c_{\beta,\alpha}$.
\begin{eqnarray}
\sum_\beta\Bigl[|K^\Omega_{\beta}\rangle
-\sum_{\gamma}|J^\Omega_{\gamma}\rangle S^\dagger_{\gamma,\beta}
+|K^I_{\beta}\rangle \Bigr]c_{\beta,\alpha}
=
|K^\Omega_{\alpha}\rangle
-\sum_{\beta}
\underbrace{\Bigl[|J^\Omega_{\beta}\rangle-|K^\Omega_{\beta}\rangle\bar{Q}_\beta\Bigr]}
_{|\bar{J}^\Omega_{\beta}\rangle} \bar{S}^\dagger_{\beta,\alpha}
+|\bar{K}^I_{\alpha}\rangle
\nonumber\\
\eqrel{eq:kbarassuperposofkbare}{\Rightarrow}\quad
\sum_\beta
|K^\Omega_{\beta}\rangle c_{\beta,\alpha}
-\sum_{\beta,\gamma}|J^\Omega_{\gamma}\rangle S^\dagger_{\gamma,\beta} c_{\beta,\alpha}
%+\sum_\beta|K^I_{\beta}\rangle c_{\beta,\alpha}
=
\sum_{\beta}|K^\Omega_{\beta}\rangle
\Bigl[\delta_{\beta,\alpha}+
\bar{Q}_\beta \bar{S}^\dagger_{\beta,\alpha}\Bigr]
-\sum_{\beta}|J^\Omega_{\beta}\rangle \bar{S}^\dagger_{\beta,\alpha}
%+|\bar{K}^I_{\alpha}\rangle
\nonumber\\
\end{eqnarray}
By comparing the coefficients, we obtain
\begin{eqnarray}
c_{\beta,\alpha}&=&\delta_{\beta,\alpha}+
\bar{Q}_\beta \bar{S}^\dagger_{\beta,\alpha}
\label{eq:definingeqsystemforsbara}
\\
\bar{S}^\dagger_{\gamma,\alpha}&=&\sum_\beta S^\dagger_{\gamma,\beta} c_{\beta,\alpha}
\,,
\label{eq:definingeqsystemforsbar}
\label{eq:definingeqsystemforsbarb}
\end{eqnarray}
which can be resolved to
\footnote{
\begin{eqnarray}
\mat{\bar{S}}^\dagger
\eqrel{eq:definingeqsystemforsbarb}{=}
\mat{S}^\dagger\mat{c}
\eqrel{eq:definingeqsystemforsbara}{=}
\mat{S}^\dagger\Bigl(\mat{1}+\mat{\bar{Q}}\mat{\bar{S}}^\dagger\Bigr)
\Rightarrow
\Bigl(\mat{1}-\mat{S}^\dagger\mat{\bar{Q}}\Bigr)\mat{\bar{S}}^\dagger
=\mat{S}^\dagger
\Rightarrow
\mat{\bar{S}}^\dagger
=\Bigl(\mat{1}-\mat{S}^\dagger\mat{\bar{Q}}\Bigr)^{-1}\mat{S}^\dagger
\end{eqnarray}
%% The following gives a wrong answer:
%% \begin{eqnarray}
%% \mat{c}&=&\mat{1}
%% +\mat{\bar{Q}}\mat{\bar{S}}^\dagger
%% =\mat{1}+\mat{\bar{Q}}\mat{S}^\dagger\mat{c}
%% \qquad
%% \Rightarrow\qquad\sum_\gamma \Bigl[\delta_{\beta,\gamma}
%% -\bar{Q}_\beta S^\dagger_{\beta,\gamma}\Bigr] c_{\gamma,\alpha}=\delta_{\beta,\alpha}
%% \qquad\Rightarrow\qquad
%% \mat{c}=[\mat{1}-\mat{\bar{Q}}\mat{S}^\dagger]^{-1}
%% \nonumber\\
%% \mat{\bar{S}}^\dagger&=&\mat{S}^\dagger\mat{c}
%% =\mat{S}^\dagger[\mat{1}-\mat{\bar{Q}}\mat{S}^\dagger]^{-1}
%% \qquad\Leftrightarrow\qquad
%% [\mat{1}-\mat{S}\mat{\bar{Q}}]\mat{\bar{S}}=\mat{S}
%% \end{eqnarray}
  } the defining
equation of the screened structure constants
\begin{myshadowminipage}{screened structure constants}
\begin{eqnarray}
\mat{\bar{S}}^\dagger=
\Bigl[\mat{1}-\mat{S}^\dagger\mat{\bar{Q}}\Bigr]^{-1}\mat{S}^\dagger
\label{eq:defscreenedstructureconstants}
\end{eqnarray}
and the expression of the screened Hankel functions 
\begin{eqnarray}
|\bar{K}^\infty_\alpha\rangle=\sum_\beta |K^\infty_\beta\rangle
\Bigl[\delta_{\beta,\alpha}+
\bar{Q}_\beta \bar{S}^\dagger_{\beta,\alpha}\Bigr]
\;.
\end{eqnarray}

Because we calculate the screened structure constants on finite
clusters, \eq{eq:defscreenedstructureconstants} should be considered
of only formal value and should not be used in the actual
calculations. Rather, the defining equations
\eq{eq:definingeqsystemforsbarb} shall be used as shown in the
following section.
\end{myshadowminipage}{}


%====================================================================
\section{Screening on finite clusters}
%====================================================================
The screened structure constants are calculated on a cluster of atomic
sites. The calculation can, in principle, be done for each single
screened Hankel function independently. In practice, we do the
calculations for all atoms centered on a given site in one step.

We go back to the defining equation system
\eq{eq:definingeqsystemforsbara}, \eq{eq:definingeqsystemforsbarb} and
rewrite it in terms of vectors, which are defined on the cluster
$B$. The index $\alpha$ labeling the vectors counts the envelope
functions centered at the central site. The dimension of the vectors
and matrices defined below is given by angular momentum channels and
sites with non-zero screening charge\footnote{Inclusion of orbitals
  with vanishing screening charge does not affect the expansion
  coefficients $c_{\alpha,\beta}$. However, it also includes the
  structure constant matrix elements for the higher angular momenta
  with vanishing screening charge. The contribution of the higher
  angular momenta can be calculated after the screening has been
  done.} $\bar{Q}$.

The equations attain the form
\begin{eqnarray}
\vec{c}_\alpha&\eqrel{eq:definingeqsystemforsbara}{=}&
\vec{e}_\alpha+\mat{\bar{Q}}\vec{s}_\alpha
\label{eq:definingeqsystemforsbarveca}
\\
\vec{s}_\alpha&\eqrel{eq:definingeqsystemforsbarb}{=}&
\mat{S}^\dagger\vec{c}_\alpha
\label{eq:definingeqsystemforsbarvecb}
\label{eq:definingeqsystemforsbarvec}
\end{eqnarray}
where the vectors $\vec{c}_\alpha$, $\vec{s}_\alpha$ and $\vec{e}_\alpha$ are
defined by its components
\begin{eqnarray}
\Bigl(\vec{c}_\alpha\Bigr)_\beta&=&c_{\beta,\alpha}
\nonumber\\
\Bigl(\vec{s}_\alpha\Bigr)_\beta&=&\bar{S}^\dagger_{\beta,\alpha}
\nonumber\\
\Bigl(\vec{e}_\alpha\Bigr)_\beta&=&\delta_{\beta,\alpha}
\end{eqnarray}
In matrix form, all these quantities are non-square matrices.

\begin{eqnarray}
\vec{c}_\alpha
&\eqrel{eq:definingeqsystemforsbarveca}{=}&
\vec{e}_\alpha+\mat{\bar{Q}}\vec{s}_\alpha
\eqrel{eq:definingeqsystemforsbarvecb}{=}
\vec{e}_\alpha+\mat{\bar{Q}}\mat{S}^\dagger\vec{c}_\alpha
\nonumber\\
\Rightarrow\qquad
\Bigl[\mat{1}-\mat{\bar{Q}}\mat{S}^\dagger\Bigr]\vec{c}_\alpha&=&\vec{e}_\alpha
\nonumber\\
\Rightarrow\qquad
\vec{c}_\alpha&=&[\mat{1}-\mat{\bar{Q}}\mat{S}^\dagger]^{-1}\vec{e}_\alpha
\nonumber\\
\vec{s}_\alpha
&\eqrel{eq:definingeqsystemforsbarvecb}{=}&
\mat{S}^\dagger\vec{c}_\alpha
=\mat{S}^\dagger[\mat{1}-\mat{\bar{Q}}\mat{S}^\dagger]^{-1}\vec{e}_\alpha
\end{eqnarray}
Interestingly, the vector on the right-hand side $\vec{e}_\alpha$ can
not be simply ignored as the matrix form seems to suggest. This is
specific to the calculation on the cluster. Because of this, we cannot
identify the contribution of these vectors with a unit matrix.



\begin{myshadowminipage}{Calculation of screened structure constants}
Thus, we first evaluate the bare structure constants $\mat{S}^\dagger$
on the cluster, and from that
$[\mat{1}-\mat{\bar{Q}}\mat{S}^\dagger]$. Then we solve the equation
\begin{eqnarray}
[\mat{1}-\mat{\bar{Q}}\mat{S}^\dagger]\vec{c}_\alpha&=&\vec{e}_\alpha
\label{eq:forscreenedsa}
\\
\vec{s}_\alpha&=&\mat{S}^\dagger\vec{c}_\alpha
\label{eq:forscreenedsb}
\end{eqnarray}
for $\vec{c}_\alpha$ first from \eq{eq:forscreenedsa} using a standard
routine for linear equation systems. Using the result for
$\vec{c}_\alpha$, we obtain the screened structure constants
$\vec{s}_\alpha$ from \eq{eq:forscreenedsb} by multiplication with the
bare structure constants.

Finally we obtain the screened structure constants as
\begin{eqnarray}
\bar{S}^\dagger_{\gamma,\alpha}=\Bigl(\vec{s}_\alpha\Bigr)_\gamma
\end{eqnarray}
\end{myshadowminipage}

Note that the screened structure constants calculated on finite
clusters are no more exactly hermitean.

%====================================================================
\section{Derivatives}
%====================================================================
In order to evaluate forces, we need to investigate the derivatives of
the structure constants with respect to changes of the atomic
structure.

\begin{eqnarray}
-\mat{\bar{Q}}d\mat{S}^\dagger\vec{c}_\alpha
+\Bigl[\mat{1}-\mat{\bar{Q}}\mat{S}^\dagger\Bigr]d\vec{c}_\alpha
&\eqrel{eq:forscreenedsa}{=}&0
\nonumber\\
\Bigl[\mat{1}-\mat{\bar{Q}}\mat{S}^\dagger\Bigr]d\vec{c}_\alpha
&=&
\mat{\bar{Q}}d\mat{S}^\dagger\vec{c}_\alpha
\nonumber\\
\Rightarrow\quad
d\vec{s}_\alpha
&\eqrel{eq:forscreenedsb}{=}&
d\mat{S}^\dagger\vec{c}_\alpha
+\mat{S}^\dagger d\vec{c}_\alpha
=\mat{\bar{Q}}^{-1}d\vec{c}_\alpha
\nonumber\\
&=&\mat{\bar{Q}}^{-1}\Bigl[\mat{1}-\mat{\bar{Q}}\mat{S}^\dagger\Bigr]^{-1}
\mat{\bar{Q}}d\mat{S}^\dagger\vec{c}_\alpha
\nonumber\\
&=&\Bigl[\mat{1}-\mat{S}^\dagger\mat{\bar{Q}}\Bigr]^{-1}
d\mat{S}^\dagger\vec{c}_\alpha
\end{eqnarray}
If this is correct, i implies that the screened structure constants
depend only in second order on a displacement of the atoms.


%====================================================================
\subsection{Lower and higher angular momenta}
%====================================================================
The angular momentum expansion of a Hankel function
$|K^\infty_\alpha\rangle$ about a neighboring size has infinitely many
angular momentum contributions $|J^\Omega_\alpha\rangle$. The Hankel
functions $|K^\Omega_\alpha\rangle$, on the other hand, are limited to
a subset. We call the angular momenta, for which the Hankel functions
$|K^\Omega_\alpha\rangle$ are taken into account, lower angular
momenta, and those for which only the Bessel function
$|J^\Omega_\alpha\rangle$ are considered, the higher angular momenta.
The distinction reflects directly to the values of $\bar{Q}$, which
are nonzero for the lower angular momenta and which vanish for the
higher angular momenta. 

To avoid the confusion between the choice of screening charges and the
selection of tight-binding orbitals, we assume that there is an
orbital considered for each non-zero screening charge. A truncation of
the basisset is postponed.  With this choice, the left index of the
expansion coefficients $c_{\beta,\alpha}$ in
\eq{eq:kbarassuperposofkbare} runs over all sites in the cluster and
angular momenta for which there is a non-zero screening charge
$\bar{Q}_\beta$. The right index includes all orbital indices on the
central site of the cluster.

If there are two orbitals with the same leading angular momentum, they
will have, per definition, the same envelope function and the same
tails.\footnote{The difference between two such orbitals is thus
  confined to the augmentation region of the central site, and can
  probably be identified with a local orbital as they are used in
  LAPW.} Thus, we consider only the angular momenta during the
construction and expand the structure-constant array after the
screening has been completed.

\begin{center}
\begin{tabular}{|c|l|}
\hline
A&indices of screened orbitals\\
B&indices of non-zero screening charges\\
C&indices of partial waves\\
D&indices of angular momenta considered one-center expansion of 
the tailed representation\\
\hline
\end{tabular}
\end{center}

Screened orbitals (A) can only be formed in channels with partial waves (C),
because they require augmentation of the envelope function.

Non-zero screening charges (B) are only possible where partial waves (C)
exist, because the irregular contribution must be augmented.

The number of screened orbitals (A) may be a subset of (B) the
channels with non0zero screening charges.

%====================================================================
\section{Augmentation and Potential parameters}
%====================================================================
%===============================================================================
\subsection{Local orbitals}
%===============================================================================
The local orbitals have the form
\begin{eqnarray}
|\chi_\alpha\rangle&=&|\phi^K_\alpha\rangle
- \sum_{R,L}|\phi^{\bar{J}}_{R,L}\rangle \bar{S}^\dagger_{R,L,R_\alpha,L_\alpha}
\nonumber\\
&+&\sum_{R',L'}|K^I_{R',L'}\rangle\Bigl[ \delta_{R',L',R_\alpha,L_\alpha}-\bar{Q}_{R',L'}
\bar{S}^\dagger_{R',L',R_\alpha,L_\alpha}\Bigr]
\end{eqnarray}
where the new partial waves $|\phi^K_\alpha\rangle$ and
$|\phi^{\bar{J}}_\alpha\rangle$ are superpositions of the valence and
scattering partial waves that match differentiably to the head and
tail functions $|K_\alpha\rangle$ and $|\bar{J}_\alpha\rangle$.
\begin{eqnarray}
|\phi^K_\alpha\rangle&=&
\overbrace{
|\phi_\alpha\rangle 
\underbrace{
\frac{W_\alpha[K,\dot{\bar{\phi}}]}{W_\alpha[\phi,\dot{\bar{\phi}}]}}_{Ktophi}
-\,|\dot{\bar{\phi}}_\alpha\rangle 
\underbrace{\frac{W_\alpha[K,\phi]}{W_\alpha[\phi,\dot{\bar{\phi}}]}}_{-Ktophidot}
}^{\rightarrow |K^\Omega_\alpha\rangle}
\nonumber\\
|\phi^{\bar{J}}_{R,L}\rangle
&=&\overbrace{
|\dot{\bar{\phi}}_\beta\rangle 
\underbrace{\biggl(-\frac{W_\beta[\bar{J},\phi]}{W_\beta[\phi,\dot{\bar{\phi}}]}\biggr)}_{JBARtophidot}
}^{\rightarrow |\bar{J}\;^\Omega_\beta\rangle}
\end{eqnarray}
Note, that in the factor $JBARTOPHIDOT$ does not depend on the choice
of $|\phi\rangle$.

With $W_\alpha[f,g]$ we denote the \textbf{Wronskian}\index{Wronskian}
\begin{eqnarray}
W_\alpha[f,g]\defas f_\alpha (\partial_r g_\alpha)-(\partial_r f_\alpha)g_\alpha
=\det\left[\begin{array}{cc}f&g\\\partial_rf&\partial_rg\end{array}\right]
\end{eqnarray}
which is used to match two functions differentiably to a third via
\begin{eqnarray}
y(x)\rightarrow f(x)\frac{W[y,g]}{W[f,g]}+g(x)\frac{W[y,f]}{W[g,f]}
\end{eqnarray}


The matrix elements
$\langle\tilde{p}_\gamma|\tilde{\chi}_\alpha\rangle$, which will be
needed later, have the form
\begin{eqnarray}
\langle\tilde{p}_\gamma|\tilde{\chi}_\alpha\rangle
=\langle\tilde{p}_\gamma|\tilde{\phi}^K_\alpha\rangle
-\sum_{R',L'}\langle\tilde{p}_\gamma|\tilde{\phi}^{\bar{J}}_{R',L'}\rangle
\bar{S}^\dagger_{R,L,R_\alpha,L_\alpha}
\nonumber\\
=\langle\tilde{p}_\gamma|\tilde{\phi}^K_\alpha\rangle
-\langle\tilde{p}_\gamma|\tilde{\phi}^{\bar{J}}_{R_\gamma,L_\gamma}\rangle
\bar{S}^\dagger_{R_\gamma,L_\gamma,R_\alpha,L_\alpha}
\end{eqnarray}

%====================================================================
\section{Coefficients of the tight-binding orbital}
%====================================================================
%====================================================================
\subsection{Introduction}
%====================================================================
In this section, I describe how to determine the wave functions in
terms of local orbitals, if the projections
$\langle\tilde{p}_\gamma|\tilde{\psi}\rangle$ onto the pseudo wave
functions are known.

The basic idea is to find a representation of the wave function in
terms of local orbitals $|\chi_\alpha\rangle$
\begin{eqnarray}
|\psi'_n\rangle=\sum_\alpha |\chi_\alpha\rangle q_\alpha\;,
\end{eqnarray}
such that the deviation from the true wave function $|\psi_n\rangle$
is as small as possible.

Ideally, this would amount to minimizing the mean-square deviation of
the orbital expansion from the wave function.
\begin{eqnarray*}
Q'[\vec{q}]:=\Bigl(\langle\psi_n|-\sum_\alpha q^*_\alpha\langle\chi_\alpha|\Bigr)
\Bigl(|\psi_n\rangle-\sum_\beta |\chi_\beta\rangle q_\beta\Bigr)
\end{eqnarray*}

Because evaluating the mean square deviation as integral over all
space is time consuming, we limit the integral to the augmentation
spheres.
\begin{eqnarray}
Q[\vec{q}]&:=&
\Bigl(\langle\tilde{\psi}_n|
-\sum_\alpha q^*_\alpha\langle\tilde{\chi}_\alpha|\Bigr)
\biggl[\sum_{\delta,\gamma}|\tilde{p}_\delta\rangle\langle\phi_\delta|
\theta_{\Omega_{R_\delta}}|\phi_\gamma\rangle\langle\tilde{p}_\gamma|\biggr]
\Bigl(|\tilde{\psi}_n\rangle
-\sum_\beta |\tilde{\chi}_\beta\rangle q_\beta\Bigr)
\nonumber\\
&=&
\sum_{\gamma}
\biggl[
\sum_{\delta}
\Bigl(\langle\tilde{\psi}_n|\tilde{p}_\delta\rangle
-\sum_\alpha q^*_\alpha\langle\tilde{\chi}_\alpha|\tilde{p}_\delta\rangle\Bigr)
\langle\phi_\delta|\theta_{\Omega_{R_\delta}}|\phi_\gamma\rangle
\biggr]
\Bigl(\langle\tilde{p}_\gamma|\tilde{\psi}_n\rangle
-\sum_\beta \langle\tilde{p}_\gamma|\tilde{\chi}_\beta\rangle q_\beta\Bigr)
\end{eqnarray}
where $\theta_{\Omega_{R_\delta}}$ is a step function that vanishes
outside the augmentation sphere at $R_\delta$.

Minimization yields
\begin{eqnarray}
\frac{\partial Q}{\partial q^*_\alpha}
&=&
-\sum_{\gamma}
\biggl[
\sum_{\delta}
\langle\tilde{\chi}_\alpha|\tilde{p}_\delta\rangle
\langle\phi_\delta|\theta_{\Omega_{R_\delta}}|\phi_\gamma\rangle
\biggr]
\Bigl(\langle\tilde{p}_\gamma|\tilde{\psi}_n\rangle
-\sum_\beta \langle\tilde{p}_\gamma|\tilde{\chi}_\beta\rangle q_\beta\Bigr)
\stackrel{!}{=}0
\nonumber
\end{eqnarray}


\begin{eqnarray}
\Rightarrow\qquad\sum_{\gamma}
\biggl[
\sum_{\delta}
\langle\tilde{\chi}_\alpha|\tilde{p}_\delta\rangle
\langle\phi_\delta|\theta_{\Omega_{R_\delta}}|\phi_\gamma\rangle
\biggr]
\langle\tilde{p}_\gamma|\tilde{\psi}_n\rangle
=
\sum_{\gamma,\beta}
\biggl[
\sum_{\delta}
\langle\tilde{\chi}_\alpha|\tilde{p}_\delta\rangle
\langle\phi_\delta|\theta_{\Omega_{R_\delta}}|\phi_\gamma\rangle
\biggr]
\langle\tilde{p}_\gamma|\tilde{\chi}_\beta\rangle q_\beta\Bigr)
\nonumber
\end{eqnarray}

\begin{eqnarray}
\Rightarrow\qquad
q_\beta=
\sum_{\beta}
\biggl[
\sum_{\gamma',\delta'}
\langle\tilde{\chi}_\alpha|\tilde{p}_{\delta'}\rangle
\langle\phi_{\delta'}|\theta_{\Omega_{R_\delta}}|\phi_{\gamma'}\rangle
\langle\tilde{p}_{\gamma'}|\tilde{\chi}_\beta\rangle 
\biggr]^{-1}
\biggl[
\sum_{\gamma\delta}
\langle\tilde{\chi}_\alpha|\tilde{p}_\delta\rangle
\langle\phi_\delta|\theta_{\Omega_{R_\delta}}|\phi_\gamma\rangle
\biggr]
\langle\tilde{p}_\gamma|\tilde{\psi}_n\rangle
\nonumber\\
\end{eqnarray}


This allows one to write the wave function in the form
\begin{eqnarray}
|\psi_n\rangle\approx
\sum_\alpha|\chi_\alpha\rangle\langle\tilde{\pi}_\alpha|\tilde{\psi}_n\rangle
\end{eqnarray}
with
\begin{eqnarray}
\langle\tilde{\pi}_\alpha|=
\sum_\gamma \biggl[
\sum_{\gamma',\delta'}
\langle\tilde{\chi}_\alpha|\tilde{p}_{\delta'}\rangle
\langle\phi_{\delta'}|\theta_{\Omega_{R_\delta}}|\phi_{\gamma'}\rangle
\langle\tilde{p}_{\gamma'}|\tilde{\chi}_\beta\rangle 
\biggr]^{-1}
\biggl[
\sum_{\delta}
\langle\tilde{\chi}_\alpha|\tilde{p}_\delta\rangle
\langle\phi_\delta|\theta_{\Omega_{R_\delta}}|\phi_\gamma\rangle
\biggr]
\langle\tilde{p}_\gamma|
\label{eq:pitilde}
\end{eqnarray}

This expression works also if the number of local orbitals
$|\chi_\alpha\rangle$ is smaller than the number of projector
functions $\langle{p}_\gamma|$. Because of the inversion, The
multicenter expansion for the projector function is long-ranged so
that this expression needs to be evaluated in a Bloch representation.

%===============================================================================
\subsection{Transformation between local-orbital and partial-wave projections}
%===============================================================================
In the previous section, I derived in \eq{eq:pitilde} a relation
between orbital and partial wave projector functions.
\begin{eqnarray}
\langle\tilde{\pi}_\alpha|\tilde{\psi}_n\rangle
&=&\sum_\beta M_{\alpha,\beta}\langle\tilde{p}_\alpha|\tilde{\psi}_n\rangle
\end{eqnarray}
This operation is performed in \verb|lmto$projtontbo| with \verb|ID='FWRD'|

The derivatives are correspondingly derived as 
\begin{eqnarray}
dE&=&\sum_{\alpha,\beta}\underbrace{\frac{dE}{d\rho_{\alpha,\beta}}}
_{=:h_{\beta,\alpha}}d\rho_{\alpha,\beta}
\nonumber\\
&=&\sum_{\alpha,\beta}h_{\beta,\alpha}
\Bigl[
\sum_n\langle\pi_\alpha|d\psi_n\rangle f_n\langle\psi_n|\pi_\beta\rangle
+\sum_n\langle\pi_\alpha|\psi_n\rangle df_n\langle\psi_n|\pi_\beta\rangle
+\sum_n\langle\pi_\alpha|\psi_n\rangle f_n\langle{d}\psi_n|\pi_\beta\rangle
\Bigr]
\nonumber\\
&=&\sum_n\sum_{\alpha}
f_n
\underbrace{\sum_{\beta}
\langle\psi_n|\pi_\beta\rangle h_{\beta,\alpha}}_{HTBC^\dagger_{n,\alpha}}
\langle\pi_\alpha|d\psi_n\rangle 
+
\sum_n\sum_{\beta}
\langle{d}\psi_n|\pi_\beta\rangle 
\underbrace{\sum_{\alpha}h_{\beta,\alpha} \langle\pi_\alpha|\psi_n\rangle}_{HTBC_{\beta,n}} f_n
\nonumber\\
&&+
\sum_n\sum_{\beta}
\langle\psi_n|\pi_\beta\rangle 
\underbrace{\sum_{\alpha}h_{\beta,\alpha} \langle\pi_\alpha|\psi_n\rangle}_{HTBC_{\beta,n}} df_n
\nonumber\\
&=&\sum_n\sum_{\gamma}
f_n
\underbrace{\sum_{\alpha}
\underbrace{\sum_{\beta}
\langle\psi_n|\pi_\beta\rangle h_{\beta,\alpha}}_{HTBC^\dagger_{n,\alpha}}
M_{\alpha,\gamma}}_{HPROJ^\dagger_{n,\gamma}}
\langle\tilde{p}_\gamma|d\tilde{\psi}_n\rangle 
+
\sum_n\sum_{\gamma}
\langle{d}\tilde{\psi}_n|\tilde{p}_\gamma\rangle 
\underbrace{
\sum_{\beta}M^\dagger_{\gamma,\beta}
\underbrace{\sum_{\alpha}h_{\beta,\alpha} \langle\pi_\alpha|\psi_n\rangle}_{HTBC_{\beta,n}}}_{HPROJ_{\gamma,n}} f_n
\nonumber\\
&+&
\sum_ndf_n\sum_{\gamma}
\langle\tilde{\psi}_n|\tilde{p}_\gamma\rangle 
\underbrace{
\sum_{\beta}M^\dagger_{\gamma,\beta}
\underbrace{\sum_{\alpha}h_{\beta,\alpha} \langle\pi_\alpha|\psi_n\rangle}_{HTBC_{\beta,n}}}_{HPROJ_{\gamma,n}} 
\end{eqnarray}

Thus, we first define the Hamiltonian $\mat{h}$ (HAMIL) 
\begin{eqnarray}
\underbrace{h_{\alpha,\beta}}_{HAMIL}&=&\frac{dE}{\rho_{\beta,\alpha}}
\nonumber\\
HTBC_{\beta,n}&=&
\sum_\alpha \underbrace{h_{\beta,\alpha}}_{HAMIL}
\underbrace{\langle\tilde{\pi}_\alpha|\tilde{\psi}_n\rangle}_{TBC_{\alpha,n}}
\nonumber\\
HPROJ_{\gamma,n}&=&\sum_\beta M^\dagger_{\gamma,\beta}\cdot HTBC_{\beta,n}
\end{eqnarray}

This operation is performed in
\verb|lmto$projtontbo| with \verb|ID='BACK'|.


Then the forces on the wave functions 
\begin{eqnarray}
\sum_{\gamma}
\frac{1}{f_n}\frac{\delta \Delta E}{\delta\langle\psi_n|}
=|\tilde{p}_\gamma\rangle HPROJ_{\gamma,n}
\end{eqnarray}
The forces acting on the occupations are, as usual, determined as
\begin{eqnarray}
\frac{\partial E}{\partial f_n}=\langle\psi_n|
\biggl[
\frac{1}{f_n}\frac{\delta E}{\delta\langle\psi_n|}
\biggr]
\end{eqnarray}

%====================================================================
\section{Tailed representation of the natural tight-binding orbitals}
%====================================================================
The tailed representation replaces the multi-center expansion of the
natural tight-binding orbitals by a one-center expansion. The
underlying idea the short-ranged tails of the screened orbitals can be
approximatd by exponential tails, that are matched to the one-center
expansion of the orbitals at the central site. Thus both the head and
the tail functions centered on the central site are extended by a
superposition of exponential tails.

%====================================================================
\section{How to choose the parameters}
%====================================================================
\begin{verbatim}
!CONTROL!DFT!NTBO 
   MODUS='HYBRID' OFFSITE=F  K2=-0.25 SCALERCUT=2. !END
!END!END!END
!STRUCTURE!SPECIES!NTBO    
   NOFL=1 1 1 1 CV=T LHFWEIGHT=0.15
   TAILLAMBDA=4.0 2.0 RAUG/RCOV=0.9 RTAIL/RCOV=1.2
!END!END!END
\end{verbatim}

%====================================================================
\subsection{Augmentation radius must be large for semi-core states}
%====================================================================
\textbf{Observation:} We had the problem that the total charge for
core states of Ca has been much larger than one.


\textbf{Explanation:} This was apparently due to a augmentation radius
that was chosen too small. The tail, represented by Hankel and Bessel
functions decayed much slower than the real core state, so that the
norm of the corresponding state was overestimated dramatically.

The augmentation radius specifies the matching radius of the Bessel
and Hankel functions to the nodeless partial waves.\footnote{It is not
  related to the matching radius of all-electron, nodeless and pseudo
  partial waves, which defines the augmentation. The parameter here
  defines the shape of the natural tight-binding orbital.} The kinetic
energy of the Hankel and Bessel function is set by the parameter
\verb|k2|. Ideally it would approximate the kinetic energy of the
partial waves at the augmentation radius.

\textbf{Remedy:} the augmentation radius must be chosen sufficiently
large so that semi-core states are well represented by their partial
wave alone, while the tail represented by Hankel and Bessel functions
is neglegible.\footnote{Special thanks to Robert Schade.}


%====================================================================
\chapter{Contributions to the Hamiltonian}
%====================================================================
%====================================================================
\section{Core-valence exchange}
%====================================================================
The exchange term between core and valence electrons acts like a
fixed, nonlocal potential acting on the electrons, of the
form\footnote{Note that this matrix $\mat{M}$ differs from the one
  with the same symbol in the previous section.}
\begin{eqnarray}
\hat{\tilde{v}}_{x,cv}=\sum_{\alpha,\beta}|\tilde{p}_\alpha\rangle 
M_{\alpha,\beta}\langle\tilde{p}_\beta|
\end{eqnarray}
The core-valence exchange is furthermore diagonal in the site indices.

\begin{eqnarray}
\langle\chi_\alpha|\hat{v}_{x,cv}|\chi_\beta\rangle
&=&
\sum_{\gamma,\delta}
\langle\chi_\alpha|p_\gamma\rangle 
M_{\gamma,\delta}\langle p_\delta|\chi_\beta\rangle
\nonumber\\
&=&
\sum_{\gamma,\delta}
\langle\tilde{\phi}^{K}_\alpha|\tilde{p}_\gamma\rangle 
M_{\gamma,\delta}
\langle\tilde{p}_\delta|\tilde{\phi}^{K}_\beta\rangle
\nonumber\\
&-&
\sum_{\gamma,\delta,\beta'}
\langle\tilde{\phi}^{K}_\alpha|\tilde{p}_\gamma\rangle 
M_{\gamma,\delta}
\langle\tilde{p}_\delta|\tilde{\phi}^{\bar{J}}_{\beta'}\rangle 
\bar{S}^\dagger_{\beta',\beta}
\nonumber\\
&-&
\sum_{\gamma,\delta,\alpha',\alpha}
\bar{S}_{\alpha,\alpha'}
\langle\tilde{\phi}^{\bar{J}}_{\alpha'}|\tilde{p}_\gamma\rangle 
M_{\gamma,\delta}
\langle\tilde{p}_\delta|\tilde{\phi}^{K}_{\beta}\rangle 
\nonumber\\
&+&
\sum_{\gamma,\delta,\alpha',\alpha}
\bar{S}_{\alpha,\alpha'}
\langle\tilde{\phi}^{\bar{J}}_{\alpha'}|\tilde{p}_\gamma\rangle 
M_{\gamma,\delta}
\langle\tilde{p}_\delta|\tilde{\phi}^{\bar{J}}_{\beta'}\rangle 
\bar{S}^\dagger_{\beta',\beta}
\end{eqnarray}
Here we used the augmented Hankel and screened Bessel fucntions,
respectively their pseudo versions.

As usual we build the expanded density matrix
\begin{eqnarray}
\left(\begin{array}{cc}
  \mat{\rho} \qquad& 
-\mat{\rho}\mat{\bar{S}}^\dagger\\
-\mat{\bar{S}}\mat{\rho} \qquad& 
\mat{\bar{S}}\mat{\rho}\mat{\bar{S}}^\dagger\\
\end{array}\right)
\end{eqnarray}

The matrix 
\begin{eqnarray}
\left(\begin{array}{cc}
\langle\tilde{\phi}^{K}|\tilde{p}\rangle 
\mat{M}\langle\tilde{p}|\tilde{\phi}^{K}\rangle &
\langle\tilde{\phi}^{K}|\tilde{p}\rangle 
\mat{M}\langle\tilde{p}|\tilde{\phi}^{\bar{J}}\rangle \\
\langle\tilde{\phi}^{\bar{J}}|\tilde{p}\rangle 
\mat{M}\langle\tilde{p}|\tilde{\phi}^{K}\rangle &
\langle\tilde{\phi}^{\bar{J}}|\tilde{p}\rangle 
\mat{M}\langle\tilde{p}|\tilde{\phi}^{\bar{J}}\rangle \\
\end{array}\right)
\end{eqnarray}
is calculated first using \verb|potpar1(isp)%prok| and
\verb|potpar1(isp)%projbar|.
\footnote{ In the earlier version the contribution from the
  $\dot{\bar{\phi}}$ has been ignored!!! It has been verified by
  temporarily switching off the jbar contributiuon to
  potpar1(isp)\%prok and potpar1(isp)\%projbar. In this old version
  only potpar(isp)\%ktophi is used to extract the $\phi$
  contribution.}

%====================================================================
\section{U-tensor}
%====================================================================
%====================================================================
\section{Double-counting correction}
%====================================================================
The double-counting correction is that of Eq.~36 of the paper by
Bl\"ochl, Walther, Pruschke\cite{bloechl11_prb84_205101}.
\begin{eqnarray}
E_{xc}^{\hat{W}_R}
&=&
\int d^3r\; n^\chi_R(\vec{r})
\epsilon_{xc}[n^\chi_{\sigma,\sigma'}(\vec{r})]
\frac{n^\chi_R(\vec{r})}{n^\chi(\vec{r})}
\nonumber\\
&=&
\int d^3r\; \biggl(n^\chi(\vec{r})
\epsilon_{xc}[n^\chi_{\sigma,\sigma'}(\vec{r})]\biggr)
\left(\frac{n^\chi_R(\vec{r})}{n^\chi(\vec{r})}\right)^2
\end{eqnarray}
The exchange correlation density is calculated with the correlated
orbitals, but there is an additional factor, which cuts of the
exchange correlation contribution at large radii, where the density
vanishes.

For hybrid functionals only the scaled Hartree-Fock energy is added,
so that the double counting is weighted as well and it is furthermore
limited to the exchange term only. \textbf{This will be different, if
  also corrections for the correlation energies are added.}

Other double-counting schemes are discussed in appendix~\ref{app:dc}.


Let us simplify the expression by introducing the symbols
$n_R=n^\chi_R(\vec{r})$ and $n_t=n^\chi_{\sigma,\sigma'}(\vec{r})$.
With $n^s_R$ and $n^s_t$ we denote the spherical parts of the
respective densities.

If the frozen-core density is employed, the contribution of the core
must not be included in the double-counting term.

The expression above can be written in the
following form:
\begin{eqnarray}
E_{xc}^{\hat{W}_R}&=&
\int d^3r\; 
\biggl[
n_t(\vec{r})
\epsilon_{xc}[n_t(\vec{r})]
-n^{core}(\vec{r})
\epsilon_{xc}[n^{core}(\vec{r})]
\biggr]
\left(\frac{n^s_R(\vec{r})}{n^s_t(\vec{r})}\right)^2
\nonumber\\
&=&
\int d^3r\; f_{xc}(\vec{r})
\left(\frac{n^s_R(\vec{r})}{n^s_t(\vec{r})}\right)^2
\end{eqnarray}
With $f_{xc}:=n_t\epsilon_{xc}[n_t]-n^{core}\epsilon_{xc}[n^{core}]$ and
$\mu_{xc}:=\frac{df_{xc}}{dn_t}$, we obtain
\begin{eqnarray}
dE_{xc}^{\hat{W}_R}
&=&\int d^3r\; 
\left(\frac{n^s_R}{n^s_t}\right)^2\mu_{xc} dn_t
+f_{xc}2\left(\frac{n^s_R}{n^s_t}\right)^2\frac{dn^s_R}{n^s_R}
-f_{xc}2\left(\frac{n^s_R}{n^s_t}\right)^2\frac{dn^s_t}{n^s_t}
\nonumber\\
&=&\int d^3r\; 
\biggl[
\left(\frac{n^s_R}{n^s_t}\right)^2\mu_{xc} 
-2f^s_{xc}\left(\frac{n^s_R}{n^s_t}\right)^2\frac{1}{n^s_t}\biggr]dn_t
+\biggl[2f^s_{xc}\left(\frac{n^s_R}{n^s_t}\right)^2\frac{1}{n^s_R}\biggr] dn^s_R
\end{eqnarray}
which yields the two potentials
\begin{eqnarray}
v_t&\defas&\left(\frac{n^s_R}{n^s_t}\right)^2
\biggl[\mu_{xc} -\frac{2f^s_{xc}}{n^s_t}\biggr]
\nonumber\\
v_R&\defas&\left(\frac{n^s_R}{n^s_t}\right)^2\frac{2f^s_{xc}}{n^s_R}
\end{eqnarray}
For the cutoff function $(n^s_R/n^s_t)^2$ we consider only the
spherical contributions of the density and we ignore the spin
contributions. This is accounted for in the derivations by only
considering the spherical part $f^s_{xc}$ of $f_{xc}$, while
maintaining the non-spherical contributions to $\mu_{xc}$.

The total density $n_t$ contains also the core density. If the core
valence Fock term is included, they are part of the correlated
electrons and need to be considered in the density $n_R$.

The double-counting correction has the negative sign, because the
DFT-expression needs to be subtracted.\footnote{This subtraction is
  done outside the routine calculating the double-counting term.}  Its
energy and the corresponding contributions to the auxiliary
Hamiltonian are
\begin{eqnarray}
E_{dc}&=&-\int d^3r\; \left(\frac{n^s_R}{n^s_t}\right)^2f^s_{xc}
\nonumber\\
d\hat{\tilde{H}}_{dc}&=&
-|\tilde{\pi}_\alpha\rangle\langle\chi_\alpha|\hat{v}_t|\chi_\beta\rangle
\langle\tilde{\pi}_\beta|
-|\tilde{p}_\alpha\rangle
\langle\phi_\alpha|\hat{v}_R|\phi_\beta\rangle
\langle\tilde{p}_\beta|
\end{eqnarray}

We start from two density matrices, which are the same at the moment.

see \verb|lmto_simpledc_new|.




%====================================================================
\subsubsection{Idea}
%====================================================================
\begin{eqnarray}
E_{xc}^{\hat{W}_R}=\sum_{a,b,c,d}\int d^3r\; 
\chi^*_a(r)\chi_b(r)\chi^*_d(r)\chi_c(r)
\frac{\epsilon_{xc}[n^\chi_{\sigma,\sigma'}(\vec{r})]}
{n^\chi(\vec{r})}
\end{eqnarray}




%====================================================================
\chapter{Description of Subroutines}
%====================================================================
%====================================================================
\section{Workflow}
%====================================================================
\begin{verbatim}
  ---initialization----- 
  POTPAR = potential parameters 
  SBAR = screened structureconstants 
  <ptilde|chitilde> tailed partial waves overlap (Onsite) 
  utensor (Onsite) 
  utensor (offsite) 
  ...  
  ----cycle-----------
  TBC=<pi-tilde|psi> from PROJ=<ptilde|psitide> 
  DENMAT density matrix in local orbitals 
  ...  
  total energy and derivatives 
  HAMIL hamiltonian matrix in tight-binding orbitals 
  ...  
  HTBC = de/dtbc * 1/f 
  HPROJ = de/dproj * 1/f
\end{verbatim}

%====================================================================
\section{LMTO\_POTPAR}
%====================================================================
All information that depends directly on the partial waves is stored
in the structure POTPAR.

\begin{tabular}{|l|l|}
\hline
\hline
\multicolumn{2}{|c|}{POTPAR Structure}\\
\hline
\hline
RAD &  augmentation radius\\
\hline
\multicolumn{2}{|l|}{Quantities connected to head functions}\\
\hline
NHEAD & number of head functions\\
LOFH(NHEAD) &  angular momentum\\
ITAIL(NHEAD) & pointer to tail function\\
LNOFH(NHEAD) & pointer to partial wave $|\phi\rangle$ and projector $\langle{p}|$\\
KTOPHI(NHEAD) & $|K^\Omega\rangle\rightarrow |\phi\rangle c_{KTOPHI}
+|\dot{\bar{\phi}}\rangle c_{KTOPHIDOT}$
\\
KTOPHIDOT(NHEAD) & \\
\hline
\multicolumn{2}{|l|}{Quantities connected to tail functions}\\
\hline
NTAIL(NTAIL) & number of tail functions\\
LOFT(NTAIL)  &  angular momentum\\
LNOFT(NTAIL) & pointer to partial wave $|\phi\rangle$ and projector $\langle{p}|$\\
QBAR(NTAIL)  & screening charge $|\bar{J}\rangle=|J\rangle-|K\rangle \bar{Q}$\\
JBARTOPHIDOT(NTAIL)  & $|\bar{J}^\Omega\rangle\rightarrow 
                |\dot{\bar{\phi}}\rangle c_{JBARTOPHIDOT}$\\
\hline
\multicolumn{2}{|l|}{Other stuff}\\
\hline
PROK(LNX,NHEAD) & $\langle\tilde{p}|\phi c_{ktopphi}+\dot{\bar{\phi}} c_{ktophidot}\rangle$\\
PROJBAR(LNX,NTAIL) & $\langle\tilde{p}|\dot{\bar{\phi}} c_{jbartophidot}\rangle$
\\
PHIOV(LNX,LNX) & $\langle\phi_\alpha|\theta_\Omega|\phi_\beta\rangle$ \\
\hline
\multicolumn{2}{|l|}{Tailed representation}\\
\hline
TAILED\%GID & grid id  for the radial grid\\
TAILED\%LNX & \\
TAILED\%LMNX & \\
TAILED\%LOX(LNX) & \\
TAILED\%AEF(NR,LNX) & \\
TAILED\%PSF(NR,LNX) & \\
TAILED\%NLF(NR,LNX) & \\
TAILED\%U(LMNX,LMNX,LMNX,LMNX) & \\
TAILED\%OVERLAP(LMNX,LMNX) & \\
TAILED\%QLN(2,LNX,LNX) & \\
\hline
\hline
\end{tabular}
The variable lnx and lmnx in the substructure TAILED differ from the
corresponding functions from the partial wave expansion.



%====================================================================
\section{LMTO\$CLUSTERSTRUCTURECONSTANTS}
%====================================================================
\verb|LMTO\$CLUSTERSTRUCTURECONSTANTS| calculates the screened
structure constants \verb|SBAR| ($\mat{\bar{S}}$) for a cluster of
\verb|NAT| atomic sites \verb|RPOS|, of which the first site is called
the central site of the cluster. The number of angular momenta on each
site is defined by \verb|LX|. The screening is defined by the vector
\verb|QBAR| ($\bar{Q}$). \verb|K2| ($\vec{k}^2=-\kappa^2$) is the
squared wave vector. (For envelope functions that fall off
exponentially, this parameter is negative.)

\begin{verbatim}
SUBROUTINE LMTO$CLUSTERSTRUCTURECONSTANTS(K2,NAT,RPOS,LX,QBAR,NORB,N,SBAR)
REAL(8)   ,INTENT(IN) :: K2          
INTEGER(4),INTENT(IN) :: NAT         ! NUMBER OF ATOMS ON THE CLUSTER
REAL(8)   ,INTENT(IN) :: RPOS(3,NAT) ! ATOMIC POSITIONS ON THE CLUSTER
INTEGER(4),INTENT(IN) :: LX(NAT)     ! X(ANGULAR MOMENTUM ON EACH CLUSTER)
INTEGER(4),INTENT(IN) :: N
REAL(8)   ,INTENT(IN) :: QBAR(N)
INTEGER(4),INTENT(IN) :: NORB
REAL(8)   ,INTENT(INOUT):: SBAR(NORB,N)
\end{verbatim}




First, the bare structure constants are evaluated on the cluster using
\verb|LMTO\$STRUCTURECONSTANTS| and then the structure constants are
screened using \verb|LMTO\$SCREEN|.

%====================================================================
\subsection{LMTO\$STRUCTURECONSTANTS}
%====================================================================
\verb|LMTO\$STRUCTURECONSTANTS| calculates the bare structure
constants for a pair of sites. The first site is at the origin, where
the Hankel function is centered, and the second site at $\vec{R}$
specified by \verb|R21|, is the center of the expansion into solid
Bessel functions.

\begin{verbatim}
subroutine lmto$structureconstants(r21,K2,L1x,L2x,S)
REAL(8)   ,INTENT(IN) :: R21(3) ! EXPANSION CENTER
INTEGER(4),INTENT(IN) :: L1X
INTEGER(4),INTENT(IN) :: L2X
REAL(8)   ,INTENT(IN) :: K2 ! 2ME/HBAR**2
REAL(8)   ,INTENT(OUT):: S((L1X+1)**2,(L2X+1)**2)
\end{verbatim}

The bare structure constants are evaluated in 
\verb|LMTO$STRUCTURECONSTANTS| as
\begin{eqnarray}
S_{RL,R'L'}\eqrel{eq:fortmulaforbarestructureconstants}{=}
(-1)^{\ell'+1} 4\pi \sum_{L''} C_{L,L',L''} 
H_{L''}(\vec{R}'-\vec{R})
\begin{cases}
(-ik)^{\ell+\ell'-\ell''}&\text{for $k^2>0$}\\
\delta_{\ell+\ell'-\ell''}&\text{for $k^2=0$}\\
\kappa^{\ell+\ell'-\ell''}&\text{for $k^2=-\kappa^2<0$}\\
\end{cases}
\label{eq:fortmulaforbarestructureconstantscopy1}
\end{eqnarray}
where $H_L(k^2,\vec{R})$ is the solid Hankel function calculated in
\verb|LMTO$SOLIDHANKEL|. The solid Hankel function is the solution of
the Helmholtz equation, \eq{eq:solidhelmholtzequation}.\footnote{The
  factors and signs of the inhomogeneity need to be confirmed. The
  equation has been taken from the methods book, chapter
  \textit{``Working with spherical Hankel and Bessel functions.}}

More information on the solid Hankel function can be found in
appendix~\ref{app:solidhankel}.

Remark: Because the Gaunt coefficients vanish for odd
$\ell+\ell'-\ell''$, the structure constants are real even for
$k^2>0$.


%====================================================================
\subsection{LMTO\$SCREEN}
%====================================================================
\textit{I describe here what has been implemented as ``version 3''.}

\verb|LMTO$SCREEN| takes the bare structure constants $S_{RL,R'L'}$
connecting all orbitals on a specific cluster with each other and the
screening constants $\bar{Q}$ for all orbitals on the cluster. It
returns the screened structure constants connecting the orbitals on
the central (first) site (1st index) with all orbitals (2nd index).

The structure constants are defined so that
\begin{eqnarray}
\langle K_{RL}|=-\sum_{L'} S_{RL,R'L'}\langle J_{R'L'}|
\qquad\text{for $R'\neq R$}
\end{eqnarray}

First we evaluate 
\begin{eqnarray}
\mat{A}=\mat{1}-\mat{\bar{Q}}\mat{S}^\dagger
\end{eqnarray}
and the vectors $\vec{e}_\alpha$ defined by
$(\vec{e}_\alpha)_\beta=\delta_{\beta,\alpha}$. Note that the number
of vectors corresponds to the number of orbitals on the central site
only. Therefore, these vectors do not build up a complete unit matrix.

Then we solve the equation system
\begin{eqnarray}
\mat{A}\vec{c}_\alpha
&\eqrel{eq:forscreenedsa}{=}&
\vec{e}_\alpha
\end{eqnarray}
for $\vec{c}_\alpha$ and
\begin{eqnarray}
\vec{s}_\alpha&\eqrel{eq:forscreenedsb}{=}&
\mat{S}^\dagger\vec{c}_\alpha
\end{eqnarray}
$\Bigl(\vec{s}_\alpha\Bigr)_\beta=\bar{S}^\dagger_{\beta,\alpha}$
contains the transposed screened structure constants. After
transposition, $\bar{S}$ is returned.

%====================================================================
\section{Waves object}
%====================================================================
The data exchange betweeen the waves object and the lmto object is
determined by the local-orbital projections
$\langle\tilde{\pi}_\alpha|\tilde{\psi}_n\rangle$ specified by the
array \verb|THIS%TBC|, which in turn is obtained from the partial-wave
projections $\langle\tilde{p}|\tilde{\psi}_n\rangle$.



In \verb|waves$etot|
\begin{verbatim}
CALL WAVES$TONTBO
-> CALL LMTO$PROJTONTBO('FWRD'...)
..
..
CALL LMTO$ETOT(LMNXX,NDIMD,NAT,DENMAT)
..
..
CALL WAVES$FROMNTBO()
-> CALL LMTO$PROJTONTBO('BACK'...)
..
..
CALL WAVES$FORCE
-> CALL WAVES_FORCE_ADDHTBC
...
CALL WAVES$HPSI
\end{verbatim}


\begin{eqnarray*}
\vec{F}&=&
-\sum_\alpha
\frac{dE}{d\langle\tilde{p}_\alpha|\psi_n\rangle}
\langle\vec{\nabla}_R\tilde{p}_\alpha|\psi_n\rangle
+\mathrm{c.c.}
\nonumber\\
&=&-\sum_{\alpha,\beta}
\frac{dE}{d\langle\tilde{\pi}_\beta|\psi_n\rangle}
\frac{d\langle\tilde{\pi}_\beta|\psi_n\rangle}
{d\langle\tilde{p}_\alpha|\psi_n\rangle}
\langle\vec{\nabla}_R\tilde{p}_\alpha|\psi_n\rangle
+\mathrm{c.c.}
\nonumber\\
&=&-\sum_{\alpha,\beta}
\frac{dE}{d\langle\tilde{\pi}_\beta|\psi_n\rangle}
\frac{d\langle\tilde{\pi}_\beta|\psi_n\rangle}
{d\langle\tilde{p}_\alpha|\psi_n\rangle}
\Bigl[-\langle\vec{\nabla}_r\tilde{p}_\alpha|\psi_n\rangle\Bigr]
+\mathrm{c.c.}
\end{eqnarray*}


%=======================================================================
\section{Offsite matrix elements}
%=======================================================================
The offsite matrix elements are kept in the data type 
\begin{verbatim}
TYPE OFFSITEX_TYPE
 INTEGER(4)         :: NDIS
 INTEGER(4)         :: NF
 REAL(8)   ,POINTER :: OVERLAP(:,:)  ! OVERLAP MATRIX ELEMENTS
 REAL(8)   ,POINTER :: X22(:,:)      !
 REAL(8)   ,POINTER :: X31(:,:)
 REAL(8)   ,POINTER :: BONDU(:,:)
 REAL(8)   ,POINTER :: DIS(:)
 REAL(8)   ,POINTER :: LAMBDA(:)
END TYPE OFFSITEX_TYPE
\end{verbatim}

The matrix elements are initialized in \verb|LMTO_initialize|
\begin{verbatim}
LMTO_TAILEDGAUSSFIT()
   GAUSSIAN_FITGAUSS(GID,NR,W,L,AUX,NE,NPOW2,E,C(:NPOW2,:,LN))
   LMTO_TAILEDGAUSSORBTOYLM()
LMTO_OFFXINT
    LMTO_OFFSITEOVERLAPSETUP  !O(AB)  ->OFFSITEX%OVERLAP
       LMTO_TWOCENTER
    LMTO_OFFSITEX22SETUP      !U(AABB) ->OFFSITEX%X22
       LMTO_TWOCENTER
    LMTO_OFFSITEX31SETUP      !U(AAAB) ->OFFSITEX%X31
       LMTO_TWOCENTER
    LMTO_TAILEDGAUSSOFFSITEU  !U(ABAB) ->OFFSITEX%BONDU
       GAUSSIAN$ZDIRECTION_FOURCENTER(NIJKA,NEA,EA,LMNXA,ORBA &
    LMTO_OFFSITEXCONVERT()
\end{verbatim}

The energy contibution is then calculated using offsitex as follows
\begin{verbatim}
LMTO_OFFSITEXEVAL_NEW(EX)  
    LMTO_EXPANDNONLOCAL
    LMTO_EXPANDLOCAL
    SPHERICAL$ROTATEYLM(LMX,ROT,YLMROT)
    LMTO_OFFSITEX22U(ISPA,ISPB, DIS,LMNXTA,LMNXTB,U22,DU22)
       lmto_offsitexvalue
    LMTO_OFFSITEX31U(ISPA,ISPB, DIS,LMNXTA,LMNXTB,U3A1B,DU3A1B)
       lmto_offsitexvalue
    LMTO_OFFSITEX31U(ISPB,ISPA,-DIS,LMNXTB,LMNXTA,U3B1A,DU3B1A)
       lmto_offsitexvalue
    LMTO_OFFSITEXBONDU(ISPA,ISPB,DIS,LMNXTA,LMNXTB,BONDU,DBONDU)
       lmto_offsitexvalue
\end{verbatim}

%=======================================================================
\section{Matrix elements using Gaussians}
%=======================================================================
\verb|LMTO_TAILEDGAUSSOFFSITEU| uses the gauss decomposition of the
tailed orbitals in \verb|potpar%tailed%gaussnlf|.  In
\verb|tailedgaussfit| the following data structure is prepared.
\begin{verbatim}
POTPAR%TAILED%GAUSSNLF%NIJK
POTPAR%TAILED%GAUSSNLF%NORB
POTPAR%TAILED%GAUSSNLF%NPOW
POTPAR%TAILED%GAUSSNLF%NE
POTPAR%TAILED%GAUSSNLF%E
POTPAR%TAILED%GAUSSNLF%C
\end{verbatim}




%=======================================================================
\section{Matrix elements on an adaptive grid}
%=======================================================================
\verb|lmto_twocenter|

\begin{verbatim}
MODULE LMTO_TWOCENTER_MODULE
LMTO_TWOCENTER
    ADAPT$EVALUATE
       ADAPTINI
       ADAPT_BASICRULE
           ADAPT_INTEGRAND
             LMTO_TWOCENTER_MYFUNC
\end{verbatim}


%=======================================================================
\section{Routines for reporting}
%=======================================================================
\begin{verbatim}
LMTO$REPORT(NFIL)
..
LMTO$REPORTOVERLAP(NFIL)
LMTO$REPORTSBAR(NFIL)
LMTO$REPORTDENMAT(NFIL)
LMTO$REPORTHAMIL(NFIL)
LMTO$REPORTPERIODICMAT(NFIL,NAME,NNS,SBAR)
...
LMTO$WRITEPHI(FILE,GID,NR,NPHI,PHI)
\end{verbatim}

%=======================================================================
\section{Routines for plotting orbitals}
%=======================================================================
There are a three routines that calculate the orbitals either in a
spherical-harmonics expansion on radial grids or directly on an array
of real-space points.
\begin{verbatim}
LMTO_TAILEDORBLM(IAT,IORB,NR,LMX,ORB)
LMTO_TAILED_NTBOOFR(IAT,iORB,NP,P,chi)
  LMTO_TAILEDORBLM(IAT,IORB,NR,LMX,ORB)
LMTO_NTBOOFR(IAT,iORB,NP,P,chi)
\end{verbatim}


\begin{verbatim}
LMTO_PLOTTAILED()  [OK]
    LMTO_TAILEDORBLM(IAT,IORB,NR,LMX,ORB)
LMTO_GRIDPLOT(type,iat)  [ok]
    LMTO_GRIDORB_CUBEGRID(R0(:,IAT0),RANGE,N1,N2,N3,ORIGIN,TVEC,P)
    LMTO_GRIDORB_STARGRID(R0(:,IAT0),RANGE,NDIR,DIR,NRAD,X1D,P)
    LMTO_TAILED_NTBOOFR(TYPE,IAT,iORB,NP,P,chi)
        LMTO_TAILEDORBLM(IAT,IORB,NR,LMX,ORB)
        LMTO_NTBOOFR(IAT,iORB,NP,P,chi)
    LMTO_WRITECUBEFILE
..
LMTO_GRIDPLOT_UNTAILED(IAT0)
    LMTO_GRIDORB_CUBEGRID(R0(:,IAT0),RANGE,N1,N2,N3,ORIGIN,TVEC,P)
    LMTO_GRIDORB_STARGRID(R0(:,IAT0),RANGE,NDIR,DIR,NRAD,X1D,P)
    LMTO_GRIDENVELOPE(RBAS,NAT,R0,IAT0,LMX,NP,P,ENV,ENV1)
    LMTO_GRIDAUGMENT(RBAS,NAT,R0,IAT0,LMX,NP,P,ORB1,ENV1)
    LMTO_GRIDGAUSS(RBAS,NAT,R0,IAT0,LMX,NP,P,ORBG)
    LMTO_WRITECUBEFILE(NFIL,NATCLUSTER,ZCLUSTER,RCLUSTER &
LMTO_PLOTLOCORB(IAT0)
    LMTO_GRIDENVELOPE(RBAS,NAT,R0,IAT0,LM1X,NP,P,ENV,ENV1)
    LMTO_GRIDAUGMENT(RBAS,NAT,R0,IAT0,LM1X,NP,P,ORB1,ENV1)
    LMTO_GRIDGAUSS(RBAS,NAT,R0,IAT0,LM1X,NP,P,ORBG)
    LMTO_WRITECUBEFILE
LMTO_PLOTNTBO(TYPE,IATORB,LMNORB)
    LMTO_GRIDORB_CUBEGRID(CENTER,RADIUS,N1,N2,N3,ORIGIN,TVEC,P)
    LMTO_GRIDORB_STARGRID(CENTER,RADIUS,NDIR,DIR,NR,X1D,P)
    LMTO_TAILED_NTBOOFR(TYPE,IATORB,LMNORB,NP,P,ORB)
    LMTO_NTBOOFR(TYPE,IATORB,LMNORB,NP,P,ORB)
..
LMTO$PLOTWAVE(NFIL,IDIM0,IB0,IKPT0,ISPIN0,NR1,NR2,NR3)
    LMTO$PLOTWAVE_TAILED(NFIL,IDIM0,IB0,IKPT0,ISPIN0,NR1,NR2,NR3)
        WRITEWAVEPLOTC(NFIL,TITLE,RBAS,NAT,R0,ZAT,Q,NAME,XK,NR1,NR2,NR3,WAVE)
\end{verbatim}






\begin{itemize}
\item \verb|LMTO_TAILEDORBLM(IAT,IORB,NR,LMX,ORB)| calculates a
  specific orbital on radial grids in a spherical harmonics
  representation.
%
\item \verb|LMTO_PLOTTAILED()| writes the tailed local orbitals in a
  spherical harmonics expanion to file, so that the componenst can be
  viewed by xmgrace. Each orbital is written to a file \verb|CHI5_3.DAT|
  where 5 is the atom index and 3 is the orbital index.
%
\item \verb|LMTO_TAILED_NTBOOFR(IAT,iORB,NP,P,chi)| calculates a
  specific tailed orbital on a set of real space points.
%
\item \verb|LMTO_GRIDPLOT_TAILED(IAT0)| writes the orbitals for the
  specified site to file. It supports a 3D representation with cube
  files, and one-dimensional representation on a star grid.
\end{itemize}

%=======================================================================
\subsubsection{Multicenter expansion}
%=======================================================================

First the interstitial orbital is determined

\begin{eqnarray}
|\bar{K}^I_\alpha\rangle&=&|\bar{K}^\infty_\alpha\rangle 
- \Bigl[ |K^\Omega_\alpha\rangle 
-\sum_\beta |\bar{J}^\Omega_\beta\rangle \bar{S}^\dagger_{\beta,\alpha}
\Bigr]
\nonumber\\
&=&\sum_\beta |K^\infty_\beta\rangle \Bigl(\delta_{\alpha,\beta}+\bar{Q}_\beta
\bar{S}^\dagger_{\beta,\alpha}\Bigr)
- \Bigl[ |K^\Omega_\alpha\rangle 
-\sum_\beta |\bar{J}^\Omega_\beta\rangle \bar{S}^\dagger_{\beta,\alpha}
\Bigr]
\nonumber\\
&=&\sum_\beta |K^\infty_\beta\rangle \Bigl(\delta_{\alpha,\beta}+\bar{Q}_\beta
\bar{S}^\dagger_{\beta,\alpha}\Bigr)
- \Bigl[ |K^\Omega_\alpha\rangle 
-\sum_\beta 
\Bigl(
|J^\Omega_\beta\rangle-|K^\Omega_\beta\rangle\bar{Q}_\beta\Bigr)
 \bar{S}^\dagger_{\beta,\alpha}
\Bigr]
\nonumber\\
&=&\sum_\beta 
\Bigl(|K^\infty_\beta\rangle -|K^\Omega_\beta\rangle \Bigr)
\Bigl(\delta_{\alpha,\beta}+\bar{Q}_\beta
\bar{S}^\dagger_{\beta,\alpha}\Bigr)
+\sum_\beta 
|J^\Omega_\beta\rangle\bar{S}^\dagger_{\beta,\alpha}
\end{eqnarray}

This implies that within the sphere centered at $R_\beta$ only the
expansion into bare bessel functions survive, while outside only the
bare Hankel function is considered.




%=======================================================================
\chapter{Benchmarks}
%=======================================================================
%=======================================================================
\section{Silicon}
%=======================================================================
The files are in \url{~/Tree/Projects/SetupTests/Si}.

I performed calculation for a silicon crystal. We used a mixing of the
lcoal exchange of $\alpha_X=0.1$. The local basiset included one
s-type and one p-type tight-binding function.

The following dependencies have been explored:
\begin{itemize}
%
%================= k2 ================================================
\item kinetic energy of the Hankel function: $K2=0,\ldots,-0.5$
\begin{figure}[h!]
\begin{center}
\includegraphics[width=0.4\linewidth,clip=true]{Figs/Eofk2si/eofk2.eps}
\includegraphics[width=0.4\linewidth,clip=true]{Figs/Gapofk2si/gapofk2.eps}
\end{center}
\caption{Energy in Hartree and band gap in eV of the silicon crystal
  as function of the kinetic energy (times -0.01) of the envelope
  function for the PBE functional (black; none), local Hartree-Fock
  (red; onsite or local) NDDO-type exchange, i.e density of one site
  with the density on a bond partner (blue, NDDO), terms with three
  orbitals on one site and one on the bond partner (green NDDO+31) and
  the exchange of the bond density with itself (orange, all bond).
  The full line is the result for one s-type and one p-type
  orbital. For the dashe4d lines also d-type orbitals are
  included. The dotted lines are calculated with double orbitals for
  s,p and single orbitals for d.}
\end{figure}
%
%================= raug ================================================
\item Augmentation radius. The augmentation radius is the radius at
  which the partial waves are matched to the envelope function. It is
  also the radius for the sphere used to derive the projector function
  onto the local orbitals.
\begin{figure}[h!]
\begin{center}
\includegraphics[width=0.5\linewidth,clip=true]{Figs/Eofraugsi/eofraug.eps}
\includegraphics[width=0.4\linewidth,clip=true]{Figs/Gapofraugsi/gapofraug.eps}\hfill
\end{center}
\caption{Energy in Hartree and band gap in eV of the silicon crystal
  as function of the augmentation radius for the PBE functional
  (black; none), local Hartree-Fock (red; onsite or local) NDDO-type
  exchange, i.e density of one site with the density on a bond partner
  (blue, NDDO), terms with three orbitals on one site and one on the
  bond partner (green NDDO+31) and the exchange of the bond density
  with itself (orange, all bond).  The full line is the result for one
  s-type and one p-type orbital. For the dashe4d lines also d-type
  orbitals are included. The dotted lines are calculated with double
  orbitals for s,p and single orbitals for d.}
\end{figure}
The procedure apparently breaks down completely, if the augmentation
radius is too large. The dependency becomes stronger, if more
non-local terms are included. This may be due to the fact that the
double-counting term is only included for the local terms.
%
%================= rtail ================================================
\item tail matching radius
\begin{center}
\includegraphics[width=0.4\linewidth,clip=true]{Figs/Eofrtailsi/eofrtail.eps}
\includegraphics[width=0.4\linewidth,clip=true]{Figs/Gapofrtailsi/gapofrtail.eps}
\\
\includegraphics[width=0.4\linewidth,clip=true]
{Figs/Eofrtailsi/eofrtailsi110hr.eps}
\end{center}
\end{itemize}
%=======================================================================
\subsection{Summary}
%=======================================================================
\begin{itemize}
\item The augmentation radius has the largest effects on the results
  both for the gap and for the total energy. Beyond a certain radius
  ($>1.2~r_{cov}$) the calculation becomes even unstable. For an
  ``overcomplete TB-basisset $2s+2p+1d$ teh calculationb also fails for
  ($<1.1~r_{cov}$).
%
\item It becomes evident that the energy is rather insensitive to the
  parameters describing the local orbitals for local exchange and the
  NDDO terms. Additional terms such as ``31'' and ``bondx'', which
  include the bond overlap density $\chi_R(r)\chi_{R'}(r)$ lead to a
  very strong dependency on the choice of orbitals. Note that
  0.05~H$\approx$1.4~eV!

  This may be due to the poor description of the bond density by the
  exponential tails ofg the tailed orbitals.

  It may also be due to the fact the these terms are not compensated
  by the double-counting term.
% 
\item It should be noted that the double-counting term is included
  only for local exchange.
%
\item Choose value of $-0.2<k^2<-0.5$~H 
%
\item Choose value of $r_{tail}=1.2 r_{cov}$
\end{itemize}





\appendix
%=======================================================================
\chapter{Definition of solid Hankel functions}
\label{app:solidhankel}
%=======================================================================
The solid Hankel function has the form
\begin{eqnarray}
H_L(\vec{R})=Y_L(\vec{R})
\begin{cases}
n_\ell(\sqrt{k^2}\cdot|\vec{R}|) \cdot \sqrt{k^2}^{\ell+1}
&\text{for $k^2>0$  (Abramovitz 10.1.26)}\\
m_\ell(\sqrt{-k^2}\cdot|\vec{R}|) \cdot \sqrt{\frac{2}{\pi}} \sqrt{-k^2}^{\ell+1}
&\text{for $k^2<0$  (Abramovitz 10.2.4)}\\
(2\ell-1)!! |\vec{R}|^{-\ell-1} 
&\text{for $k^2=0$  (Abramovitz 10.2.5)}\\
\end{cases}
\end{eqnarray}
The solid Hankel function is defined such that the boundary conditions
at the origin are independent of $k^2$.

\begin{itemize}
\item the function
\begin{eqnarray}
n_\ell(r)=r^\ell\Bigl(-\frac{1}{r}\partial_r\Bigr)^\ell\frac{1}{r}\cos(r)
\label{eq:defneumann}
\end{eqnarray}
is the spherical Neumann function (see Eq. 8.175 of Cohen Tannoudhi
Band 2), which is also called the spherical Bessel function of the
second kind. Abramowitz defines $n_\ell(r)=-y_\ell(r)$ (compare
Abramowitz Eq. 10.1.26)

The spherical Neumann function obeys the radial Helmholtz equation
(Abramowitz Eq. 10.1.1) for positive kinetic energy
\begin{eqnarray}
r^2\partial^2_r n_\ell+2r\partial_r n_\ell+\Bigl(r^2-\ell(\ell+1)\Bigr) n_\ell=0
\nonumber\\
\Rightarrow\quad
\Bigl[-\frac{1}{r}\partial_r r +\frac{\ell(\ell+1)}{r^2}\Bigr]
n_\ell(r)=+n_\ell(r)
\end{eqnarray}

\textbf{Note that the subroutine SPFUNCTION\$NEUMANN returns the
  Neumann function with the opposite sign, namely what Abramowitz
  defines as Bessel function of the second kind. The minus sign is
  added in the calling routine.}
%
\item The function 
\begin{eqnarray}
m_\ell(r)=r^\ell\Bigl(-\frac{1}{r}\partial_r\Bigr)^\ell\frac{1}{r}\e{-r}
\label{eq:defml}
\end{eqnarray}
used for $k^2<0$ is obeys the radial Helmholtz equation (Abramowitz
Eq. 10.2.1) for negative kinetic energy
\begin{eqnarray}
r^2\partial^2_r m_\ell
+2r\partial_r m_\ell
-\Bigl(r^2+\ell(\ell+1)\Bigr) m_\ell=0
\nonumber\\
\Rightarrow\quad
\Bigl[-\frac{1}{r}\partial_r r +\frac{\ell(\ell+1)}{r^2}\Bigr]
m_\ell(r)=-m_\ell(r)
\end{eqnarray}
They are solutions for negative energy and therefore they fall off
exponentially.  The solution $m_\ell(r)$ is proportional to the
modified spherical Bessel functions of the third kind as defined by
Abramowitz\cite{abramowitz64_book} in their Eq. 10.2.4.
\begin{eqnarray}
m_\ell(r)=\frac{2}{\pi} \Bigl[\sqrt{\frac{\pi}{2r}} K_{\ell+1}(r)\Bigr]
\end{eqnarray}
which can be verified by comparing the defining equation \eq{eq:defml}
with equations 10.2.24-25 and the definition Eq. 10.2.4 of Abramowitz.
\end{itemize}


%=======================================================================
\section{Bare structure constants}
%=======================================================================
\textit{ This section is copied from Methods-book, Section ``Working
  with spherical Hankel and Bessel functions'', Peter Bl\"ochl, private
  communication.}

The bare structure constants have been determined first by
Segall\cite{segall57_pr105_108}. He uses the
theorem\cite{korringa47_physica13_392} that supposedy goes back to
Kasterin (N. Kasterin, Proc. Acad. Sci Amsterdam 6, 460 (1897/98));
see Seegall\cite{segall57_pr105_108}, Eq. B4)
\begin{eqnarray}
h_\ell^{(1)}(\kappa r)Y_{\ell,m}(\vec{r})=i^{-\ell}
\mathcal{Y}_{\ell,m}(\vec{\nabla}_r)h_0^{(1)}(\kappa r)
\label{eq:hlfromh0}
\end{eqnarray}
where $h^{(1)}_\ell(x)$ i the spherical Hankel function of the first
kind (see \eq{eq:defsphericalhankelfirstkind} below)
and where (Eq. B5 of Segall\cite{segall57_pr105_108})
\begin{eqnarray}
\mathcal{Y}_{\ell,m}(\vec{\nabla})
&=&\sqrt{\frac{2\ell+1}{4\pi}\frac{(\ell-m)!}{(\ell+m)!}}
\biggl(\frac{1}{ik}\biggr)^{|m|}\biggl(\partial_x\pm i\partial_y\biggr)
\mathcal{P}_\ell^{|m|}\left(\frac{1}{ik}\partial_z\right)
\end{eqnarray}
where the positive sign applies for nonzero $m$ and the negative sign
for negative $m$. Furthermore (see Segall\cite{segall57_pr105_108}
Eq.~B5)
\begin{eqnarray*}
\mathcal{P}_\ell^{|m|}(z)=\frac{d^{|m|}P_\ell(z)}{dz^{|m|}}
\end{eqnarray*}
where $P_\ell(z)$ is the conventional Legendre polynomial.

In addition Segall\cite{segall57_pr105_108} refers in his Eq. B7 to
Morse and Feshbach\cite{morse53_book} (part II, p. 1574) for
\begin{eqnarray}
h_0^{(1)}(\kappa|\vec{r}-\vec{r'}|)&=&
4\pi\sum_L \biggl( h^{(1)}_\ell(\kappa|\vec{r'}|)Y_L(\vec{r'})\biggr)
j_\ell(\kappa|\vec{r}|)Y_L^*(\vec{r})
\label{eq:structureconstant0}
\end{eqnarray}
which is valid for $|\vec{r'}|>|\vec{r}|$.

The two equations, \eq{eq:hlfromh0} and \eq{eq:structureconstant0},
can be combined into
\begin{eqnarray*}
h_\ell^{(1)}(\kappa |\vec{r}|)Y_{\ell,m}(\vec{r})
&\eqrel{eq:hlfromh0}{=}&
i^{-\ell}\mathcal{Y}_{\ell,m}(\vec{\nabla}_r)h_0^{(1)}(\kappa |\vec{r}|)
\\
&=&
i^{-\ell}\mathcal{Y}_{\ell,m}(\vec{\nabla}_r)
h_0^{(1)}(\kappa |(\vec{r}-\vec{R})+\vec{R}|)
\\
&\eqrel{eq:structureconstant0}{=}&
i^{-\ell}\mathcal{Y}_{\ell,m}(\vec{\nabla}_r)
\biggl[
4\pi\sum_{L'}\biggl( h^{(1)}_{\ell'}(\kappa|\vec{R}|)Y_{L'}(-\vec{R})\biggr)
j_{\ell'}(\kappa|\vec{r}-\vec{R}|)Y_{L'}^*(\vec{r}-\vec{R})
\biggr]
\\
&\eqrel{eq:helper1forprestructureconstants}{=}&
4\pi\sum_{L'}\biggl(i^{-\ell}
\mathcal{Y}_{\ell,m}(\vec{\nabla}_R) 
h^{(1)}_{\ell'}(\kappa|\vec{R}|)Y_{L'}(-\vec{R})\biggr)
j_{\ell'}(\kappa|\vec{r}-\vec{R}|)Y_{L'}^*(\vec{r}-\vec{R})
\end{eqnarray*}
Here we used that 
\begin{eqnarray}
\vec{\nabla}_r \bigl[f(\vec{R})g(\vec{r}-\vec{R})\bigr]
&=&f(\vec{R})\vec{\nabla}_rg(\vec{r}-\vec{R})
=-f(\vec{R})\vec{\nabla}_Rg(\vec{r}-\vec{R})
\\
&=&-\underbrace{\vec{\nabla}_R\overbrace{\bigl[f(\vec{R})g(\vec{r}-\vec{R})\bigr]
}^{h(\vec{r})}}_{=0}
+\Bigl[\vec{\nabla}_Rf(\vec{R})\Bigr]g(\vec{r}-\vec{R})\bigr]
\label{eq:helper1forprestructureconstants}
\end{eqnarray}

We summarize the final result
\begin{myshadowminipage}{Condition for structure constants (positive energies)}
\begin{eqnarray}
h_\ell^{(1)}(\kappa |\vec{r}|)Y_{\ell,m}(\vec{r})
&=&
4\pi\sum_{L'}\biggl(i^{-\ell}
\mathcal{Y}_{\ell,m}(\vec{\nabla}_R) 
h^{(1)}_{\ell'}(\kappa|\vec{R}|)Y_{L'}(-\vec{R})\biggr)
j_{\ell'}(\kappa|\vec{r}-\vec{R}|)Y_{L'}^*(\vec{r}-\vec{R})
\nonumber\\
\label{eq:prestructureconstants}
\end{eqnarray}
where $h^{(1)}(x)$ is the spherical Hankel function of the first kind
defined in Abramowitz and Stegun (AS)\cite{abramowitz}
\begin{eqnarray}
h_\ell^{(1)}(x)\stackrel{AS 10.1.1}{=}j_\ell(x)+iy_\ell(x)
\stackrel{AS10.1.26}{=}
x^\ell\biggl(-\frac{1}{x}\partial_x\biggr)^\ell\frac{\sin(x)-i\cos(x)}{x}
\label{eq:defsphericalhankelfirstkind}
\end{eqnarray}
\end{myshadowminipage}



%
%====================================================================
\subsubsection{Expression for the structure constants}
%====================================================================
By comparing our notation to that of Daniel Grieger and using his
expression for the Structure constants, we arrive at the following
expression for the structure constants in our notation.

There was a misunderstanding with the sign of the structure
constants. Here I follow the signconvention $K=-\sum JS$, which is
opposite to the one I and Daniel had earlier.

\begin{eqnarray}
S_{R',L',R,L}
&=&
-4\pi \sum_{L''} H^B_{L''}(\vec{R'}-\vec{R}) C_{L,L'',L} 
\left\lbrace
\begin{array}{c}
(-1)^{\ell'}
(-ik)^{\ell+\ell'-\ell''}
\\
(-1)^{\ell'}\delta_{\ell+\ell',\ell''}
\\
(-1)^{\ell'}\kappa^{\ell+\ell'-\ell''}
\end{array}\right\rbrace 
\label{eq:expressionstructureconstants}
\end{eqnarray}




%=======================================================================
\section{Consistency checks}
%=======================================================================
We consider the case with $\kappa=0$, for which the solid Bessel and Hankel functions are
\begin{eqnarray}
K_{\vec{0},L}^\infty(\vec{r})&=&(2\ell-1)!! \frac{1}{|\vec{r}|^{\ell+1}}Y_L(\vec{r})
\\
J_{\vec{0},L}(\vec{r})&=&\frac{1}{(2\ell+1)!!}|\vec{r}|^\ell Y_L(\vec{r})
\end{eqnarray}

The explicit form of the first few is
\begin{eqnarray}
K_{\vec{0},s}^\infty(\vec{r})&=&\frac{1}{\sqrt{4\pi}}\frac{1}{|\vec{r}|}
\\
K_{\vec{0},p_x}^\infty(\vec{r})&=&\sqrt{\frac{3}{4\pi}}\frac{x}{|\vec{r}|^3}
\\
J_{\vec{0},s}(\vec{r})&=&\frac{1}{\sqrt{4\pi}}
\\
J_{\vec{0},p_x}(\vec{r})&=&\frac{1}{3}\sqrt{\frac{3}{4\pi}} x
\end{eqnarray}

Now we extract the structure constants from the off-site expansion
\begin{eqnarray}
K_{\vec{0},s}^\infty(\vec{r})&=&
-S_{\vec{0},s;\vec{R},s} J_s(\vec{r}-\vec{R}) 
\nonumber\\
&&-S_{\vec{0},s;\vec{R},p_x} J_{p_x}(\vec{r}-\vec{R}) 
-S_{\vec{0},s;\vec{R},p_y} J_{p_y}(\vec{r}-\vec{R}) 
-S_{\vec{0},s;\vec{R},p_z} J_{p_z}(\vec{r}-\vec{R}) 
\end{eqnarray}
which allows us to evaluate the structure constants directly
calculating value and derivatives at the second center and by
exploiting selection rules\footnote{Only an s-function has a finite
  value at the origin, only a p-function has a finite first derivative
  at the center, etc.}
\begin{eqnarray}
K_{\vec{0},s}^\infty(\vec{R})&=&\frac{1}{\sqrt{4\pi}}\frac{1}{|\vec{R}|}
=-
\underbrace{\Bigl(-\frac{1}{|\vec{R}|}\Bigr)}_{S_{\vec{0},s,\vec{R},s}}
\underbrace{\frac{1}{\sqrt{4\pi}}}_{J_{\vec{R},s}(\vec{R})} 
\\
\left.\partial_x\right|_{\vec{R}}K_{\vec{0},s}^\infty&=&
-\frac{1}{\sqrt{4\pi}}\frac{X}{|\vec{R}|^3}
=-
\underbrace{\sqrt{3}\frac{X}{|\vec{R}|^3}}_{S_{\vec{0},s;\vec{R},p_x}}
\underbrace{\frac{1}{3}\sqrt{\frac{3}{4\pi}}}_{\partial_x J_{\vec{R},p_x}(\vec{R})}
\\
K_{\vec{0},p_x}(\vec{R})&=&\sqrt{\frac{3}{4\pi}}\frac{X}{|\vec{R}|^3}
=-
\underbrace{\Bigl(-\sqrt{3}\frac{X}{|\vec{R}|^3}\Bigr)
}_{S_{\vec{0},p_x;\vec{R},s}}
\underbrace{\frac{1}{\sqrt{4\pi}}}_{J_{\vec{R},s}(\vec{R})}
\nonumber\\
\left.\partial_x\right|_{\vec{R}}K_{\vec{0},p_x}^\infty&=&
\sqrt{\frac{3}{4\pi}}
\left(\frac{1}{|\vec{R}|^3}-3\frac{X^2}{|\vec{R}|^5}\right)
=
-\underbrace{3\frac{3X^2-\vec{R}^2}{|\vec{R}|^2}}_{S_{\vec{0},p_x,\vec{R},p_x}}
\underbrace{\frac{1}{3}\sqrt{\frac{3}{4\pi}}|\vec{R}|^{-3}
}_{\left.\partial_x\right|_{\vec{R}} J_{\vec{R},p_x}}
\end{eqnarray}

Thus, the matrix of structure constants in the (s,p$_x$) subspace is
\begin{eqnarray}
\mat{S}_{\vec{0},\vec{R}}=\left(\begin{array}{cc}
-|\vec{R}|^{-1} & \sqrt{3}X/|\vec{R}|^3\\
-\sqrt{3}X/|\vec{R}|^3 & 
3[3X^2/R^2-1]\\
\end{array}\right)
\end{eqnarray}

We compare this result now for the one obtained from the direct
formula for $\kappa=0$. These structure constants have the form
\begin{eqnarray}
S_{RL,R'L'}=(-1)^{\ell'+1} 4\pi \sum_{L''} C_{L,L',L''} 
H_{L''}(\vec{R}'-\vec{R})
\delta^{\ell+\ell'-\ell''}
\end{eqnarray}
The structure constants obtained from this equation are
\begin{eqnarray}
S_{\vec{0},s;\vec{R},s}&=&(-1) 4\pi \underbrace{\frac{1}{\sqrt{4\pi}}}_{C_{sss}}
\cdot\underbrace{\frac{1}{\sqrt{4\pi}}\frac{1}{|\vec{R}|}}_{H_s(\vec{R})}
=-\frac{1}{|\vec{R}|}
\nonumber\\
S_{\vec{0},s;\vec{R},p_x}&=& 4\pi
\underbrace{\frac{1}{\sqrt{4\pi}}}_{C_{p_x,s,p_x}}
\underbrace{
\sqrt{\frac{3}{4\pi}}\frac{X}{|\vec{R}|^3}}
_{H_{p_x}(\vec{R})}
=\sqrt{3}\frac{X}{|\vec{R}|^3}
\nonumber\\
S_{\vec{0},p_x;\vec{R},s}&=& (-1)4\pi
\underbrace{\frac{1}{\sqrt{4\pi}}}_{C_{p_x,s,s}}\sqrt{\frac{3}{4\pi}}
\frac{X}{|\vec{R}|^3}=-\sqrt{3}\frac{X}{|\vec{R}|^3}
\nonumber\\
S_{\vec{0},p_x;\vec{R},p_x}&=& 4\pi
\underbrace{\frac{1}{\sqrt{5\pi}}}_{C_{p_x,p_x,d_{3x^2-r^2}}}
\underbrace{
\overbrace{\sqrt{\frac{5}{16\pi}} \frac{3X^2-R^2}{|\vec{R}|^2}}^{Y_{3x^2-r^2}}
\frac{3}{|\vec{R}|^{3}}
}_{H_{3x^2-r^2}(\vec{R})}
=3\frac{3X^2-R^2}{|R^5|}
\end{eqnarray}
For Gaunt coefficients see footnote.\footnote{
\begin{eqnarray}
Y_{p_x}Y_{p_x}=\frac{3}{4\pi}\frac{x^2}{r^2}
=\frac{1}{4\pi}\frac{x^2}{r^2} +\frac{1}{4\pi}\frac{3x^2-r^2}{r^2}
=\frac{1}{\sqrt{4\pi}}Y_s +\frac{1}{4\pi}\sqrt{\frac{16\pi}{5}}
Y_{3x^2-r^2}
=\frac{1}{\sqrt{4\pi}}Y_s +\sqrt{\frac{1}{5\pi}}Y_{3x^2-r^2}
\nonumber\\
\Rightarrow 
C_{p_x,p_x,s}=\frac{1}{\sqrt{4\pi}}\qquad\text{and}\qquad
C_{p_x,p_x,d_{3x^2-r^2}}=\frac{1}{\sqrt{5\pi}}
\end{eqnarray}}




%====================================================================
\chapter{Bloch theorem revisited}
%====================================================================
The Bloch states are eigenstates of the discrete lattice translation 
\begin{eqnarray}
\hat{S}(\vec{t})=\int d^3r\;|\vec{r}+\vec{t}\rangle\langle\vec{r}|
\end{eqnarray}
for the discrete lattice vectors $\vec{t}$. The eigenvalue equation has the form
\begin{eqnarray}
\hat{S}(\vec{t})|\psi_{\vec{k}}\rangle=|\psi_{\vec{k}}\rangle
\e{i\vec{k}\vec{r}}
\end{eqnarray}
This eigenvalue equation can be recast into the form
\begin{eqnarray}
\langle\vec{r}-\vec{t}|\psi_{\vec{k}}\rangle=\langle\vec{r}|\psi_{\vec{k}}\rangle
\e{i\vec{k}\vec{t}}
\end{eqnarray}
This implies that the states can be written  as product of a periodic function and a phase factor
\begin{eqnarray}
\langle\vec{r}|\psi_{\vec{k}}\rangle=u_{\vec{k}}(\vec{r})\e{i\vec{k}\vec{r}}
\end{eqnarray}
with
\begin{eqnarray}
u_{\vec{k}}(\vec{r})=u_{\vec{k}}(\vec{r}+\vec{t})
\end{eqnarray}

%====================================================================
\subsubsection{Bloch theorem in a local orbital basis}
%====================================================================
With $q_\alpha\defas \langle\pi_\alpha|\psi\rangle$, we obtain
\begin{eqnarray}
\hat{S}(\vec{t})\sum_\alpha|\chi_\alpha\rangle q_{\alpha,n}
&=&
\sum_\alpha|\chi_\alpha\rangle q_{\alpha,n}\e{i\vec{k}_n\vec{t}}
\nonumber\\
\int d^3r\;|\vec{r}+\vec{t}\rangle\langle\vec{r}|
\sum_\alpha|\chi_\alpha\rangle q_{\alpha,n}
&=&
\int d^3r\;|\vec{r}\rangle\langle\vec{r}|
\sum_\alpha|\chi_\alpha\rangle q_{\alpha,n}\e{i\vec{k}_n\vec{t}}
\nonumber\\
\sum_\alpha\langle\vec{r}-\vec{t}|\chi_\alpha\rangle q_{\alpha,n}
&=&
\sum_\alpha
\langle\vec{r}|\chi_\alpha\rangle q_{\alpha,n}\e{i\vec{k}_n\vec{t}}
\nonumber\\
\sum_\alpha\langle\vec{r}|\chi_{\alpha+\vec{t}}\rangle q_{\alpha,n}
&=&
\sum_\alpha
\langle\vec{r}|\chi_\alpha\rangle q_{\alpha,n}\e{i\vec{k}_n\vec{t}}
\nonumber\\
\sum_{\alpha'}\langle\vec{r}|\chi_{\alpha'}\rangle q_{\alpha'-\vec{t},n}
&=&
\sum_\alpha
\langle\vec{r}|\chi_\alpha\rangle q_{\alpha,n}\e{i\vec{k}_n\vec{t}}
\nonumber\\
q_{\alpha+\vec{t},n}&=&q_{\alpha,n}\e{-i\vec{k}_n\vec{t}}
\end{eqnarray}

%====================================================================
\subsubsection{Density matrix}
%====================================================================
\begin{eqnarray}
\rho_{\alpha,\beta+\vec{t}}
&=&
\sum_n \langle\pi_\alpha|\psi_n\rangle f_n
\langle\psi_n|\pi_\beta\rangle
\e{+i\vec{k}_n\vec{t}}
\end{eqnarray}

%=======================================================================
\chapter{Offsite matrix elements using Gaussian integrals}
%=======================================================================
In \verb|LMTO_INITIALIZE| there are non-functional calls for doing the
integrations in a representation of Gauss orbitals. The routines are
no more present, except for 
\begin{itemize}
\item \verb|LMTO_TAILEDPRODUCTS|. They have been
removed in 7501a0g from Feb.2, 2013 (svn revision 1116 from Nov. 20,
2011). \verb|LMTO_TAILEDPRODUCTS| has been removed after 9803b02 on from
 Mar. 1, 2014.
\item \verb|LMTO_EXPANDPRODS|
\end{itemize}

The routines related to Gaussians have been moved into the file
\verb|paw_lmto_stuffwithgaussians.f90|

\begin{verbatim}
!!$      CALL LMTO_TAILEDPRODUCTS()
              gaussian_fitgauss
!!$      CALL LMTO_GAUSSFITKPRIME()
!!$      CALL LMTO_GAUSSFITKAUGMENT()
!!$      CALL LMTO_GAUSSFITKJTAILS()
!!$      CALL LMTO_ONSITEOVERLAP()
\end{verbatim}

%====================================================================
\chapter{Double counting}
\label{app:dc}
%====================================================================
%====================================================================
\section{Other double-counting schemes}
%====================================================================
An excellent review of double-counting terms has been discussed by
Nekrasov et al.\cite{nekrasov12_arxiv1208_4732}.  


%=====================================================================
\subsubsection{Around-mean-field (AMF) and fully-localized limit (FLL)}
%=====================================================================
In LDA+U, two double-counting schemes are in common use: one is called
around mean-field (AMF)\cite{anisimov91_prb44_943} and the other is
called fully localized limit (FLL)\cite{czyzyk94_prb49_14211}.

In the Hubbard model, the two approximations have the
form\cite{nekrasov12_arxiv1208_4732}. 
\begin{eqnarray}
\hat{H}^{dc}_{AMF}&=&\frac{1}{2}U\sum_\sigma\hat{n}_{d,\sigma}(\hat{n}_d-n_\sigma^0)
-\frac{1}{2}J\sum_\sigma\hat{n}_{d,\sigma}(\hat{n}_{d,\sigma}-n_\sigma^0)
\nonumber\\
\hat{H}^{dc}_{FLL}&=&\frac{1}{2}Un_d(n_d-1)
-\frac{1}{2}J\sum_\sigma\hat{n}_{d,\sigma}(\hat{n}_{d,\sigma}-1)
\end{eqnarray}
with
\begin{eqnarray}
n_{d,\sigma}&=&\sum_m \langle \hat{n}_{m,\sigma}\rangle
\nonumber\\
n_{d}&=&\sum_\sigma n_{d,\sigma}
\nonumber\\
n^0_\sigma&=&\frac{1}{2(2\ell+1)}\sum_{m,\sigma} n_{m,\sigma}
\nonumber\\
n^0&=&\sum_\sigma n^0_\sigma
\end{eqnarray}
\texttt{It is still unclear what are operators and what are numbers,
  but I believe all are numbers.}

%=====================================================================
\subsubsection{Nekrasov, Pavlov Sadovskii scheme}
%=====================================================================
Nekrasov et al.\cite{nekrasov12_arxiv1208_4732} proposes a new scheme
which amounts to removing the density of the correlated orbitals from
the integral for the exchange-correlation term.  The corresponding
double-counting energy would be
\begin{eqnarray}
E^{DC}=E_{xc}[n_t-n_{corr}]-E_{xc}[n_t]
\end{eqnarray}
where $n_t$ is the density of all orbitals, while $n_{corr}$ is the
density of the correlated orbitals only.


%=====================================================================
\subsubsection{Bl\"ochl-Walther-Pruschke scheme}
%=====================================================================
Our scheme\cite{bloechl11_prb84_205101} differs in that it divides the
DFT exchange correlation energy based on a partitioning of the
two-particle density, where $n(\vec{r})=\sum_R n_R(\vec{r})$
\begin{eqnarray}
E_{xc}
&=&\int d^3r\;n(\vec{r})
\frac{1}{2}\int d^3r'\frac{e^2 \int_0^1d\lambda\;h_\lambda (\vec{r},\vec{r'})}
{4\pi\epsilon|\vec{r}-\vec{r'}|}
\nonumber\\
&=&\frac{1}{2}\sum_{R,R'}\int d^3r\int d^3r'
\left(\frac{n_R(\vec{r})}{n(\vec{r})}\right)
\biggl(
n(\vec{r})
\frac{e^2 \int_0^1d\lambda\;h_\lambda (\vec{r},\vec{r'})}
{4\pi\epsilon|\vec{r}-\vec{r'}|}\biggr)
\left(\frac{n_{R'}(\vec{r'})}{n(\vec{r'})}\right)
\end{eqnarray}
This formulation takes into account that the electrons are
indistinguishable and that each electron contributes equally to the
exchange correlation energy. Furthermore, the partitions of the
expression add up to the total exchange-correlation energy (when also
the inter-site terms are included).

The model rests on a well defined expression for the two-particle
density, namely
\begin{eqnarray}
n^{(2)}(\vec{r},\vec{r'})=n^{(1)}(\vec{r})n^{(1)}(\vec{r'})
+
\frac{1}{2}\sum_{R,R'}
\left(\frac{n_R(\vec{r})}{n(\vec{r})}\right)
\biggl(
n(\vec{r})h_{\lambda=1} (\vec{r},\vec{r'})
\biggr)
\left(\frac{n_{R'}(\vec{r'})}{n(\vec{r'})}\right)
\end{eqnarray}
and an expression for the kinetic-energy correction 
\begin{eqnarray}
T-T_s
&=&\frac{1}{2}\sum_{R,R'}\int d^3r\int d^3r'
\left(\frac{n_R(\vec{r})}{n(\vec{r})}\right)
\biggl(
n(\vec{r})
\frac{e^2 \int_0^1d\lambda\;\Bigl(h_\lambda (\vec{r},\vec{r'})
-h_{\lambda=1} (\vec{r},\vec{r'})\Bigr)}
{4\pi\epsilon|\vec{r}-\vec{r'}|}\biggr)
\left(\frac{n_{R'}(\vec{r'})}{n(\vec{r'})}\right)
\nonumber\\
\end{eqnarray}
Note, however, that only the total exchange-correlation energy
including Coulomb and kinetic energy contributions is a density
functional, whereas both contributions can be expressed individually
as density-matrix functional.



Thus, a double-counting correction for a set of correlated orbitals,
which contribute $n_R$ to the total density would be
\begin{eqnarray}
E_{dc}
&=&-\frac{1}{2}\sum_{R}\int d^3r\int d^3r'
\left(\frac{n_R(\vec{r})}{n(\vec{r})}\right)
\biggl(
n(\vec{r})
\frac{e^2 \int_0^1d\lambda\;h_\lambda (\vec{r},\vec{r'})}
{4\pi\epsilon|\vec{r}-\vec{r'}|}\biggr)
\left(\frac{n_{R}(\vec{r'})}{n(\vec{r'})}\right)
\end{eqnarray}



This integral can be approximated further as
\begin{eqnarray}
E_{xc}
&=&-\frac{1}{2}\sum_{R}\int d^3r\; n(\vec{r})\epsilon[n(\vec{r})]
\left(\frac{n_R(\vec{r})}{n(\vec{r})}\right)^2
\end{eqnarray}


%====================================================================
\subsection{Around mean field (AMF)}
%====================================================================
The original expression, AMF, for the double-counting term is derived
from the mean-field approximation of the Hubbard model.

\textbf{This section is unfinished. It is an attempt to derive the AMF
  limit consistent with our U-tensor}
We start out from the interaction
\begin{eqnarray}
\hat{W}&=&
\frac{1}{2}\sum_{i,j,k,l}W_{i,j,l,k} 
\hat{c}^\dagger_i\hat{c}^\dagger_j\hat{c}_k\hat{c}_l
\end{eqnarray}
with 
\begin{eqnarray}
W_{i,j,k,l}&=&
\int d^4x\int d^4x'\; \frac{e^2
\chi_i^*(\vec{x})\chi_j^*(\vec{x'})\chi_k(\vec{x})\chi_l(\vec{x})}
{4\pi\epsilon_0|\vec{r}-\vec{r'}|}
\nonumber\\
&=&
\delta_{\sigma_i,\sigma_k}
\delta_{\sigma_j,\sigma_l}
\int d^3r\int d^3r'\; \frac{e^2
\phi_i^*(\vec{r})\phi_j^*(\vec{r'})\phi_k(\vec{r})\phi_l(\vec{r})}
{4\pi\epsilon_0|\vec{r}-\vec{r'}|}
=\delta_{\sigma_i,\sigma_k}
\delta_{\sigma_j,\sigma_l}
U_{i,j,k,l}
\end{eqnarray}
where the spin-orbitals
$\chi_i(\vec{r},\sigma)=\phi_i(\vec{r})\delta_{\sigma,\sigma_i}$ are
decomposed into a outer product of a spatial orbital $\phi_i(\vec{r})$
and a spinor. The interaction has the internal symmetry
\begin{eqnarray}
W_{i,j,k,l}=W_{j,i,l,k}=W^*_{k,l,i,j}=W^*_{l,k,j,i}
\end{eqnarray}

In order to arrive at the mean-field Hamiltonian, we express the wave
function in terms of wave function of a non-interacting system,
i.e. by Slater determinants. This non-interacting system is then
optimized to minimize the expectation value of the Hamiltonian
containing the interaction. (See Bogoljubov inequality) The result is
the Hartree-Fock approximation.

The ground state is
\begin{eqnarray}
|\Phi_0\rangle=
\prod_n \left(\hat{a}_n^\dagger\right)^{\sigma_n}|\mathcal{O}\rangle
\end{eqnarray}
where $\vec{\sigma}\in\{0,1\}^\infty$ is the occupation number
representation of the Slater determinant.  The orbitals making up the
Slater determinant are $|\phi_n\rangle$
\begin{eqnarray}
\hat{a}_n^\dagger=\sum_j\hat{c}^\dagger_j\langle\pi_j|\psi_n\rangle
\end{eqnarray}
with a complete set of one-particle orbitals and the bi-orthogonality
condition this relationship can be inverted
\begin{eqnarray}
\sum_n\hat{a}_n^\dagger\langle\psi_n|\chi_i\rangle
=\sum_j\hat{c}^\dagger_j
\underbrace{\langle\pi_j|
\underbrace{\sum_n|\psi_n\rangle\langle\psi_n|}_{=1}|\chi_i\rangle}_{\delta_{i,j}}
=\hat{c}^\dagger_i
\end{eqnarray}

\begin{eqnarray}
\langle \Phi|\hat{a}^\dagger_m\hat{a}^\dagger_n\hat{a}_o\hat{a}_p|\Phi\rangle
&=&\delta_{m,o}\delta_{n,p}
\langle \Phi|\hat{a}^\dagger_m
\underbrace{\hat{a}^\dagger_n\hat{a}_m}_{\delta_{m,n}-\hat{a}_m\hat{a}^\dagger_n}
\hat{a}_n
|\Phi\rangle
+\delta_{m,p}\delta_{n,o}
\langle \Phi|\hat{a}^\dagger_m\underbrace{\hat{a}^\dagger_n\hat{a}_n\hat{a}_m
}_{-(\delta_{m,n}-\hat{a}_m\hat{a}^\dagger_n)\hat{a}_n}
|\Phi\rangle
\nonumber\\
&&-\delta_{m,p}\delta_{n,o}\delta_{m,n}
\langle \Phi|\hat{a}^\dagger_m\hat{a}^\dagger_m\underbrace{\hat{a}_m\hat{a}_m}
_{=0}|\Phi\rangle
\nonumber\\
&=& \delta_{m,o}\delta_{n,p}
\Bigl[
\delta_{m,n}\langle \Phi|\hat{a}^\dagger_m
\hat{a}_n|\Phi\rangle
-
\langle \Phi|\hat{a}^\dagger_m
\hat{a}_m|\Phi\rangle\langle \Phi|\hat{a}^\dagger_n
\hat{a}_n|\Phi\rangle\Bigr]
\nonumber\\
&-& \delta_{m,p}\delta_{n,o}
\Bigl[
\delta_{m,n}\langle \Phi|\hat{a}^\dagger_m
\hat{a}_n|\Phi\rangle
-
\langle \Phi|\hat{a}^\dagger_m
\hat{a}_m|\Phi\rangle\langle \Phi|\hat{a}^\dagger_n
\hat{a}_n|\Phi\rangle\Bigr]
\nonumber\\
&=& -\delta_{m,o}\delta_{n,p}
\langle \Phi|\hat{a}^\dagger_m
\hat{a}_m|\Phi\rangle\langle \Phi|\hat{a}^\dagger_n
\hat{a}_n|\Phi\rangle
+ \delta_{m,p}\delta_{n,o}
\langle \Phi|\hat{a}^\dagger_m
\hat{a}_m|\Phi\rangle\langle \Phi|\hat{a}^\dagger_n
\hat{a}_n|\Phi\rangle
\nonumber\\
&=&f_mf_n\Bigl[\delta_{m,o}\delta_{n,p}-\delta_{m,p}\delta_{n,o}\Bigr]
\end{eqnarray}
Here the occupations are those of the reverence Slater determinant,
the mean field.

Now we can evaluate the matrix elements for the interaction as
\begin{eqnarray}
\langle\Phi_0|\hat{c}^\dagger_i\hat{c}^\dagger_j\hat{c}_k\hat{c}_l|\Phi_0\rangle
&=&\sum_{m,n,o,p}\langle\psi_m|\chi_i\rangle\langle\psi_n|\chi_j\rangle
\langle\Phi_0|\hat{a}^\dagger_m\hat{a}^\dagger_n\hat{a}_o\hat{a}_p|\Phi_0\rangle
\langle\chi_k|\psi_o\rangle\langle\chi_l|\psi_p\rangle
\nonumber\\
&=&\sum_{m,n,o,p}\langle\psi_m|\chi_i\rangle\langle\psi_n|\chi_j\rangle
\Bigl[
f_mf_n\Bigl[\delta_{m,o}\delta_{n,p}-\delta_{m,p}\delta_{n,o}\Bigr]
\Bigr]
\langle\chi_k|\psi_o\rangle\langle\chi_l|\psi_p\rangle
\nonumber\\
&=&
\sum_{m,n}
f_mf_n
\langle\psi_m|\chi_i\rangle\langle\psi_n|\chi_j\rangle
\langle\chi_k|\psi_m\rangle\langle\chi_l|\psi_n\rangle
-
\sum_{m,n}
f_mf_n
\langle\psi_m|\chi_i\rangle\langle\psi_n|\chi_j\rangle
\langle\chi_k|\psi_n\rangle\langle\chi_l|\psi_m\rangle
\nonumber\\
&=&
\underbrace{\sum_{m}
\langle\chi_k|\psi_m\rangle f_m
\langle\psi_m|\chi_i\rangle}_{\rho_{k,i}}
\underbrace{\sum_{m}
\langle\chi_l|\psi_n\rangle f_n\langle\psi_n|\chi_j\rangle 
}_{\rho_{l,j}}
-
\underbrace{\sum_{m}
\langle\chi_l|\psi_m\rangle f_m\langle\psi_m|\chi_i\rangle}_{\rho_{l,i}}
\underbrace{\sum_{n}
\langle\chi_k|\psi_n\rangle f_n\langle\psi_n|\chi_j\rangle}_{\rho_{k,j}}
\nonumber\\
&=&\rho_{k,i}\rho_{l,j}-\rho_{l,i}\rho_{k,j}
\nonumber\\
&=&
\langle\Phi_0|\hat{c}^\dagger_i\hat{c}_k|\Phi_0\rangle
\langle\Phi_0|\hat{c}^\dagger_j\hat{c}_l|\Phi_0\rangle
-\langle\Phi_0|\hat{c}^\dagger_i\hat{c}_l|\Phi_0\rangle
\langle\Phi_0|\hat{c}^\dagger_j\hat{c}_k|\Phi_0\rangle
\end{eqnarray}
This is the expectation value of a Slater determinant which is the
zero'th order term for an expansion in the deviation from these mean
values.


The density matrix has the form
\begin{eqnarray}
\rho_{i,j}&=&\langle\Phi_0|\hat{c}^\dagger_j\hat{c}_{i}|\Phi_0\rangle
=\sum_{m,n}\langle\psi_m|\chi_j\rangle\langle\chi_i|\psi_n\rangle
\underbrace{\langle\Phi_0|\hat{a}^\dagger_m\hat{c}_{n}|\Phi_0\rangle}_{\delta_{m,n}f_n}
=\sum_{n}\chi_i|\psi_n\rangle f_n\langle\psi_n|\chi_j\rangle\langle
\end{eqnarray}



\begin{eqnarray}
\hat{c}^\dagger_i\hat{c}^\dagger_j\hat{c}_k\hat{c}_l
&=&\hat{c}^\dagger_i\hat{c}^\dagger_j\hat{c}_k\hat{c}_l
\nonumber\\
&=&
\langle\Phi_0|\hat{c}^\dagger_i\hat{c}_k|\Phi_0\rangle
\langle\Phi_0|\hat{c}^\dagger_j\hat{c}_l|\Phi_0\rangle
-\langle\Phi_0|\hat{c}^\dagger_i\hat{c}_l|\Phi_0\rangle
\langle\Phi_0|\hat{c}^\dagger_j\hat{c}_k|\Phi_0\rangle
\end{eqnarray}




interaction in terms of products of one-particle operators
$\hat{c}^\dagger_i\hat{c}_j$. There are two equivalent forms for this, namely
\begin{eqnarray}
\hat{c}^\dagger_i\hat{c}^\dagger_j\hat{c}_k\hat{c}_l
&=&\delta_{j,k}\hat{c}^\dagger_i\hat{c}_l
-\hat{c}^\dagger_i\hat{c}_k\hat{c}^\dagger_j\hat{c}_l 
\nonumber\\
&=&\delta_{j,l}\hat{c}^\dagger_i\hat{c}_k
-\hat{c}^\dagger_i\hat{c}_l\hat{c}^\dagger_j\hat{c}_k 
\end{eqnarray}





Similarly the products of the field operators can be rewritten in a
number of different ways
\begin{eqnarray}
\hat{c}^\dagger_i\hat{c}^\dagger_j\hat{c}_k\hat{c}_l
&=&-\hat{c}^\dagger_i\hat{c}^\dagger_j\hat{c}_l\hat{c}_k
=-\hat{c}^\dagger_j\hat{c}^\dagger_i\hat{c}_k\hat{c}_l
=\hat{c}^\dagger_j\hat{c}^\dagger_i\hat{c}_l\hat{c}_k
\nonumber\\
=\delta_{j,k}\hat{c}^\dagger_i\hat{c}_l
-\hat{c}^\dagger_i\hat{c}_k\hat{c}^\dagger_j\hat{c}_l &=&
\end{eqnarray}



\begin{eqnarray}
\nonumber\\
&=&
\frac{1}{2}\sum_{i,j,k,l}W_{i,j,l,k} 
\frac{1}{2}\biggl[-\biggl(
\hat{c}^\dagger_i\hat{c}_k\hat{c}^\dagger_j\hat{c}_l
-\delta_{k,j}\hat{c}^\dagger_i\hat{c}_l\biggr)
+\biggl(
\hat{c}^\dagger_i\hat{c}_l\hat{c}^\dagger_j\hat{c}_k
-\delta_{j,l}\hat{c}^\dagger_i\hat{c}_k\biggr)\biggr]
\nonumber\\
&\stackrel{i\leftrightarrow l}{=}&
\frac{1}{2}\sum_{i,j,k,l}\frac{W_{i,j,k,l}-W_{i,j,l,k}}{2} 
\biggl(\hat{n}_{i,k}\hat{n}_{j,l}-\delta_{k,j}\hat{n}_{i,l}\biggr)
\end{eqnarray}

In the mean-field approximation, terms that are quadratic in
$\hat{n}-\langle\hat{n}\rangle$ are ignored.
\begin{eqnarray}
\hat{W}_{MF}&=&
\frac{1}{2}\sum_{i,j,k,l}\frac{W_{i,j,k,l}-W_{i,j,l,k}}{2} 
\biggl(\langle\hat{n}_{i,k}\rangle\langle\hat{n}_{j,l}\rangle-\delta_{k,j}
\langle\hat{n}_{i,l}\rangle\biggr)
\nonumber\\
&+&
\frac{1}{2}\sum_{i,j,k,l}\frac{W_{i,j,k,l}-W_{i,j,l,k}}{2} 
\biggl([\hat{n}_{i,k}-\langle\hat{n}_{i,k}\rangle]\langle\hat{n}_{j,l}\rangle
+\langle\hat{n}_{i,k}\rangle[\hat{n}_{j,l}-\langle\hat{n}_{j,l}\rangle]
-\delta_{k,j}[\hat{n}_{i,l}-\langle\hat{n}_{i,l}\rangle]\biggr)
\nonumber\\
&=&
\frac{1}{2}\sum_{i,j,k,l}\frac{W_{i,j,k,l}-W_{i,j,l,k}}{2} 
\biggl(\hat{n}_{i,k}\langle\hat{n}_{j,l}\rangle
+\langle\hat{n}_{i,k}\rangle\hat{n}_{j,l}
-\delta_{k,j}\hat{n}_{i,l}
-\langle\hat{n}_{i,k}\rangle\langle\hat{n}_{j,l}\rangle\biggr)
\nonumber\\
\end{eqnarray}
The resulting Hamiltonian is a one-particle hamiltonian, which depends
parametrically on the occupations., i.e the density matrix.

This Hamiltonian can be derived consistently\footnote{Both, the the
  total energy and the derived one-particle hamiltonian is consistent
  with the mean-field form of the interaction.} from the
density-matrix functional with the density matrix
$\rho_{i,j}=\langle\hat{c}^\dagger_i\hat{c}_j\rangle$
\begin{eqnarray}
E^{DC}_{AMF}
&=&
\frac{1}{2}\sum_{i,j,k,l}\frac{W_{i,j,k,l}-W_{i,j,l,k}}{2} 
\biggl(\rho_{i,k}\rho_{j,l}-\delta_{k,j}\rho_{i,l}\biggr)
\end{eqnarray}

In order to arrive at the common expression, we consider the
\textbf{density-density approximation}\index{density-density
  approximation} , that is we consider only density matrices that are
diagonal in the orbital and spin coordinates
\begin{eqnarray}
E^{DC}_{AMF}
&\approx&
\frac{1}{2}\sum_{\sigma,\sigma'}\sum_{i,j}\frac{W_{i,j,i,j}-W_{i,j,j,i}}{2} 
\rho_{i,i}\rho_{j,j}
-
\frac{1}{2}\sum_{i,j}\frac{W_{i,j,i,j}-W_{i,j,j,i}}{2} 
\rho_{i,i}
\nonumber\\
&=&
\frac{1}{2}\sum_{i\neq j}\frac{W_{i,j,j,i}-W_{i,j,i,j}}{2} 
\rho_{i,i}(\rho_{j,j}+1)
\end{eqnarray}
The terms with $i=j$ cancel.
\begin{eqnarray}
E^{DC}_{AMF}
&\approx&
\frac{1}{2}\sum_{m,m'}\frac{U_{m,m',m,m'}-U_{m,m',m',m}\delta_{\sigma,\sigma'}}{2} 
n_{m,\sigma}n_{m',\sigma'}
\nonumber\\
&-&
\frac{1}{2}\sum_{i}
\biggl(\sum_{m',\sigma'}\frac{U_{m,m',m',m}-U_{m,m',m,m'}
\delta_{\sigma,\sigma'}}{2}\biggr)
n_{m,\sigma}
\nonumber\\
&=&
\frac{1}{2}\sum_\sigma\sum_{m}\frac{U_{m,m,m,m}}{2}n_{m,\sigma}n_{m,\bar{\sigma}}
\nonumber\\
&+&\frac{1}{2}\sum_{m\neq m'}\sum_{\sigma,\sigma'}
\frac{U_{m,m',m,m'}-U_{m,m',m',m}\delta_{\sigma,\sigma'}}{2} 
n_{m,\sigma}n_{m',\sigma'}
\nonumber\\
&-&
\frac{1}{2}\sum_{i}
\biggl(\sum_{m'}
\frac{2U_{m,m',m',m}-U_{m,m',m,m'}}{2}\biggr)n_{m,\sigma}
\end{eqnarray}


%====================================================================
\subsection{Definition of U and J parameters}
%====================================================================
The U- and J-parameters for one angular-momentum shell are defined by
the U-tensor as follows\cite{shick99_prb60_10763} (see Eq. 5 of Shick
et al.)
\begin{eqnarray}
U&:=&\frac{1}{(2\ell+1)^2}\sum_{m_1,m_2} U_{m_1,m_2,m_1,m_2}
\label{eq:ldaplusudefu}
\\
J&:=&U-\frac{1}{2\ell(2\ell+1)}\sum_{m_1,m_2}\left( U_{m_1,m_2,m_1,m_2}
-U_{m_1,m_2,m_2,m_1}\right)
\label{eq:ldaplusudefj}
\end{eqnarray}
where the Coulomb matrix elements are
\begin{eqnarray}
U_{i,j,k,l}&=&\int d^3r\int d^3r'\;
\frac{e^2\chi^*_i(\vec{r})\chi^*_j(\vec{r'})\chi_k(\vec{r})\chi_l(\vec{r'})}
{4\pi\epsilon_0|\vec{r}-\vec{r'}|}
\label{eq:defcoulombmatrixelement}
\end{eqnarray}
The indices used here are spatial-orbital indices.



%=====================================================================
\chapter{Changelog, Bugfixes}
%=====================================================================
\begin{itemize}
%
\item the core-valence exchange contribution differs from the old
  version, because it also includes the projection on the phidot
  functions.
%
\item there has been a bug in \verb|lmto$screen|, which has been fixed
  with version 3. It may be better to rewrite all structure constants
  routines with teh transposed structure constants.
%
\item in \verb|lmto$makestructureconstants|, the structure constants
  have not been calculated because the parallelzation was wrong.  in
  13403f6.
\begin{verbatim}
-       IF(MOD(IAT1-1,NTASKS).NE.THISTASK-1) THEN
+       IF(MOD(IAT1-1,NTASKS).eq.THISTASK-1) THEN
\end{verbatim}
%
\end{itemize}


\clearpage
\bibliographystyle{unsrtnat} \bibliography{../all}
\end{document}  
