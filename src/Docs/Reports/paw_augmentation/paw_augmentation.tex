\documentclass[11pt,a4paper]{report}
%%%%%%%%%%%%%%%%%%%%%%%%%%%%%%%%%%%%%%%%%%%%%%%%%%%%%%%%%%%%%%%%%%%%%
%%                                                                 %%
%%    Header file for the Phi-S-X Series                           %%
%%                                                                 %%
%%    german version header_gm.tex is derived from header.tex      %%
%%    by uncommenting the line ``\setboolean{german}{true}'' below %%
%%                                                                 %%
%%    Never edit the german version! all changes must be done      %%
%%    in the english version header.tex                            %%
%%                                                                 %%
%%%%%%%%%%%%%%%%%%%%%%%%%%%%%%%%%%%%%%%%%%%%%%%%%%%%%%%%%%%%%%%%%%%%%
%%%%%%%%%%%%%%%%%%%%%%%%%%%%%%%%%%%%%%%%%%%%%%%%%%%%%%%%%%%%%%%%%%%%%
%%                                                                 %%
%%    Header file for the Phi-S-X Series                           %%
%%                                                                 %%
%%    german version header_gm.tex is derived from header.tex      %%
%%    by uncommenting the line ``\setboolean{german}{true}'' below %%
%%                                                                 %%
%%    Never edit the german version! all changes must be done      %%
%%    in the english version header.tex                            %%
%%                                                                 %%
%%%%%%%%%%%%%%%%%%%%%%%%%%%%%%%%%%%%%%%%%%%%%%%%%%%%%%%%%%%%%%%%%%%%%
%====================================================================
%-- define flag for language adaptations
\usepackage{ifthen}   % allows to select only certain text
\provideboolean{german}
\setboolean{german}{false}
%\setboolean{german}{true}  % uncomment this line for german editions
%====================================================================
%
% Textschriftart: Computer modern Bright
% body:            CM-Bright 10pt
% section titles:  CM-Bright Bold
% formulas:        CM-Bright Math Oblique
%
\usepackage[standard-baselineskips]{cmbright}
\usepackage{cmbright}
\usepackage[T1]{fontenc}
\def\usedfonts{CM-Bright}
\usepackage{typearea}
%\typearea[current]{calc} % benutzt die aktuelle 
       % bindekorrektur (BCOR angabe als parameter in koma usepackage)
       % und berechnet satzspiegel neu
\typearea[current]{11} %fixed div value

\usepackage{textcomp} % special symbols
\usepackage{amsfonts} % special symols
                      % see ftp://ftp.ams.org/pub/tex/doc/amsfonts/amsfndoc.pdf
\usepackage{amssymb}  % CM-Bright provides the AMS symbols
\usepackage{exscale}  % allows to scale math expressions to big fonts, 
                      % e.g. \Huge
\usepackage{curves}
\usepackage{braket}
\usepackage{miller}     % miller indices
\usepackage{chemmacros} % http://www.mychemistry.eu/mychemistry/
\usepackage[numbers]{natbib}     % bibliography style
\usepackage{url}\urlstyle{tt}
\usepackage{float}
\usepackage{bm}       % provides the command \bm{} that makes bold math symbols
\usepackage{amsmath}
\usepackage{amsbsy}   % allows bold mathematical symbols
\usepackage{amscd}
 \usepackage{a4wide}  % it is better to use the ``geometry'' package
\usepackage{array}    % 
\usepackage{fancyhdr} %  defines pagestyle fancy
\usepackage{epsfig}   % include graphics with epsfig
\usepackage{graphicx} % includegraphics
\usepackage{epstopdf}
\usepackage{wrapfig}
\usepackage{fancybox} % allows shadow-boxes
\usepackage{color}    % allows to use color in the text
%\usepackage{eepic}
\usepackage{flafter}  % places picture next to its reference
\usepackage{makeidx}  % make an index
%\usepackage{MnSymbol}  % 
%\usepackage{marvosym}  % 
\usepackage{textcase}
\usepackage{ulem} % defines strikeout \sout{}; underline \uline{}
                  % double underline \uuline{}; wave underline \uwave{}
                  % cross out \xout{}
%
%==========================================================================
%==  page layout  =========================================================
%==========================================================================
% eqnarray environment: reduce with of space in place of each ``&''
\setlength\arraycolsep{1.4pt}
\pagestyle{fancy}
%\renewcommand{\chaptermark}[1]{\markboth{\thechapter\ #1}{}}
\renewcommand{\chaptermark}[1]{\markboth{\MakeUppercase{\thechapter\ #1}}{}}
\fancyhf{} 
\fancyhead[LE]{\textsc{\thepage}\qquad\textsc{\leftmark}}
\fancyhead[RO]{\textsc{\leftmark}\qquad\textsc{\thepage}}
\renewcommand{\headrulewidth}{0.5pt}
\renewcommand{\footrulewidth}{0pt} 
\addtolength{\headheight}{2.5pt}
\fancypagestyle{plain}{\fancyhead{}
   \renewcommand{\headrulewidth}{0pt}
   \fancyfoot[CO]{\bfseries\thepage}}

% Line spacing -----------------------------------------------------------
\newlength{\defbaselineskip}
\setlength{\defbaselineskip}{\baselineskip}
\newcommand{\setlinespacing}[1]%
           {\setlength{\baselineskip}{#1 \defbaselineskip}}
\newcommand{\doublespacing}{\setlength{\baselineskip}%
                           {2.0 \defbaselineskip}}
\newcommand{\singlespacing}{\setlength{\baselineskip}{\defbaselineskip}}

% Absatz einr\"ucken ------------------------------------------------------
%\setlength{\parindent}{0pt}
\setlength{\parskip}{2pt}
% -------------------------------------------------------------------------
\ifthenelse{\boolean{german}}
  {\def\figurename{Abb.}}
  {\def\figurename{Fig.}}
%--------------------------------------------------------------------------
\renewcommand{\arraystretch}{1.15}  % skaliert den Zeilen abstand in der 
    % tabular und array umgebung
%
%==========================================================================
%==  boxes etc ============================================================
%==========================================================================
%== minipage in a shadowbox ===============================================
\newenvironment{myshadowminipage}[1]%
  {\par\noindent\begin{Sbox}\begin{minipage}{\linewidth}\vspace{0.1cm}\begin{center}\uppercase{#1}\end{center}}%
  {\vspace{0.1cm}\end{minipage}\end{Sbox}\shadowbox{\TheSbox}}
%
%== minipage in a framedbox ===============================================
\newenvironment{myframedminipage}%
  {\par\noindent\begin{Sbox}\begin{minipage}\linewidth\vspace{0.1cm}}%
  {\vspace{0.1cm}\end{minipage}\end{Sbox}\fbox{\TheSbox}}
%
\newcommand{\myshadowbox}[1]{\noindent\shadowbox{\parbox{\linewidth}{\smallskip #1\smallskip}}}
\newcommand{\myfbox}[1]{\noindent\fbox{\parbox{\linewidth}{\smallskip #1\smallskip}}\medskip}
%== minipage in a framedbox ===============================================
\newtheorem{defi}{Definition}[chapter]
\newenvironment{definition}[1]%
  {\par\noindent\begin{Sbox}\begin{minipage}{\linewidth}\vspace{0.1cm}\begin{defi}\uppercase{#1}\\\vspace{0.1cm}}%
  {\vspace{0.1cm}\end{defi}\end{minipage}\end{Sbox}\shadowbox{\TheSbox}}
%
%=========================================================================
% color used to point out information to the teacher
\definecolor{highlight}{rgb}{1.0,0.7,0.}
\newcommand{\Special}[1]{\textbf{\textcolor{highlight}{#1}}}
%=========================================================================
%  switch certain parts on and off. uses ifthen package
\newboolean{teacher}\setboolean{teacher}{false}
% this parameter can be changed in the manuscript again
\setboolean{teacher}{true} %private version if true!
\newcommand{\teacheronly}[1]{\ifthenelse{\boolean{teacher}}{#1\hfill\\ }}
\newcommand{\editor}[1]{\textcolor{blue}{\texttt{Editor: #1}}}
\newcommand{\MARK}[1]{\textcolor{blue}{#1}} 
\newcommand{\RED}[1]{\textcolor{red}{#1}} 
%
%==========================================================================
%==  define new symbols                                                 ===
%==========================================================================
% define \stat (stationary state) as an operator like \min
\DeclareMathOperator*{\stat}{stat}
\let\Vec=\mathbold   % cmbright.sty provides a bold/italic math alphabet
\let\Dot=\mathbold   % cmbright.sty provides a bold/italic math alphabet
\let\Ddot=\mathbold   % cmbright.sty provides a bold/italic math alphabet
%
\newcommand{\e}[1]{\mathrm{e}^{#1}}% exponential function
\renewcommand{\Re}{\mathrm{Re}}    % real part
\renewcommand{\Im}{\mathrm{Im}}    % imaginary part
\newcommand{\lagr}{\ell}           % Lagrange dichte
\newcommand{\Lagr}{\mathcal{L}}    % Lagrange Funktion
\newcommand{\erf}{{\rm erf}}       %
\newcommand{\atan}{{\rm atan}}     % arcus tangens
\newcommand{\mat}[1]{\bm{#1}}  % Matrix
\newcommand{\gmat}[1]{{\boldsymbol #1}}  % Matrix(symbol)
\newcommand{\defas}{\stackrel{\text{def}}{=}}  %  is defined as
\ifthenelse{\boolean{german}}
  {\newcommand{\rot}{{\rm\bf rot}}}    % curl
  {\newcommand{\rot}{{\rm\bf curl}}}   % curl
\newcommand{\sgn}{{\rm sgn}}       % sign
\ifthenelse{\boolean{german}}
   {\newcommand{\Tr}{\mathrm{Sp}}}      % trace
   {\newcommand{\Tr}{\mathrm{Tr}}}      % trace
\ifthenelse{\boolean{german}}
   {\newcommand{\grmn}[2]{\footnote{``#2'' hei{\ss}t in englisch ``#1''}}}
   {\newcommand{\grmn}[2]{\footnote{``#1'' translates as ``#2'' into German}}}
% define the equation reference
\ifthenelse{\boolean{german}}
   {\newcommand{\eq}[1]{\text{Gl.}~\ref{#1}}}
   {\newcommand{\eq}[1]{\text{Eq.}~\ref{#1}}}
% define a relation with an equation number ontop
\newcommand{\eqrel}[2]{\stackrel{\eq{#1}}{#2}}
\newcommand{\zero}{\varnothing}
%\newcommand{\ket}[1]{|#1\rangle} % contained in package braket
\newcommand{\sumint}{\int\hspace{-15pt}\sum}
\newcommand{\marker}[1]{\textcolor{blue}{\emph{#1}}}
\renewcommand*{\dot}[1]{\overset{\mbox{\large\bfseries .}}{#1}}
\renewcommand*{\ddot}[1]{\overset{\mbox{\large\bfseries\hspace{+0.1ex}.\hspace{-0.1ex}.}}{#1}}
%
%==========================================================================
%==                                                                     ===
%==========================================================================
% Prevent figures from appearing on a page by themselves
% from http://dcwww.camd.dtu.dk/~schiotz/comp/LatexTips/LatexTips.html
\renewcommand{\topfraction}{0.85}
\renewcommand{\textfraction}{0.1}
\renewcommand{\floatpagefraction}{0.75}
%
%==========================================================================
%==                                                                     ===
%==========================================================================
\makeindex    % make index. uses makeidx package.

%== allow links between documents ============================================
\usepackage{xr}
\usepackage{xr-hyper}
%==  hyperref package (must be last package)
\usepackage[colorlinks=true]{hyperref} %specify this as last package
\hypersetup{citecolor=blue}
\hypersetup{menucolor=magenta}
\hypersetup{urlcolor=blue}      % 
\hypersetup{filecolor=green}    % file links
\hypersetup{linkcolor=magenta}  %table of contents
\hypersetup{pdfauthor={Peter E. Bl\"ochl}}
\hypersetup{pdfdisplaydoctitle=true}
\externaldocument[phisx1-]{/Users/ptpb/Tree/PhiSX/ClassicalMechanics/Book/cm-gm}
\externaldocument[phisx2-]{/Users/ptpb/Tree/PhiSX/Electrodynamics/Book/el-gm}
\externaldocument[phisx3-]{/Users/ptpb/Tree/PhiSX/QuantumMechanics/Book/qm}
\externaldocument[phisx4-]{/Users/ptpb/Tree/PhiSX/StatisticalMechanics/Book/sm}
\externaldocument[phisxqm2-]{/Users/ptpb/Tree/PhiSX/QuantumMechanicsII/Book/qm2}
\externaldocument[phisxsm2-]{/Users/ptpb/Tree/PhiSX/StatisticalMechanicsII/Book/sm2}
\externaldocument[phisxcb-]{/Users/ptpb/Tree/PhiSX/Chemicalbond/Book/cb}
% Example: Figure~PhiSX:Quantum
% Mechanics-\ref{phisx3-fig:doubleslitwave} on page
% \pageref{phisx3-fig:doubleslitwave}


\hypersetup{pdftitle=paw_brillouin}
\begin{document}
\begin{titlepage}
\begin{center}
\vspace*{3.5cm}
{\huge \textbf{The Augmentation object of the CP-PAW code}}\\
\vspace{0.5cm}
{\large Peter E. Bl\"ochl}
\vspace{0.5cm} 
\end{center}

\vfill
\begin{center}
Copyright Peter E. Bl\"ochl; Sept.2, 2013-\today\\
{\small
Institute of Theoretical Physics;
Clausthal University of Technology;\\ 
D-38678 Clausthal Zellerfeld; Germany;\\
http://www.pt.tu-clausthal.de/atp/}
\end{center}
\end{titlepage}
\noindent            
\tableofcontents
%====================================================================
\chapter{Todo}
%====================================================================
%====================================================================
\chapter{Purpose and theoretical background of the Augmentation Object}
%====================================================================
The AUGMENTATION object calculates the energies related to the
one-center expansions of the wave functions.

Look up my notes ``Methods'' section ``augmentation object'' in part
objects.tex.

%====================================================================
\section{Exchange and correlation energy}
%====================================================================
\subsection{The energy}

The exchange energy is expressed as a function of the electron density
$n_t$, spin-density $n_s$, and the squares of the total and spin
densities.
\begin{equation}
E_{xc}\Bigl(n_t,n_s,(\nabla n_t)^2,(\nabla n_s)^2,\nabla n_s\nabla n_t\Bigr)
\end{equation}
Because of the requirement of isotropy, the exchange correlation
energy can only depend on the absolute values of the gradients, and on
the relative angle between gradients of total and spin density.  

The angle between the magnetization of a non-collinear spin density
and the gradient of the densities is not taken into account. The
rationale is that spin and lattice are decoupled in a non-relativistic
description. A relativistic description however would also need to
account for current densities.

For reasons of simplicity we introduce a new symbol the five
dimensional vector
\begin{equation}
u=\Bigl(n_t,n_s,(\vec{\nabla} n_t)^2,(\vec{\nabla}n_s)^2
,\vec{\nabla}n_s\vec{\nabla}n_t\Bigr)
\end{equation}
The vector $u$ is expanded into real spherical harmonics, so that
\begin{equation}
u_j(\vec{r})=\sum_L u_{j,L}(|\vec{r}|)Y_L(\vec{r})
\end{equation}
In the code, $u_j$ is the variable \verb|XVAL|.

The total energy is expanded to second order about the spherical
contributions $u_{j,L=0}(\vec{r})Y_0$ of $u_j(\vec{r})$. The first-order term in the expansion vanishes during the angular average.
\begin{eqnarray}
E&=&\int d^3r\;
E_{xc}\Bigl(u_{j,L=0}Y_0+\sum_{L\neq0} u_{j,L}Y_L\Bigr)
\nonumber\\
&=&\int d^3r\; 
\biggl\lbrace
E_{xc}(u_{L=0}Y_0)
+\sum_{L\neq 0} 
\sum_m\frac{dE_{xc}}{d(u_mY_0)}
u_{m,L}
Y_L(\vec{r})
\nonumber\\
&&\hspace{1cm}
+\frac{1}{2}\sum_{L,L'\neq 0} 
\sum_{m,n}\frac{d^2E_{xc}}{d(u_mY_0)d(u_nY_0)}
u_{m,L}u_{n,L'}
Y_LY_{L'}
\biggr\rbrace
\nonumber\\
&=&\int dr\;r^2\; \biggl\lbrace
4\pi E_{xc}(u_{L=0}Y_0)
+
\frac{1}{2}
\sum_{L\neq 0} 
\sum_{m,n}
\frac{d^2E_{xc}}{d(u_mY_0)d(u_nY_0)} u_{m,L}u_{n,L}
\biggr\rbrace
\end{eqnarray}
Here we have used that $d\Omega Y_LY_L=4\pi Y_0 C_{0LL}=1$

%===================================================================
\subsection{The potential}
%===================================================================

The potential is defined via the variation of the total energy as
\begin{eqnarray}
dE&=&
\int d^3r\; \frac{dE}{dn(\vec{r})}dn(\vec{r})
=\int d^3r\; v(\vec{r})dn(\vec{r})
=\int d^3r\sum_L 
\Bigl[v_L(r) Y_L(\vec{r})\Bigr] \Bigl[dn_L(r) Y_L(\vec{r})\Bigr]
\nonumber\\ 
&=& \int dr\; r^2\;  \sum_L v_L(r)dn_L(r)
\end{eqnarray}

Thus, if the energy has the form of a radial integral
\begin{eqnarray}
E=\int dr\;r^2\; F[\{n_L(r)\}]
\end{eqnarray}
the angular momentum components of the potential are
\begin{eqnarray}
v_L(r)=\frac{\partial F}{\partial n_L(r)}
\end{eqnarray}

Hence, the exchange-correlation potential is obtained as follows:
First we evaluate the variation of the total energy
\begin{eqnarray}
dE&=&\int dr\;r^2\; \biggl\lbrace
4\pi \sum_j\frac{\partial E_{xc}}{\partial (u_{j,L=0}Y_0)}
du_{j,L=0}Y_0
+
\frac{1}{2}
\sum_{L'\neq 0}\sum_{m,n,j} 
\frac{\partial^3E_{xc}}{\partial(u_mY_0)\partial(u_nY_0)\partial(u_jY_0)} u_{m,L'}u_{n,L'}du_{j,L=0}Y_0
\nonumber\\
&&\hspace{1cm}
+\frac{1}{2}
\sum_{L\neq 0} 
\sum_{m,n}
\frac{\partial^2E_{xc}}{\partial(u_mY_0)\partial(u_nY_0)} u_{m,L}du_{n,L}
+
\frac{1}{2}
\sum_{L\neq 0} 
\sum_{m,n}
\frac{\partial^2E_{xc}}{\partial(u_mY_0)\partial(u_nY_0)} u_{m,L}du_{n,L}
\biggr\rbrace
\end{eqnarray}


Thus the potential of $u$, defined as
\begin{eqnarray}
dE=\int dr\;r^2\sum_L\sum_j K_{j,L}(r)du_{j,L}(r)
\end{eqnarray}
is
\begin{eqnarray}
K_{j,L=0}&=&
4\pi\frac{\partial E_{xc}}{\partial (u_{j,L=0}Y_0)}Y_0
+
\frac{1}{2}
\sum_{L'\neq 0} 
\sum_{m,n}
\frac{d^3E_{xc}}{\partial(u_mY_0)\partial(u_nY_0)\partial(u_jY_0)} u_{m,L'}u_{n,L'}Y_0
\nonumber\\
K_{j,L\neq0}&=&
\sum_m
\frac{d^2E_{xc}}{\partial(u_mY_0)\partial(u_jY_0)} u_{m,L}
\end{eqnarray}
The potentials are represented in the code by the variable $XDER$.


%===================================================================
\subsection{The gradient terms and their derivatives}
%===================================================================
The gradient correction term of the one-center contribution to
exchange and correlation energy is expressed by the radial functions
for various $L$'s for total $n_t$ and spin $n_s$ density, and the
square of the gradients, $w_{tt}=|\nabla n_t|^2$, $w_{ss}=|\nabla
n_s|^2$ and $w_{ts}=\nabla n_t \nabla n_s$.

The total energy is expressed as a Taylor expansion about the
spherical parts, including the nonspherical parts up to second order.
In this way the first order contributions drop out, because the energy
is the scalar product of different angular momenta.

%====================================================================
\subsection{The gradient dependency}
%====================================================================
Firstly, we evaluate the square of the gradients. We use the indices
$\alpha,\beta\in\{t,s\}$ to indicate the total and spin densities.
\begin{eqnarray}
w_{\alpha,\beta}(\vec{r})&=&\sum_L w_{\alpha,\beta,L}(r) Y_L(\vec{r})
=(\nabla n_{\alpha}(r))(\nabla n_{\beta}(r))
\nonumber\\
&=&\sum_{L^\prime L^{\prime\prime}}
[\nabla n_{\alpha,L^\prime}(r)Y_{L^\prime}(r))]
[\nabla n_{\beta,L^{\prime\prime}}(r)Y_{L^{\prime\prime}}(r))]
.
\end{eqnarray}
Note that the gradient acts on the radial function and the spherica
harmonics.

This expression contains terms of the form
\begin{equation}
[\nabla{f}][\nabla{g}]=\frac{1}{2} \Bigl[ \nabla^2(fg)-(\nabla^2 f)g
-f(\nabla^2 g)\Bigr]
\end{equation}
where each of the two functions $f$ and $g$ are expressed in a
spherical harmonics expansion.  The expression can be verified by
evaluating $\nabla^2(fg)$.

Next we use the Laplacian in an angular momentum representation
\begin{equation}
\nabla^2 f_L(r)Y_L(r)= \Bigl[{1\over{r}}\partial_r^2 r f_L
 -{{\ell(\ell+1)}\over{r^2}}f_L\Bigr]Y_L.
\end{equation}
and the Gaunt coefficients $C_{L,L',L''}$ defined by
$Y_LY_L'=\sum_{L''}C_{L,L',L''}Y_{L''}$

\begin{eqnarray*}
[\nabla{f}][\nabla{g}]
&=&\frac{1}{2} \Bigl[\nabla^2(fg)-(\nabla^2 f)g-f(\nabla^2 g)\Bigr]
\\
&=&\frac{1}{2}\sum_{L,L',L''}C_{L,L',L''} 
\biggl\lbrace 
\underbrace{
\biggl(
\frac{1}{r}\partial^2_r r (f_{L'}g_{L''})-\frac{\ell(\ell+1}{r^2}f_{L'}g_{L''}
\biggr)Y_L
}_{\nabla^2\Bigl(f_{L'}g_{L''}Y_L\Bigr)}
\\
&&-
\underbrace{
\biggl(
\frac{1}{r}\partial_2 r f_{L'}-\frac{\ell'(\ell'+1)}{r^2}f_{L'}\biggr)g_{L''}
Y_L
}_{\leftarrow\Bigl(\nabla^2f_{L'}Y_L'\Bigr)g_{L''}Y_{L''}}
-\underbrace{
f_{L'}\biggl(\frac{1}{r}\partial_2 r g_{L''}-\frac{\ell''(\ell''+1)}{r^2}g_{L''}
\biggr)Y_L
}_{\leftarrow f_{L'}Y_L\nabla^2\Bigl(g_{L''}Y_{L''}\Bigr)}
\\
&=&\frac{1}{2}\sum_{L,L',L''}C_{L,L',L''} Y_L(\vec{r})
\biggl\lbrace
\frac{1}{r}\Bigl[ 
\left(\partial^2_r r f_{L'}g_{L''}\right)
-\left(\partial^2 r f_{L'}\right)g_{L''}
-f_{L'}\left(\partial^2 r g_{L''}\right)\Bigr]
\\
&&\hspace{3cm}
-\Bigl[ \frac{\ell(\ell+1)}{r^2}-\frac{\ell'(\ell'+1)}{r^2}-\frac{\ell''(\ell''+1)}{r^2}\Bigr]
f_{L'}g_{L''}\biggr\rbrace
\end{eqnarray*}

Let us simplify the terms with the derivatives
\begin{eqnarray}
\partial^2_r r f_{L'}g_{L''}
&=&\partial_r\Bigl[ 
\Bigl(\partial_r r f_{L'}\Bigr) g_{L''}
+f_{L'}\Bigl(\partial_r r g_{L''}\Bigr)-f_{L'}g_{L''}\Bigr]
\nonumber\\
&=&
\underbrace{
\Bigl(\partial^2_r r f_{L'}\Bigr) g_{L''}
}_{(A)}
+\underbrace{
\Bigl(\partial_r r f_{L'}\Bigr)\Bigl( \partial_r g_{L''}\Bigr)
}_{(C_1)}
+
\underbrace{
\Bigl(\partial_rf_{L'}\Bigr)\Bigl(\partial_r r g_{L''}\Bigr)
}_{(D_1)}
\nonumber\\
&+&\underbrace{
f_{L'}\Bigl(\partial^2_r r g_{L''}\Bigr)
}_{(B)}
-\underbrace{
\Bigl(\partial_rf_{L'}\Bigr)g_{L''}
}_{(D_2)}
-\underbrace{
f_{L'}\Bigl(\partial_rg_{L''}\Bigr)
}_{(C_2)}
\nonumber\\
&=&
\underbrace{2r\Bigl(\partial_r f_{L'}\Bigr)\Bigl( \partial_r g_{L''}\Bigr)
}_{(C_1-C_2+D_1-D_2)}
+\underbrace{\Bigl(\partial^2_r r f_{L'}\Bigr) g_{L''}}_{(A)}
+\underbrace{f_{L'}\Bigl(\partial^2_r r g_{L''}\Bigr)}_{(B)}
\end{eqnarray}
Thus we obtain
\begin{eqnarray*}
[\nabla{f}][\nabla{g}]
&=&\sum_{L,L',L''}\Bigl[ 
\Bigl(\partial_r f_{L'}\Bigr)\Bigl(\partial_rg_{L''}\Bigr)
-\frac{1}{2}\Bigl( \frac{\ell(\ell+1)}{r^2}-\frac{\ell'(\ell'+1)}{r^2}-\frac{\ell''(\ell''+1)}{r^2}\Bigr)
f_{L'}g_{L''}\Bigr]C_{L,L',L''}Y_L(\vec{r})
\end{eqnarray*}


If we apply this to the expression for the squared gradient, we obtain
\begin{eqnarray}
w_{\alpha,\beta}(\vec{r})&=&\sum_L w_{\alpha,\beta,L}(r) Y_L(\vec{r})
\nonumber\\
&=&\sum_{L^\prime L^{\prime\prime}}
Y_L C_{LL^\prime L^{\prime\prime}}
\Bigl[
(\partial_r n_{\alpha,L^\prime})(\partial_r n_{\beta,L^{\prime\prime}})
+{{\gamma(ll^{\prime}l^{\prime\prime})}\over{r^2}}
n_{L^\prime}n_{L^{\prime\prime}}
\Bigr]
\end{eqnarray}
where 
\begin{equation}
\gamma(ll^{\prime}l^{\prime\prime})=-
\frac{1}{2} 
\left({{\ell(\ell+1)}\over{r^2}}-{{\ell_{L^\prime}(\ell_{L^\prime}+1)}\over{r^2}}
-{{\ell_{L^{\prime\prime}}(\ell_{L^{\prime\prime}}+1)}\over{r^2}}\right)
\end{equation}

We divide $w_{\alpha,\beta}(\vec{r})$ into three contributions, 
\begin{itemize}
\item a spherical one, 
\item a non-spherical one, which is linear in $n_{L{\neq}0}$ and 
\item a non-spherical term
quadratic in $n_{L{\neq}0}$. 
\end{itemize}
The last term is excluded, because it
will not contribute to the total energy if we exclude terms higher
than quadratic in the non-spherical contributions to the density
$n_L$ with $L\neq0$, and those terms only contribute if combined
with another contribution with $L\neq 0$.

\begin{eqnarray}
w_{\alpha,\beta}(\vec{r})&=&\sum_{L}Y_0 C_{0LL}
\Bigl[(\partial_r n_{\alpha,L})(\partial_r n_{\beta,L})
+\frac{\ell(\ell+1)}{r^2}n_{\alpha,L}n_{\beta,L}
\Bigr]
\nonumber\\
&+&\sum_{L\neq0}Y_L C_{0LL}\Bigl[
(\partial_r n_{\alpha,0})(\partial_r n_{\beta,L}) 
+(\partial_r n_{\alpha,L})(\partial_r n_{\beta,0}) 
\Bigr]
\nonumber\\
&+&
\underbrace{\sum_{L\neq0}Y_L 
\sum_{L'\neq0,L''\neq0}
C_{LL'L''}\Bigl[
(\partial_r n_{\alpha,L'})(\partial_r n_{\beta,L''}) 
+\frac{\gamma(\ell,\ell',\ell'')}{r^2}n_{\alpha,L}n_{\beta,L}
\Bigr]}_{\text{ignored}}
\end{eqnarray}


This yields,
\begin{eqnarray}
w(\vec{r})&=&\sum_L w_{L}(|\vec{r}|)Y_{L}(\vec{r})
\\
w_{\alpha,\beta,L}(r)&=&\frac{1}{\sqrt{4\pi}}
\begin{cases}
\sum_{L}
\Bigl[(\partial_r n_{\alpha,L})(\partial_r n_{\beta,L})
+\frac{\ell(\ell+1)}{r^2}n_{\alpha,L}n_{\beta,L}
\Bigr]&\text{for $L=0$}\\
\Bigl[
(\partial_r n_{\alpha,0})(\partial_r n_{\beta,L}) 
+(\partial_r n_{\alpha,L})(\partial_r n_{\beta,0}) 
\Bigr]
&\text{for $L\neq0$}
\end{cases}
\end{eqnarray}
We used $C_{0,L,L}=Y_0=1/\sqrt{4\pi}$.

%====================================================================
\subsection{The derivatives}
%====================================================================
\begin{eqnarray}
dE&=&\int dr\; r^2 \sum_LK_{\alpha,\beta,L}(r)dw_{\alpha,\beta,L}(r)
\nonumber\\
&=&\int dr\; r^2
\biggl\lbrace
 K_{\alpha,\beta,0}(r)dw_{\alpha,\beta,0}(r)
+\sum_{L\neq0} K_{\alpha,\beta,L}(r)dw_{\alpha,\beta,L}(r)
\biggr\rbrace
\nonumber\\
&=&\int dr\; r^2
\biggl\lbrace
 K_{\alpha,\beta,0}(r)\frac{1}{\sqrt{4\pi}}
\sum_L\Bigl[(\partial_r dn_{\alpha,L})(\partial_r n_{\beta,L})
+(\partial_r n_{\alpha,L})(\partial_r dn_{\beta,L})
\nonumber\\
&&\hspace{3cm}+\frac{\ell(\ell+1)}{r^2}
\Bigl(dn_{\alpha,L}n_{\beta,L}+n_{\alpha,L}dn_{\beta,L}\Bigr)
\Bigr]
\nonumber\\
&&+\sum_{L\neq0} K_{\alpha,\beta,L}(r)\frac{1}{\sqrt{4\pi}}
\Bigl[
(\partial_r dn_{\alpha,0})(\partial_r n_{\beta,L}) 
+(\partial_r n_{\alpha,0})(\partial_r dn_{\beta,L}) 
\nonumber\\
&&\hspace{3cm}+(\partial_r dn_{\alpha,L})(\partial_r n_{\beta,0}) 
+(\partial_r n_{\alpha,L})(\partial_r dn_{\beta,0}) 
\Bigr]
\biggr\rbrace
%
%
%
\nonumber\\
&=&\int dr\; r^2\biggl\lbrace
\sum_L\biggl[-\frac{1}{r^2}\partial_rr^2\biggl(
\frac{K_{\alpha,\beta,0}(r)}{\sqrt{4\pi}} (\partial_r n_{\beta,L})
\biggr) 
+
\frac{\ell(\ell+1)}{r^2}\frac{K_{\alpha,\beta,0}(r)}{\sqrt{4\pi}}n_{\beta,L}
\biggr]dn_{\alpha,L}
\nonumber\\
&&\hspace{1.3cm}
+\biggl[-\frac{1}{r^2}\partial_rr^2\sum_{L\neq0}\biggl(
\frac{K_{\alpha,\beta,L}(r)}{\sqrt{4\pi}} \partial_r n_{\beta,L}
\biggr) 
\biggr]dn_{\alpha,0}
+\sum_{L\neq0}\biggl[-\frac{1}{r^2}\partial_rr^2\biggl(
\frac{K_{\alpha,\beta,L}(r)}{\sqrt{4\pi}} \partial_r n_{\beta,0}
\biggr) 
\biggr]dn_{\alpha,L}
\nonumber\\
&&\hspace{1.3cm}+\sum_L\biggl[-\frac{1}{r^2}\partial_rr^2\biggl(
\frac{K_{\alpha,\beta,0}(r)}{\sqrt{4\pi}} \partial_r n_{\alpha,L}
\biggr) 
+
\frac{\ell(\ell+1)}{r^2}\frac{K_{\alpha,\beta,0}(r)}{\sqrt{4\pi}}n_{\alpha,L}
\biggr]dn_{\beta,L}
\nonumber\\
&&\hspace{1.3cm}
+\biggl[-\frac{1}{r^2}\partial_rr^2\sum_{L\neq0}\biggl(
\frac{K_{\alpha,\beta,L}(r)}{\sqrt{4\pi}} \partial_r n_{\alpha,L}
\biggr) 
\biggr]dn_{\beta,0}
+\sum_{L\neq0}\biggl[-\frac{1}{r^2}\partial_rr^2\biggl(
\frac{K_{\alpha,\beta,L}(r)}{\sqrt{4\pi}} \partial_r n_{\alpha,0}
\biggr) 
\biggr]dn_{\beta,L}
\biggr\rbrace
\nonumber\\
%
%
%
&=&\int dr\; r^2\biggl\lbrace
\sum_{L\neq0}\biggl[-\frac{1}{r^2}\partial_rr^2\biggl(
\frac{K_{\alpha,\beta,0}(r)}{\sqrt{4\pi}} (\partial_r n_{\beta,L})
+\frac{K_{\alpha,\beta,L}(r)}{\sqrt{4\pi}} (\partial_r n_{\beta,0})
\biggr) 
+
\frac{\ell(\ell+1)}{r^2}\frac{K_{\alpha,\beta,0}(r)}{\sqrt{4\pi}}n_{\beta,L}
\biggr]dn_{\alpha,L}
\nonumber\\
&&\hspace{1.3cm}
+\biggl[-\frac{1}{r^2}\partial_rr^2\sum_{L}\biggl(
\frac{K_{\alpha,\beta,L}(r)}{\sqrt{4\pi}} \partial_r n_{\beta,L}
\biggr) 
\biggr]dn_{\alpha,0}
\nonumber\\
&&\hspace{1.3cm}+\sum_{L\neq0}\biggl[-\frac{1}{r^2}\partial_rr^2\biggl(
\frac{K_{\alpha,\beta,0}(r)}{\sqrt{4\pi}} \partial_r n_{\alpha,L}
+\frac{K_{\alpha,\beta,L}(r)}{\sqrt{4\pi}} \partial_r n_{\alpha,0}
\biggr) 
+
\frac{\ell(\ell+1)}{r^2}\frac{K_{\alpha,\beta,0}(r)}{\sqrt{4\pi}}n_{\alpha,L}
\biggr]dn_{\beta,L}
\nonumber\\
&&\hspace{1.3cm}
+\biggl[-\frac{1}{r^2}\partial_rr^2\sum_{L}\biggl(
\frac{K_{\alpha,\beta,L}(r)}{\sqrt{4\pi}} \partial_r n_{\alpha,L}
\biggr) 
\biggr]dn_{\beta,0}
\biggr\rbrace
\nonumber\\
\end{eqnarray}



\clearpage
\bibliographystyle{unsrtnat} \bibliography{../all}
\end{document}  
