\documentclass[11pt,a4paper]{report}
%%%%%%%%%%%%%%%%%%%%%%%%%%%%%%%%%%%%%%%%%%%%%%%%%%%%%%%%%%%%%%%%%%%%%
%%                                                                 %%
%%    Header file for the Phi-S-X Series                           %%
%%                                                                 %%
%%    german version header_gm.tex is derived from header.tex      %%
%%    by uncommenting the line ``\setboolean{german}{true}'' below %%
%%                                                                 %%
%%    Never edit the german version! all changes must be done      %%
%%    in the english version header.tex                            %%
%%                                                                 %%
%%%%%%%%%%%%%%%%%%%%%%%%%%%%%%%%%%%%%%%%%%%%%%%%%%%%%%%%%%%%%%%%%%%%%
%%%%%%%%%%%%%%%%%%%%%%%%%%%%%%%%%%%%%%%%%%%%%%%%%%%%%%%%%%%%%%%%%%%%%
%%                                                                 %%
%%    Header file for the Phi-S-X Series                           %%
%%                                                                 %%
%%    german version header_gm.tex is derived from header.tex      %%
%%    by uncommenting the line ``\setboolean{german}{true}'' below %%
%%                                                                 %%
%%    Never edit the german version! all changes must be done      %%
%%    in the english version header.tex                            %%
%%                                                                 %%
%%%%%%%%%%%%%%%%%%%%%%%%%%%%%%%%%%%%%%%%%%%%%%%%%%%%%%%%%%%%%%%%%%%%%
%====================================================================
%-- define flag for language adaptations
\usepackage{ifthen}   % allows to select only certain text
\provideboolean{german}
\setboolean{german}{false}
%\setboolean{german}{true}  % uncomment this line for german editions
%====================================================================
%
% Textschriftart: Computer modern Bright
% body:            CM-Bright 10pt
% section titles:  CM-Bright Bold
% formulas:        CM-Bright Math Oblique
%
\usepackage[standard-baselineskips]{cmbright}
\usepackage{cmbright}
\usepackage[T1]{fontenc}
\def\usedfonts{CM-Bright}
\usepackage{typearea}
%\typearea[current]{calc} % benutzt die aktuelle 
       % bindekorrektur (BCOR angabe als parameter in koma usepackage)
       % und berechnet satzspiegel neu
\typearea[current]{11} %fixed div value

\usepackage{textcomp} % special symbols
\usepackage{amsfonts} % special symols
                      % see ftp://ftp.ams.org/pub/tex/doc/amsfonts/amsfndoc.pdf
\usepackage{amssymb}  % CM-Bright provides the AMS symbols
\usepackage{exscale}  % allows to scale math expressions to big fonts, 
                      % e.g. \Huge
\usepackage{curves}
\usepackage{braket}
\usepackage{miller}     % miller indices
\usepackage{chemmacros} % http://www.mychemistry.eu/mychemistry/
\usepackage[numbers]{natbib}     % bibliography style
\usepackage{url}\urlstyle{tt}
\usepackage{float}
\usepackage{bm}       % provides the command \bm{} that makes bold math symbols
\usepackage{amsmath}
\usepackage{amsbsy}   % allows bold mathematical symbols
\usepackage{amscd}
 \usepackage{a4wide}  % it is better to use the ``geometry'' package
\usepackage{array}    % 
\usepackage{fancyhdr} %  defines pagestyle fancy
\usepackage{epsfig}   % include graphics with epsfig
\usepackage{graphicx} % includegraphics
\usepackage{epstopdf}
\usepackage{wrapfig}
\usepackage{fancybox} % allows shadow-boxes
\usepackage{color}    % allows to use color in the text
%\usepackage{eepic}
\usepackage{flafter}  % places picture next to its reference
\usepackage{makeidx}  % make an index
%\usepackage{MnSymbol}  % 
%\usepackage{marvosym}  % 
\usepackage{textcase}
\usepackage{ulem} % defines strikeout \sout{}; underline \uline{}
                  % double underline \uuline{}; wave underline \uwave{}
                  % cross out \xout{}
%
%==========================================================================
%==  page layout  =========================================================
%==========================================================================
% eqnarray environment: reduce with of space in place of each ``&''
\setlength\arraycolsep{1.4pt}
\pagestyle{fancy}
%\renewcommand{\chaptermark}[1]{\markboth{\thechapter\ #1}{}}
\renewcommand{\chaptermark}[1]{\markboth{\MakeUppercase{\thechapter\ #1}}{}}
\fancyhf{} 
\fancyhead[LE]{\textsc{\thepage}\qquad\textsc{\leftmark}}
\fancyhead[RO]{\textsc{\leftmark}\qquad\textsc{\thepage}}
\renewcommand{\headrulewidth}{0.5pt}
\renewcommand{\footrulewidth}{0pt} 
\addtolength{\headheight}{2.5pt}
\fancypagestyle{plain}{\fancyhead{}
   \renewcommand{\headrulewidth}{0pt}
   \fancyfoot[CO]{\bfseries\thepage}}

% Line spacing -----------------------------------------------------------
\newlength{\defbaselineskip}
\setlength{\defbaselineskip}{\baselineskip}
\newcommand{\setlinespacing}[1]%
           {\setlength{\baselineskip}{#1 \defbaselineskip}}
\newcommand{\doublespacing}{\setlength{\baselineskip}%
                           {2.0 \defbaselineskip}}
\newcommand{\singlespacing}{\setlength{\baselineskip}{\defbaselineskip}}

% Absatz einr\"ucken ------------------------------------------------------
%\setlength{\parindent}{0pt}
\setlength{\parskip}{2pt}
% -------------------------------------------------------------------------
\ifthenelse{\boolean{german}}
  {\def\figurename{Abb.}}
  {\def\figurename{Fig.}}
%--------------------------------------------------------------------------
\renewcommand{\arraystretch}{1.15}  % skaliert den Zeilen abstand in der 
    % tabular und array umgebung
%
%==========================================================================
%==  boxes etc ============================================================
%==========================================================================
%== minipage in a shadowbox ===============================================
\newenvironment{myshadowminipage}[1]%
  {\par\noindent\begin{Sbox}\begin{minipage}{\linewidth}\vspace{0.1cm}\begin{center}\uppercase{#1}\end{center}}%
  {\vspace{0.1cm}\end{minipage}\end{Sbox}\shadowbox{\TheSbox}}
%
%== minipage in a framedbox ===============================================
\newenvironment{myframedminipage}%
  {\par\noindent\begin{Sbox}\begin{minipage}\linewidth\vspace{0.1cm}}%
  {\vspace{0.1cm}\end{minipage}\end{Sbox}\fbox{\TheSbox}}
%
\newcommand{\myshadowbox}[1]{\noindent\shadowbox{\parbox{\linewidth}{\smallskip #1\smallskip}}}
\newcommand{\myfbox}[1]{\noindent\fbox{\parbox{\linewidth}{\smallskip #1\smallskip}}\medskip}
%== minipage in a framedbox ===============================================
\newtheorem{defi}{Definition}[chapter]
\newenvironment{definition}[1]%
  {\par\noindent\begin{Sbox}\begin{minipage}{\linewidth}\vspace{0.1cm}\begin{defi}\uppercase{#1}\\\vspace{0.1cm}}%
  {\vspace{0.1cm}\end{defi}\end{minipage}\end{Sbox}\shadowbox{\TheSbox}}
%
%=========================================================================
% color used to point out information to the teacher
\definecolor{highlight}{rgb}{1.0,0.7,0.}
\newcommand{\Special}[1]{\textbf{\textcolor{highlight}{#1}}}
%=========================================================================
%  switch certain parts on and off. uses ifthen package
\newboolean{teacher}\setboolean{teacher}{false}
% this parameter can be changed in the manuscript again
\setboolean{teacher}{true} %private version if true!
\newcommand{\teacheronly}[1]{\ifthenelse{\boolean{teacher}}{#1\hfill\\ }}
\newcommand{\editor}[1]{\textcolor{blue}{\texttt{Editor: #1}}}
\newcommand{\MARK}[1]{\textcolor{blue}{#1}} 
\newcommand{\RED}[1]{\textcolor{red}{#1}} 
%
%==========================================================================
%==  define new symbols                                                 ===
%==========================================================================
% define \stat (stationary state) as an operator like \min
\DeclareMathOperator*{\stat}{stat}
\let\Vec=\mathbold   % cmbright.sty provides a bold/italic math alphabet
\let\Dot=\mathbold   % cmbright.sty provides a bold/italic math alphabet
\let\Ddot=\mathbold   % cmbright.sty provides a bold/italic math alphabet
%
\newcommand{\e}[1]{\mathrm{e}^{#1}}% exponential function
\renewcommand{\Re}{\mathrm{Re}}    % real part
\renewcommand{\Im}{\mathrm{Im}}    % imaginary part
\newcommand{\lagr}{\ell}           % Lagrange dichte
\newcommand{\Lagr}{\mathcal{L}}    % Lagrange Funktion
\newcommand{\erf}{{\rm erf}}       %
\newcommand{\atan}{{\rm atan}}     % arcus tangens
\newcommand{\mat}[1]{\bm{#1}}  % Matrix
\newcommand{\gmat}[1]{{\boldsymbol #1}}  % Matrix(symbol)
\newcommand{\defas}{\stackrel{\text{def}}{=}}  %  is defined as
\ifthenelse{\boolean{german}}
  {\newcommand{\rot}{{\rm\bf rot}}}    % curl
  {\newcommand{\rot}{{\rm\bf curl}}}   % curl
\newcommand{\sgn}{{\rm sgn}}       % sign
\ifthenelse{\boolean{german}}
   {\newcommand{\Tr}{\mathrm{Sp}}}      % trace
   {\newcommand{\Tr}{\mathrm{Tr}}}      % trace
\ifthenelse{\boolean{german}}
   {\newcommand{\grmn}[2]{\footnote{``#2'' hei{\ss}t in englisch ``#1''}}}
   {\newcommand{\grmn}[2]{\footnote{``#1'' translates as ``#2'' into German}}}
% define the equation reference
\ifthenelse{\boolean{german}}
   {\newcommand{\eq}[1]{\text{Gl.}~\ref{#1}}}
   {\newcommand{\eq}[1]{\text{Eq.}~\ref{#1}}}
% define a relation with an equation number ontop
\newcommand{\eqrel}[2]{\stackrel{\eq{#1}}{#2}}
\newcommand{\zero}{\varnothing}
%\newcommand{\ket}[1]{|#1\rangle} % contained in package braket
\newcommand{\sumint}{\int\hspace{-15pt}\sum}
\newcommand{\marker}[1]{\textcolor{blue}{\emph{#1}}}
\renewcommand*{\dot}[1]{\overset{\mbox{\large\bfseries .}}{#1}}
\renewcommand*{\ddot}[1]{\overset{\mbox{\large\bfseries\hspace{+0.1ex}.\hspace{-0.1ex}.}}{#1}}
%
%==========================================================================
%==                                                                     ===
%==========================================================================
% Prevent figures from appearing on a page by themselves
% from http://dcwww.camd.dtu.dk/~schiotz/comp/LatexTips/LatexTips.html
\renewcommand{\topfraction}{0.85}
\renewcommand{\textfraction}{0.1}
\renewcommand{\floatpagefraction}{0.75}
%
%==========================================================================
%==                                                                     ===
%==========================================================================
\makeindex    % make index. uses makeidx package.

%== allow links between documents ============================================
\usepackage{xr}
\usepackage{xr-hyper}
%==  hyperref package (must be last package)
\usepackage[colorlinks=true]{hyperref} %specify this as last package
\hypersetup{citecolor=blue}
\hypersetup{menucolor=magenta}
\hypersetup{urlcolor=blue}      % 
\hypersetup{filecolor=green}    % file links
\hypersetup{linkcolor=magenta}  %table of contents
\hypersetup{pdfauthor={Peter E. Bl\"ochl}}
\hypersetup{pdfdisplaydoctitle=true}
\externaldocument[phisx1-]{/Users/ptpb/Tree/PhiSX/ClassicalMechanics/Book/cm-gm}
\externaldocument[phisx2-]{/Users/ptpb/Tree/PhiSX/Electrodynamics/Book/el-gm}
\externaldocument[phisx3-]{/Users/ptpb/Tree/PhiSX/QuantumMechanics/Book/qm}
\externaldocument[phisx4-]{/Users/ptpb/Tree/PhiSX/StatisticalMechanics/Book/sm}
\externaldocument[phisxqm2-]{/Users/ptpb/Tree/PhiSX/QuantumMechanicsII/Book/qm2}
\externaldocument[phisxsm2-]{/Users/ptpb/Tree/PhiSX/StatisticalMechanicsII/Book/sm2}
\externaldocument[phisxcb-]{/Users/ptpb/Tree/PhiSX/Chemicalbond/Book/cb}
% Example: Figure~PhiSX:Quantum
% Mechanics-\ref{phisx3-fig:doubleslitwave} on page
% \pageref{phisx3-fig:doubleslitwave}


\hypersetup{pdftitle=paw_brillouin}
\newcommand{\petertt}[1]{\textcolor{red}{\texttt{#1}}}
\makeindex
\begin{document}
\begin{titlepage}
\begin{center}
\vspace*{3.5cm}
{\huge \textbf{The SETUPS object of the CP-PAW code}}\\
\vspace{0.5cm}
{\large Peter E. Bl\"ochl}
\vspace{0.5cm} 
\end{center}

\vfill
\begin{center}
Copyright Peter E. Bl\"ochl; Sept.2, 2013-\today\\
{\small
Institute of Theoretical Physics;
Clausthal University of Technology;\\ 
D-38678 Clausthal Zellerfeld; Germany;\\
http://www.pt.tu-clausthal.de/atp/}
\end{center}
\end{titlepage}
\noindent            
\tableofcontents
%====================================================================
\chapter{Purpose and theoretical background}
%====================================================================
The nodeless construction is proposed by Bl\"ochl and
F\"orst\cite{bloechl12_arxiv1210_5937}.

%====================================================================
\section{Nodeless constructiuon}
%====================================================================
%====================================================================
\subsection{Basic definitions}
%====================================================================
%====================================================================
\subsubsection{Energy-dependent partial waves}
%====================================================================
Let us consider a spherical one-particle Hamiltonian $\hat{h}$. We
construct energy-dependent partial waves by radially integrating the
radial Schr\"odinger equation 
\begin{eqnarray}
(\hat{h}-\epsilon)|\phi(\epsilon)\rangle=0
\end{eqnarray}
outward with the boundary
conditions 
\begin{eqnarray}
\lim_{|\vec{r}|\rightarrow0}\frac{\phi(\epsilon,\vec{r})}
{|\vec{r}|^\ell Y_{\ell,m}(\vec{r})}
\end{eqnarray}
at the origin.

The only difference to the common construction is that we do not
normalize the partial waves within a sphere, but by the boundary
conditions at the origin. That is, we effectively shrink the
normalization sphere to zero.



%====================================================================
\subsubsection{Node-less bound states}
%====================================================================
We define\cite{bloechl12_arxiv1210_5937} a sequence of
\textbf{node-less bound states}\index{node-less bound states}
$|u_n\rangle$
\begin{eqnarray}
(\hat{h}-\epsilon_n)|u_n\rangle=-|u_{n-1}\rangle
\label{eq:nodelesseq}
\end{eqnarray}
The first wave function, $|u_1\rangle$ is the lowest bound state of
the hamiltonian. The other ones are calculated by radially integrating
the Schr\"odinger equation outward, starting with vanishing value and
derivative, so that
\begin{eqnarray}
\lim_{\vec{r}\rightarrow0}|\frac{\langle\vec{r}|u_n\rangle}{|\vec{r}|^{\ell+1}}=0
\qquad\text{for $n>1$}
\end{eqnarray}


The node-less wave functions do not
have any nodes but look like Slater orbitals, in the sense that they
grow at the origin as a power and that they decay exponentially at
large distance. Their behavior at the origin
is\cite{bloechl12_arxiv1210_5937}
\begin{eqnarray}
u_{n}(\vec{r})
=\left(\frac{2m_e}{\hbar^2}\right)^{n-1}
\frac{(2\ell+1)!!}{(\ell-1+2n)!!(2n-2)!!}
|\vec{r}|^{\ell+2(n-1)}Y_{\ell,m}(\vec{r})
\end{eqnarray}
where $Y_{\ell,m}$ is a real spherical harmonics.

The n lowest node-less wave functions span the same Hilbert space as
the n lowest true wave functions. The all-electron wave functions can
be recovered from the node-less wave functions
via\cite{bloechl12_arxiv1210_5937}
\begin{eqnarray}
|\phi_n\rangle=\sum_{m=1}^n|u_m\rangle \prod_{j=1}^{m-1}(\epsilon_j-\epsilon_n)
=\biggl[\sum_{m=1}^n|u_m\rangle 
\prod_{j=m}^{n-1}\frac{1}{(\epsilon_j-\epsilon_n)}
\biggr]\prod_{k=1}^{n-1}(\epsilon_j-\epsilon_n)
\end{eqnarray}

%====================================================================
\subsubsection{Node-reduced partial waves}
%====================================================================
Secondly, we define \cite{bloechl12_arxiv1210_5937} a sequence of
energy-dependent \textbf{node-reduced partial wave}
\index{node-reduced partial wave} through
\begin{eqnarray}
(\hat{h}-\epsilon)|q_n(\epsilon)\rangle=-|u_{n-1}\rangle
\end{eqnarray}
with the boundary conditions
\begin{eqnarray}
\lim_{\vec{r}\rightarrow0}|
\frac{\langle\vec{r}|q_n(\epsilon)\rangle}
{\langle\vec{r}|u_n\rangle}=1
\qquad\text{for $n>1$}
\end{eqnarray}

The node-reduced wave functions have $n-1$ nodes less than the true
wave function. Near the origin, they are very similar to $|u_n\rangle$
over nearly the entire energy range.


The all-electron partial wave can be constructed from the node-reduced
partial wave and the lower node-less bound states
as\cite{bloechl12_arxiv1210_5937}
\begin{eqnarray}
|\phi(\epsilon)\rangle=\biggl[|q_n(\epsilon)\rangle
+\sum_{m=1}^{n-1}|u_m\rangle\prod_{j=m}^{n-1}\frac{1}{\epsilon_j-\epsilon}
\biggr]
\prod_{k=1}^{n-1}(\epsilon_j-\epsilon)
\label{eq:aephifromnodereduced}
\end{eqnarray}


%====================================================================
\subsubsection{Series expansion of the node-reduced partial wave}
%====================================================================
Let us now define \cite{bloechl12_arxiv1210_5937} a new sequence of
nodeless functions via
\begin{eqnarray}
(\hat{h}-\epsilon)|\bar{q}_{n}^{(j)}(\epsilon)\rangle&=&
-|\bar{q}_n^{(j-1)}(\epsilon_{n+j-1})\rangle
\label{eq:qbardiffgl}
\\
(\hat{h}-\epsilon)|\bar{q}_{n}^{(0)}(\epsilon)\rangle&=&-|u_{n-1}\rangle
\end{eqnarray}
which is nearly identical to that of the nodeless equations and that
of the Taylor coefficients.

This sequence can be constructed by the recursive equation
\begin{eqnarray}
|\bar{q}_{n}^{(j+1)}(\epsilon)\rangle&=&
\frac{-1}{\epsilon-\epsilon_{n+j}}
\Bigl(|\bar{q}_n^{(j)}(\epsilon)\rangle
-|\bar{q}_n^{(j)}(\epsilon_{n+j})\rangle\Bigr)
\qquad\text{for $\epsilon\ne\epsilon_{n+j}$}
\nonumber\\
|\bar{q}_{n}^{(j+1)}(\epsilon_{n+j})\rangle&=&
-\left.\partial_\epsilon\right|_{\epsilon_{n+j}}|\bar{q}_n^{(j)}(\epsilon)\rangle
\label{eq:recursionqbar}
\end{eqnarray}

In the special case that all energies $\epsilon_{n+k}$ are chosen to
be identical, this series expansion produces the Taylor expansion
coefficients
\begin{eqnarray}
|q_n^{(j)}(\epsilon)\rangle=
\frac{(-1)^j}{j!}\partial_\epsilon^j|q_n(\epsilon)\rangle
\label{eq:taylorexpansioncoefficientsqn}
\end{eqnarray}



\eq{eq:recursionqbar} is shown recursively: We assume that
\eq{eq:qbardiffgl} holds for $j$ and show that  \eq{eq:recursionqbar}
obeys \eq{eq:qbardiffgl} for $j+1$.
\begin{eqnarray}
(\hat{h}-\epsilon)|\bar{q}^{(j+1)}_{n}(\epsilon)\rangle
&\eqrel{eq:recursionqbar}{=}&
\frac{-1}{\epsilon-\epsilon_{n+j}}
\Bigl((\hat{h}-\epsilon)|\bar{q}_n^{(j)}(\epsilon)\rangle
-(\hat{h}-\epsilon_{n+j})|\bar{q}_n^{(j)}(\epsilon_{n+j})\rangle
+(\epsilon-\epsilon_{n+j})|\bar{q}_n^{(j)}(\epsilon_{n+j})\rangle
\Bigr)
\nonumber\\
&\eqrel{eq:qbardiffgl}{=}&
\frac{-1}{\epsilon-\epsilon_n}
\Bigl(
-|\bar{q}_n^{(j-1)}(\epsilon_{n+j-1})\rangle
+|\bar{q}_n^{(j-1)}(\epsilon_{n+j-1})\rangle
+(\epsilon-\epsilon_{n+j})|\bar{q}_n^{(j)}(\epsilon_{n+j})\rangle
\Bigr)
\nonumber\\
&=&
-|\bar{q}_n^{(j)}(\epsilon_{n+j})\rangle
\end{eqnarray}
which is \eq{eq:qbardiffgl} for the next higher value of $j$. This
completes the proof that $|q_{n}^{(j)}(\epsilon)\rangle$ defined in
\eq{eq:qbardiffgl} obeyes the recursion \eq{eq:recursionqbar}.



From this series we can again recover the node-reduced partial wave as
\begin{eqnarray}
|q_n(\epsilon)\rangle&=&|\bar{q}_n^{(0)}(\epsilon)\rangle
\nonumber\\
&=&|\bar{q}_n^{(0)}(\epsilon_n)\rangle
+|\bar{q}_{n}^{(1)}(\epsilon)\rangle(\epsilon_n-\epsilon_)
\nonumber\\
&=&|\bar{q}_n^{(0)}(\epsilon_n)\rangle
+|\bar{q}_{n}^{(1)}(\epsilon_{n+1})\rangle(\epsilon_n-\epsilon)
+|\bar{q}_{n}^{(2)}(\epsilon)\rangle
(\epsilon_{n+1}-\epsilon)(\epsilon_{n}-\epsilon)
\nonumber\\
&=&\sum_{k=0}^{j-1}
|\bar{q}_{n}^{(k)}(\epsilon_{n+k})\rangle
\prod_{m=0}^{k-1}(\epsilon_{n+m}-\epsilon)
+|\bar{q}_n^{(j)}(\epsilon)\rangle
\prod_{m=0}^{j-1}(\epsilon_{n+m}-\epsilon)
\end{eqnarray}

In this derivation, no assumptions have been made regarding the
energies $\epsilon_{n+j}$. Using all energies equal, results in the
Taylor expanstion given above. Using bound state energies in sequence
results in a nodeless construction.

This flexibility can be exploited to set up an expansion with bound
states and with scattering states. This is important if we wish to
describe semi-core states, valence states and scattering states on the
same footing.

From \eq{eq:aephifromnodereduced}, we obtain the expression for the
full energy-dependent wave function.
\begin{eqnarray}
|\phi(\epsilon)\rangle
\frac{1}{\prod_{k=1}^{n-1}(\epsilon_j-\epsilon)}
&=&
\underbrace{
|\bar{q}_n^{(j)}(\epsilon)\rangle
\left(\prod_{m=0}^{j-1}(\epsilon_{n+m}-\epsilon)\right)
+
\sum_{k=0}^{j-1}
|\bar{q}_{n}^{(k)}(\epsilon_{n+k})\rangle
\left(\prod_{m=0}^{k-1}(\epsilon_{n+m}-\epsilon)\right)
}_{|q_n(\epsilon)\rangle}
\nonumber\\
&+&
\underbrace{
\left(\sum_{m=1}^{n-1}|u_m\rangle\prod_{j=m}^{n-1}\frac{1}{\epsilon_j-\epsilon}\right)
}_{\text{core contribution}}
\end{eqnarray}

Interesting is that the energy-dependent function
$|\bar{q}^{(j)}(\epsilon)\rangle$ does not contribute to the full wave
function at the energies $\epsilon_{j}$ of the core states and at the
energies $\epsilon_{n+m}$ with $m<j$.



%====================================================================
\subsection{Augmentation from node-reduced wave functions}
%====================================================================
%====================================================================
\subsubsection{Pseudization}
%====================================================================

The node-reduced wave function have a similar shape over the entire
energy region. This suggests to change them by the same,
energy-independent function $|k\rangle$, because this is likely not to
introduce extra nodes and to produce wave functions that are equally
suited for a plane wave expansion.

Thus we obtain a definition of our pseudo partial waves over the
entire energy region
\begin{eqnarray}
|\tilde{\phi}(\epsilon)\rangle
\frac{1}{\prod_{k=1}^{n-1}(\epsilon_j-\epsilon)}
=|q_n(\epsilon)\rangle+|k\rangle
\end{eqnarray}
Having a well behaved representation of the pseudo partial waves, we
can be quite sure that we do not add ghost states. If each partial
wave is pseudized individually, one easily obtains a polynomial
representation of the energy dependent partial wave that contains
divergences.  The construction of $|k\rangle$ is described in
section~\ref{sec:determinedeltepseudo}.

%====================================================================
\subsubsection{Truncation of the series expansion}
%====================================================================
In order to obtain individual partial waves, we truncate the expansion
for the energy dependent partial wave to $N_j$ terms
\begin{eqnarray}
|\phi(\epsilon)\rangle
\frac{1}{\prod_{k=1}^{n-1}(\epsilon_j-\epsilon)}
&\approx&
\sum_{k=0}^{N_j-1}
|\bar{q}_{n}^{(k)}(\epsilon_{n+k})\rangle
\prod_{m=0}^{k-1}(\epsilon_{n+m}-\epsilon)
+
\underbrace{
\sum_{m=1}^{n-1}|u_m\rangle\prod_{j=m}^{n-1}\frac{1}{\epsilon_j-\epsilon}
}_{\text{core contribution}}
\nonumber\\
|\tilde{\phi}(\epsilon)\rangle\frac{1}{\prod_{k=1}^{n-1}(\epsilon_j-\epsilon)}
&\approx&
\sum_{k=0}^{N_j-1}
|\bar{q}_{n}^{(k)}(\epsilon_{n+k})\rangle
\prod_{m=0}^{k-1}(\epsilon_{n+m}-\epsilon)
+|k\rangle
\label{eq:truncatedpspartialwithk}
\end{eqnarray}
This representation of the energy dependent partial wave is exact for
the energies $\epsilon_{n+j}$ for $0\le j\le N_j-1$.

A basis set describing the pseudo partial waves is
\begin{eqnarray}
|\tilde{\phi}^{(0)}\rangle&=&|q_n(\epsilon_n)\rangle+|k\rangle
\nonumber\\
|\tilde{\phi}^{(j)}\rangle&=&|\bar{q}_n^{(j)}(\epsilon_{n+j})\rangle
\qquad\text{for $1\le j\le N_j-1$}
\label{eq:tildephijfromqbar}
\end{eqnarray}
Note, that only the first pseudo partial wave differs from its
node-reduced partial wave. This is because the difference $|k\rangle$
has been chosen energy independent.

%====================================================================
\subsubsection{Core orthogonalization}
%====================================================================
The all-electron partial waves are obtained from the node-reduced
partial waves $|q_n^{(j)}(\epsilon_{n+j})\rangle$ by explicit
orthogonalization to the nodeless core states. Here we begin to
orthogonalize to the highest core state first and then proceed to the
lowest core state.

Note that we do not include a pseudo core into the pseudo partial
waves even though we currently still define a pseudo core state. The
latter shall be used to construct a pseudo core density.

%====================================================================
\subsubsection{Construction of bare projector functions}
%====================================================================
The projector functions must obey the biorthogonality condition and
the bare projector functions must fulfill the closure relation
\begin{eqnarray}
|\bar{p}^{(j)}\rangle=(\tilde{h}-\epsilon_{n+j})
|\tilde{\phi}(\epsilon_{n+j})\rangle \qquad\text{for $j=0,\ldots,j_x$}
\end{eqnarray}
which implies that the scattering properties shall be exactly obeyed at
the chosen energies $\epsilon_{n+j}$.

The pseudo partial waves at the energies are
\begin{eqnarray}
|\tilde{\phi}(\epsilon_{n+p})\rangle
&\eqrel{eq:truncatedpspartialwithk}{=}&
\sum_{k=0}^{N_j-1}|\bar{q}_n^{(k)}(\epsilon_{n+k})\rangle
\prod_{m=0}^{k-1}(\epsilon_{n+m}-\epsilon_{n+p})+|k\rangle
\nonumber\\
&=&\sum_{k=0}^{p}|\bar{q}_n^{(k)}(\epsilon_{n+k})\rangle
\prod_{m=0}^{k-1}(\epsilon_{n+m}-\epsilon_{n+p})+|k\rangle
\nonumber\\
&\eqrel{eq:tildephijfromqbar}{=}&\sum_{k=0}^{p}|\tilde{\phi}^{(k)}\rangle
\prod_{m=0}^{k-1}(\epsilon_{n+m}-\epsilon_{n+p})
\label{eq:tildepsienplusp}
\end{eqnarray}
(the terms that drop out in the second line vanish because of the
energy factor contains a zero term.)

The expression {eq:tildepsienplusp} holds strictly, as long as there
are no degernate valued in the sequence $\epsilon_{n+p}$. If two or
more $\epsilon_{n+p}$ are identical, there is only a single
independent partial wave for this set of energies. As a result the
space of bare projector functions derived from these partial waves is
too small. For each degenerate set of partial waves we need to
introduce first and higher derivatives of the partial waves.

The bare projector functions are obtained as follows
\begin{eqnarray}
(\hat{\tilde{h}}-\epsilon_{n+p})
|\tilde{\phi}(\epsilon_{n+p})\rangle
&=&\sum_{k=0}^{p}
(\hat{\tilde{h}}-\epsilon_{n+p})
|\tilde{\phi}^{(k)}\rangle
\prod_{m=0}^{k-1}(\epsilon_{n+m}-\epsilon_{n+p})
\nonumber\\
&=&
(\hat{\tilde{h}}-\epsilon_{n})|\tilde{\phi}^{(0)}\rangle
+|\tilde{\phi}^{(0)}\rangle(\epsilon_n-\epsilon_{n+p})
\nonumber\\
&-&
\underbrace{\sum_{k=1}^{p}
\overbrace{|q_n^{(k-1)}(\epsilon_{n+k-1})\rangle
}^{-(\hat{h}-\epsilon_{n+k})|q_n^{(k)}\rangle}
\prod_{m=0}^{k-1}(\epsilon_{n+m}-\epsilon_{n+p})}
_{
\sum_{k=0}^{p-1}
|q_n^{(k)}(\epsilon_{n+k})\rangle(\epsilon_{n+k}-\epsilon_{n+p})
\prod_{m=0}^{k-1}(\epsilon_{n+m}-\epsilon_{n+p})
}
\nonumber\\
&+&\sum_{k=1}^{p-1}
|q_n^{(k)}(\epsilon_{n+k})\rangle(\epsilon_{n+k}-\epsilon_{n+p})
\prod_{m=0}^{k-1}(\epsilon_{n+m}-\epsilon_{n+p})
\nonumber\\
&+&\sum_{k=1}^{p}
(\hat{\tilde{h}}-\hat{h})
|q_n^{(k)}(\epsilon_{n+k})\rangle
\prod_{m=0}^{k-1}(\epsilon_{n+m}-\epsilon_{n+p})
\nonumber\\
&=&
(\hat{\tilde{h}}-\epsilon_{n})|\tilde{\phi}^{(0)}\rangle
+|\tilde{\phi}^{(0)}\rangle(\epsilon_n-\epsilon_{n+p})
-|q_n^{(0)}(\epsilon_{n})\rangle(\epsilon_{n}-\epsilon_{n+p})
\nonumber\\
&+&\sum_{k=1}^{p}
(\hat{\tilde{h}}-\hat{h})
|q_n^{(k)}(\epsilon_{n+k})\rangle
\prod_{m=0}^{k-1}(\epsilon_{n+m}-\epsilon_{n+p})
\nonumber\\
&=&
(\hat{\tilde{h}}-\epsilon_{n})|\tilde{\phi}^{(0)}\rangle
+
\Bigl[|\tilde{\phi}^{(0)}\rangle-|q_n^{(0)}(\epsilon_{n})\rangle\Bigr]
(\epsilon_n-\epsilon_{n+p})
\nonumber\\
&+&\sum_{k=1}^{p}
(\hat{\tilde{h}}-\hat{h})
|q_n^{(k)}(\epsilon_{n+k})\rangle
\prod_{m=0}^{k-1}(\epsilon_{n+m}-\epsilon_{n+p})
\nonumber\\
&=&
(\hat{\tilde{h}}-\epsilon_{n})|\tilde{\phi}^{(0)}\rangle
+
(\epsilon_n-\epsilon_{n+p})
\biggl(
|\tilde{\phi}^{(0)}\rangle-|q_n^{(0)}(\epsilon_{n})\rangle
+(\hat{\tilde{h}}-\hat{h})
|q_n^{(1)}(\epsilon_{n+1})\rangle
\nonumber\\
&+&\sum_{k=2}^{p}
(\hat{\tilde{h}}-\hat{h})
|q_n^{(k)}(\epsilon_{n+k})\rangle
\prod_{m=1}^{k-1}(\epsilon_{n+m}-\epsilon_{n+p})\biggr)
\end{eqnarray}

The bare projector functions can be spanned by the $N_j$ independent functions
\begin{eqnarray}
|\bar{p}^{(0)}\rangle&=&(\hat{\tilde{h}}-\epsilon_{n})|\tilde{\phi}^{(0)}\rangle
\nonumber\\
|\bar{p}^{(1)}\rangle&=&|\tilde{\phi}^{(0)}\rangle-|q_n^{(0)}(\epsilon_n)\rangle
+(\hat{\tilde{h}}-\hat{h})
|q_n^{(1)}(\epsilon_{n+1})\rangle
\nonumber\\
|\bar{p}^{(j)}\rangle&=&(\hat{\tilde{h}}-\hat{h})|q_n^{(j)}(\epsilon_{n+j})\rangle
\qquad\text{for $j\ge2$}
\end{eqnarray}
This selection is independent from the choice of the energies
$\epsilon_{n+p}$ used to define the sequence of node-reduced wave
functions. That is the result is equally valid if all these energies
are identical.


%====================================================================
\subsubsection{Bi-orthogonalization}
%====================================================================
The bi-orthogonalization shall establish 
\begin{eqnarray}
\langle p^{(j)}|\tilde{\phi}^{(j')}\rangle=\delta_{j,j'}
\end{eqnarray}

We do this while maintaining the partial waves, i.e.
\begin{eqnarray}
|p^{(j)}\rangle=\sum_{n}|p'^{(n)}\rangle
\Bigl(\langle p'^{(j)}|\tilde{\phi}^{(n)}\rangle\Bigr)_{n,j}
\end{eqnarray}

\textbf{The biorthogonalization deteriorates quickly, if the number of
  partial waves is increased.}


\petertt{Question: does the integrand for the overlap between
  projector function and partial wave fall off sufficiently fast with
  distance? Note, that the partial waves increase exponentially.}


%====================================================================
\subsubsection{Augmentation contribution to matrix elements}
%====================================================================
\begin{eqnarray}
dO_{\alpha,\beta}&=&\langle\phi_\alpha|\phi_\beta\rangle-
\langle\tilde{\phi}_\alpha|\tilde{\phi}_\beta\rangle
\nonumber\\
dT_{\alpha,\beta}&=&\langle\phi_\alpha|\hat{t}|\phi_\beta\rangle-
\langle\tilde{\phi}_\alpha|\hat{t}|\tilde{\phi}_\beta\rangle
\end{eqnarray}
The calculation of these matrix elements becomes substantially more
involved in the relativistic case.

\petertt{Problem: In order to cancel the tails of the core wave
  functions, the pseudo partial waves must include a pseudo-core
  contribution. This, however, is likely to produce ghost states. A
  remedy could be to change the exponentially increasing behavior of
  the the node-reduced partial waves artificially such that they do no
  more grow. In that case, the pseudo-core contribution would not harm.}

%====================================================================
\subsection{Pseudo wave functions from node-reduced wave functions}
\label{sec:determinedeltepseudo}
%====================================================================
Here, the routine \verb|SETUPS_MAKEPSPHI_MINE| is described. It is used
to construct a pseudo wave function for the lowest node-reduced
partial wave.

Starting from a wave function $\phi(\vec{r})$, which shall be
pseudized, we invert the Schr\"odinger equation to determine
\begin{eqnarray}
v-\epsilon_\nu=
\begin{cases}
-\frac{1}{\phi(\vec{r})}\hat{t}\phi(\vec{r})&\qquad\text{for $r>r_c$.}
\\
-\frac{1}{\phi(\vec{r}_c)}\hat{t}\phi(\vec{r}_c)&\qquad\text{for $r<r_c$.}
\end{cases}
\label{eq:invertschrgl}
\end{eqnarray}
where $r_c$ is the pseudization radius. $\hat{t}|\phi\rangle$ is a
short-hand notation for the result of the kinetic energy operator
acting on the partial wave. In the non-relativistic theory
$\hat{t}=\frac{\hat{\vec{p}}^2}{2m_0}$, while the expression is more
complex in the relativistic theory.

With \eq{eq:invertschrgl}, we obtain a potential that is shallow and
flat inside the pseudization radius. Outside the pseudization radius
it is chosen such that partial wave obeys the non-relativistic
Schr\"odinger equation for this potential.

Now, we derive for this potential the radial wave function and its
first two energy derivatives
\begin{eqnarray}
\Bigl[\frac{\vec{p}^2}{2m_0}+v-\epsilon_\nu\Bigr]|f\rangle&=&0
\nonumber\\
\Bigl[\frac{\vec{p}^2}{2m_0}+v-\epsilon_\nu\Bigr]|\dot{f}\rangle&=&|f\rangle
\nonumber\\
\Bigl[\frac{\vec{p}^2}{2m_0}+v-\epsilon_\nu\Bigr]|\ddot{f}\rangle
&=&2|\dot{f}\rangle
\end{eqnarray}

At the same time we determine 
\begin{eqnarray}
\hat{t}|f\rangle&=&\frac{\hat{\vec{p}}^2}{2m_e}|f\rangle
=-(\epsilon-v)|f\rangle
\nonumber\\
\hat{t}|\dot{f}\rangle&=&\frac{\hat{\vec{p}}^2}{2m_e}|\dot{f}\rangle
=|f\rangle-(\epsilon-v)|\dot{f}\rangle
\nonumber\\
\hat{t}|\ddot{f}\rangle&=&\frac{\hat{\vec{p}}^2}{2m_e}|\dot{f}\rangle
=2|\dot{f}\rangle-(\epsilon-v)|\ddot{f}\rangle
\end{eqnarray}

Finally, we match these three functions onto the node-reduced wave
function so that
\begin{eqnarray}
|\tilde{\phi}\rangle=|f\rangle c_1 
+|\dot{f}\rangle c_2+|\ddot{f}\rangle c_3
\end{eqnarray}
so that value, derivative and (if requested) the kinetic energy
density at the cutoff radius agree.

%====================================================================
\section{Relativistic effects}
%====================================================================
I recommend to read the doctoral thesis entitled ``The ZORA equation''
of Eric van Lenthe\cite{lenthe96_thesis}, if you can get your hands on
it.


%====================================================================
\subsection{Dirac equation}
%====================================================================
We start from the Dirac equation of an electron in an electric field.
First we divide the wave function into an upper part
$\phi=(\Psi_1,\Psi_2)$ and a lower part $\chi=(\Psi_3,\Psi_4)$
\begin{eqnarray}
\left(\begin{array}{cc}
v-\epsilon &\quad \vec{\sigma}\vec{p}c\\
\vec{\sigma}\vec{p}c &\quad -2m_0c^2+v-\epsilon\end{array}\right)
\left(\begin{array}{c}|\phi\rangle\\|\chi\rangle\end{array}\right)=0
\label{eq:dirac2by2withepsilon}
\end{eqnarray}
where $\epsilon$ is the energy of the electron relative to the
rest-energy $m_0c^2$. $\vec{\sigma}$ is a vector of $2\times2$
matrices formed from the three Pauli matrices
$\sigma_x,\sigma_y,\sigma_z$.

It will be convenient to introduce the function
\begin{eqnarray}
1+D(\vec{r})\defas\Bigl[1+\frac{\epsilon-v(\vec{r})}{2m_0c^2}\Bigr]^{-1}
\end{eqnarray}
so that the Dirac equation obtains the form
\begin{eqnarray}
\left(\begin{array}{cc}
v-\epsilon &\quad \vec{\sigma}\vec{p}c\\
\vec{\sigma}\vec{p}c &\quad \frac{-2m_0c^2}{1+D}\end{array}\right)
\left(\begin{array}{c}|\phi\rangle\\|\chi\rangle\end{array}\right)=0
\label{eq:dirac2by2withepsilonb}
\end{eqnarray}

The small contribution can be extracted from the second equation of
\eq{eq:dirac2by2withepsilonb}
\begin{eqnarray}
|\chi\rangle
=\frac{1}{2m_0c^2+\epsilon-v}
\vec{\sigma}\vec{p}c |\phi\rangle
=\frac{1+D}{2m_0c}
\vec{\sigma}\vec{p} |\phi\rangle
\end{eqnarray}
Thus, the small component can be eliminated from the first equation of
\eq{eq:dirac2by2withepsilonb}, which is turned into a second-order
differential equation for the large component.
\begin{eqnarray}
\Bigl[\vec{\sigma}\vec{p} \frac{1+D}{2m_0}
\vec{\sigma}\vec{p} +v-\epsilon\Bigr]|\phi\rangle=0
\end{eqnarray}

Using the magic equation
\begin{eqnarray}
(\vec{\sigma}\vec{a})(\vec{\sigma}\vec{b})
=\vec{a}\vec{b}+i\vec{\sigma}(\vec{a}\times\vec{b})
\end{eqnarray}
we obtain the form
\begin{eqnarray}
\Bigl[\vec{p} \frac{1+D}{2m_0}\vec{p} 
+i\vec{\sigma} 
\Bigl(\vec{p} \frac{1+D}{2m_0}\times\vec{p} \Bigr)
+v-\epsilon\Bigr]|\phi\rangle=0
\nonumber\\
\Bigl[\frac{1+D}{2m_0}\vec{p}^2-i\frac{\hbar}{2m_0}(\vec{\nabla}D)\vec{p}
+\vec{S}\Bigl(\frac{\vec{\nabla}D}{m_0}\times\vec{p} \Bigr)
+v-\epsilon\Bigr]|\phi\rangle=0
\end{eqnarray}
where $\vec{S}=\frac{\hbar}{2}\vec{\sigma}$ is the spin.  For a
rotationally symmetric $D(\vec{r})$ , we can
introduce the orbit angular momentum $\vec{L}=\vec{r}\times\vec{p}$
and we obtain the Dirac equation in the form
\begin{eqnarray}
\Bigl[
\frac{1+D}{2m_0}\vec{p}^2 
-i\frac{\hbar\partial_rD}{2m_0|\vec{r}|}\vec{r}\vec{p} 
+i\frac{2}{\hbar}\vec{S} 
\Bigl(\frac{\hbar}{i}(\underbrace{\vec{\nabla}\frac{1+D}{2m_0}}
_{\frac{(\partial_r D)}{2m_0}\frac{\vec{r}}{|\vec{r}|}}\times\vec{p} \Bigr)
+v-\epsilon\Bigr]|\phi\rangle=0
\nonumber\\
\Rightarrow\Bigl[
\frac{1+D}{2m_0}\vec{p}^2 
-i\frac{\hbar\partial_rD}{2m_0|\vec{r}|}\vec{r}\vec{p} 
+ \underbrace{\frac{(\partial_r D)}{m_0|\vec{r}|}\vec{S}\vec{L}}_{\text{spin-orbit}}
+v-\epsilon\Bigr]|\phi\rangle=0
\nonumber\\
\end{eqnarray}


%====================================================================
\subsection{Relativistic augmentation}
%====================================================================
It is our goal to separate all relativistic effects out into the
augmentation contribution, so that we need not consider the
relativistic effects in the plane-wave part. The plane-wave part does
not involve large kinetic energies so that their contribution to the
relativistic effects can be ignored with good confidence.

In the PAW method, we make the following Ansatz for the wave function:
\begin{eqnarray}
\left(\begin{array}{c}|\phi\rangle\\|\chi\rangle\end{array}\right)
=
\left(
\begin{array}{c}|\tilde{\psi}\rangle\\\frac{1}{2m_0c}\vec{\sigma}\vec{p}|\tilde\psi\rangle\end{array}\right)
+
\sum_\alpha
\left[
\left(
\begin{array}{c}|\phi_\alpha\rangle\\
\frac{1+D}{2m_0c}\vec{\sigma}\vec{p}|\phi_\alpha\rangle\end{array}\right)
-
\left(
\begin{array}{c}|\tilde{\phi}_\alpha\rangle\\
\frac{1}{2m_0c}\vec{\sigma}\vec{p}|\tilde{\phi}_\alpha\rangle\end{array}\right)
\right]
\langle\tilde{p}_\alpha|\tilde\psi\rangle
\end{eqnarray}
This ansatz ensures that the wave function is continuous--if $D(r)$
vanishes smoothly beyond the augmentation radius--and that all-electron and
pseudo partial waves become identical. 

This mapping from a two-dimensional Pauli spinor onto a four component
spinor wave function is reminiscent of the Foldy-Wouthuysen
transformation\cite{foldy50_pr78_29}, which diagonalizes the matrix
form of the Dirac Hamiltonian.\footnote{Here we distinguish
  $\vec{\sigma}\vec{p}$ and its hermitean conjugate
  $(\vec{\sigma}\vec{p})^\dagger$. This refers to the fact that the
  identity is not valid on a point-per-point basis, but only in
  connection with certain boundary conditions.}
\begin{eqnarray}
\left(\begin{array}{c}\langle\phi|\\
\langle\chi|\end{array}\right)
\left(\begin{array}{cc}
v-\epsilon &\quad (\vec{\sigma}\vec{p}c)^\dagger\\
\vec{\sigma}\vec{p}c &\quad -2m_0c^2+v-\epsilon\end{array}\right)
\left(\begin{array}{c}|\phi\rangle\\|\chi\rangle\end{array}\right)=0
\end{eqnarray}


For the pseudo contribution, we change the all-electron potential to
the pseudo potential $\tilde{v}$ and we set the relativistic factor
$D$ to zero. 

In addition we introduce the small components of the partial waves 
\begin{eqnarray}
|\chi_\alpha\rangle&=&(1+D_\alpha)\frac{\vec{\sigma}\vec{p}}{2m_0c}
|\phi_\alpha\rangle
\nonumber\\
|\tilde{\chi}_\alpha\rangle&=&\frac{\vec{\sigma}\vec{p}}{2m_0c}
|\tilde{\phi}_\alpha\rangle
\end{eqnarray}
The index on the $D_\alpha$ indicates that it has been evaluated with
the energy $\epsilon_\alpha$, the energy used in the nodeless equation
for $|\phi_\alpha\rangle$.

For the small pseudo partial wave we made the choice to set $D$ to
zero, while we keep the finite speed of light.

To extract Hamiltonian and Overlap operator from the Dirac equation, I
form the matrix element of the Dirac equation with the corresponding
Bra.
\begin{eqnarray}
\Rightarrow
0&=&\left(\begin{array}{c}\langle\tilde{\psi}|\\
\langle\tilde{\psi}|\frac{(\vec{\sigma}\vec{p})^\dagger}{2m_0c}\end{array}\right)
\left(\begin{array}{cc}
\tilde{v}-\epsilon &\quad (\vec{\sigma}\vec{p}c)^\dagger \\
\vec{\sigma}\vec{p}c &\quad -2m_0c^2+\tilde{v}-\epsilon\end{array}\right)
\left(\begin{array}{c}|\psi\rangle\\
\frac{\vec{\sigma}\vec{p}}{2m_0c}|\tilde{\psi}\rangle\end{array}\right)
\nonumber\\
&+&\sum_{\alpha,\beta}\langle\tilde{\psi}|\tilde{p}_\alpha\rangle
\biggl[
%
\left(\begin{array}{c}\langle\phi_\alpha|\\
\langle\chi_\alpha|
\end{array}\right)
\left(\begin{array}{cc}
v-\epsilon &\quad (\vec{\sigma}\vec{p}c)^\dagger\\
\vec{\sigma}\vec{p}c &\quad -2m_0c^2+v-\epsilon\end{array}\right)
\left(\begin{array}{c}|\phi_\beta\rangle\\
|\chi_\beta\rangle\end{array}\right)
%
\nonumber\\
&&\hspace{2cm}-\left(\begin{array}{c}\langle\tilde{\phi}_\alpha|\\
\langle\tilde{\chi}_\alpha|
\end{array}\right)
\left(\begin{array}{cc}
\tilde{v}-\epsilon &\quad (\vec{\sigma}\vec{p}c)^\dagger\\
\vec{\sigma}\vec{p}c &\quad -2m_0c^2+\tilde{v}-\epsilon\end{array}\right)
\left(\begin{array}{c}|\tilde{\phi}_\beta\rangle\\
|\tilde{\chi}_\beta\rangle\end{array}\right)
%
\biggr]
\langle\tilde{p}_\beta|\tilde{\psi}\rangle
\nonumber\\  %==========================================================
%
&=&\langle\tilde\psi|
\frac{\hat{\vec{p}}^2}{2m_0}+\tilde{v}-\epsilon
+
\underbrace{
\textcolor{red}{(\vec{\sigma}\vec{p})^\dagger
\frac{(\tilde{v}-\epsilon)}{(2m_0c)^2}(\vec{\sigma}\vec{p})}}_{\textrm{ignored}}
|\tilde{\psi}\rangle
\nonumber\\
&+&\sum_{\alpha,\beta}\langle\tilde{\psi}|\tilde{p}_\alpha\rangle
\biggl\lbrace
\underbrace{\Bigl[
\overbrace{
\langle\phi_\alpha|(\vec{\sigma}\vec{p})^\dagger
\frac{1+D_\alpha}{2m_0}
}^{\langle\chi_\alpha|}
(\vec{\sigma}\vec{p})|\phi_\beta\rangle
+\langle\phi_\alpha|
(\vec{\sigma}\vec{p})^\dagger
\overbrace{\frac{1+D_\beta}{2m_0}(\vec{\sigma}\vec{p})
|\phi_\beta\rangle}^{|\chi_\beta\rangle}
-2m_0c^2\langle\chi_\alpha|\chi_\beta\rangle\Bigr]}_{T_{\alpha,\beta}}
\nonumber\\
%
&&\hspace{1.5cm}
-\underbrace{\Bigl[
\overbrace{\langle\tilde{\phi}_\alpha|
(\vec{\sigma}\vec{p})^\dagger\frac{1}{2m_0}
}^{\langle\tilde{\chi}_\alpha|}(\vec{\sigma}\vec{p})
+(\vec{\sigma}\vec{p})^\dagger
\overbrace{
\frac{1}{2m_0}(\vec{\sigma}\vec{p})
|\tilde{\phi}_\beta\rangle}^{|\tilde{\chi}_\beta\rangle}
-2m_0c^2\langle\tilde{\chi}_\alpha
|\tilde{\chi}_\beta\rangle\Bigr]}_{\tilde{T}_{\alpha,\beta}}
\nonumber\\
&&\hspace{1.5cm}+\langle\phi_\alpha|v-\epsilon|\phi_\beta\rangle
+\textcolor{red}{\langle\chi_\alpha|v-\epsilon|\chi_\beta\rangle}
-\langle\tilde{\phi}_\alpha|
\tilde{v}-\epsilon|\tilde{\phi}_\beta\rangle
-
\textcolor{red}{
\langle\tilde{\chi}_\alpha|\tilde{v}-\epsilon|\tilde{\chi}_\beta\rangle}
\biggr\rbrace\langle
\tilde{p}_\beta|\tilde{\psi}\rangle
\nonumber\\
\end{eqnarray}
We used that $(\vec{\sigma}\vec{p})^2=\vec{p}^2$.

The three terms marked in red are problematic, because
the first one contributes to plane wave part. The all-electron term is
not neglegible in the core region, but it also does not vanish in the
tail region, if the partial waves, and with it their small components,
diverge with increasing radius. This diverging part does, however, not
contribute, because it is cancelled by the corresponding one-center
pseudo term. This is also the reason that the pseudo term needs to be
included.


%========================================================================
\subsection{Approximation of relativistic effects for the PAW-Hamiltonian}
%========================================================================
The pseudo term of the plane wave part marked in red
\begin{eqnarray}
\langle\tilde\psi|
(\vec{\sigma}\vec{p})\frac{(\tilde{v}-\epsilon)}{(2m_0c)^2}(\vec{\sigma}\vec{p})
|\tilde{\psi}\rangle &=& \langle\tilde\psi|
\frac{1}{(2m_0c)^2}\frac{\hbar}{i}(\vec{\nabla}\tilde{v})\vec{p}
+\frac{2}{(2m_0c)^2}
\vec{S}\Bigl((\vec{\nabla}\tilde{v})\times\vec{p}\Bigr)
+\frac{(\tilde{v}-\epsilon)}{c^2} \frac{\vec{p}^2}{2m_0}
|\tilde{\psi}\rangle \nonumber\\ &=&
\frac{1}{2m_0c^2}\langle\tilde\psi|
\frac{-i\hbar}{2m_0}(\vec{\nabla}\tilde{v})\vec{p} +\frac{1}{m_0}
\vec{S}\Bigl((\vec{\nabla}\tilde{v})\times\vec{p}\Bigr)
+(\tilde{v}-\epsilon)\vec{p}^2 |\tilde{\psi}\rangle
\label{eq:neglectedrelpart}
\end{eqnarray}
is considered to be small and is ignored in the current
implementation. Let us investigate the size of the error.

In order to estimate its size let us consider the free electron case,
where the wave functions are simple plane waves.
\begin{eqnarray*}
\langle\tilde\psi|
(\vec{\sigma}\vec{p})\frac{(\tilde{v}-\epsilon)}{(2m_0c)^2}(\vec{\sigma}\vec{p})
|\tilde{\psi}\rangle
&=&\frac{(\hbar\vec{G})^2}{2m_0}\frac{\epsilon-v}{2m_0c^2}
\approx \frac{E_{kin}^2}{2m_0c^2}\approx E^2_{kin} \cdot 2.5
\times10^{-5}\frac{1}{H}
\nonumber\\
&\approx&
\begin{cases}
5.00 mH\cdot N_e &\text{for $E_{kin}=30$~Ry}\\
0.60 mH\cdot N_e &\text{for $E_{kin}=10$~Ry}\\
0.06 mH\cdot N_e &\text{for $E_{kin}=5$~Ry}\\
\end{cases}
\end{eqnarray*}
where $N_e$ is the number of electrons.

Alternatives: Conceivable is to remove the first term from the pseudo
one-center part along with ignoring the complete plane wave part of
the small component. The rational is that this term captures the
dominant term of the pseudo plane wave part, while it does not have
the exponentially increasing behavior that needs to be canceled in the
all-electron term.

%========================================================================
\subsection{Relativistic PAW-Hamiltonian}
%========================================================================
With the neglect of the term Eq.~\ref{eq:neglectedrelpart}, we obtain
\begin{eqnarray}
0&=&
\langle\tilde{\Psi}|\hat{\tilde{H}}-\hat{\tilde{O}}\epsilon|\tilde{\Psi}\rangle
\nonumber\\
&=&\langle\tilde\psi|
\frac{\hat{\vec{p}}^2}{2m_0}+\tilde{v}-\epsilon 
+\sum_{\alpha,\beta}|\tilde{p}_\alpha\rangle
\biggl(dT_{\alpha,\beta}+dV_{\alpha,\beta}-\epsilon dO_{\alpha,\beta}\biggr)
\langle\tilde{p}_\beta|
|\tilde{\psi}\rangle
\nonumber\\
&=&\langle\tilde\psi|
\frac{\hat{\vec{p}}^2}{2m_0}+\tilde{v}-\epsilon
|\tilde{\psi}\rangle
\nonumber\\
&+&\sum_{\alpha,\beta}\langle\tilde{\psi}|\tilde{p}_\alpha\rangle
\biggl\lbrace
\underbrace{\Bigl[
\overbrace{
\langle\phi_\alpha|(\vec{\sigma}\vec{p})^\dagger
\frac{1+D_\alpha}{2m_0}
}^{\langle\chi_\alpha|}
(\vec{\sigma}\vec{p})|\phi_\beta\rangle
+\langle\phi_\alpha|
(\vec{\sigma}\vec{p})^\dagger
\overbrace{\frac{1+D_\beta}{2m_0}(\vec{\sigma}\vec{p})
|\phi_\beta\rangle}^{|\chi_\beta\rangle}
-2m_0c^2\langle\chi_\alpha|\chi_\beta\rangle\Bigr]}_{T_{\alpha,\beta}}
\nonumber\\
%
&&\hspace{1.5cm}
-\underbrace{\Bigl[
\overbrace{\langle\tilde{\phi}_\alpha|
(\vec{\sigma}\vec{p})^\dagger\frac{1}{2m_0}
}^{\langle\tilde{\chi}_\alpha|}(\vec{\sigma}\vec{p})
+(\vec{\sigma}\vec{p})^\dagger
\overbrace{
\frac{1}{2m_0}(\vec{\sigma}\vec{p})
|\tilde{\phi}_\beta\rangle}^{|\tilde{\chi}_\beta\rangle}
-2m_0c^2\langle\tilde{\chi}_\alpha
|\tilde{\chi}_\beta\rangle\Bigr]}_{\tilde{T}_{\alpha,\beta}}
\nonumber\\
&&\hspace{1.5cm}+\langle\phi_\alpha|v-\epsilon|\phi_\beta\rangle
+\textcolor{red}{\langle\chi_\alpha|v-\epsilon|\chi_\beta\rangle}
-\langle\tilde{\phi}_\alpha|
\tilde{v}-\epsilon|\tilde{\phi}_\beta\rangle
-
\textcolor{red}{
\langle\tilde{\chi}_\alpha|\tilde{v}-\epsilon|\tilde{\chi}_\beta\rangle}
\biggr\rbrace\langle
\tilde{p}_\beta|\tilde{\psi}\rangle
\nonumber\\
\end{eqnarray}

\begin{itemize}
\item Thus we obtain pseudo kinetic energy operator as
\begin{eqnarray}
\hat{\tilde{T}}&=&
\frac{\hat{\vec{p}}^2}{2m_0}
+\sum_{\alpha,\beta}|\tilde{p}_\alpha\rangle 
dT_{\alpha,\beta}\langle\tilde{p}_\beta|
\nonumber\\
dT_{\alpha,\beta}&=&
\underbrace{
\langle\phi_\alpha|
(\vec{\sigma}\vec{p})\frac{1+D_\alpha}{2m_0}(\vec{\sigma}\vec{p})
}_{\langle g_\alpha|-\langle\phi_\alpha|(v-\epsilon_\alpha)}
|\phi_\beta\rangle
+\langle\phi_\alpha|
\underbrace{(\vec{\sigma}\vec{p})\frac{1+D_\beta}{2m_0}(\vec{\sigma}\vec{p})
|\phi_\beta\rangle}_{|g_\beta\rangle-(v-\epsilon_\beta)|\phi_\beta\rangle}
-2m_0c^2\langle\chi_\alpha|\chi_\beta\rangle
\nonumber\\
&-&
\underbrace{
\langle\tilde{\phi}_\alpha|\frac{\hat{\vec{p}}^2}{2m_0}
}_{\langle \tilde{g}_\alpha|-\langle\tilde{\phi}_\alpha|(\tilde{v}-\epsilon_\alpha)}
|\tilde{\phi}_\beta\rangle
-\langle\tilde{\phi}_\alpha|
\underbrace{
\frac{\hat{\vec{p}}^2}{2m_0}|\tilde{\phi}_\beta\rangle
}_{|\tilde{g}_\beta\rangle-(\tilde{v}-\epsilon_\beta)|\tilde{\phi}_\beta\rangle}
+2m_0c^2\langle\tilde{\chi}_\alpha|\tilde{\chi}_\beta\rangle
\nonumber\\
\end{eqnarray}
Here we exploit that the partial waves are constructed from a radial
Dirac equation of the form
\begin{eqnarray}
\biggl[(\vec{\sigma}\vec{p})\frac{1+D_\beta}{2m_0}(\vec{\sigma}\vec{p})
+v-\epsilon_\beta\biggr]|\phi_\beta\rangle&=&|g_\beta\rangle
\nonumber\\
\biggl[
\frac{\hat{\vec{p}}^2}{2m_0}+\tilde{v}-\epsilon_\beta\biggr]
|\tilde{\phi}_\beta\rangle
&=&|\tilde{g}_\beta\rangle
\end{eqnarray}

\item The relativistic pseudo overlap operator has the form
\begin{eqnarray}
\hat{\tilde{O}}&=&1+
\sum_{\alpha,\beta}|\tilde{p}_\alpha\rangle dO_{\alpha,\beta}\langle\tilde{p}_\beta|
\nonumber\\
dO_{\alpha,\beta}&=&
\langle\phi_\alpha|\phi_\beta\rangle+\langle\chi_\alpha|\chi_\beta\rangle
-\langle\tilde{\phi}_\alpha|\tilde{\phi}_\beta\rangle
-\langle\tilde{\chi}_\alpha|\tilde{\chi}_\beta\rangle
\end{eqnarray}

\item The potential energy has the form
\begin{eqnarray}
\hat{\tilde{v}}&=&\hat{v}+
\sum_{\alpha,\beta}|\tilde{p}_\alpha\rangle dV_{\alpha,\beta}\langle\tilde{p}_\beta|
\nonumber\\
dV_{\alpha,\beta}&=&
\langle\phi_\alpha|\hat{v}|\phi_\beta\rangle
+\langle\chi_\alpha|\hat{v}|\chi_\beta\rangle
-\langle\tilde{\phi}_\alpha|\tilde{v}|\tilde{\phi}_\beta\rangle
-\langle\tilde{\chi}_\alpha|\tilde{v}|\tilde{\chi}_\beta\rangle
\end{eqnarray}


From the expression of the potential energy, we obtain a
representation for the density, namely
\begin{eqnarray}
n(\vec{r})=
\underbrace{
\sum_n f_n \tilde{\psi}_n^2(\vec{r})+\tilde{n}_c}_{\tilde{n}(\vec{r})}
&+&\underbrace{
\sum_{\alpha,\beta}\theta_{\alpha,\beta}
\biggl(\phi^*_\beta(\vec{r})\phi^*_\alpha(\vec{r})
+\chi^*_\beta(\vec{r})\chi^*_\alpha(\vec{r})\biggr)
+n_c(\vec{r})}_{n^1(\vec{r})}
\nonumber\\
&-&\underbrace{
\sum_{\alpha,\beta}\theta_{\alpha,\beta}
\biggl(\tilde{\phi}^*_\beta(\vec{r})\tilde{\phi}^*_\alpha(\vec{r})
+\tilde{\chi}^*_\beta(\vec{r})\tilde{\chi}^*_\alpha(\vec{r})
\biggr)+\tilde{n}_c(\vec{r})}_{\tilde{n}^1(\vec{r})}
\end{eqnarray}
Here $n_c(\vec{r})$ is the core electron density and
$\tilde{n}_c(\vec{r})$ is the core pseudo electron density.

The one-center density matrix $\theta_{\alpha,\beta}$ is defined as
\begin{eqnarray}
\theta_{\alpha,\beta}=\sum_n\langle\tilde{p}_\alpha|\tilde{\psi}_n\rangle f_n
\langle\tilde{p}_\beta\rangle
\end{eqnarray}
\end{itemize}


%====================================================================
\subsection{Nodeless construction for the Dirac equation}
%====================================================================
We generalize \eq{eq:nodelesseq} to the Dirac equation, which yields
\begin{eqnarray}
\left(\begin{array}{cc}
v-\epsilon_n &\quad \vec{\sigma}\vec{p}c\\
\vec{\sigma}\vec{p}c &\quad -2m_0c^2+v-\epsilon_n\end{array}\right)
\left(\begin{array}{c}|u_n\rangle\\|v_n\rangle\end{array}\right)
&=&-\left(\begin{array}{c}|u_{n-1}\rangle\\|v_{n-1}\rangle\end{array}\right)
\end{eqnarray}
where the small component of a nodeless function is denoted by $|v_n\rangle$.

We resolve for the small component $|v_n\rangle$
\begin{eqnarray}
|v_n\rangle=\frac{1+D_n}{2m_0c}(\vec{\sigma}\vec{p})|u_n\rangle
+\frac{1+D_n}{2m_0c^2}|v_{n-1}\rangle
\label{eq:nodelessdiracsmall}
\end{eqnarray}
and insert the result into the first equation
\begin{eqnarray}
(v-\epsilon)|u_n\rangle
+(\vec{\sigma}\vec{p})c|v_n\rangle&=&-|u_{n-1}\rangle 
\nonumber\\
\Bigl[(\vec{\sigma}\vec{p})
\frac{1+D_n}{2m_0}(\vec{\sigma}\vec{p})
+v-\epsilon_n\Bigr]|u_n\rangle
&=&-|u_{n-1}\rangle -(\vec{\sigma}\vec{p})\frac{1+D_n}{2m_0c}|v_{n-1}\rangle
\nonumber\\
\Bigl[
(1+D_n)\frac{\vec{p}^2}{2m_0}
+\frac{\hbar}{i}
\frac{\vec{\nabla}D_n}{2m_0}\vec{p}
+\vec{S}\Bigl(\frac{\vec{\nabla}D_n}{m_0}\times\vec{p}\Bigr)
+v-\epsilon_n\Bigr]|u_n\rangle
&=&-|u_{n-1}\rangle -(\vec{S}\vec{p})\frac{1+D_n}{\hbar m_0c}|v_{n-1}\rangle
\nonumber\\
\label{eq:nodelessdirac1}
\end{eqnarray}


%====================================================================
\subsection{Spinor harmonics}
%====================================================================
For a spherical atom, the spin and orbit angular momenta are no more
conserved independently, but only the total angular momentum is
conserved. Therefore we need to introduce spinor harmonics.

The spinor harmonics obey the eigenvalue equations
\begin{eqnarray}
\left(\hat{\vec{L}}+\hat{\vec{S}}\right)^2|\chi_{\kappa,j_z}\rangle
&=&|\chi_{\kappa,j_z}\rangle\quad
\underbrace{\hbar^2(\kappa^2-\frac{1}{2})}_{\hbar^2j(j+1)}
\\
\hat{\vec{L}}^2|\chi_{\kappa,j_z}\rangle
&=&|\chi_{\kappa,j_z}\rangle\quad\hbar^2\kappa(\kappa+1)
\\
\hat{\vec{S}}^2|\chi_{\kappa,j_z}\rangle
&=&|\chi_{\kappa,j_z}\rangle\quad\hbar^2\frac{3}{4}
\\
\hat{\vec{L}}\hat{\vec{S}}|\chi_{\kappa,j_z}\rangle
&=&|\chi_{\kappa,j_z}\rangle
\quad\left(-\hbar^2\frac{\kappa+1}{2}\right)
\label{eq:lsspinorharmonic}
\\
\left(\hat{L}_z+\hat{S}_z\right)|\chi_{\kappa,j_z}\rangle
&=&|\chi_{\kappa,j_z}\rangle\quad\hbar j_z
\\
\frac{1}{|\vec{r}|}\vec{r}\;\vec{\sigma}\chi_{\kappa,j_z}&=&-\chi_{-\kappa,j_z}
\end{eqnarray}
where the quantum numbers $\kappa$ can assume any integer value except zero.
\begin{eqnarray}
\kappa&=&\ldots,-2,-1,\qquad 1,2,\ldots
\\
j_z&=&-|\kappa|+\frac{1}{2},-|\kappa|+\frac{3}{2},\ldots,|\kappa|-\frac{1}{2}
\\
j&=& |\kappa|-\frac{1}{2}
\\
\ell&=&|\kappa+\frac{1}{2}|-\frac{1}{2}
\\
\left(\begin{array}{c} m_{\uparrow}\\m_{\downarrow}\end{array}\right)
&=&\left(\begin{array}{c} j_z-\frac{1}{2}\\j_z+\frac{1}{2}\end{array}\right)
\end{eqnarray}
Furthermore, 
\begin{eqnarray}
\vec{L}\vec{S}
\begin{cases}
<0 & \text{ for $\kappa >0$}\\
=0 & \text{ for $\kappa =-1$, i.e. for $\ell=0$}\\
>0 & \text{ for $\kappa <-1$}
\end{cases}
\end{eqnarray}

\begin{center}
\begin{tabular}{|c|c|c|c|c|c|c|c|c|}
\hline
$\kappa$ & -4 & -3 & -2 & -1 & - & 1 & 2 & 3 \\
\hline
$\ell$   &  3 &  2 &  1 &  0 & - & 1 & 2 & 3 \\
\hline
$\ell$   &  f &  d &  p &  s & - & p & d & f \\
\hline
\end{tabular}
\end{center}


The spinor harmonics have the form
\begin{eqnarray}
|\chi_{\kappa,j_z}\rangle
\stackrel{\kappa<0}{=}\left(
\begin{array}{c}
|Y_{-\kappa-1,j_z-\frac{1}{2}}\rangle
\sqrt{\frac{-\kappa+j_z-\frac{1}{2}}{-2\kappa-1}}\\
|Y_{-\kappa-1,j_z+\frac{1}{2}}\rangle
\sqrt{\frac{-\kappa-j_z-\frac{1}{2}}{-2\kappa-1}}\end{array}\right)
\qquad
|\chi_{\kappa,j_z}\rangle
\stackrel{\kappa>0}{=}\left(
\begin{array}{c}
-|Y_{\kappa,j_z-\frac{1}{2}}\rangle
\sqrt{\frac{\kappa-j_z+\frac{1}{2}}{2\kappa+1}}\\
|Y_{\kappa,j_z+\frac{1}{2}}\rangle
\sqrt{\frac{\kappa+j_z+\frac{1}{2}}{2\kappa+1}}\end{array}\right)
\label{eq:spinorharmonicskappa_a}
\\
\underbrace{\chi_{-\ell-1,m+\frac{1}{2}}}_{\Omega_{\ell+\frac{1}{2},\ell,m+\frac{1}{2}}}
=\left(
\begin{array}{c}
|Y_{\ell,m}\rangle\sqrt{\frac{\ell+m+1}{2\ell+1}}\\
|Y_{\ell,m+1}\rangle\sqrt{\frac{\ell-m}{2\ell+1}}\end{array}\right)
\qquad
\underbrace{\chi_{\ell,m+\frac{1}{2}}}_{\Omega_{\ell-\frac{1}{2},\ell,m+\frac{1}{2}}}
=\left(
\begin{array}{c}
-|Y_{\ell,m}\rangle\sqrt{\frac{\ell-m}{2\ell+1}}\\
|Y_{\ell,m+1}\rangle\sqrt{\frac{\ell+m+1}{2\ell+1}}\end{array}\right)
\label{eq:spinorharmonicskappa}
\end{eqnarray}

Then spinor harmonics describe states with the spin parallel or
antiparallel to the orbit angular momentum.

%====================================================================
\subsection{Spherical Dirac equations in spinor harmonics}
%====================================================================
The wave function of an electron is a four-component spinor function
consisting of a two-dimensional large and a two-dimensional small
component. Each has again a spin-up and a spin-down component.
\begin{eqnarray}
\left(\begin{array}{c}\phi(\epsilon,\vec{r})\\\chi(\epsilon,\vec{r})
\end{array}\right) =
\left(\begin{array}{c}g_{\kappa,j_z}(\epsilon,\vec{r})
\chi_{\kappa,j_z}(\vec{r})\\ if_{-\kappa,j_z}(\epsilon,\vec{r})
\chi_{-\kappa,j_z}(\vec{r})\\ \end{array}\right) 
\end{eqnarray} 

This Ansatz can be inserted into \eq{eq:nodelessdirac1}
and \eq{eq:nodelessdiracsmall}


\begin{eqnarray}
\Bigl[
(1+D_n)\frac{\vec{p}^2}{2m_0}
+\frac{\hbar}{i}
\frac{\partial_rD_n}{2m_0|\vec{r}}\vec{r}\vec{p}
+\frac{\partial_rD_n}{m_0|\vec{r}|} \vec{S}\underbrace{
\Bigl(\vec{r}\times\vec{p}\Bigr)}_{\vec{L}}
+v-\epsilon_n\Bigr]|u_n\rangle
&=&-|u_{n-1}\rangle -(\vec{S}\vec{p})\frac{1+D_n}{\hbar m_0c}|v_{n-1}\rangle
\nonumber\\
|v_n\rangle=\frac{1+D_n}{2 m_0}
\Bigl[\frac{2}{\hbar c}
(\vec{S}\vec{p})|u_n\rangle
+\frac{1}{c^2}|v_{n-1}\rangle\Bigr]
\end{eqnarray}
which yields with \eq{eq:lsspinorharmonic}

\begin{eqnarray}
&&\hspace{-2cm}\biggl(
(1+D)\frac{-\hbar^2}{2m_0}\biggl(\frac{1}{r}\partial_r^2r
-\frac{\ell(\ell+1)}{r^2}\biggr)
-\frac{\hbar^2(\partial_r D)}{2m_0}
\biggl[\partial_r
+\frac{\kappa+1}{|\vec{r}|}\biggr]
+v-\epsilon \biggr)g^{(n)}_{\kappa,j_z}(\epsilon,r)
\nonumber\\
&=&
-g^{(n-1)}_{\kappa,j_z}(\epsilon_{n-1},r)
+
\frac{\hbar}{c}\biggl[\partial_r+\frac{-\kappa+1}{|\vec{r}|}\biggr]
\frac{1+D}{ 2m_0 }
f_{-\kappa,j_z}^{(n-1)}(\epsilon_{n-1},r)
\nonumber\\
f_{-\kappa,j_Z}^{(n)}(\epsilon,r)
&=&\frac{1+D(\epsilon)}{2 m_0}
\biggl(\frac{\hbar}{c}\biggl[\partial_r+\frac{\kappa+1}{|\vec{r}|}\biggr]
g_{\kappa,j_z}^{(n)}(\epsilon,r)
+
\frac{1}{c^2}f_{-\kappa,j_z}^{(n-1)}(\epsilon_{n-1},r)\biggr)
\end{eqnarray}
Compare with Eq.~5.49-5.51 of the master thesis of Robert
Schade\cite{schade12_thesis}.  (Note, however, that I changed the sign
of $f_{n}$ relative to that of $f_{n-1}$, etc. relative to my earlier
notes.)

%======================================================================
\section{Pseudization}
\label{sec:pseudiation}
%======================================================================
%======================================================================
\subsection{Pseudization of core density and potential}
%======================================================================
For an all-electron quantity $f(\vec{r})$ we determine a pseudized
quantity $\tilde{f}(\vec{r})$ according to the following procedure.
\begin{eqnarray*}
\tilde{f}(r)=
\left\lbrace\begin{array}{ll}
a+br^\lambda+cr^{\lambda+2}&\quad\textrm{for}\qquad r<r_c\\
f(r)&\quad\textrm{for}\qquad r>r_c
\end{array}
\right.
\end{eqnarray*}
The parameters $a$, $b$ and $c$ are determined such that the pseudo
quantity $\tilde{f}$ is differentiable.

The construction depends on three parameters:
\begin{itemize}
\item $\tilde{v}(r=0)$
\item $r_c$
\item $\lambda$
\end{itemize}

%======================================================================
\section{Nucleus with finite size}
%======================================================================
The nucleus is considered as a homogeneously charged sphere. The
volume of the nucleus is proportional to the number of nucleons.  This
allows to relate the radius directly to the mass of the
nucleus.\cite{cooper53_pr92_801,hofstadter56_rmp28_214}
\begin{eqnarray*}
r_{nuc}=\sqrt[3]{\frac{M}{u}}\cdot1.2\cdot 10^{15} m 
=\sqrt[3]{\frac{M}{m_e}}\cdot1.85635065215\cdot 10^{-6}  a_0
\end{eqnarray*}
where $M$ is the mass of the nucleus and
$u=\frac{1}{12}m(C^{12})$. 

The potential of the nucleus is therefore
\begin{eqnarray*}
v_{nuc}(r)=
\left\lbrace
\begin{array}{cc}
\frac{-Ze^2}{r_{nuc}}\left(\frac{3}{2}-\frac{1}{2}\frac{r^2}{r_{nuc}^2}\right)
&\textrm{for}\qquad r<r_{nuc}\\
\frac{-Ze^2}{r}
&\textrm{for}\qquad r>r_{nuc}
\end{array}\right.
\end{eqnarray*}

%====================================================================
\section{Fock term}
%====================================================================
We wish to replace part of the exchange correlation potential by the
full exchange potential, that is by the Fock operator. We face the
problem that, now, there is a fully non-local potential\footnote{The
  energy for a fully nonlocal potential must be expressed as $\int
  d^3r\int d^3r'\;
  \psi^*(\vec{r})v(\vec{r},\vec{r'})\psi(\vec{r})$.} in the
  Hamiltonian, which makes the radial Schr\"odinger equation
  substantially more complex.

The total energy has the form
\begin{eqnarray*}
E&=&E_{GGA}[n]+\alpha\left(-\frac{1}{2}\sum_{n,m,\sigma,\sigma'}
\int dr\int dr'\;
\frac{e^2\Psi^*_n(\vec{r},\sigma)\Psi^*_m(\vec{r}\,',\sigma')
\Psi_m(\vec{r},\sigma)\Psi_n(\vec{r}\,',\sigma')}{4\pi\epsilon_0|\vec{r}-\vec{r'}|}
-E_{GGA,X}[n]\right)
\\
n(\vec{r})&=&\sum_{n,\sigma}\Psi_n^*(\vec{r},\sigma)\Psi_n(\vec{r},\sigma)
\end{eqnarray*}
We treat each wave function as a Pauli spinor wave function
with a spin up and a spin down component. Usually (without spin-orbit
coupling) a state has either a pure spin up or a spin down component.
The sum over $n$ runs over all electrons.
The term $E_{GGA,x}$ is the exchange only contribution to the exchange
functional.

Formulate problem in the form $h_0$, $w$. etc. Define all variables.
%====================================================================
\subsection{Generalized perturbation theory}
\label{sec:gpt}
%====================================================================
While we are able to obtain the solutions of the Schr\"odinger
equation for a spherical, local or semi-local potential efficiently,
the solution for a completely non-local potential is cumbersome even
if it is spherical. Here we use an iterative approach, which obtains
the result from an approximate solution. 

The basic idea is inspired by perturbation theory.  In contrast to
first order perturbation theory however, we do not require that the
approximate solution obeys a Schr\"odinger equation for some kind of
local hamiltonian exactly. This makes it more convenient for a
self-consistent, iterative procedure.

Probably, it is more related to the Green's function approach.

Our goal is to find a solution for the inhomogeneous Schr\"odinger
equation
\begin{eqnarray*}
\left(\hat{h}_0+\hat{w}-\varepsilon\right)
|\phi\rangle&=&|I\rangle
\end{eqnarray*}
The hamiltonian shall be divided into an approximate hamiltonian
$\hat{h}_0$, having a local potential only, and $\hat{w}$ is a
correction including the non-local potential.

We start with an approximate solution $|\phi_0\rangle$, and
approximate energy $\varepsilon_0$. 
\begin{eqnarray*}
\left(\hat{h}_0+\hat{w}-\varepsilon_0-\delta\varepsilon\right)
|\phi_0+\delta\phi\rangle&=&|I\rangle
\\
\Rightarrow\qquad
\left(\hat{h}_0+\hat{w}-\varepsilon_0-\delta\varepsilon\right)
|\delta\phi\rangle
&=&-\left(\hat{h}_0+\hat{w}-\epsilon_0-\delta\varepsilon\right)
|\phi_0\rangle
+|I\rangle
\end{eqnarray*}
Now we assume that $(\hat{w}-\delta\varepsilon)|\delta\phi\rangle$
is small. We neglect it and arrive at
\begin{eqnarray*}
\left(\hat{h}_0-\varepsilon_0\right)
|\delta\phi\rangle
=-\left(\hat{h}_0+\hat{w}-\epsilon_0-\delta\varepsilon\right)
|\phi_0\rangle
+|I\rangle
\end{eqnarray*}
The value of $\delta\varepsilon$ is determined by the boundary
conditions. 

\begin{myshadowminipage}{Iterative solution of the Schr\"odinger equation}
In practice we determine $|\phi'\rangle$ and $|\partial_\epsilon
\phi\rangle$ from
\begin{eqnarray*}
\left(\hat{h}_0-\varepsilon_0\right)
|\phi\prime\rangle
&=&-\left(\hat{h}_0+\hat{w}-\varepsilon_0\right)
|\phi_0\rangle
+|I\rangle
\\
\left(\hat{h}_0-\varepsilon_0\right)
|\partial_\varepsilon\phi\rangle
&=&|\phi_0\rangle
\\
(\hat{h}_0-\epsilon_0)|\phi_{hom}\rangle&=&0
\end{eqnarray*}
We have explicitely also included the homogeneous solution, which may
be admixed into the two states $|\phi'\rangle$ and
$|\partial_\epsilon\phi\rangle$.

With these objects we can improve our solution
\begin{eqnarray*}
|\phi\rangle=|\phi_0\rangle+|\phi'\rangle
+|\partial_\varepsilon\phi\rangle\delta\epsilon+|\phi_{hom}\rangle \alpha
\end{eqnarray*}
The parameters $\delta\epsilon$ and $\alpha$ are determined according
to the boundary conditions: value, derivative at the integration
bounds or energy or norm.
The new state $|\phi\rangle$ is not yet the correct result, because we
neglected second-order terms, but we can use it as new trial solution
$|\phi_0\rangle$ in an iterative procedure.
\end{myshadowminipage}

Let us test the result
\begin{eqnarray*}
&&\left(\hat{h}_0+\hat{w}-\varepsilon_0-\delta\varepsilon\right)
\Bigl(|\phi_0\rangle+|\phi'\rangle
+|\partial_\varepsilon\phi\rangle\delta\varepsilon
+|\phi_{hom}\rangle\alpha\Bigr)-|I\rangle
\\
&=&\left(\hat{h}_0+\hat{w}-\varepsilon_0-\delta\varepsilon\right)|\phi_0\rangle
-|I\rangle
\\
&+&
\underbrace{\left(\hat{h}_0-\varepsilon_0\right)|\phi'\rangle}
_{-(\hat{h}_0+\hat{w}-\epsilon_0)|\phi_0\rangle+|I\rangle}
+\left(\hat{w}-\delta\varepsilon\right)|\phi'\rangle
\\
&+&
\underbrace{\left(\hat{h}_0-\varepsilon_0\right)
|\partial_\varepsilon\phi\rangle}_{|\phi_0\rangle}\delta\epsilon
+\left(\hat{w}-\delta\varepsilon\right)
|\partial_\varepsilon\phi\rangle\delta\epsilon
\\
&+&
\underbrace{\left(\hat{h}_0-\varepsilon_0\right)
|\phi_{hom}\rangle}_{=0}\alpha
+\left(\hat{w}-\delta\varepsilon\right)
|\phi_{hom}\rangle\alpha
\\
&=&
\left(\hat{w}-\delta\varepsilon\right)
\Bigl(|\phi'\rangle
+|\partial_\varepsilon\phi\rangle\delta\epsilon
+|\phi_{hom}\rangle\alpha\Bigr)
\end{eqnarray*}
We can use various boundary conditions: If we set
$\delta\varepsilon=0$ then we obtain a fixed energy solution at
$\epsilon=\varepsilon_0$. 


On the other hand, we can also adjust $\delta\varepsilon$ to enforce
boundary condition at some outer radius. This boundary condition could
be that the value at some radius vanishes, or it could be that the
solution has a specified logarithmic derivative
$D=\partial_r\phi/\phi$ at some radius.

Before we start imposing the boundary conditions we decompose the
logarithmic derivative into a value $\phi_0$ and a derivative
$\partial_r\phi$ at the given radius. These values do not need to be
extracted from the initial wave function $\phi_0$ used above. We work
with value and derivative in order to be able to deal with infinite
logarithmic derivatives, which correspond to the particyularly
interesting boundary condition of a hard box.
\begin{eqnarray*}
\frac{\partial_r\phi}{\phi}&\stackrel{!}{=}&\frac{\partial_r\phi_0}{\phi_0}
\\
\frac{\partial_r\phi_0+\partial_r\phi\prime
+\partial_r\partial_\epsilon\phi\delta\epsilon
+\partial_r\phi_{hom}\alpha}
{\phi_0+\phi\prime+\partial_\epsilon\phi\delta\epsilon+\phi_{hom}\alpha}
&=&\frac{\partial_r\phi_0}{\phi_0}
\\
\phi_0\partial_r\phi_0+\phi_0\partial_r\phi\prime
+\phi_0\partial_r\partial_\epsilon\phi\delta\epsilon
+\phi_0\partial_r\phi_{hom}\alpha
&=&
\phi_0\partial_r\phi_0+\phi\prime\partial_r\phi_0
+\partial_\epsilon\phi\partial_r\phi_0\delta\epsilon+\phi_{hom}\partial_r\phi_0 \alpha
\end{eqnarray*}
With the definition of the \textbf{Wronskian}\index{Wronskian}
\begin{eqnarray*}
W[f,g]\defas f\partial_r g-g\partial_r f
\end{eqnarray*}
we obtain
\begin{eqnarray*}
0&=&W[\phi_0,\phi_0+\phi']
+W[\phi_0,\partial_\epsilon\phi]\delta \epsilon
+W[\phi_0,\phi_{hom}]\alpha
\\
\delta\epsilon&=&
-\frac{W[\phi_0,\phi_0+\phi']}{W[\phi_0,\partial_\epsilon\phi]}
-\frac{W[\phi_0,\phi_{hom}]}{W[\phi_0,\partial_\epsilon\phi]}\alpha
\end{eqnarray*}
It is important to resolve the equation for $\delta\epsilon$ because
both $W[\phi_0,\phi_0+\phi']$ and $W[\phi_0,\phi_{hom}]$ approach zero
upon convergence, which would create a divide-by-zero.

Thus the solution can be written as
\begin{eqnarray*}
|\phi\rangle&=&
\Bigl(|\phi_0\rangle+|\phi'\rangle-|\partial_\epsilon\phi\rangle
\frac{W[\phi_0,\phi_0+\phi']}{W[\phi_0,\partial_\epsilon\phi]}\Bigr)
+\Bigl(|\phi_{hom}\rangle-|\partial_\epsilon\phi\rangle
\frac{W[\phi_0,\phi_{hom}]}{W[\phi_0,\partial_\epsilon\phi]}\Bigr)\alpha
\end{eqnarray*}
The boundary condition at the outer boundary is fulfilled for any
value of $\alpha$. The value of $\alpha$ is then determined such that
the wave function starts with the lowest possible order. The
homogeneous solution always starts with $r^\ell$. Thus we can only
remove that contribution from the inhomogeneous solution.


%================================================================
\subsubsection{Relation to nodeless wave functions}
%================================================================
We can look upon the nodeless construction in the following way: For
each wave function we superimpose the wave functions with less nodes,
so that the wave function starts with the lowest possible power.

For our normal Hamiltonian, this implies that the node-less wave
function starts as $r^{\ell+2n}$. The Knotensatz garantees that these
wave functions are also nodeless (not proven!).

With the Fock term, we can still form wave functions that begin like
$r^{\ell+2n}$. We assume here that the radial part of a wave function
only contains even orders in its power series expansion, which is a
consequence of inversion symmetry (weak argument!). However it is not
clear from this argument if the resulting state is also nodeless.





%================================================================
\subsubsection{Relation to Greens functions}
%================================================================
For the sake of completeness let me show here the relation in terms of
Green's functions. This section does not produce funcamentally new
insights.

First we define a full Green's function $\hat{G}(\varepsilon)$ and an
approximate Green's function $\hat{G}_0(\varepsilon)$ as follows:
\begin{eqnarray*}
\left(\hat{h}_0+\hat{w}-\varepsilon\right)\hat{G}(\varepsilon)&=&\hat{1}
\\
\left(\hat{h}_0-\varepsilon\right)\hat{G}_0(\varepsilon)&=&\hat{1}
\end{eqnarray*}

Then we express the full Green's function by the approximate one.
\begin{eqnarray*}
(\hat{G}_0^{-1}+\hat{w})\hat{G}&=&\hat{1}
\\
(1+\hat{G}_0\hat{w})\hat{G}&=&\hat{G}_0
\\
\hat{G}&=&\hat{G}_0-\hat{G}_0\hat{w}\hat{G}
\\
\hat{G}^{(n+1)}&=&\hat{G}_0-\hat{G}_0\hat{w}\hat{G}^{(n)}
\end{eqnarray*}
Here we introduced a series of Green's function, which converges to
the full Green's function, if it converges.

The inhomogeneous Schr\"odinger equation has the form
\begin{eqnarray*}
|\phi\rangle&=&\hat{G}|I\rangle
\\
|\phi^{(n+1)}\rangle&=&\hat{G}^{(n+1)}|I\rangle
\\
&=&\left(\hat{G}_0-\hat{G}_0\hat{w}\hat{G}^{(n)}\right)|I\rangle
\\
&=&\hat{G}_0|I\rangle-\hat{G}_0\hat{w}|\phi^{(n)}\rangle
\\
\hat{G}_0^{-1}
\underbrace{\left(|\phi^{(n+1)}\rangle-|\phi^{(n)}\rangle\right)
}_{|\delta\phi\rangle}
&=&
|I\rangle-\hat{w}|\phi^{(n)}\rangle
+\hat{G}_0^{-1}|\phi^{(n)}\rangle
=
-\Bigl[\left(\hat{G}_0^{-1}+\hat{w}\right)|\phi^{(n)}\rangle
-|I\rangle\Bigr]
\\
(\hat{h}_0-\varepsilon)|\delta\phi\rangle&=&-
\Bigl[\left(\hat{h}_0+\hat{w}-\varepsilon\right)|\phi^{(n)}\rangle
-|I\rangle\Bigr]
\end{eqnarray*}




%====================================================================
\subsection{Apply Fock potential to a function}
%====================================================================
\petertt{The Fock potential is not yet generalized to include the
  small component nor spin-orbit coupling.}



The Fock operator can be written in the form
\begin{eqnarray*}
V_{X}(\vec{r},\sigma,\vec{r'},\sigma')
&=&-\sum_{j=1}^N \frac{e^2 \phi_j(\vec{r},\sigma)\phi^*_j(\vec{r'},\sigma')}
{4\pi\epsilon_0|\vec{r}-\vec{r'}|}
\end{eqnarray*}
Here $\phi_j(\vec{r},\sigma)$ are spin
orbitals. $\sigma\in\{\uparrow,\downarrow\}$. Usually, the wave
functions have nonzero elements either for spin up or for
spin-down components, but not both.

If there are partially filled angular momentum shells, 
the occupied and the unoccupied wave functions in the same shell
have different shape and energies. Because only the occupied states
determine energy and density, we use only the equations for the
occupied states. 

Now we want to apply this non-local potential to a function
$f(\vec{r},\sigma)=f(|\vec{r}|)Y_{L_f}(\vec{r})\delta_{\sigma,\sigma_f}$
with defined angular momentum and $s_z$ character.  Thus
\begin{eqnarray*}
f(\vec{r},\sigma)&=&\delta_{\sigma,\sigma_f}f(|\vec{r}|)Y_{L_f}(\vec{r})
\\
\phi_j(\vec{r},\sigma)&=&\delta_{\sigma,\sigma_j}\phi_j(|\vec{r}|)Y_{L_j}(\vec{r})
\\
g(\vec{r},\sigma)&=&\delta_{\sigma,\sigma_f}f(|\vec{r}|)Y_{L_f}(\vec{r})
\end{eqnarray*}
where it needts to be confirmed that $g$ can be represented by a
single angular momentum and spin channel.


\begin{eqnarray*}
g(\vec{r},\sigma)&=&\sum_{\sigma'}\int d^3r'\;
V_X(\vec{r},\sigma,\vec{r'},\sigma') f(\vec{r'},\sigma')
\\
&=&-\sum_{\sigma'}\int d^3r'\; 
\sum_{j=1}^N\frac{e^2 \phi_j(\vec{r},\sigma)\phi^*_j(\vec{r'},\sigma')}
{4\pi\epsilon_0|\vec{r}-\vec{r'}|}
f(\vec{r'},\sigma')
\\
&=&
-\sum_{\sigma'}\sum_{j=1}^N
\underbrace{\phi_j(|\vec{r}|)Y_{L_j}(\vec{r})\delta_{\sigma,\sigma_j}
}_{\phi_j(\vec{r},\sigma)}
\int d^3r'\; \frac{e^2}
{4\pi\epsilon_0|\vec{r}-\vec{r'}|}
\underbrace{
\phi^*_j(|\vec{r'}|)Y^*_{L_j}(\vec{r'})\delta_{\sigma',\sigma_j}
}_{\phi^*_j(\vec{r'},\sigma')}
\underbrace{
f(|\vec{r'}|)Y_{L_f}(\vec{r'})\delta_{\sigma',\sigma_f}
}_{f(\vec{r'},\sigma')}
 \\
&=&
-\sum_{j=1}^N
\underbrace{
\underbrace{
\left[\sum_{\sigma'}\delta_{\sigma',\sigma_j}\delta_{\sigma',\sigma_f}\right]
}_{\delta_{\sigma_j,\sigma_f}}
\delta_{\sigma,\sigma_j}
}_{\delta_{\sigma_j,\sigma_f}\delta_{\sigma,\sigma_f}}
\phi_j(|\vec{r}|)Y_{L_j}(\vec{r})
\int d^3r'\; \frac{e^2}
{4\pi\epsilon_0|\vec{r}-\vec{r'}|}
\phi_j(|\vec{r'}|)
f(|\vec{r'}|)
\underbrace{
Y^*_{L_j}(\vec{r'})Y_{L_f}(\vec{r'})
}_{\sum_{L_\rho}C^*_{L_f,L_j,L_\rho}Y^*_{L_\rho}(\vec{r'})}
\\
&=&
-\delta_{\sigma,\sigma_f}\sum_{j=1}^N\delta_{\sigma_j,\sigma_f}
\phi_j(|\vec{r}|)Y_{L_j}(\vec{r})
\sum_{L_\rho}C^*_{L_f,L_j,L_\rho}
\int d^3r'\; \frac{e^2}
{4\pi\epsilon_0|\vec{r}-\vec{r'}|}
\phi^*_j(|\vec{r'}|)f(|\vec{r'}|)Y^*_{L_\rho}(\vec{r'})
\\
&=&
-\delta_{\sigma,\sigma_f}\sum_{j=1}^N\delta_{\sigma_j,\sigma_f}
\phi_j(|\vec{r}|)Y_{L_j}(\vec{r})
\sum_{L_\rho}C^*_{L_f,L_j,L_\rho} Y_{L_\rho}(\vec{r})
\\
&&\times
\underbrace{
\int d^3r'\;\Bigl[Y^*_{L_\rho}(\vec{r'})\frac{1}{|\vec{r'}|^2}\delta(|\vec{r}|-|\vec{r'}|)
\Bigl[\int d^3r''\; \frac{e^2}
{4\pi\epsilon_0|\vec{r'}-\vec{r''}|}
\phi_j(|\vec{r''}|)f(|\vec{r''}|)Y_{L_\rho}(\vec{r''})\Bigr]\Bigr]^*
}_{=:v_{L_\rho}^{(j,f)}(|\vec{r}|)}
\\
&=&
-\delta_{\sigma,\sigma_f}\sum_{j=1}^N\delta_{\sigma_j,\sigma_f}
\phi_j(|\vec{r}|)
\sum_{L_\rho}C^*_{L_f,L_j,L_\rho} 
\underbrace{Y_{L_j}(\vec{r})Y^*_{L_\rho}(\vec{r})}_{\sum_{L_g}C_{L_\rho,L_j,L_g}Y_{L_g}}
v_{L_\rho}^{(j,f)}(|\vec{r}|)
\\
&=&-\sum_{L_g}Y_{L_g}(\vec{r})\delta_{\sigma,\sigma_f}
\sum_{j=1}^N\delta_{\sigma_j,\sigma_f}
\sum_{L_\rho}C^*_{L_f,L_j,L_\rho} C_{L_\rho,L_j,L_g}
\phi_j^*(|\vec{r}|)v_{L_\rho}^{(j,f)}(|\vec{r}|)
\end{eqnarray*}
where $C_{L,L',L''}:=\int d\Omega\;
Y^*_L(\vec{r})Y_{L'}(\vec{r})Y_{L''}(\vec{r})$ is the Gaunt
coefficient, so that
\begin{eqnarray*}
Y^*_L(\vec{r})Y_{L'}(\vec{r})=\sum_{L''}C_{L,L',L''}Y_{L''}(\vec{r})
\end{eqnarray*}
and the potential is defined as
\begin{eqnarray*}
v_{L_\rho}^{j,f}(\vec{r})&=&\int d^3r'\;
\frac{e^2\phi_j(|\vec{r'}|)f(|\vec{r'}|)Y_{L_{\rho}(\vec{r'})}}
{4\pi\epsilon_0|\vec{r}-\vec{r'}|}
\end{eqnarray*}
The potential has a pure $L_\rho$ character and is obtained using
RADIAL\$POISSON.

Furthermore we used the identity
\begin{eqnarray*}
\int d^3r'\;  \Bigl[Y^*_L(\vec{r'})\frac{1}{|\vec{r'}|^2}
\delta(r-|\vec{r'}|)\Bigr]f(\vec{r})
&=&\int d^3r'\;  
\Bigl[Y^*_L(\vec{r'})\frac{1}{|\vec{r'}|^2}\delta(r-|\vec{r'}|)\Bigr]
\Bigl[\sum_{L'}f_{L'}(|\vec{r'}|)Y_{L'}(\vec{r'})\Bigr]
\\
&=&\sum_{L'}
\underbrace{\Bigl[\int d\Omega\; Y_{L'}(\vec{r'}) Y^*_L(\vec{r'})\Bigr]}_{\delta_{L,L'}}
\Bigl[\int dr'\;r'^2 f_L'(r')\frac{1}{{r'}^2}\delta(r-r')\Bigr]
\\
&=&\Bigl[\int dr'\;f_L(r')\delta(r-r')\Bigr]
=f_L(r)
\end{eqnarray*}


Now we exploit that the radial functions $\phi_j$ do not depend on
$m_j$ and $\sigma_j$, so that we can average over $m_j$ and
$\sigma_j$. Note that here we assume that we average over complete
angular momentm shells, which is not necessarily the case. Here
caution is required.
\begin{eqnarray*}
g(\vec{r},\sigma)
&=&-\sum_{L_g}Y_{L_g}(\vec{r})\delta_{\sigma,\sigma_f}
\sum_{j=1}^N
\sum_{\ell_\rho}
\left(\frac{1}{2(2\ell_j+1)}
\sum_{m_j}\sum_{m_\rho}C_{L_j,L_f,L_\rho} C_{L_j,L_g,L_\rho}\right)
\phi_j^*(|\vec{r}|)v_{L_\rho}^{(j,f)}(|\vec{r}|)
\end{eqnarray*}

The next step is only physically motivated and implies
\begin{eqnarray*}
\frac{4\pi}{ (2\ell_1+1)(2\ell_2+1)}
\sum_{m_1,m_2}C_{L_1,L_2,L_3} C_{L_1,L_2,L'_3}&=&K_{\ell_1,\ell_2,\ell_3}\delta_{L_3,L'_3}
\\
\qquad\text{with}\qquad K_{\ell_1,\ell_2,\ell_3}&=&
\frac{4\pi\sum_{m_1,m_2,m_3}C_{L_1,L_2,L_3}^2}
{(2\ell_1+1)(2\ell_2+1)(2\ell_3+1)}
\end{eqnarray*}
It has been tested empirically using the routine SPHERICAL\$TESTGAUNTREL.

\begin{eqnarray*}
g(\vec{r},\sigma)
&=&-\sum_{L_g}Y_{L_g}(\vec{r})\delta_{\sigma,\sigma_f}
\sum_{j=1}^N
\sum_{\ell_\rho}\frac{2\ell_\rho+1}{8\pi}K_{\ell_j,\ell_\rho,\ell_f}\delta_{L_f,L_g}
\phi_j^*(|\vec{r}|)v_{L_\rho}^{(j,f)}(|\vec{r}|)
\\
&=&\left[-\sum_{j=1}^N
\sum_{\ell_\rho}\frac{2\ell_\rho+1}{8\pi}K_{\ell_j,\ell_\rho,\ell_f}
\phi_j^*(|\vec{r}|)v_{L_\rho}^{(j,f)}(|\vec{r}|)\right]
Y_{L_g}(\vec{r})\delta_{\sigma,\sigma_f}
\\
&=&\left[-\sum_{j=1}^N
\sum_{\ell_\rho}\frac{2\ell_\rho+1}{8\pi}K_{\ell_j,\ell_\rho,\ell_f}
\phi_j^*(|\vec{r}|)v_{L_\rho}^{(j,f)}(|\vec{r}|)\right]
Y_{L_g}(\vec{r})\delta_{\sigma,\sigma_f}
\end{eqnarray*}
In the last step we exploited that the radial functions of
$|\phi_j\rangle$ do not depend on the spin index. Because ths sum is
always performed over complete shells including the spin multiplet,
the $\delta$-function has been replaced by a factor $\frac{1}{2}$.
This step needs to be reconsidered when spin orbit coupling is
introduced.

Thus we find, as expected from the rotational symmetry, that the
potential preserves angular momentum and spin and that
\begin{eqnarray*}
g_{L_f,\sigma_f}(|\vec{r}|)&=&
-\sum_{j=1}^N
\sum_{\ell_\rho}
\left(\frac{2\ell_\rho+1}{8\pi}K_{\ell_j,\ell_f,\ell_\rho}\right)
\phi_j^*(|\vec{r}|) v_{L_\rho}^{(j,f)}(|\vec{r}|)
\end{eqnarray*}

The recipe goes as follows 
\begin{enumerate}
\item First we determine
\begin{eqnarray*}
K(\ell_j,\ell_\rho,\ell_f)\defas
\frac{4\pi}{(2\ell_j+1)(2\ell_\rho+1)}\sum_{m_j,m_\rho}C_{L_j,L_\rho,L_f}^2
\end{eqnarray*}

\item Then we determine
\begin{eqnarray*}
v_{L_\rho}(|\vec{r}|)&=&
\int d^3r'\;\Bigl[Y^*_{L_\rho}(\vec{r'}\frac{1}{|\vec{r'}|^2}\delta(|\vec{r}|-|\vec{r'}|)\Bigr]
\Bigl[\int d^3r''\; \frac{e^2}
{4\pi\epsilon_0|\vec{r}-\vec{r'}|}
\phi_j(|\vec{r''}|)f(|\vec{r''}|)Y_{L_\rho}(\vec{r''})\Bigr]
\end{eqnarray*}

The routine radial\$poisson solves
\begin{eqnarray*}
v(\vec{r})&=&\int d^3r'\;\frac{e^2\rho(\vec{r'})}{4\pi\epsilon_0|\vec{r}-\vec{r'}|}
\\
\rho(\vec{r})&=&\rho_L(|\vec{r}|)Y_L(\vec{r})
\\
v(\vec{r})&=&v_L(|\vec{r}|)Y_L(\vec{r})
\\
v_L(\vec{r})
&=&\int d^3r'; \Bigl[Y^*_L(\vec{r})\frac{1}{|\vec{r'}^2}
\delta(|\vec{r}|-|\vec{r'}|)\Bigr]
v(\vec{r})
\\
&=&\int d^3r'; \Bigl[Y^*_L(\vec{r})\frac{1}{|\vec{r'}^2}
\delta(|\vec{r}|-|\vec{r'}|)\Bigr]
\int d^3r^{''}\;\frac{e^2\rho_L(|\vec{r''}|)Y_L(\vec{r''})}{4\pi\epsilon_0|\vec{r'}-\vec{r''}|}
\end{eqnarray*}

\item Finally we obtain the radial part of the result as
\begin{eqnarray*}
g_{L_f}(|\vec{r}|)&=&-\sum_{j}f_j\sum_{\ell_\rho}
\frac{(2\ell_\rho+1)}{8\pi}K_{\ell_j,\ell_\rho,\ell_f}
\phi_j(|\vec{r}|)v_{L_\rho}(|\vec{r}|)
\end{eqnarray*}
In our notation the
occupation $f_j$ is unity, because the sum over j runs over all
states.  The occupation $f_j$ has been introduced so that the sum over
j can also be considered a sum over all shells.
\end{enumerate}

%=====================================================================
\chapter{Code structure}
%=====================================================================
%=====================================================================
\section{Flowchart of the paw\_setups object}
%=====================================================================
\petertt{This is lifted from my notes and needs to be updated.}


\begin{enumerate}
\item collect input data
\begin{itemize}
\item AEZ: atomic number
\item ZV:  Number of valence electros
\item $r_{c,\ell}$,$\lambda_\ell$: Parameters for pseudo-partial-wave
  construction
\item $r_{c,small}$: The compensation density is proportional to
  $\e{-(r/r_{c,small})^2}$
\item $r_{c,big}=1/\sqrt{0.218}$. The extended compensation density is
  proportional to $\e{-(r/r_{c,big})^2}$ (currently hardwired,
  probably too small.)
\item $\lambda,r_c,f(r=0)$: Parameters for pseudopotential construction 
\item $\lambda,r_c,f(r=0)$: Parameters for pseudocore construction 
\item The radial grid is encoded in grid-id ``GID''
\item Atomic mass
\item PSG2,PSG4
\item NPRO number of partial waves per l-channel
\item LRHOX density is expanded up to maximum angular momentum LRHOX
\end{itemize}
\item AESCF performs an all-electron self-consistent DFT calculation: The
  boundary condition is a hard box with radius equal to the third-last
  radial grid point. The calculation considers only spherical
  densities and ignores spin-polarization. This is important so that the
  setup construction does not artificially break the symmetry of the
  environment.

  One obtains the potential $v(\vec{r})$ of the all-electron atom, the
  all-electron wave functions $|\psi_n\rangle$., the one-particle
  energies $\epsilon_n$

  The procedure is described in more detail in section
  \ref{sec:atomlibaescf}.

\item ISCATT is a variable that is zero for the occupied partial wave
  with the highest energy for a given angular momentum. ISCATT=-1
  identifies a semi-core state, and ISCATT=1 identifies a scattering
  state.
\item calculate all-electron core density $n^C(\vec{r})$.
\item calculate pseudo core density pscore $\tilde{n}^C(\vec{r})$ from
  the all-electron core density. The method is described in section
  \ref{sec:pseudiation}.
\item MAKEPARTIALWAVES
\begin{itemize}
\item \textbf{pseudo potential:} construct pseudopotential PSPOT from
  the all-electron potential in a hard box with radius ROUT, provided
  on input. The method is described in section~\ref{sec:pseudiation}.
%
\item \textbf{nodeless atomic wave functions:} construct nodeless wave
  functions UOFI.  Even for calculations with a Fock contribution,
  UOFI will be determined here only for the local potential and it
  will be updated later with the Fock potential.
\begin{eqnarray*}
\left[\hat{h}_{loc}-\epsilon_n\right]|u_n\rangle&=&|u_{n-1}\rangle
\\
u_n(0)=\partial_r u_n(0)&=&0\qquad\text{for}\qquad n>0
\end{eqnarray*}
%
\item \textbf{nodeless partial waves:} construct nodeless partial
  waves NLPHI.  The lowest partial wave for each $\ell$ will be
  constructed with a hard sphere potential at radius ROUT, just as the
  nodeless wave function constructed above.  The higher partial waves
  will be constructed with the same logarithmic derivative at RBND as
  the first partial wave for the same $\ell$.

  The lowest partial wave for each $\ell$ uses the highest core state
  as inhomogneity, while the higher partial waves use the next lower
  partial wave as inhomogeneity. Thus a sequence of nodeless partial
  waves in introduced. One potential disadvantage of this
  construction is that the inhomogeneity extends further out with each
  partial wave.

\begin{eqnarray*}
\left[\hat{h}_{loc}-\bar{\epsilon}_n\right]|\phi^{nl}_{n}\rangle&=&
|\phi^{nl}_{n-1}\rangle
\\
\hat{t}|\phi^{nl}_n\rangle
&=&|\phi^{nl}_{n-1}\rangle+(\bar{\epsilon}_n-v)|\phi^{nl}_n\rangle
\end{eqnarray*}
where $|\phi^{nl}_{-1}\rangle=|u_c\rangle$ is the nodeless wave
function of the highest core state. The energies $\bar{\epsilon}_n$
for the partial waves are EOFLN. (The energies of the atomic wave
functions are EOFI).

  (Using the parameter TSMALLBOX=T the boundary conditions can be
  changed so that all partial waves experience a hard sphere at
  RBND. This choice has the disadvantage that the lowest partial wave
  is chosen at a fairly high energy.)
%
\item \textbf{add Fock term to nodeless wave functions:} This step is
  only done for Fock contribution in the potential: Starting from the
  nodeless wave function obtained for the local potential only, the
  nodeless atomic wave functions are constructed with the Fock
  contribution.
  \begin{eqnarray*}
  \left(\hat{h}_{loc}-\epsilon_0\right)|\phi'\rangle&=&
  -\left(\hat{h}_{loc}+v_{nl}-\epsilon_0\right)|\phi_0\rangle+|g\rangle
\\
  \left(\hat{h}_{loc}-\epsilon'\right)|\dot{\phi}\rangle&=&|\phi_0\rangle
\\
  \left(\hat{h}_{loc}-\epsilon'\right)|\phi_{hom}\rangle&=&0
\\
|\phi\rangle&=&|\phi_0\rangle+|\phi'\rangle
+|\dot{\phi}\rangle\delta\epsilon+|\phi_{hom}\rangle\alpha
  \end{eqnarray*}
  The variables are adjusted to fulfill the boundary conditions.  The
  inhomogeneity is adapted as well.  The basic equation is derived
  later in Section~\ref{sec:gpt}.
%
\item \textbf{add Fock term to partial waves:} This step is only done
  for Fock contribution in the potential: Starting from the nodeless
  partial waves and the inhomogeneity constructed consistently with
  the Fock potential the nodeless partial waves are updated.
%
\item \textbf{rescale:} rescale wave functions and partial waves such
  that the first partial wave is normalized. Only one scale factor per
  $\ell$ is allowed!
%
\item For plotting purposes, we introduce factors
  $f^\phi_n=$\verb|PHISCALE| for the partial waves and
  $f^\psi_n=$\verb|PSISCALE| for the energy eigenstates, so that
  $|u_n\rangle f_n$. have about the same size.
   \begin{eqnarray*}
    f^{\psi}_n=\prod_{j=1}^n(\epsilon_j-\epsilon_n)
   \\
    f^{\phi}_\alpha=\prod_{j=c+1}^\alpha(\bar{\epsilon}_j-\bar{\epsilon}_\alpha)
  \end{eqnarray*}
   where the sum includs only states with the same angular momentum.
%
\item \textbf{node-reduced partial waves:} construct
  $|q_{c+1}(\epsilon_{n})\rangle$ functions, named \verb|QN|. Here
  $c$ is the index of the highest state treated explicitely as core
  state. Thus the index $c+1$ refers to the first wave function
  included in the valence shell. The functions
  $|q_{c+1}(\epsilon_{n+i})\rangle$ are not necessarily nodeless.  The
  number of nodes for the $|q_{c+1}(\epsilon)\rangle$ function is
  equal to the number of nodes of the corresponding all-electron
  partial wave $|\phi(\epsilon)\rangle$ minus the number of core
  states with the same angular momentum.

We use \eq{eq:qmofepsilonanandnodeless} which is repeated below:
\begin{eqnarray*}
|q_{c+1}(\epsilon_n)\rangle&\eqrel{eq:qmofepsilonanandnodeless}{=}&
\sum_{i=c+1}^n|\phi^{nl}_i\rangle
\prod_{j=c+1}^{i-1}(\bar{\epsilon}_n-\bar{\epsilon}_j)
\end{eqnarray*}
This implies that the $|q_{c+1}(\epsilon_{n})\rangle$ and $|\phi^{nl}_n\rangle$
functions are scaled such that their long-range tails differ.
The long range parts behave as
\begin{eqnarray*}
|q_{c+1}(\epsilon_{n})\rangle\leftrightarrow
|\phi^{nl}_{n}\rangle
\prod_{j=c+1}^{n-1}(\epsilon_{n}-\epsilon_j)
\end{eqnarray*}
Thus we introduce a factor
$qbyu_n=\prod_{j=c+1}^{n-1}\frac{1}{\bar{\epsilon}_{n}-\bar{\epsilon}_j}$.  


A matrix $\mat{T}$ is constructed that describes the transformation
from the nodeless partial waves to the $|q_{c+1}(\epsilon_n)\rangle$
functions.
\begin{eqnarray*}
|q_{c+1}(\epsilon_n)\rangle&=&\sum_m|\phi^{nl}_m\rangle T_{m,n}
\end{eqnarray*}
This matrix will later allow to perform the back transform.
%
\item \textbf{rescale nodeless partial waves:} The nodeless
  $|\phi^{nl}_n\rangle$ functions are now rescaled so that that their
  long-range-behavior matches that of
  $|q_{c+1}(\epsilon_{n+i})\rangle$. The scale factor is
  \verb|ubyq|$=1/T_{n,n}$,
  i.e. $|\phi^{nl}_n\rangle\leftarrow|\phi^{nl}_n\rangle T_{n,n}$.

  The scaling of $|q_{c+1}(\epsilon_n)\rangle$ has been adopted
  because the pseudization has to depend on the
  q-function. Pseudization to nodeless functions does not work over
  several bands.

  From now on we \textbf{must no more use} the relation
  $(\hat{h}-\epsilon_n)|\phi^{nl}_n\rangle=|\phi^{nl}_{n-1}\rangle$!

\item \textbf{all-electron partial waves:} Construct all-electron
  partial waves by mixing in the core states. We use the nodeless
  construction instead of reorthogonalization. (The two methods differ
  because the small component is ignored.)

  We use \eq{eq:fromqntophi} which has the form
  \begin{eqnarray*}
    |\phi(\epsilon_n)\rangle&=& |q_{c+1}(\epsilon_n)\rangle+\sum_{i=1}^{c}
    |u_i\rangle\prod_{j=i}^{c}\frac{1}{\bar{\epsilon}_n-\epsilon_j}
    \label{eq:fromqntophi}
  \end{eqnarray*}
%
\item \textbf{pseudo partial waves:} Construct pseudo partial waves
\begin{itemize}
\item Type HBS: The technique is analogous to the procedure of Hamann
  Bachelet Schl\"uter.
The equation 
\begin{eqnarray*}
\left[\frac{\vec{p}^2}{2m}+\tilde{v}
+ A\e{-\left(\frac{r}{r_{c,\alpha}}\right)^{\lambda_\alpha}}-\epsilon_\alpha\right]
|\tilde\phi_\alpha\rangle=0
\end{eqnarray*}
is solved iteratively with differing $A$ until the logaritmic
derivative and the number of nodes of the pseudo and true partial wave
are identical. The logarithmic derivative is taken at a radius beyond
which the following two conditions are fulfilled
\begin{eqnarray*}
\left|\frac{1}{\langle\vec{r}|q_n(\epsilon_\alpha)\rangle}
\langle\vec{r}|
\frac{p^2}{2m}+\tilde{v}-\epsilon_\alpha|q_n(\epsilon_\alpha)\rangle\right|
<10^{-5}
\\
\e{-\left(\frac{r}{r_{c,\alpha}}\right)^{\lambda_\alpha}}<10^{-8}
\end{eqnarray*}
Note that this does not require the all-electron and the pseudo
potential or partial waves to be identical! The partial waves may
differ by an admixture of the core wave functions.

Number of nodes and logarithmic derivative are encoded in the function
\begin{eqnarray*}
\alpha(\epsilon)\defas
\frac{1}{2}-\frac{1}{\pi}
\atan(\frac{\partial_r\phi(\epsilon,r)}{\phi(\epsilon,r)})+NN
\end{eqnarray*}
which I will name generalized phaseshift.  According to the Wigner
rule, a band would lie between an half-integer and an integer value of
this generalized phaseshift. 
\begin{eqnarray*}
\partial_r\phi=0&\qquad\Rightarrow\qquad&\alpha= NN+\frac{1}{2}\qquad\text{bond}
\\
\phi=0&\qquad\Rightarrow\qquad&\alpha=NN+1\qquad\text{antibond}
\end{eqnarray*}


\item Type Kerker:
\end{itemize}
\item construct bare projectors $\langle\bar{p}_\alpha|$:
\begin{eqnarray*}
|\bar{p}_\alpha\rangle
&=&\left[\frac{\vec{p}^2}{2m}+\tilde{v}-\epsilon_\alpha\right]
|\tilde{\phi}_\alpha\rangle
\end{eqnarray*}
This is a result of the closure relation that the PAW equations are
exactly fulfilled for the pseudo partial waves.
\item Biorthogonality
  $\langle\tilde{p}_\alpha|\tilde{\phi}_\beta\rangle=\delta_{\alpha,\beta}$:

The biorthogonality is enforced by a Gram-Schmidt-like procedure
\begin{eqnarray*}
|\tilde{\phi}'_\alpha\rangle&=&\sum_\beta |\tilde{\phi}_\beta\rangle A_{\beta\alpha}
\\
|\tilde{p}'_\alpha\rangle&=&\sum_\beta |\bar{p}_\beta\rangle B_{\beta,\alpha}
\end{eqnarray*}
so that 
\begin{eqnarray*}
\langle\tilde{p'}_\alpha|\tilde{\phi'}_\beta\rangle=\delta_{\alpha,\beta}
\end{eqnarray*}
The matrices $\mat{A}$ and $\mat{B}$ are triangular, i.e. $A_{i,j}=0$
for $i>j$ and similar for $\mat{B}$.

So-far, only the matrices $\mat{A}$ and $\mat{B}$ have been computed. The partial waves and
projector functions are not updated.  
Once the matrices $A$ and $B$ have been computed we cal
\begin{eqnarray*}
|\tilde{\phi}^{new}_\alpha\rangle&=&|\tilde{\phi}_\alpha\rangle
\\
|\tilde{p}^{new}_\alpha\rangle&=&\sum_\beta |\bar{p}_\beta\rangle C_{\beta,\alpha}
\qquad\text{with $\mat{C}=\mat{B}\mat{A}^{\dagger}$}
\end{eqnarray*}


Only the projector functions are transformed. The partial waves remain
unchanged and keep their physical meaning.

\item Determine 
\begin{eqnarray*}
dT_{\alpha,\beta}&=&
\langle\phi_\alpha|\frac{\vec{p}^2}{2m}|\phi_\beta\rangle
-\langle\tilde{\phi}_\alpha|\frac{\vec{p}^2}{2m}|\tilde{\phi}_\beta\rangle
\\
dO_{\alpha,\beta}&=&
\langle\phi_\alpha|\phi_\beta\rangle
-\langle\tilde{\phi}_\alpha|\tilde{\phi}_\beta\rangle
\\
dH_{\alpha,\beta}&=&
\langle\phi_\alpha|\frac{\vec{p}^2}{2m}+v|\phi_\beta\rangle
-\langle\tilde{\phi}_\alpha|\frac{\vec{p}^2}{2m}+\tilde{v}|\tilde{\phi}_\beta\rangle
\end{eqnarray*}
In practive we do not use the kinetic energy operator
$\hat{t}=\frac{\hat{\vec{p}}^2}{2m}$, but the expression
$\hat{t}|\phi_\alpha
\rangle=(\epsilon_\alpha-v)|\phi_\alpha\rangle$. There are two reasons
for it. Applying a differential operator to a function stored on a
grid, introduces numerical noise. The differential operator as used by
us is strictly consistent with the Schr\"odinger equation including
all numeric errors. Last but not least, our method automatically
incorporates relativistic effects in the PAW method.
\item construct scattering wave functions:

First the nodeless scattering wave function is constructed
\begin{eqnarray*}
(\hat{h}-\epsilon_\gamma)|u^{scatt}\rangle=|u_n\rangle
\end{eqnarray*}
The scattering wave function $|q^{scatt}_{c+1}\rangle$
and the pseudo version of the scattering wave function are set equal
to $|u^{scatt}\rangle$. The reason is that both must not include any
contribution from the head function. Remember that they do not obey
the equations of the corresponding energy derivative wave functions!
Then we project out the core wave function to obtain the all-electron
version of scattering wave function
\begin{eqnarray*}
|q_{c+1}^{scatt}\rangle&=&|u^{scatt}\rangle
\\
|\tilde{\phi}^{scatt}\rangle&=&|u^{scatt}\rangle
\\
|\phi^{scatt}\rangle&=&|u^{scatt}\rangle-\sum_{i=1}^c |\phi_i\rangle\langle\phi_i|u^{scatt}\rangle
\end{eqnarray*}

\item calculate pseudo density: 
\begin{itemize}
\item We determine the PAW bound states
  using the same boundary conditions as the all-electron calculation
  (typically a hard box with radius equal to the third grid point from
  the outside.)
\begin{eqnarray*}
\left[\frac{\vec{p}^2}{2m}+v-\epsilon_n
\right]|\psi_n\rangle&=&0
\\
\left[\frac{\vec{p}^2}{2m}+\tilde{v}-\epsilon_n
+\sum_{\alpha,\beta}|\tilde{p}_\alpha\Bigl(
dH_{\alpha,\beta}-\epsilon_n dO_{\alpha,\beta}\Bigr)
\langle\tilde{p}_\beta|\right]|\tilde\psi_n\rangle&=&0
\end{eqnarray*}
The energies are determined independently. A deviation of the PAW
bound energy from the original all-electron energy larger than
$10^{-2}$~H, will cause an error message.

\item Determine projections $\langle\tilde{p}_\alpha|\tilde\psi_n\rangle$.

\item Normalize the all-electron and pseudo wave functions so that
\begin{eqnarray*}
\langle\tilde{\psi}_n|\tilde{\psi}_n\rangle
+\sum_{\alpha,\beta}\langle\tilde{\psi}_n|\tilde{p}_\alpha\rangle dO_{\alpha,\beta}
\langle\tilde{p}_\beta|\tilde{\psi}_n\rangle &=&1
\\
\langle\psi_n|\psi_n\rangle&=&1
\end{eqnarray*}
The sign of the all-electron wave function is changed if it is
inconsistent with the corresponding pseudo wave function.

\item The densities are determined
\begin{eqnarray*}
n(\vec{r})&=&\sum_n f_n\psi^*_n(\vec{r})\psi_n(\vec{r})
\\
\tilde{n}(\vec{r})&=&\sum_n f_n\tilde{\psi}^*_n(\vec{r})\tilde{\psi}_n(\vec{r})
\end{eqnarray*}
\end{itemize}
\item unscreening: The potential $\bar{v}(\vec{r})$ is constructed sich that
\begin{eqnarray*}
\tilde{v}(\vec{r})=\bar{v}(\vec{r})+\int d^3r'\;
\frac{e^2\tilde{n}(\vec{r'})+e^2Z(\vec{r'})}{4\pi\epsilon_0|\vec{r}-\vec{r'}|}
+\mu_{xc}([\tilde{n}],\vec{r})
\end{eqnarray*}
\end{itemize}
\end{enumerate}




%======================================================================
\section{ATOMLIB\$AESCF}
\label{sec:atomlibaescf}
%======================================================================
This routine performs a selfconsistent all-electron calculation for a
spherical, non-spin-polarized all-electron atom.

The boundary conditions are chosen such that there is a node at ROUT,
which is currently set (outside the routine) to the third radial grid
point from the end.

The operation of the subroutine is directed by a text variable ``key''.
\begin{itemize}
\item NONREL or REL: specified a relativistic or non-relativistic
  calculation.
\item NONSO or SO: switches spin orbit coupling on or off. (the option
  SO is not implemented.)
\item START: initializes occupations, angular momenta, starting
  potential, etc.
\end{itemize}

The orbitals are filled in the sequence
\begin{center}
\begin{tabular}{|l|l|}
\hline
n & $\ell$\\
\hline
1 & 0 \\
2 & 0,1\\
3 & 0,1 \\
4 & 0,2,1 \\
5 & 0,2,1 \\
6 & 0,3,2,1 \\
7 & 0,3,2,1\\
\hline
\end{tabular}
\end{center}
for spin-orbit coupling each multiplet is divided into a $2\ell$
states with antiparallel spin and orbit and $2\ell+2$ parallel states.

In the end of the self-consistent calculations the states are ordered
according to increasing energues.

%======================================================================
\subsubsection{Dirac equation}
%======================================================================
The Dirac equation for the large component has the form
\begin{eqnarray*}
\left\lbrace
(1+D)\frac{\hat{p}^2}{2m}+V-\epsilon
-\frac{\hbar^2}{2m_0}[\partial_r,D]_-\partial_r
+\frac{[\partial_r,D]_-}{m_0|\vec{r}|}\vec{L}\vec{S}
\right\rbrace|\phi\rangle=0
\end{eqnarray*}
where $D$ is the measure for relativistic effects
\begin{eqnarray*}
D(r)=\frac{m_0}{M}-1=\frac{-1}{1+\frac{2m_0c^2}{\epsilon-V(\vec{r})}}
\end{eqnarray*}
with $M$ the relativistic mass. $D(\vec{r})$ is recalculated in each
step for the corresponding energy. For non-relativistic calculations
$D(r)$ is set to zero.

For a spherical potential we obtain the radial Dirac equation for the
large component.
\begin{eqnarray*}
\left\lbrace
(1+D)\left[-\frac{1}{2r}\partial_r^2 r+\frac{\ell(\ell+1)}{2r^2}\right]
-\frac{1}{2}D'\partial_r+\frac{X}{2r}D'+V-e\right\rbrace R(r)=0
\end{eqnarray*}
where $D'(r)=\partial_rD(r)$ and
\begin{eqnarray}
X=\left\lbrace\begin{array}{ll}
\ell &\qquad\textrm{for parallel spin and orbital angular momentum}\\
-\ell-1 &\qquad\textrm{for anti-parallel spin and orbital angular momentum}\\
0&\qquad\textrm{for a scalar relativistic equation}
	      \end{array}\right.
\end{eqnarray}

\textbf{Attention!} At this point the small component is neglected
both for the normalization and for the charge density.

%======================================================================
\subsubsection{Potential of the nucleus}
%======================================================================
The nucleus is considered as a homogeneously charged sphere. The
volume of the nucleus is proportional to the number of nucleons.  This
allows to relate the radius directly to the mass of the nucleus.\cite{cooper53_pr92_801,hofstadter56_rmp28_214}
\begin{eqnarray*}
r_{nuc}=\sqrt[3]{\frac{M}{u}}\cdot1.2\cdot 10^{15} m 
=\sqrt[3]{\frac{M}{m_e}}\cdot1.85635065215\cdot 10^{-6}  a_0
\end{eqnarray*}
where $M$ is the mass of the nucleus and
$u=\frac{1}{12}m(C^{12})$. 

The potential of the nucleus is therefore
\begin{eqnarray*}
v_{nuc}(r)=
\left\lbrace
\begin{array}{cc}
\frac{-Ze^2}{r_{nuc}}\left(\frac{3}{2}-\frac{1}{2}\frac{r^2}{r_{nuc}^2}\right)
&\textrm{for}\qquad r<r_{nuc}\\
\frac{-Ze^2}{r}
&\textrm{for}\qquad r>r_{nuc}
\end{array}\right.
\end{eqnarray*}

%======================================================================
\subsubsection{Potential: ATOMLIB\$BOXVOFRHO}
%======================================================================
Calculates the output potential for a given chargedensity. 

The integrations are performed only up to a selected radius RAD. In
order to do the interpolation properly, the density must have a zero
at the radius, and it must be specified for two grid points beyond
RAD.


First we determine Hartree energy and potential:

Determine total charge
\begin{eqnarray*}
Q=\int d^3r\;\rho(\vec{r})-Z
\end{eqnarray*}
and then determine 
\begin{eqnarray*}
v_H(\vec{r})&=&\left\lbrace
\begin{array}{ll}
v_{nuc}(\vec{r})+\int d^3r'\;
\frac{e^2\rho(\vec{r'})}{4\pi|\vec{r}-\vec{r'}|}
+\Delta_H &\qquad\text{for}\qquad r<RAD\\
\\
-\frac{Q}{|\vec{r}|} &\qquad\text{for}\qquad r>RAD\\
\end{array}\right.
\\
E_H&=&
\frac{1}{2}\int d^3r\; \frac{e^2\rho(\vec{r})\rho(\vec{r'})}
{4\pi|\vec{r}-\vec{r'}|}
+\int d^3r\; \rho(\vec{r})v_{nuc}(\vec{r})
\\
&=& \frac{1}{2}\int d^3r\; \rho(\vec{r})v_{H}(\vec{r})
+\frac{1}{2}\int d^3r\; \rho(\vec{r})v_{nuc}(\vec{r})
\end{eqnarray*}
The variable $\Delta_H$ is determined such that $v_{nuc}$ is
continuous at RAD. The first expression for the Hartree energy is one
that is easily recognized, while the second expression is the way it
is actually calculated.

Now we determine the exchange-correlation energy and potential:

The routine that determines exchange-correlation and potential and
energy takes the arguments: $\rho_t,\rho_s. (\vec{\nabla}\rho_t)^2
,(\vec{\nabla}\rho_s)^2,(\vec{\nabla}\rho_t)(\vec{\nabla}\rho_s)$.
Because the calculation does not include spin, only the arguments
$\rho_t$ and $(\vec{\nabla}\rho_t)^2$ will be required. Because the
density is spherical we can furthermore use
$(\vec{\nabla}\rho_t)^2=(\partial_r\rho_t)^2$.

\begin{eqnarray*}
E_{xc}&=&\int d^3r\; F(\rho,(\vec{\nabla}\rho)^2)
\\
dE_{xc}
&=&\int d^3r\; \left[
\frac{\partial F}{\partial\rho}d\rho
+\frac{\partial F}{\partial(\vec{\nabla}\rho)^2)}(2\vec{\nabla}\rho)
\vec{\nabla}d\rho\right]
\\
&=&\int_\Omega d^3r\; \left[
\frac{\partial F}{\partial\rho}d\rho
-\vec{\nabla}\left(\frac{\partial F}{\partial(\vec{\nabla}\rho)^2)}(2\vec{\nabla}\rho)\right)
d\rho
+
\vec{\nabla}\left(\frac{\partial F}{\partial(\vec{\nabla}\rho)^2)}(2\vec{\nabla}\rho)d\rho\right)
\right]
\\
&=&\int d^3r\; \theta_\Omega(\vec{r})\left[
 \frac{\partial F}{\partial\rho}d\rho
-\vec{\nabla}\left(\frac{\partial F}{\partial(\vec{\nabla}\rho)^2)}(2\vec{\nabla}\rho)\right)
d\rho\right]
-\int d^3r\;\Bigl(\vec{\nabla}\theta_\Omega(\vec{r})\Bigr)
\left(\frac{\partial F}{\partial(\vec{\nabla}\rho)^2)}
(2\vec{\nabla}\rho)d\rho\right)
\end{eqnarray*}
Here $\theta_\Omega(\vec{r})$ is a step function which is equal to one
within the integration volume $\Omega$ and zero outside. Its gradient
is a $\delta$ function on the surface of the integration volume
multiplies with an \textbf{inward-pointing} normal vector.  The delta
like contribution to the potential at the sphere surface is ignored.
The simple reason is that the density vanishes at the surface, and it
is hoped that $F$, and its derivative behave similarly.  The more
solid argument, which however is not very straightforward, is that the
variation of the density at the sphere surface vanishes. This implies
that the Lagrange parameter, the potential, is not needed at this
point, and that any potential would not contribute to energy
eigenvalues, for example.

The potential is determined as
\begin{eqnarray*}
v_{xc}(\vec{r})
&=& \frac{\partial F}{\partial\rho}
-\vec{\nabla}\left(\frac{\partial F}{\partial(\vec{\nabla}\rho)^2)}2\vec{\nabla}\rho\right)
\\
&=& \frac{\partial F}{\partial\rho}
- \vec{\nabla}\left(\frac{\partial F}{\partial(\vec{\nabla}\rho)^2)}
2\frac{\vec{r}}{|\vec{r}|}\partial_r\rho\right)
\\
&=& \frac{\partial F}{\partial\rho}
- \left[\vec{\nabla}\frac{\vec{r}}{|\vec{r}|}\right]
\left(\frac{\partial F}{\partial(\vec{\nabla}\rho)^2)}
2\partial_r\rho\right)
- \frac{\vec{r}}{|\vec{r}|}\vec{\nabla}
\left(\frac{\partial F}{\partial(\vec{\nabla}\rho)^2)}
2\partial_r\rho\right)
\\
&=& \frac{\partial F}{\partial\rho}
- \left[\frac{2}{r}\right]
\left(\frac{\partial F}{\partial(\vec{\nabla}\rho)^2)}
2\partial_r\rho\right)
- \partial_r
\left(\frac{\partial F}{\partial(\vec{\nabla}\rho)^2)}
2\partial_r\rho\right)
\end{eqnarray*}

The exchange correlation potential is set to zero if the density falls
below a minimum of $10^{-6}~a_0^{-3}$.



%=====================================================================
\section{Setups\_newpro}
%=====================================================================
%=====================================================================
\subsection{Input variables}
%=====================================================================
The major input parameters are:
\begin{center}
\begin{tabular}{|l|l|}
\hline
L    & main angular-momentum quantum number\\
SO   & spin-orbit allignment $\sgn(\vec{S}\vec{L})$ (SO$\in\{-1,0,1\}$)\\
ROUT & bound states are calculates in a box with radius ROUT\\
RC   & cutoff for pseudization of partial waves\\
ENU  & energy for Taylor expansion of partial waves\\
\hline
\end{tabular}
\end{center}

%=====================================================================
\subsection{Flow chart}
%=====================================================================
The flow of the subroutine is as follows:
\begin{enumerate}
\item nodeless core wave functions \verb|UCORE|
\item coefficients \verb|QN| for the expansion of node-reduced partial waves 
\item energy dertivative partial wave of highest partial wave \verb|QNDOT|
\item pseudo core wave functions \verb|PSCORE|
\item pseudo partial waves \verb|PSPHI| (without core tails)
\item all-electron partial waves \verb|AEPHI| by core orthogonalization
\item bare projector functions \verb|PRO|
\item bi-orthogonalization
\item matrix elements \verb|DTKIN| \verb|DOVER|
\end{enumerate}

%=====================================================================
\subsubsection{Pseudo core wave functions}
%=====================================================================
We construct the function
\begin{eqnarray}
f_1&=&r^\ell
\nonumber\\
f_2&=& r^{\ell+2}
\nonumber\\
f_3&=& r^{\ell+4}
\end{eqnarray}
to the nodeless core wave function so that value and derivative agree
at the pseudization radius.

\appendix
%=====================================================================
\chapter{Remarks}
%=====================================================================
\begin{itemize}
\item The small contribution introduces nodes for the nodeless wave
  functions that lie near the nucleus which must not be counted. It is
  a consequence of treating the small component. This is taken care
  off by changing \verb|schroedinger$phaseshift| so that nodes are
  counted starting with a minimum radius. Schade\cite{schade12_thesis}
  gives the minimum radus of 0.07~a$_0$ for the core states and
  0.09~a$_0$ for the valence states.
%
\item Zora avoids the small component.?? Scalar relativistic
  calculations should treat the small component.
%
\item Currently the pseudo partial waves do not contain a pseudo core
  contribution. The pseudo core contribution can introduce ghost
  states.  On the other hand the pseudo core part lets the
  all-electron and pseudo partial waves to deviate at larger
  distances. Does this affect introduce an effect between core states
  and exponentially increasing partial waves?
%
\item The Taylor and the Secant equation are closely related. The
  equations differ only by the value of the chosen energy which is
  $\epsilon_\nu$ in one case and $\epsilon_{n+j}$ in the other. Can
  this be exploited?
%
\item we need a criterion for the quality of the augmentation: The
  Taylor expansion of $|q_n(\epsilon)\rangle$ may have a radially
  dependent convergence radius $\epsilon_c(r)$. For a truncated Taylor
  expansion there is a radius where it is better to leave out a term
  than to include it. Similar problems occur for the secant
  construction usually employed in the PAW method.

  We could use something like
  \begin{eqnarray}
  Q(\epsilon)\defas\min_{\vec{c}}
\sum_n\Bigl|\langle f_n|\Bigl[\tilde{h}-\epsilon+
\sum_\alpha|\tilde{p}_\alpha\rangle(dH_{\alpha,\beta}-\epsilon dO_{\alpha,\beta})
\langle\tilde{p}_\beta|\Bigr]|q'_n(\vec{c},\epsilon)\rangle\Bigr|^2
  \end{eqnarray}
  where $|f_n\rangle$ is some orthonormal basisset and
  $q'_n(\vec{c},\epsilon)\rangle$ is some kind of expansion for the
  partial waves with coefficients $\vec{c}$.
%
\item I need a section about the finite nuclear size
%
\item The fock contribution is not yet included in the new version.
%
\item for a semi-core state one should include a bound state for
  semi-core and for the valence state. The equations are very similar
  so that formulations can be integrated well. 
%
\end{itemize}




%=====================================================================
\chapter{Useful formulas}
%=====================================================================
\begin{eqnarray}
(\vec{\sigma}\vec{a})(\vec{\sigma}\vec{b})
&=&\vec{a}\vec{b}+i\vec{\sigma}(\vec{a}\times\vec{b})
\\
\vec{r}\vec{\sigma}\chi_{\kappa,j_z}&=&-|\vec{r}|\chi_{-\kappa,j_z}
\\
\vec{S}\vec{p}R(|\vec{r}|)\chi_{\kappa,j_z}
&=&\frac{i\hbar^2}{2}\Bigl[\partial_r+\frac{\kappa+1}{|\vec{r}|}\Bigr]
R(|\vec{r}|)\chi_{-\kappa,j_z}
\\
1+D&=&\frac{1}{1+\frac{\epsilon-v}{2m_0c^2}}
\\
\kappa(\ell,so)&=&-1+so\cdot\Bigl(\ell+\frac{so-1}{2}\Bigr)
=
\begin{cases}
-\ell-1&\text{for $\vec{L}\vec{S}\ge0$ .i.e. $so=1$ }\\
\ell&\text{for $\vec{L}\vec{S}<0$ .i.e. $so=-1$}\\
-1&\text{for $\vec{L}\vec{S}=0$ .i.e. $so=0$}\\
\end{cases}
\end{eqnarray}

%=====================================================================
\chapter{Taylor expansion of node-reduced partial waves}
\label{app:tayloraephi} 
%=====================================================================
Here we derive the coefficients $c_{m,j}$ used to determine the Taylor
coeffcieients \eq{eq:taylorexpansioncoefficientsaephi} of the
all-electron partial waves from the nodeless core wave functions and
the Taylor expansion coefficients \eq{eq:taylorexpansioncoefficientsqn}
of the node-reduced wave functions.

\begin{eqnarray}
|\phi_n^{(j)}(\epsilon_\nu)\rangle
&\eqrel{eq:taylorexpansioncoefficientsaephi}{=}&
\frac{(-1)^j}{j!}\left.\partial_\epsilon^{j}\right|_{\epsilon_\nu}
\biggl[|\phi_n(\epsilon)\rangle
\frac{1}{\prod_{k=1}^{n-1}(\epsilon_j-\epsilon)}\biggr]
\nonumber\\
&\eqrel{eq:aephifromnodereduced}{=}&
\frac{(-1)^j}{j!}\left.\partial_\epsilon^{j}\right|_{\epsilon_\nu}
\biggl[
|q_n(\epsilon)\rangle
+\sum_{m=1}^{n-1}|u_m\rangle\prod_{j=m}^{n-1}\frac{1}{\epsilon_j-\epsilon}
\biggr]
\nonumber\\
&\eqrel{eq:taylorexpansioncoefficientsqn}{=}&
|q_n^{(j)}(\epsilon_\nu)\rangle
+\sum_{m=1}^{n-1}|u_m\rangle
\biggl(
\underbrace{\frac{(-1)^j}{j!}\left.\partial_\epsilon^{j}\right|_{\epsilon_\nu}
\biggl[\prod_{j=m}^{n-1}\frac{1}{\epsilon_j-\epsilon}
\biggr]}_{c_{m,j}}\Biggr)
\end{eqnarray}

It is our goal to work out the coeffcients $c_{m,j}$. To explore the
structure of the expressions let us evaluate the first two
derivatives of the product terms
\begin{eqnarray}
\underbrace{\left.\partial_\epsilon^{0}\right|_{\epsilon_\nu}
\biggl[\prod_{j=m}^{n-1}\frac{1}{\epsilon_j-\epsilon}\biggr]}_{b_{m,0}}
&=&
\underbrace{\biggl[\prod_{j=m}^{n-1}\frac{1}{\epsilon_j-\epsilon_\nu}\biggr]
}_{b_{m,0}}
\nonumber\\
\underbrace{\left.\partial_\epsilon^{1}\right|_{\epsilon_\nu}
\biggl[\prod_{j=m}^{n-1}\frac{1}{\epsilon_j-\epsilon}\biggr]}_{b_{m,1}}
&=&\underbrace{\biggl[\prod_{j=m}^{n-1}\frac{1}{\epsilon_j-\epsilon_\nu}\biggr]
}_{b_{m,0}}
\underbrace{\biggl[\sum_{k=m}^{n-1}\frac{1}{\epsilon_j-\epsilon_\nu}\biggr]
}_{a_{m,0}}
\label{eq:taylorphicm1}
\end{eqnarray}
Let us now introduce the new symbols
\begin{eqnarray}
b_{m,j}&\defas&\left.\partial_\epsilon^{j}\right|_{\epsilon_\nu}
\biggl[\prod_{j=m}^{n-1}\frac{1}{\epsilon_j-\epsilon}\biggr]
\nonumber\\
a_{m,j}&\defas&\left.\partial_\epsilon^{j}\right|_{\epsilon_\nu}
\biggl[\sum_{k=m}^{n-1}\frac{1}{\epsilon_j-\epsilon}\biggr]
\end{eqnarray}

Thus we obtain
\begin{eqnarray}
b_{m,1}&=&b_{m,0}a_{m,0}
\nonumber\\
b_{m,2}&=&b_{m,1}a_{m,0}+b_{m,0}a_{m,1}
\nonumber\\
b_{m,2}&=&b_{m,2}a_{m,0}+2b_{m,1}a_{m,1}+b_{m,0}a_{m,2}
\nonumber\\
b_{m,j}&=&\sum_{k=0}^{j-1}\binom{j-1}{k}b_{m,j-k-1}a_{m,k}
\nonumber\\
\underbrace{\frac{(-1)^j}{j!}b_{m,j}}_{c_{m,j}}
&=&
\sum_{k=0}^{j-1}
\frac{(-1)^j}{j!}\underbrace{\frac{j-1)!}{(j-k-1)!k!}}
_{\binom{j-1}{k}}b_{m,j-k-1}a_{m,k}
\nonumber\\
&=&
\sum_{k=0}^{j-1}
\frac{+1}{j}
\underbrace{\Bigl[\frac{(-1)^{j-k-1}}{(j-k-1)!}b_{m,j-k-1}\Bigr]}_{c_{m,j-k-1}}
\Bigl[-\frac{(-1)^k}{k!}a_{m,k}\Bigr]
\end{eqnarray}

The coefficients $a_{m,k}$ are obtained as follows
\begin{eqnarray}
a_{m,j}&=&\left.\partial_\epsilon^{j}\right|_{\epsilon_\nu}
\biggl[\sum_{k=m}^{n-1}\frac{1}{\epsilon_j-\epsilon}\biggr]
=\biggl[\sum_{k=m}^{n-1}\frac{j!}{(\epsilon_j-\epsilon_\nu)^{j+1}}\biggr]
\end{eqnarray}
so that
\begin{eqnarray}
-\frac{(-1)^j}{j!}a_{m,j}&=&
+\sum_{k=m}^{n-1}\frac{1}{(\epsilon_\nu-\epsilon_j)^{j+1}}
\end{eqnarray}

Thus we obtain the following recursive set of equations
\begin{eqnarray}
c_{m,0}&=&\prod_{j=m}^{n-1}\frac{1}{\epsilon_j-\epsilon}
\\
c_{m,j}&=&\frac{1}{j}\sum_{k=0}^{j-1}  c_{m,j-k-1}d_{m,k}
\qquad\text{for $j=1,\ldots,\infty$}
\nonumber\\
d_{m,j}&=&
\biggl[\sum_{k=m}^{n-1}\frac{1}{(\epsilon_j-\epsilon_\nu)^{j+1}}\biggr]
\qquad\text{for $j=0,\ldots,\infty$}
\end{eqnarray}
with which we can evaluate the true wave functions in the form
\begin{eqnarray}
|\phi_n^{(j)}(\epsilon_\nu)\rangle
&=&
|q_n^{(j)}(\epsilon_\nu)\rangle+\sum_{m=1}^{n-1}|u_m\rangle c_{j,m}
\end{eqnarray}

%=====================================================================
\chapter{Derivation of inhomogeneity for the radial Dirac equation}
\label{app:inhomraddirac}
%=====================================================================
Here I make the derivation of the inhomogenity of the radial Dirac
equation very explicit so that one can follow all the signs. This is
because I changed the sign convention of the nodeless construction,
which may cause a mixup with earlier results.

\begin{eqnarray}
&&-|u_{n-1}\rangle-(\vec{S}\vec{p})\frac{1+D}{\hbar m_0c}|v_{n-1}\rangle
\nonumber\\
&&
-g_{\kappa,j_z}^{n-1}-\frac{i\hbar^2}{2}
\Bigl(\partial_r+\frac{1-\kappa}{|\vec{r}|}\Bigr)\frac{1+D}{\hbar m_0c} 
\Bigl(if_{-\kappa,j_z}^{n-1}\Bigr)
\nonumber\\
&&\Bigl(-g_{\kappa,j_z}^{n-1}\Bigr)-\frac{\hbar}{c}
\Bigl(\partial_r+\frac{1-\kappa}{|\vec{r}|}\Bigr)\frac{1+D}{2m_0} 
\Bigl(-f_{-\kappa,j_z}^{n-1}\Bigr)
\end{eqnarray}

\begin{eqnarray}
|v_n\rangle&=&\frac{1+D}{2m}
\Bigl[\frac{2}{\hbar c}(\vec{S}\vec{p})|u_n\rangle+\frac{1}{c}v_n\rangle
\nonumber\\
if_{-\kappa,j_z}^{n}&=&\frac{1+D}{2m}
\Bigl[\frac{2}{\hbar c}\frac{i\hbar^2}{2}
\Bigl[\partial_r+\frac{1+\kappa|}{|\vec{r}|}\Bigr]
g_{\kappa,j_z}^{(n)}
+\frac{1}{c^2}\Bigl(if_{-\kappa,j_z}^{(n)}\Bigr)\Bigr]
\nonumber\\
f_{-\kappa,j_z}^{n}&=&\frac{1+D}{2m}
\Bigl[\frac{\hbar}{c}
\Bigl[\partial_r+\frac{1+\kappa|}{|\vec{r}|}\Bigr]
g_{\kappa,j_z}^{(n)}
+\frac{1}{c^2}\Bigl(f_{-\kappa,j_z}^{(n)}\Bigr)\Bigr]
\nonumber\\
\end{eqnarray}

%======================================================================
\chapter{Parameters for the Setup construction}
%======================================================================
%======================================================================
\subsection{Parameters for the HBS-type construction}
%======================================================================
\begin{verbatim}
  !SETUP ID='CA_HBS' EL='CA'  ZV=2   
     RBOX/RCOV=2.0  RCSM/RCOV=0.25     
     TYPE='HBS' 
     RCL/RCOV=0.75 0.75 0.75 LAMBDA=6. 6. 6. 
     !GRID DMIN=5.E-6 DMAX=0.1 RMAX=7.2 !END
     !POT   POW=3. VAL0=-1.2 RC/RCOV=0.67 !END
     !CORE  POW=2. VAL0= 0.1 RC/RCOV=0.67 !END
  !END

  !SETUP ID='CA_SC_HBS' EL='CA'  ZV=10.  
     RBOX/RCOV=2. RCSM/RCOV=0.25 
     TYPE='HBS'
     RCL/RCOV=0.5 0.5 0.5 0.5  LAMBDA=6. 6. 6. 6.
     !GRID DMIN=5.E-6 DMAX=0.1 RMAX=7. !END
     !POT   POW=3. VAL0=-2.2 RC/RCOV=0.5 !END
     !CORE  POW=2. VAL0= 0.1 RC/RCOV=0.5 !END
  !END
\end{verbatim}

For valence-only setups use the following set for the partial waves
$r_c=0.75 r_{cov}$, $\lambda=6$.

For semi-core setups use the following set ofr the partial waves
$r_c=0.55 r_{cov}$, $\lambda=6$.

The decay parameter for the compensation charge density should be set
to $0.25 r_{cov}$.

It is beneficial if the pseudopotential follows the all-electron
potential inward further than the covalent radius.

Usually, we do not specify the parameter VAL0 for the potential. This
however causes problems for transition metals, where we obtain ghost
states.


\printindex
\bibliographystyle{unsrtnat}
\bibliography{all}
\end{document}  



