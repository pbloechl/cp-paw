\documentclass[11pt,a4paper]{report}
%%%%%%%%%%%%%%%%%%%%%%%%%%%%%%%%%%%%%%%%%%%%%%%%%%%%%%%%%%%%%%%%%%%%%
%%                                                                 %%
%%    Header file for the Phi-S-X Series                           %%
%%                                                                 %%
%%    german version header_gm.tex is derived from header.tex      %%
%%    by uncommenting the line ``\setboolean{german}{true}'' below %%
%%                                                                 %%
%%    Never edit the german version! all changes must be done      %%
%%    in the english version header.tex                            %%
%%                                                                 %%
%%%%%%%%%%%%%%%%%%%%%%%%%%%%%%%%%%%%%%%%%%%%%%%%%%%%%%%%%%%%%%%%%%%%%
%%%%%%%%%%%%%%%%%%%%%%%%%%%%%%%%%%%%%%%%%%%%%%%%%%%%%%%%%%%%%%%%%%%%%
%%                                                                 %%
%%    Header file for the Phi-S-X Series                           %%
%%                                                                 %%
%%    german version header_gm.tex is derived from header.tex      %%
%%    by uncommenting the line ``\setboolean{german}{true}'' below %%
%%                                                                 %%
%%    Never edit the german version! all changes must be done      %%
%%    in the english version header.tex                            %%
%%                                                                 %%
%%%%%%%%%%%%%%%%%%%%%%%%%%%%%%%%%%%%%%%%%%%%%%%%%%%%%%%%%%%%%%%%%%%%%
%====================================================================
%-- define flag for language adaptations
\usepackage{ifthen}   % allows to select only certain text
\provideboolean{german}
\setboolean{german}{false}
%\setboolean{german}{true}  % uncomment this line for german editions
%====================================================================
%
% Textschriftart: Computer modern Bright
% body:            CM-Bright 10pt
% section titles:  CM-Bright Bold
% formulas:        CM-Bright Math Oblique
%
\usepackage[standard-baselineskips]{cmbright}
\usepackage{cmbright}
\usepackage[T1]{fontenc}
\def\usedfonts{CM-Bright}
\usepackage{typearea}
%\typearea[current]{calc} % benutzt die aktuelle 
       % bindekorrektur (BCOR angabe als parameter in koma usepackage)
       % und berechnet satzspiegel neu
\typearea[current]{11} %fixed div value

\usepackage{textcomp} % special symbols
\usepackage{amsfonts} % special symols
                      % see ftp://ftp.ams.org/pub/tex/doc/amsfonts/amsfndoc.pdf
\usepackage{amssymb}  % CM-Bright provides the AMS symbols
\usepackage{exscale}  % allows to scale math expressions to big fonts, 
                      % e.g. \Huge
\usepackage{curves}
\usepackage{braket}
\usepackage{miller}     % miller indices
\usepackage{chemmacros} % http://www.mychemistry.eu/mychemistry/
\usepackage[numbers]{natbib}     % bibliography style
\usepackage{url}\urlstyle{tt}
\usepackage{float}
\usepackage{bm}       % provides the command \bm{} that makes bold math symbols
\usepackage{amsmath}
\usepackage{amsbsy}   % allows bold mathematical symbols
\usepackage{amscd}
 \usepackage{a4wide}  % it is better to use the ``geometry'' package
\usepackage{array}    % 
\usepackage{fancyhdr} %  defines pagestyle fancy
\usepackage{epsfig}   % include graphics with epsfig
\usepackage{graphicx} % includegraphics
\usepackage{epstopdf}
\usepackage{wrapfig}
\usepackage{fancybox} % allows shadow-boxes
\usepackage{color}    % allows to use color in the text
%\usepackage{eepic}
\usepackage{flafter}  % places picture next to its reference
\usepackage{makeidx}  % make an index
%\usepackage{MnSymbol}  % 
%\usepackage{marvosym}  % 
\usepackage{textcase}
\usepackage{ulem} % defines strikeout \sout{}; underline \uline{}
                  % double underline \uuline{}; wave underline \uwave{}
                  % cross out \xout{}
%
%==========================================================================
%==  page layout  =========================================================
%==========================================================================
% eqnarray environment: reduce with of space in place of each ``&''
\setlength\arraycolsep{1.4pt}
\pagestyle{fancy}
%\renewcommand{\chaptermark}[1]{\markboth{\thechapter\ #1}{}}
\renewcommand{\chaptermark}[1]{\markboth{\MakeUppercase{\thechapter\ #1}}{}}
\fancyhf{} 
\fancyhead[LE]{\textsc{\thepage}\qquad\textsc{\leftmark}}
\fancyhead[RO]{\textsc{\leftmark}\qquad\textsc{\thepage}}
\renewcommand{\headrulewidth}{0.5pt}
\renewcommand{\footrulewidth}{0pt} 
\addtolength{\headheight}{2.5pt}
\fancypagestyle{plain}{\fancyhead{}
   \renewcommand{\headrulewidth}{0pt}
   \fancyfoot[CO]{\bfseries\thepage}}

% Line spacing -----------------------------------------------------------
\newlength{\defbaselineskip}
\setlength{\defbaselineskip}{\baselineskip}
\newcommand{\setlinespacing}[1]%
           {\setlength{\baselineskip}{#1 \defbaselineskip}}
\newcommand{\doublespacing}{\setlength{\baselineskip}%
                           {2.0 \defbaselineskip}}
\newcommand{\singlespacing}{\setlength{\baselineskip}{\defbaselineskip}}

% Absatz einr\"ucken ------------------------------------------------------
%\setlength{\parindent}{0pt}
\setlength{\parskip}{2pt}
% -------------------------------------------------------------------------
\ifthenelse{\boolean{german}}
  {\def\figurename{Abb.}}
  {\def\figurename{Fig.}}
%--------------------------------------------------------------------------
\renewcommand{\arraystretch}{1.15}  % skaliert den Zeilen abstand in der 
    % tabular und array umgebung
%
%==========================================================================
%==  boxes etc ============================================================
%==========================================================================
%== minipage in a shadowbox ===============================================
\newenvironment{myshadowminipage}[1]%
  {\par\noindent\begin{Sbox}\begin{minipage}{\linewidth}\vspace{0.1cm}\begin{center}\uppercase{#1}\end{center}}%
  {\vspace{0.1cm}\end{minipage}\end{Sbox}\shadowbox{\TheSbox}}
%
%== minipage in a framedbox ===============================================
\newenvironment{myframedminipage}%
  {\par\noindent\begin{Sbox}\begin{minipage}\linewidth\vspace{0.1cm}}%
  {\vspace{0.1cm}\end{minipage}\end{Sbox}\fbox{\TheSbox}}
%
\newcommand{\myshadowbox}[1]{\noindent\shadowbox{\parbox{\linewidth}{\smallskip #1\smallskip}}}
\newcommand{\myfbox}[1]{\noindent\fbox{\parbox{\linewidth}{\smallskip #1\smallskip}}\medskip}
%== minipage in a framedbox ===============================================
\newtheorem{defi}{Definition}[chapter]
\newenvironment{definition}[1]%
  {\par\noindent\begin{Sbox}\begin{minipage}{\linewidth}\vspace{0.1cm}\begin{defi}\uppercase{#1}\\\vspace{0.1cm}}%
  {\vspace{0.1cm}\end{defi}\end{minipage}\end{Sbox}\shadowbox{\TheSbox}}
%
%=========================================================================
% color used to point out information to the teacher
\definecolor{highlight}{rgb}{1.0,0.7,0.}
\newcommand{\Special}[1]{\textbf{\textcolor{highlight}{#1}}}
%=========================================================================
%  switch certain parts on and off. uses ifthen package
\newboolean{teacher}\setboolean{teacher}{false}
% this parameter can be changed in the manuscript again
\setboolean{teacher}{true} %private version if true!
\newcommand{\teacheronly}[1]{\ifthenelse{\boolean{teacher}}{#1\hfill\\ }}
\newcommand{\editor}[1]{\textcolor{blue}{\texttt{Editor: #1}}}
\newcommand{\MARK}[1]{\textcolor{blue}{#1}} 
\newcommand{\RED}[1]{\textcolor{red}{#1}} 
%
%==========================================================================
%==  define new symbols                                                 ===
%==========================================================================
% define \stat (stationary state) as an operator like \min
\DeclareMathOperator*{\stat}{stat}
\let\Vec=\mathbold   % cmbright.sty provides a bold/italic math alphabet
\let\Dot=\mathbold   % cmbright.sty provides a bold/italic math alphabet
\let\Ddot=\mathbold   % cmbright.sty provides a bold/italic math alphabet
%
\newcommand{\e}[1]{\mathrm{e}^{#1}}% exponential function
\renewcommand{\Re}{\mathrm{Re}}    % real part
\renewcommand{\Im}{\mathrm{Im}}    % imaginary part
\newcommand{\lagr}{\ell}           % Lagrange dichte
\newcommand{\Lagr}{\mathcal{L}}    % Lagrange Funktion
\newcommand{\erf}{{\rm erf}}       %
\newcommand{\atan}{{\rm atan}}     % arcus tangens
\newcommand{\mat}[1]{\bm{#1}}  % Matrix
\newcommand{\gmat}[1]{{\boldsymbol #1}}  % Matrix(symbol)
\newcommand{\defas}{\stackrel{\text{def}}{=}}  %  is defined as
\ifthenelse{\boolean{german}}
  {\newcommand{\rot}{{\rm\bf rot}}}    % curl
  {\newcommand{\rot}{{\rm\bf curl}}}   % curl
\newcommand{\sgn}{{\rm sgn}}       % sign
\ifthenelse{\boolean{german}}
   {\newcommand{\Tr}{\mathrm{Sp}}}      % trace
   {\newcommand{\Tr}{\mathrm{Tr}}}      % trace
\ifthenelse{\boolean{german}}
   {\newcommand{\grmn}[2]{\footnote{``#2'' hei{\ss}t in englisch ``#1''}}}
   {\newcommand{\grmn}[2]{\footnote{``#1'' translates as ``#2'' into German}}}
% define the equation reference
\ifthenelse{\boolean{german}}
   {\newcommand{\eq}[1]{\text{Gl.}~\ref{#1}}}
   {\newcommand{\eq}[1]{\text{Eq.}~\ref{#1}}}
% define a relation with an equation number ontop
\newcommand{\eqrel}[2]{\stackrel{\eq{#1}}{#2}}
\newcommand{\zero}{\varnothing}
%\newcommand{\ket}[1]{|#1\rangle} % contained in package braket
\newcommand{\sumint}{\int\hspace{-15pt}\sum}
\newcommand{\marker}[1]{\textcolor{blue}{\emph{#1}}}
\renewcommand*{\dot}[1]{\overset{\mbox{\large\bfseries .}}{#1}}
\renewcommand*{\ddot}[1]{\overset{\mbox{\large\bfseries\hspace{+0.1ex}.\hspace{-0.1ex}.}}{#1}}
%
%==========================================================================
%==                                                                     ===
%==========================================================================
% Prevent figures from appearing on a page by themselves
% from http://dcwww.camd.dtu.dk/~schiotz/comp/LatexTips/LatexTips.html
\renewcommand{\topfraction}{0.85}
\renewcommand{\textfraction}{0.1}
\renewcommand{\floatpagefraction}{0.75}
%
%==========================================================================
%==                                                                     ===
%==========================================================================
\makeindex    % make index. uses makeidx package.

%== allow links between documents ============================================
\usepackage{xr}
\usepackage{xr-hyper}
%==  hyperref package (must be last package)
\usepackage[colorlinks=true]{hyperref} %specify this as last package
\hypersetup{citecolor=blue}
\hypersetup{menucolor=magenta}
\hypersetup{urlcolor=blue}      % 
\hypersetup{filecolor=green}    % file links
\hypersetup{linkcolor=magenta}  %table of contents
\hypersetup{pdfauthor={Peter E. Bl\"ochl}}
\hypersetup{pdfdisplaydoctitle=true}
\externaldocument[phisx1-]{/Users/ptpb/Tree/PhiSX/ClassicalMechanics/Book/cm-gm}
\externaldocument[phisx2-]{/Users/ptpb/Tree/PhiSX/Electrodynamics/Book/el-gm}
\externaldocument[phisx3-]{/Users/ptpb/Tree/PhiSX/QuantumMechanics/Book/qm}
\externaldocument[phisx4-]{/Users/ptpb/Tree/PhiSX/StatisticalMechanics/Book/sm}
\externaldocument[phisxqm2-]{/Users/ptpb/Tree/PhiSX/QuantumMechanicsII/Book/qm2}
\externaldocument[phisxsm2-]{/Users/ptpb/Tree/PhiSX/StatisticalMechanicsII/Book/sm2}
\externaldocument[phisxcb-]{/Users/ptpb/Tree/PhiSX/Chemicalbond/Book/cb}
% Example: Figure~PhiSX:Quantum
% Mechanics-\ref{phisx3-fig:doubleslitwave} on page
% \pageref{phisx3-fig:doubleslitwave}


\hypersetup{pdftitle=paw_brillouin}
\newcommand{\petertt}[1]{\textcolor{red}{\texttt{#1}}}
\makeindex
\begin{document}
\begin{titlepage}
\begin{center}
\vspace*{3.5cm}
{\huge \textbf{The SETUPS object of the CP-PAW code}}\\
\vspace{0.5cm}
{\large Peter E. Bl\"ochl}
\vspace{0.5cm} 
\end{center}

\vfill
\begin{center}
Copyright Peter E. Bl\"ochl; Sept.2, 2013-\today\\
{\small
Institute of Theoretical Physics;
Clausthal University of Technology;\\ 
D-38678 Clausthal Zellerfeld; Germany;\\
http://www.pt.tu-clausthal.de/atp/}
\end{center}
\end{titlepage}
\noindent            
\tableofcontents
%====================================================================
\chapter{Purpose and theoretical background}
%====================================================================
%====================================================================
\section{Nodeless constructiuon}
%====================================================================
%====================================================================
\subsection{Basic definitions}
%====================================================================
%====================================================================
\subsubsection{Node-less bound states}
%====================================================================
We define\cite{bloechl12_arxiv1210_5937} a sequence of
\textbf{node-less bound states}\index{node-less bound states}
$|u_n\rangle$
\begin{eqnarray}
(\hat{h}-\epsilon_n)|u_n\rangle=-|u_{n-1}\rangle
\label{eq:nodelesseq}
\end{eqnarray}
The first wave function, $|u_1\rangle$ is the lowest bound state of
the hamiltonian. The other ones are calculated by radially integrating
the Schr\"odinger equation outward, starting with vanishing value and
derivative, so that
\begin{eqnarray}
\lim_{\vec{r}\rightarrow0}|\frac{\langle\vec{r}|u_n\rangle}{|\vec{r}|^{\ell+1}}=0
\qquad\text{for $n>1$}
\end{eqnarray}


The node-less wave functions do not
have any nodes but look like Slater orbitals, in the sense that they
grow at the origin as a power and that they decay exponentially at
large distance. Their behavior at the origin
is\cite{bloechl12_arxiv1210_5937}
\begin{eqnarray}
u_{n}(\vec{r})
=\left(\frac{2m_e}{\hbar^2}\right)^{n-1}
\frac{(2\ell+1)!!}{(\ell-1+2n)!!(2n-2)!!}
|\vec{r}|^{\ell+2(n-1)}Y_{\ell,m}(\vec{r})
\end{eqnarray}
where $Y_{\ell,m}$ is a real spherical harmonics.

The n lowest node-less wave functions span the same Hilbert space as
the n lowest true wave functions. The all-electron wave functions can
be recovered from the node-less wave functions
via\cite{bloechl12_arxiv1210_5937}
\begin{eqnarray}
|\psi_n\rangle=\sum_{m=1}^n|u_m\rangle \prod_{j=1}^{m-1}(\epsilon_j-\epsilon_n)
=\biggl[\sum_{m=1}^n|u_m\rangle 
\prod_{j=m}^{n-1}\frac{1}{(\epsilon_j-\epsilon_n)}
\biggr]\prod_{k=1}^{n-1}(\epsilon_j-\epsilon_n)
\end{eqnarray}

%====================================================================
\subsubsection{Node-reduced partial waves}
%====================================================================
Secondly we define a sequence of energy-dependent 
\textbf{node-reduced partial wave}
\index{node-reduced partial wave}
through
\begin{eqnarray}
(\hat{h}-\epsilon)|q_n(\epsilon)\rangle=-|u_{n-1}\rangle
\end{eqnarray}
with the boundary conditions
\begin{eqnarray}
\lim_{\vec{r}\rightarrow0}|
\frac{\langle\vec{r}|q_n(\epsilon)\rangle}
{\langle\vec{r}|u_n\rangle}=1
\qquad\text{for $n>1$}
\end{eqnarray}

The node-reduced wave functions have $n-1$ nodes less than the true
wave function. Near the origin, it is very similar to $|u_n\rangle$
over nearly the entire energy range.


The all-electron partial wave can be constructed from the node-reduced
partial wave and the lower node-less bound states
as\cite{bloechl12_arxiv1210_5937}
\begin{eqnarray}
|\phi(\epsilon)\rangle=\biggl[|q_n(\epsilon)\rangle
+\sum_{m=1}^{n-1}|u_m\rangle\prod_{j=m}^{n-1}\frac{1}{\epsilon_j-\epsilon}
\biggr]
\prod_{k=1}^{n-1}(\epsilon_j-\epsilon)
\end{eqnarray}


%====================================================================
\subsubsection{Taylor expansion of node-reduced partial waves}
%====================================================================
Finally we introduce a Taylor expansion coefficents of the
node-reduced wave function in energy.
\begin{eqnarray}
|q_n^{(j)}(\epsilon)\rangle=\partial_\epsilon^j|q_n(\epsilon)\rangle
\end{eqnarray}


The Taylor expansion coefficients can be obtained from an inhomogeneus
Schr\"odinger equation
\begin{eqnarray}
(\hat{h}-\epsilon_\nu)|q_n^{(0)}(\epsilon)\rangle&=&-|u_{n-1}\rangle
\nonumber\\
(\hat{h}-\epsilon_\nu)|q_n^{(j)}(\epsilon)\rangle&=&j|q_n^{(j-1)}\rangle
\qquad\text{for $j>0$}
\end{eqnarray}

The Taylor coefficients start near the origin with a high order, $\sim
r^{\ell+2n+2j}$, and they usually increase exponentially at large
distances.

The all-electron partners
are obtains as follows from the node-reduced partial waves
\begin{eqnarray}
|\phi_n^{(j)}(\epsilon_\nu)\rangle
&\defas&\left.\partial_\epsilon^{j}\right|_{\epsilon_\nu}
\biggl[|\phi_n(\epsilon)\rangle
\frac{1}{\prod_{k=1}^{n-1}(\epsilon_j-\epsilon)}\biggr]
=|q_n^{(j)}(\epsilon_\nu)\rangle+\sum_{m=1}^{n-1}|u_m\rangle c_{j,m}
\end{eqnarray}
with coefficients $c_{j,m}$ derived in appendix~\ref{app:tayloraephi}.

%====================================================================
\subsection{Augmentation from node-reduced wave functions}
%====================================================================
The node-reduced wave function have a similar shape over the entire
energy region. This suggests to change them by the same,
energy-independent function $|k\rangle$, because this is likely not to
introduce extra nodes and to produce wave functions that are equally
suited for a plane wave expansion.

Thus we obtain a definition of our pseudo partual waves over the
entire energy region
\begin{eqnarray}
|\tilde{\phi}(\epsilon)\rangle=|q_n(\epsilon)\rangle+|k\rangle
\end{eqnarray}
Having a well behaved representation of the pseudo partial waves, we
can be quite sure that we do not add ghost states. If each partial
wave is pseudized individually, one easily obtains a polynomial
representation of the energy dependent partial wave that contains
divergences.

The construction of $|k\rangle$ is described in
section~\ref{sec:determinedeltepseudo}.

In order to obtain individual partial waves, we chose an expansion
energy $e_\nu$ and determine the Taylor expansion coefficients in
energy.
\begin{eqnarray}
|\tilde{\phi}^{(0)}\rangle&=&|q_n(\epsilon_\nu)\rangle+|k\rangle
\nonumber\\
|\tilde{\phi}^{(j)}\rangle&=&|q_n^{(j)}(\epsilon_\nu)\rangle
\qquad\text{for $j>0$}
\end{eqnarray}
Note, that only the first pseudo partial wave differs from its
node-reduced partial wave. This is because the difference $|k\rangle$
has been chosen energy independent.

%====================================================================
\subsubsection{Construct bare projector functions}
%====================================================================
The projector functions are derived from a energy dependent definition
of a projector, namely
\begin{eqnarray}
|p'(\epsilon)\rangle&=&
\left(\frac{\vec{p}^2}{2m_e}+\tilde{v}-\epsilon\right)
|\tilde\phi(\epsilon)\rangle
\nonumber\\
&=&-|u_{n-1}\rangle
+\left(\tilde{v}-v\right)|q_n(\epsilon)\rangle
+\left(\frac{\vec{p}^2}{2m_e}+\tilde{v}-\epsilon\right)
|k\rangle
\end{eqnarray}

The Taylor expansion in the energy leads to the bare projector
functions
\begin{eqnarray}
|p'^{(j)}\rangle&=&\partial_\epsilon^j|p'(\epsilon)\rangle
\nonumber\\
&=&
\Bigl[-|u_{n-1}\rangle
+\left(\frac{\vec{p}^2}{2m_e}+\tilde{v}-\epsilon_\nu\right)
|k\rangle\Bigr]\delta_{j,0}
\nonumber\\
&&
-|k\rangle\delta_{j,1}
+\left(\tilde{v}-v\right)|q_n^{(j)}(\epsilon_\nu)\rangle
\end{eqnarray}

The first term (j=0) is best calculated directly from
\begin{eqnarray}
|p'^{(0)}\rangle&=&
\left(\frac{\vec{p}^2}{2m_e}+\tilde{v}-\epsilon_\nu\right)
|\tilde\phi^{(0)}\rangle
\\
|p'^{(1)}\rangle&=&
|q_{n}^{(0)}(\epsilon_\nu)\rangle-|\tilde\phi^{(0)}\rangle
+\left(\tilde{v}-v\right)|q_n^{(1)}(\epsilon_\nu)\rangle
\\
|p'^{(j)}\rangle&=&
\left(\tilde{v}-v\right)|q_n^{(j)}(\epsilon_\nu)\rangle
\qquad\text{for $j>1$}
\end{eqnarray}

%====================================================================
\subsubsection{Bi-orthogonalization}
%====================================================================
The bi-orthogonalization shall establish 
\begin{eqnarray}
\langle p^{(j)}|\tilde{\phi}^{(j')}\rangle=\delta_{j,j'}
\end{eqnarray}

We do this while maintaining the partial waves, i.e.
\begin{eqnarray}
|p^{(j)}\rangle=\sum_{n}|p'^{(n)}\rangle
\Bigl(\langle p'^{(j)}|\tilde{\phi}^{(n)}\rangle\Bigr)_{n,j}
\end{eqnarray}

\textbf{The biorthogonalization deteriorates quickly, if the number of
  partial waves is increased.}


\petertt{Question: does the integrand for the overlap between
  projector function and partial wave fall off sufficiently fast with
  distance? Note, that the partial waves increase exponentially.}


%====================================================================
\subsubsection{Augmentation contribution to matrix elements}
%====================================================================
\begin{eqnarray}
dO_{\alpha,\beta}&=&\langle\phi_\alpha|\phi_\beta\rangle-
\langle\tilde{\phi}_\alpha|\tilde{\phi}_\beta\rangle
\nonumber\\
dT_{\alpha,\beta}&=&\langle\phi_\alpha|\hat{t}|\phi_\beta\rangle-
\langle\tilde{\phi}_\alpha|\hat{t}|\tilde{\phi}_\beta\rangle
\end{eqnarray}
The calculation of these matrix elements becomes substantially more
involved in the relativistic case.

\petertt{Problem: In order to cancel the tails of the core wave
  functions, the pseudo partial waves must include a pseudo-core
  contribution. This, however, is likely to produce ghost states. A
  remedy could be to change the exponentially increasing behavior of
  the the node-reduced partial waves artificially such that they do no
  more grow. In that case, the pseudo-core contribution would not harm.}

%====================================================================
\subsection{Pseudo wave functions from node-reduced wave functions}
\label{sec:determinedeltepseudo}
%====================================================================
Here, the routine \verb|SETUPS_MAKEPSPHI_MINE| is described. It is used
to construct a pseudo wave function for the lowest node-reduced
partial wave.

Starting from a wave function $\phi(\vec{r})$, which shall be
pseudized, we invert the Schr\"odinger equation to determine
\begin{eqnarray}
v-\epsilon_\nu=
\begin{cases}
-\frac{1}{\phi(\vec{r})}\hat{t}\phi(\vec{r})&\qquad\text{for $r>r_c$.}
\\
-\frac{1}{\phi(\vec{r}_c)}\hat{t}\phi(\vec{r}_c)&\qquad\text{for $r<r_c$.}
\end{cases}
\label{eq:invertschrgl}
\end{eqnarray}
where $r_c$ is the pseudization radius. $\hat{t}|\phi\rangle$ is a
short-hand notation for the result of the kinetic energy operator
acting on the partial wave. In the non-relativistic theory
$\hat{t}=\frac{\hat{\vec{p}}^2}{2m_0}$, while the expression is more
complex in the relativistic theory.

With \eq{eq:invertschrgl}, we obtain a potential that is shallow and
flat inside the pseudization radius. Outside the pseudization radius
it is chosen such that partial wave obeys the non-relativistic
Schr\"odinger equation for this potential.

Now, we derive for this potential the radial wave function and its
first two energy derivatives
\begin{eqnarray}
\Bigl[\frac{\vec{p}^2}{2m_0}+v-\epsilon_\nu\Bigr]|f\rangle&=&0
\nonumber\\
\Bigl[\frac{\vec{p}^2}{2m_0}+v-\epsilon_\nu\Bigr]|\dot{f}\rangle&=&|f\rangle
\nonumber\\
\Bigl[\frac{\vec{p}^2}{2m_0}+v-\epsilon_\nu\Bigr]|\ddot{f}\rangle
&=&2|\dot{f}\rangle
\end{eqnarray}

At the same time we determine 
\begin{eqnarray}
\hat{t}|f\rangle&=&\frac{\hat{\vec{p}}^2}{2m_e}|f\rangle
=-(\epsilon-v)|f\rangle
\nonumber\\
\hat{t}|\dot{f}\rangle&=&\frac{\hat{\vec{p}}^2}{2m_e}|\dot{f}\rangle
=|f\rangle-(\epsilon-v)|\dot{f}\rangle
\nonumber\\
\hat{t}|\ddot{f}\rangle&=&\frac{\hat{\vec{p}}^2}{2m_e}|\dot{f}\rangle
=2|\dot{f}\rangle-(\epsilon-v)|\ddot{f}\rangle
\end{eqnarray}

Finally, we match these three functions onto the node-reduced wave
function so that
\begin{eqnarray}
|\tilde{\phi}\rangle=|f\rangle c_1 
+|\dot{f}\rangle c_2+|\ddot{f}\rangle c_3
\end{eqnarray}
so that value, derivative and (if requested) the kinetic energy
density at the cutoff radius agree.

%====================================================================
\section{Relativistic effects}
%====================================================================
I recommend to read the doctoral thesis entitled ``The ZORA equation''
of Eric van Lenthe\cite{lenthe96_thesis}, if you can get your hands on
it.


%====================================================================
\subsection{Dirac equation}
%====================================================================
We start from the Dirac equation of an electron in an electric field.
First we divide the wave function into an upper part
$\phi=(\Psi_1,\Psi_2)$ and a smaller part $\chi=(\Psi_3,\Psi_4)$
\begin{eqnarray}
\left(\begin{array}{cc}
v-\epsilon &\quad \vec{\sigma}\vec{p}c\\
\vec{\sigma}\vec{p}c &\quad -2m_0c^2+v-\epsilon\end{array}\right)
\left(\begin{array}{c}|\phi\rangle\\|\chi\rangle\end{array}\right)=0
\label{eq:dirac2by2withepsilon}
\end{eqnarray}
where $\epsilon$ is the energy of the electron relative to the
rest-energy $m_0c^2$. $\vec{\sigma}$ is a vector formed from the three
Pauli matrices $\sigma_x,\sigma_y,\sigma_z$.

It will be connvenient to introduce the function
\begin{eqnarray}
1+D(\vec{r})\defas\Bigl[1+\frac{\epsilon-v(\vec{r})}{2m_0c^2}\Bigr]^{-1}
\end{eqnarray}
so that the Dirac equation obtains the form
\begin{eqnarray}
\left(\begin{array}{cc}
v-\epsilon &\quad \vec{\sigma}\vec{p}c\\
\vec{\sigma}\vec{p}c &\quad \frac{-2m_0c^2}{1+D}\end{array}\right)
\left(\begin{array}{c}|\phi\rangle\\|\chi\rangle\end{array}\right)=0
\label{eq:dirac2by2withepsilonb}
\end{eqnarray}

The small contribution can be extracted from the second equation of
\eq{eq:dirac2by2withepsilonb}
\begin{eqnarray}
|\chi\rangle
=\frac{1}{2m_0c^2+\epsilon-v}
\vec{\sigma}\vec{p}c |\phi\rangle
=\frac{1+D}{2m_0c}
\vec{\sigma}\vec{p} |\phi\rangle
\end{eqnarray}
Thus it can be eliminated from the first equation, which is turned
into a second-order differntial equation for the large component.
\begin{eqnarray}
\Bigl[\vec{\sigma}\vec{p} \frac{1+D}{2m_0}
\vec{\sigma}\vec{p} +v-\epsilon\Bigr]|\phi\rangle=0
\end{eqnarray}

Using the magic equation
\begin{eqnarray}
(\vec{\sigma}\vec{a})(\vec{\sigma}\vec{b})
=\vec{a}\vec{b}+i\vec{\sigma}(\vec{a}\times\vec{b})
\end{eqnarray}
we obtain the form
\begin{eqnarray}
\Bigl[\vec{p} \frac{1+D}{2m_0}\vec{p} 
+i\vec{\sigma} 
\Bigl(\vec{p} \frac{1+D}{2m_0}\times\vec{p} \Bigr)
+v-\epsilon\Bigr]|\phi\rangle=0
\end{eqnarray}
With the spin $\vec{S}=\frac{\hbar}{2}\vec{\sigma}$ and the orbit
angular momentum $\vec{L}=\vec{r}\times\vec{p}$, we obtain for a
rotationally symmetric $D(\vec{r})$ the Dirac equation for a spherical
system
\begin{eqnarray}
\Bigl[
\frac{1+D}{2m_0}\vec{p}^2 
+\frac{\hbar}{i}\frac{\partial_rD}{2m_0|\vec{r}|}\vec{r}\vec{p} 
+i\frac{2}{\hbar}\vec{S} 
\Bigl(\frac{\hbar}{i}(\underbrace{\vec{\nabla}\frac{1+D}{2m_0}}
_{\frac{(\partial_r D)}{2m_0}\frac{\vec{r}}{|\vec{r}|}}\times\vec{p} \Bigr)
+v-\epsilon\Bigr]|\phi\rangle=0
\nonumber\\
\Rightarrow\Bigl[
\frac{1+D}{2m_0}\vec{p}^2 
+\frac{\hbar}{i}\frac{\partial_rD}{2m_0|\vec{r}|}\vec{r}\vec{p} 
+ \underbrace{\frac{(\partial_r D)}{m_0|\vec{r}|}\vec{S}\vec{L}}_{\text{spin-orbit}}
+v-\epsilon\Bigr]|\phi\rangle=0
\nonumber\\
\end{eqnarray}


%====================================================================
\subsection{Relativistic augmentation}
%====================================================================
It is our goal to separate all relativistic effects out into the
augmentation contribution, so that we need not consider the
relativistic effects in the plane wave part. The latter does not
involve large kinetic energies so that they can be ignored with good
confidence.

In the PAW method, we make the following Ansatz for the wave function:
\begin{eqnarray}
\left(\begin{array}{c}|\phi\rangle\\|\chi\rangle\end{array}\right)
=
\left(
\begin{array}{c}|\tilde{\psi}\rangle\\\frac{1}{2m_0c}\vec{\sigma}\vec{p}|\tilde\psi\rangle\end{array}\right)
+
\sum_\alpha
\left[
\left(
\begin{array}{c}|\phi_\alpha\rangle\\
\frac{1+D}{2m_0c}\vec{\sigma}\vec{p}|\phi_\alpha\rangle\end{array}\right)
-
\left(
\begin{array}{c}|\tilde{\phi}_\alpha\rangle\\
\frac{1}{2m_0c}\vec{\sigma}\vec{p}|\tilde{\phi}_\alpha\rangle\end{array}\right)
\right]
\langle\tilde{p}_\alpha|\tilde\psi\rangle
\end{eqnarray}
This ansatz ensures that the wave function is continuous--if $D(r)$
vanishes smoothly beyond the augmentation radius--and that all-electron and
pseudo partial waves become identical. 

This mapping from a two-dimensional Pauli spinor onto a four component
spinor wave function is reminiscent of the Foldy-Wouthuysen
transformation\cite{foldy50_pr78_29}, which diagonalizes the matrix
form of the Dirac Hamiltonian.
\begin{eqnarray}
\left(\begin{array}{c}\langle\phi|\\
\langle\chi|\end{array}\right)
\left(\begin{array}{cc}
v-\epsilon &\quad \vec{\sigma}\vec{p}c\\
\vec{\sigma}\vec{p}c &\quad -2m_0c^2+v-\epsilon\end{array}\right)
\left(\begin{array}{c}|\phi\rangle\\|\chi\rangle\end{array}\right)=0
\end{eqnarray}


For the pseudo contribution, we change the all-electron potential to
the pseudo potential $\tilde{v}$ and we set the relativistic factor
$D$ to zero. 

In addition we introduce the small components of the partial waves 
\begin{eqnarray}
|\chi_\alpha\rangle&=&(1+D_\beta)\frac{\vec{\sigma}\vec{p}}{2m_0c}
|\phi_\alpha\rangle
\nonumber\\
|\tilde{\chi}_\alpha\rangle&=&\frac{\vec{\sigma}\vec{p}}{2m_0c}
|\tilde{\phi}_\alpha\rangle
\end{eqnarray}
The motivation for the small pseudo partial wave is that we have set
$D$ to zero, while we kept the speed of light finite.


\begin{eqnarray}
\Rightarrow
0&=&\left(\begin{array}{c}\langle\tilde{\psi}|\\
\langle\tilde{\psi}|\frac{\vec{\sigma}\vec{p}}{2m_0c}\end{array}\right)
\left(\begin{array}{cc}
\tilde{v}-\epsilon &\quad \vec{\sigma}\vec{p}c\\
\vec{\sigma}\vec{p}c &\quad -2m_0c^2+\tilde{v}-\epsilon\end{array}\right)
\left(\begin{array}{c}|\psi\rangle\\
\frac{\vec{\sigma}\vec{p}}{2m_0c}|\tilde{\psi}\rangle\end{array}\right)
\nonumber\\
&+&\sum_{\alpha,\beta}\langle\tilde{\psi}|\tilde{p}_\alpha\rangle
\biggl[
%
\left(\begin{array}{c}\langle\phi_\alpha|\\
\langle\chi_\alpha|
\end{array}\right)
\left(\begin{array}{cc}
v-\epsilon &\quad \vec{\sigma}\vec{p}c\\
\vec{\sigma}\vec{p}c &\quad -2m_0c^2+v-\epsilon\end{array}\right)
\left(\begin{array}{c}|\phi_\beta\rangle\\
|\chi_\beta\rangle\end{array}\right)
%
\nonumber\\
&&\hspace{2cm}-\left(\begin{array}{c}\langle\tilde{\phi}_\alpha|\\
\langle\tilde{\chi}_\alpha|
\end{array}\right)
\left(\begin{array}{cc}
\tilde{v}-\epsilon &\quad \vec{\sigma}\vec{p}c\\
\vec{\sigma}\vec{p}c &\quad -2m_0c^2+\tilde{v}-\epsilon\end{array}\right)
\left(\begin{array}{c}|\tilde{\phi}_\beta\rangle\\
|\tilde{\chi}_\beta\rangle\end{array}\right)
%
\biggr]
\langle\tilde{p}_\beta|\tilde{\psi}\rangle
\nonumber\\
%
&=&\langle\tilde\psi|
\frac{\hat{\vec{p}}^2}{2m_0}+\tilde{v}-\epsilon
+
\underbrace{
\textcolor{red}{(\vec{\sigma}\vec{p})\frac{(\tilde{v}-\epsilon)}{(2m_0c)^2}(\vec{\sigma}\vec{p})}}_{\textrm{ignored}}
|\tilde{\psi}\rangle
\nonumber\\
&+&\sum_{\alpha,\beta}\langle\tilde{\psi}|\tilde{p}_\alpha\rangle
\Bigl[
\underbrace{\langle\phi_\alpha|(\vec{\sigma}\vec{p})|\chi_\beta\rangle
+\langle\chi_\alpha|(\vec{\sigma}\vec{p})|\phi_\beta\rangle
-2m_0c^2\langle\chi_\alpha|\chi_\beta\rangle}_{T_{\alpha,\beta}}
+\langle\phi_\alpha|v-\epsilon|\phi_\beta\rangle
+\textcolor{red}{\langle\chi_\alpha|v-\epsilon|\chi_\beta\rangle}
\nonumber\\
%
&&\hspace{2cm}-\langle\tilde{\phi}_\alpha|
\frac{\hat{\vec{p}}^2}{2m_0}
+\tilde{v}-\epsilon|\tilde{\phi}_\beta\rangle
-
\textcolor{red}{
\langle\tilde{\chi}_\alpha|\tilde{v}-\epsilon|\tilde{\chi}_\beta\rangle}
\Bigr]\langle
\tilde{p}_\beta|\tilde{\psi}\rangle
%
\nonumber\\
&=&\langle\tilde\psi|
\frac{\hat{\vec{p}}^2}{2m_0}+\tilde{v}-\epsilon
+
\underbrace{
\textcolor{red}{(\vec{\sigma}\vec{p})\frac{(\tilde{v}-\epsilon)}{(2m_0c)^2}(\vec{\sigma}\vec{p})}}_{\textrm{ignored}}
|\tilde{\psi}\rangle
\nonumber\\
&+&\sum_{\alpha,\beta}\langle\tilde{\psi}|\tilde{p}_\alpha\rangle
\Bigl[
\underbrace{\langle\phi_\alpha|
(\vec{\sigma}\vec{p})\frac{1+D_\alpha}{2m_0}(\vec{\sigma}\vec{p})
+(\vec{\sigma}\vec{p})\frac{1+D_\beta}{2m_0}(\vec{\sigma}\vec{p})
|\phi_\beta\rangle
-2m_0c^2\langle\chi_\alpha|\chi_\beta\rangle}_{T_{\alpha,\beta}}
\nonumber\\
&&+\langle\phi_\alpha|v-\epsilon|\phi_\beta\rangle
+\textcolor{red}{\langle\chi_\alpha|v-\epsilon|\chi_\beta\rangle}
-\langle\tilde{\phi}_\alpha|
\frac{\hat{\vec{p}}^2}{2m_0}
+\tilde{v}-\epsilon|\tilde{\phi}_\beta\rangle
-
\textcolor{red}{
\langle\tilde{\chi}_\alpha|\tilde{v}-\epsilon|\tilde{\chi}_\beta\rangle}
\Bigr]\langle
\tilde{p}_\beta|\tilde{\psi}\rangle
\end{eqnarray}
We used that $(\vec{\sigma}\vec{p})^2=\vec{p}^2$.

The three terms marked in red are problematic, because
the first one contrbutes to plane wave part. The all-electron term is
not neglegible in the core region, but it also does not vanish in the
tail region, if the partial waves, and with it their small components,
diverge with increasing radius. This diverging part does, however, not
contribute, because it is cancelled by the corresponding one-center
pseudo term. This is also the reason that the pseudo term needs to be
included.

Approximation: 
The pseudo term of the plane wave part, however, is considered to be
small and is ignored in the current implementation. Let us investigate
the size of the error.
\begin{eqnarray*}
\langle\tilde\psi|
(\vec{\sigma}\vec{p})\frac{(\tilde{v}-\epsilon)}{(2m_0c)^2}(\vec{\sigma}\vec{p})
|\tilde{\psi}\rangle &=& \langle\tilde\psi|
\frac{1}{(2m_0c)^2}\frac{\hbar}{i}(\vec{\nabla}\tilde{v})\vec{p}
+\frac{2}{(2m_0c)^2}
\vec{S}\Bigl((\vec{\nabla}\tilde{v})\times\vec{p}\Bigr)
+\frac{(\tilde{v}-\epsilon)}{c^2} \frac{\vec{p}^2}{2m_0}
|\tilde{\psi}\rangle \nonumber\\ &=&
\frac{1}{2m_0c^2}\langle\tilde\psi|
\frac{-i\hbar}{2m_0}(\vec{\nabla}\tilde{v})\vec{p} +\frac{1}{m_0}
\vec{S}\Bigl((\vec{\nabla}\tilde{v})\times\vec{p}\Bigr)
+(\tilde{v}-\epsilon)\vec{p}^2 |\tilde{\psi}\rangle
\end{eqnarray*}


In order to estimate its size let us consider the free electron case,
where the wave functions are simple plane waves.
\begin{eqnarray*}
\langle\tilde\psi|
(\vec{\sigma}\vec{p})\frac{(\tilde{v}-\epsilon)}{(2m_0c)^2}(\vec{\sigma}\vec{p})
|\tilde{\psi}\rangle
&=&\frac{(\hbar\vec{G})^2}{2m_0}\frac{\epsilon-v}{2m_0c^2}
\approx \frac{E_{kin}^2}{2m_0c^2}\approx E^2_{kin} \cdot 2.5
\times10^{-5}\frac{1}{H}
\nonumber\\
&\approx&
\begin{cases}
5.00 mH\cdot N_e &\text{for $E_{kin}=30$~Ry}\\
0.60 mH\cdot N_e &\text{for $E_{kin}=10$~Ry}\\
0.06 mH\cdot N_e &\text{for $E_{kin}=5$~Ry}\\
\end{cases}
\end{eqnarray*}
where $N_e$ is the number of electrons.

Alternatives: Conceivable is to remove the first term from the pseudo
one-center part along with ignoring the complete plane wave part of
the small component. The rational is that this term captures the
dominant term of the pseudo plane wave part, while it does not have
the exponentially increasing behavior that needs to be canceled in the
all-electron term.


Thus we obtain Hamiltonian and overlap matrices of the form
\begin{eqnarray}
\hat{\tilde{T}}&=&
\frac{\hat{\vec{p}}^2}{2m_0}
+\sum_{\alpha,\beta}|\tilde{p}_\alpha\rangle 
dT_{\alpha,\beta}\langle\tilde{p}_\beta|
\nonumber\\
dT_{\alpha,\beta}&=&
\underbrace{
\langle\phi_\alpha|
(\vec{\sigma}\vec{p})\frac{1+D_\alpha}{2m_0}(\vec{\sigma}\vec{p})
}_{\langle g_\alpha|-\langle\phi_\alpha|(v-\epsilon_\alpha)}
|\phi_\beta\rangle
+\langle\phi_\alpha|
\underbrace{(\vec{\sigma}\vec{p})\frac{1+D_\beta}{2m_0}(\vec{\sigma}\vec{p})
|\phi_\beta\rangle}_{|g_\beta\rangle-(v-\epsilon_\beta)|\phi_\beta\rangle}
-2m_0c^2\langle\chi_\alpha|\chi_\beta\rangle
\nonumber\\
&-&
\underbrace{
\langle\tilde{\phi}_\alpha|\frac{\hat{\vec{p}}^2}{2m_0}
}_{\langle \tilde{g}_\alpha|-\langle\tilde{\phi}_\alpha|(\tilde{v}-\epsilon_\alpha)}
|\tilde{\phi}_\beta\rangle
-\langle\tilde{\phi}_\alpha|
\underbrace{
\frac{\hat{\vec{p}}^2}{2m_0}|\tilde{\phi}_\beta\rangle
}_{|\tilde{g}_\beta\rangle-(\tilde{v}-\epsilon_\beta)|\tilde{\phi}_\beta\rangle}
+2m_0c^2\langle\tilde{\chi}_\alpha|\tilde{\chi}_\beta\rangle
\nonumber\\
\hat{\tilde{O}}&=&1+
\sum_{\alpha,\beta}|\tilde{p}_\alpha\rangle dO_{\alpha,\beta}\langle\tilde{p}_\beta|
\nonumber\\
dO_{\alpha,\beta}&=&
\langle\phi_\alpha|\phi_\beta\rangle+\langle\chi_\alpha|\chi_\beta\rangle
-\langle\tilde{\phi}_\alpha|\tilde{\phi}_\beta\rangle
-\langle\tilde{\chi}_\alpha|\tilde{\chi}_\beta\rangle
\nonumber\\
\hat{\tilde{v}}&=&\hat{v}+
\sum_{\alpha,\beta}|\tilde{p}_\alpha\rangle dV_{\alpha,\beta}\langle\tilde{p}_\beta|
\nonumber\\
dV_{\alpha,\beta}&=&
\langle\phi_\alpha|\hat{v}|\phi_\beta\rangle
+\langle\chi_\alpha|\hat{v}|\chi_\beta\rangle
-\langle\tilde{\phi}_\alpha|\tilde{v}|\tilde{\phi}_\beta\rangle
-\langle\tilde{\chi}_\alpha|\tilde{v}|\tilde{\chi}_\beta\rangle
\end{eqnarray}

%====================================================================
\subsection{Nodeless construction for the Dirac equation}
%====================================================================
We generalize \eq{eq:nodelesseq} to the Dirac equation, which yields
\begin{eqnarray}
\left(\begin{array}{cc}
v-\epsilon_n &\quad \vec{\sigma}\vec{p}c\\
\vec{\sigma}\vec{p}c &\quad -2m_0c^2+v-\epsilon_n\end{array}\right)
\left(\begin{array}{c}|u_n\rangle\\|v_n\rangle\end{array}\right)
&=&-\left(\begin{array}{c}|u_{n-1}\rangle\\|v_{n-1}\rangle\end{array}\right)
\end{eqnarray}
where the small component of a nodeless function is denoted by $|v_n\rangle$.

We resolve for the small component $|v_n\rangle$
\begin{eqnarray}
|v_n\rangle=\frac{1+D_n}{2m_0c}(\vec{\sigma}\vec{p})|u_n\rangle
+\frac{1+D_n}{2m_0c^2}|v_{n-1}\rangle
\label{eq:nodelessdiracsmall}
\end{eqnarray}
and insert the result into the first equation
\begin{eqnarray}
(v-\epsilon)|u_n\rangle
+(\vec{\sigma}\vec{p})c|v_n\rangle&=&-|u_{n-1}\rangle 
\nonumber\\
\Bigl[(\vec{\sigma}\vec{p})
\frac{1+D_n}{2m_0}(\vec{\sigma}\vec{p})
+v-\epsilon_n\Bigr]|u_n\rangle
&=&-|u_{n-1}\rangle -(\vec{\sigma}\vec{p})\frac{1+D_n}{2m_0c}|v_{n-1}\rangle
\nonumber\\
\Bigl[
(1+D_n)\frac{\vec{p}^2}{2m_0}
+\frac{\hbar}{i}
\frac{\vec{\nabla}D_n}{2m_0}\vec{p}
+\vec{S}\Bigl(\frac{\vec{\nabla}D_n}{m_0}\times\vec{p}\Bigr)
+v-\epsilon_n\Bigr]|u_n\rangle
&=&-|u_{n-1}\rangle -(\vec{S}\vec{p})\frac{1+D_n}{\hbar m_0c}|v_{n-1}\rangle
\nonumber\\
\label{eq:nodelessdirac1}
\end{eqnarray}


%====================================================================
\subsection{Spinor harmonics}
%====================================================================
For a spherical atom, the spin and orbit angular momenta are no more
conserved independently, but only the total angular momentum is
conserved. Therefore we need to introduce spinor harmonics.

The spinor harmonics obey the eigenvalue equations
\begin{eqnarray}
\left(\hat{\vec{L}}+\hat{\vec{S}}\right)^2|\Omega_{j,\ell,j_z}\rangle
&=&|\Omega_{j,\ell,j_z}\rangle\quad\hbar^2j(j+1)
\\
\hat{\vec{L}}^2|\Omega_{j,\ell,j_z}\rangle
&=&|\Omega_{j,\ell,j_z}\rangle\quad\hbar^2\ell(\ell+1)
\\
\hat{\vec{S}}^2|\Omega_{j,\ell,j_z}\rangle
&=&|\Omega_{j,\ell,j_z}\rangle\quad\hbar^2\frac{3}{4}
\\
\hat{\vec{L}}\hat{\vec{S}}|\Omega_{j,\ell,j_z}\rangle
&=&|\Omega_{j,\ell,j_z}\rangle\quad\hbar^2\frac{1}{2}\left[j(j+1)-\ell(\ell+1)-\frac{3}{4}\right]
\label{eq:lsspinorharmonic}
\\
\left(\hat{L}_z+\hat{S}_z\right)|\Omega_{j,\ell,j_z}\rangle
&=&|\Omega_{j,\ell,j_z}\rangle\quad\hbar(m+\frac{1}{2})
\end{eqnarray}
where the quantum numbers $\kappa$ can assume any value except zero.
\begin{eqnarray}
\kappa&=&\ldots,-2,-1,1,2,\ldots
\\
j_z&=&-|\kappa|+\frac{1}{2},-|\kappa|+\frac{3}{2},\ldots,|\kappa|-\frac{1}{2}
\\
j&=& |\kappa|-\frac{1}{2}
\\
\ell&=&|\kappa+\frac{1}{2}|-\frac{1}{2}
\end{eqnarray}

The spinor harmonics have the form
\begin{eqnarray}
|\chi_{\kappa,j_z}\rangle
\stackrel{\kappa<0}{=}\left(
\begin{array}{c}
|Y_{-\kappa-1,j_z-\frac{1}{2}}\rangle
\sqrt{\frac{-\kappa+j_z-\frac{1}{2}}{-2\kappa-1}}\\
|Y_{-\kappa-1,j_z+\frac{1}{2}}\rangle
\sqrt{\frac{-\kappa-j_z-\frac{1}{2}}{-2\kappa-1}}\end{array}\right)
\qquad
|\chi_{\kappa,j_z}\rangle
\stackrel{\kappa>0}{=}\left(
\begin{array}{c}
-|Y_{\kappa,j_z-\frac{1}{2}}\rangle
\sqrt{\frac{\kappa-j_z+\frac{1}{2}}{2\kappa+1}}\\
|Y_{\kappa,j_z+\frac{1}{2}}\rangle
\sqrt{\frac{\kappa+j_z+\frac{1}{2}}{2\kappa+1}}\end{array}\right)
\label{eq:spinorharmonicskappa_a}
\\
\underbrace{\chi_{-\ell-1,m+\frac{1}{2}}}_{\Omega_{\ell+\frac{1}{2},\ell,m+\frac{1}{2}}}
=\left(
\begin{array}{c}
|Y_{\ell,m}\rangle\sqrt{\frac{\ell+m+1}{2\ell+1}}\\
|Y_{\ell,m+1}\rangle\sqrt{\frac{\ell-m}{2\ell+1}}\end{array}\right)
\qquad
\underbrace{\chi_{\ell,m+\frac{1}{2}}}_{\Omega_{\ell-\frac{1}{2},\ell,m+\frac{1}{2}}}
=\left(
\begin{array}{c}
-|Y_{\ell,m}\rangle\sqrt{\frac{\ell-m}{2\ell+1}}\\
|Y_{\ell,m+1}\rangle\sqrt{\frac{\ell+m+1}{2\ell+1}}\end{array}\right)
\label{eq:spinorharmonicskappa}
\end{eqnarray}

Then spinor harmonics describe states with the spin parallel or
antiparallel to the orbit angular momentum.

%====================================================================
\subsection{Spherical Dirac equations in spinor harmonics}
%====================================================================
The wave function of an electron is a four-component spinor function
consisting of a two-dimensional large and a two-dimensional small
component. Each has again a spin-up and a spin-down component.
\begin{eqnarray}
\left(\begin{array}{c}\phi(\epsilon,\vec{r})\\\chi(\epsilon,\vec{r})
\end{array}\right) =
\left(\begin{array}{c}g_{\kappa,j_z}(\epsilon,\vec{r})
\chi_{\kappa,j_z}(\vec{r})\\ if_{-\kappa,j_z}(\epsilon,\vec{r})
\chi_{-\kappa,j_z}(\vec{r})\\ \end{array}\right) 
\end{eqnarray} 

This Ansatz can be inserted into \eq{eq:nodelessdirac1}
and \eq{eq:nodelessdiracsmall}


\begin{eqnarray}
\Bigl[
(1+D_n)\frac{\vec{p}^2}{2m_0}
+\frac{\hbar}{i}
\frac{\partial_rD_n}{2m_0|\vec{r}}\vec{r}\vec{p}
+\frac{\partial_rD_n}{m_0|\vec{r}|} \vec{S}\underbrace{
\Bigl(\vec{r}\times\vec{p}\Bigr)}_{\vec{L}}
+v-\epsilon_n\Bigr]|u_n\rangle
&=&-|u_{n-1}\rangle -(\vec{S}\vec{p})\frac{1+D_n}{\hbar m_0c}|v_{n-1}\rangle
\nonumber\\
|v_n\rangle=\frac{1+D_n}{2 m_0}
\Bigl[\frac{2}{\hbar c}
(\vec{S}\vec{p})|u_n\rangle
+\frac{1}{c^2}|v_{n-1}\rangle\Bigr]
\end{eqnarray}
which yields with \eq{eq:lsspinorharmonic}

\begin{eqnarray}
&&\hspace{-2cm}\biggl(
(1+D)\frac{-\hbar^2}{2m_0}\biggl(\frac{1}{r}\partial_r^2r
-\frac{\ell(\ell+1)}{r^2}\biggr)
-\frac{\hbar^2(\partial_r D)}{2m_0}
\biggl[\partial_r
+\frac{\kappa+1}{|\vec{r}|}\biggr]
+v-\epsilon \biggr)g^{(n)}_{\kappa,j_z}(\epsilon,r)
\nonumber\\
&=&
-g^{(n-1)}_{\kappa,j_z}(\epsilon_{n-1},r)
+
\frac{\hbar}{c}\biggl[\partial_r+\frac{-\kappa+1}{|\vec{r}|}\biggr]
\frac{1+D}{ 2m_0 }
f_{-\kappa,j_z}^{(n-1)}(\epsilon_{n-1},r)
\nonumber\\
f_{-\kappa,j_Z}^{(n)}(\epsilon,r)
&=&\frac{1+D(\epsilon)}{2 m_0}
\biggl(\frac{\hbar}{c}\biggl[\partial_r+\frac{\kappa+1}{|\vec{r}|}\biggr]
g_{\kappa,j_z}^{(n)}(\epsilon,r)
+
\frac{1}{c^2}f_{-\kappa,j_z}^{(n-1)}(\epsilon_{n-1},r)\biggr)
\end{eqnarray}
Compare with Eq.~5.49-5.51 of the master thesis of Robert
Schade\cite{schade12_thesis}.  (Note, however, that I changed the sign
of $f_{n}$ relative to that of $f_{n-1}$, etc. relative to my earlier
notes.)

%=====================================================================
\chapter{Code structure}
%=====================================================================
%=====================================================================
\section{Remarks}
%=====================================================================
\begin{itemize}
\item The small contribution introduces nodes for the nodeless wave
  functions that lie near the nucleus which must not be counted. It is
  a consequence of treating the small component. This is taken care
  off by changing \verb|schroedinger$phaseshift| so that nodes are
  counted starting with a minimum radius. Schade\cite{schade12_thesis}
  gives the minimum radus of 0.07~a$_0$ for the core states and
  0.09~a$_0$ for the valence states.
%
\item Zora avoids the small component.?? Scalar relativistic
  calculations should treat the small component.
%
\item Currently the pseudo partial waves contain a pseudo core
  contribution so that they match teh all-electron partial waves
  exactly. This is likely to cause ghost states! 
%
\end{itemize}

%=====================================================================
\section{Setups\_newpro}
%=====================================================================
%=====================================================================
\subsection{Input variables}
%=====================================================================
The major input parameters are:
\begin{center}
\begin{tabular}{|l|l|}
\hline
L    & main angular-momentum quantum number\\
SO   & spin-orbit allignment $\sgn(\vec{S}\vec{L})$ (SO$\in\{-1,0,1\}$)\\
ROUT & bound states are calculates in a box with radius ROUT\\
RC   & cutoff for pseudization of partial waves\\
ENU  & energy for Taylor expansion of partial waves\\
\hline
\end{tabular}
\end{center}

%=====================================================================
\subsection{Flow chart}
%=====================================================================
The flow of the subroutine is as follows:
\begin{enumerate}
\item nodeless core wave functions \verb|UCORE|
\item Taylor expansion coefficients of node-reduced partial waves \verb|QN|
\item energy dertivative partial wave of highest partial wave \verb|QNDOT|
\item pseudo core wave functions \verb|PSUCORE|
\item pseudo partial waves \verb|PSPHI| (without core tails)
\item all-electron partial waves \verb|AEPHI| and inclusion of ps-core
  tails in pseudo partial waves
\item bare projector functions \verb|PRO|
\item bi-orthogonalization
\item matrix elements \verb|DTKIN| \verb|DOVER|
\end{enumerate}

%=====================================================================
\subsubsection{pseudo core wave functions}
%=====================================================================
We construct the function
\begin{eqnarray}
f_1&=&r^\ell
\nonumber\\
f_2&=& r^{\ell+2}
\nonumber\\
f_3&=& r^{\ell+4}
\end{eqnarray}
to the nodeless core wave function os that value and derivative agree
at the pseudization radius.

\appendix
%=====================================================================
\chapter{Useful formulas}
%=====================================================================
\begin{eqnarray}
(\vec{\sigma}\vec{a})(\vec{\sigma}\vec{b})
&=&\vec{a}\vec{b}+i\vec{\sigma}(\vec{a}\times\vec{b})
\\
\vec{r}\vec{\sigma}\chi_{\kappa,j_z}&=&-|\vec{r}|\chi_{-\kappa,j_z}
\\
\vec{S}\vec{p}R(|\vec{r}|)\chi_{\kappa,j_z}
&=&\frac{i\hbar^2}{2}\Bigl[\partial_r+\frac{\kappa+1}{|\vec{r}|}\Bigr]
R(|\vec{r}|)\chi_{-\kappa,j_z}
\\
1+D&=&\frac{1}{1+\frac{\epsilon-v}{2m_0c^2}}
\\
\kappa(\ell,so)&=&-1+so\cdot\Bigl(\ell+\frac{so-1}{2}\Bigr)
=
\begin{cases}
-\ell-1&\text{for $\vec{L}\vec{S}\ge0$ .i.e. $so=1$ }\\
\ell&\text{for $\vec{L}\vec{S}<0$ .i.e. $so=-1$}\\
-1&\text{for $\vec{L}\vec{S}=0$ .i.e. $so=0$}\\
\end{cases}
\end{eqnarray}

%=====================================================================
\chapter{Taylor expansion of node-reduced partial waves}
\label{app:tayloraephi} 
%=====================================================================

\begin{eqnarray}
|\phi_n^{(j)}(\epsilon_\nu)\rangle
&\defas&\left.\partial_\epsilon^{j}\right|_{\epsilon_\nu}
\biggl[|\phi_n(\epsilon)\rangle
\frac{1}{\prod_{k=1}^{n-1}(\epsilon_j-\epsilon)}\biggr]
\nonumber\\
&=&\left.\partial_\epsilon^{j}\right|_{\epsilon_\nu}
\biggl[
|q_n(\epsilon)\rangle
+\sum_{m=1}^{n-1}|u_m\rangle\prod_{j=m}^{n-1}\frac{1}{\epsilon_j-\epsilon}
\biggr]
\nonumber\\
&=&
|q_n^{(j)}(\epsilon_\nu)\rangle
+\sum_{m=1}^{n-1}|u_m\rangle
\left.\partial_\epsilon^{j}\right|_{\epsilon_\nu}
\biggl[\prod_{j=m}^{n-1}\frac{1}{\epsilon_j-\epsilon}
\biggr]
\end{eqnarray}


\begin{eqnarray}
\left.\partial_\epsilon^{0}\right|_{\epsilon_\nu}
\biggl[\prod_{j=m}^{n-1}\frac{1}{\epsilon_j-\epsilon}\biggr]
&=&
\biggl[\prod_{j=m}^{n-1}\frac{1}{\epsilon_j-\epsilon_\nu}\biggr]
\nonumber\\
\left.\partial_\epsilon^{1}\right|_{\epsilon_\nu}
\biggl[\prod_{j=m}^{n-1}\frac{1}{\epsilon_j-\epsilon}\biggr]
&=&\biggl[\prod_{j=m}^{n-1}\frac{1}{\epsilon_j-\epsilon_\nu}\biggr]
\biggl[\sum_{k=m}^{n-1}\frac{1}{\epsilon_j-\epsilon_\nu}\biggr]
\end{eqnarray}

Let us first form the derivatives 
\begin{eqnarray}
a_{j,m}&\defas&\left.\partial_\epsilon^{j}\right|_{\epsilon_\nu}
\biggl[\sum_{k=m}^{n-1}\frac{1}{\epsilon_j-\epsilon}\biggr]
=\biggl[\sum_{k=m}^{n-1}\frac{j!}{(\epsilon_j-\epsilon_\nu)^{j+1}}\biggr]
\qquad\text{for $j=0,\infty$}
\end{eqnarray}

Then we can evaluate the derivatives of the product
\begin{eqnarray}
c_{j,m}&\defas&\left.\partial_\epsilon^{j}\right|_{\epsilon_\nu}
\biggl[\prod_{j=m}^{n-1}\frac{1}{\epsilon_j-\epsilon}\biggr]
\nonumber\\
c_{0,m}&=&\prod_{j=m}^{n-1}\frac{1}{\epsilon_j-\epsilon_\nu}
\nonumber\\
c_{1,m}&=&c_{0,m}a_{0,m}
\nonumber\\
c_{2,m}&=&(c_{0,m}a_{0,m})a_{0,m}+c_{0,m}a_{1,m}
=c_{1,m}a_{0,m}+c_{0,m}a_{1,m}
\nonumber\\
c_{3,m}&=&c_{2,m}a_{0,m}+2c_{1,m}a_{1,m}+c_{0,m}a_{2,m}
\nonumber\\
c_{j,m}&=&\sum_{k=0}^{j-1}\binom{j-1}{k}c_{j-k-1,m}a_{k,m}
\end{eqnarray}


\begin{eqnarray}
|\phi_n^{(j)}(\epsilon_\nu)\rangle
&=&
|q_n^{(j)}(\epsilon_\nu)\rangle+\sum_{m=1}^{n-1}|u_m\rangle c_{j,m}
\end{eqnarray}



\printindex
\bibliographystyle{unsrtnat}
\bibliography{all}
\end{document}  
