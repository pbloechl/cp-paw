\documentclass[11pt,a4paper]{report}
%%%%%%%%%%%%%%%%%%%%%%%%%%%%%%%%%%%%%%%%%%%%%%%%%%%%%%%%%%%%%%%%%%%%%
%%                                                                 %%
%%    Header file for the Phi-S-X Series                           %%
%%                                                                 %%
%%    german version header_gm.tex is derived from header.tex      %%
%%    by uncommenting the line ``\setboolean{german}{true}'' below %%
%%                                                                 %%
%%    Never edit the german version! all changes must be done      %%
%%    in the english version header.tex                            %%
%%                                                                 %%
%%%%%%%%%%%%%%%%%%%%%%%%%%%%%%%%%%%%%%%%%%%%%%%%%%%%%%%%%%%%%%%%%%%%%
%%%%%%%%%%%%%%%%%%%%%%%%%%%%%%%%%%%%%%%%%%%%%%%%%%%%%%%%%%%%%%%%%%%%%
%%                                                                 %%
%%    Header file for the Phi-S-X Series                           %%
%%                                                                 %%
%%    german version header_gm.tex is derived from header.tex      %%
%%    by uncommenting the line ``\setboolean{german}{true}'' below %%
%%                                                                 %%
%%    Never edit the german version! all changes must be done      %%
%%    in the english version header.tex                            %%
%%                                                                 %%
%%%%%%%%%%%%%%%%%%%%%%%%%%%%%%%%%%%%%%%%%%%%%%%%%%%%%%%%%%%%%%%%%%%%%
%====================================================================
%-- define flag for language adaptations
\usepackage{ifthen}   % allows to select only certain text
\provideboolean{german}
\setboolean{german}{false}
%\setboolean{german}{true}  % uncomment this line for german editions
%====================================================================
%
% Textschriftart: Computer modern Bright
% body:            CM-Bright 10pt
% section titles:  CM-Bright Bold
% formulas:        CM-Bright Math Oblique
%
\usepackage[standard-baselineskips]{cmbright}
\usepackage{cmbright}
\usepackage[T1]{fontenc}
\def\usedfonts{CM-Bright}
\usepackage{typearea}
%\typearea[current]{calc} % benutzt die aktuelle 
       % bindekorrektur (BCOR angabe als parameter in koma usepackage)
       % und berechnet satzspiegel neu
\typearea[current]{11} %fixed div value

\usepackage{textcomp} % special symbols
\usepackage{amsfonts} % special symols
                      % see ftp://ftp.ams.org/pub/tex/doc/amsfonts/amsfndoc.pdf
\usepackage{amssymb}  % CM-Bright provides the AMS symbols
\usepackage{exscale}  % allows to scale math expressions to big fonts, 
                      % e.g. \Huge
\usepackage{curves}
\usepackage{braket}
\usepackage{miller}     % miller indices
\usepackage{chemmacros} % http://www.mychemistry.eu/mychemistry/
\usepackage[numbers]{natbib}     % bibliography style
\usepackage{url}\urlstyle{tt}
\usepackage{float}
\usepackage{bm}       % provides the command \bm{} that makes bold math symbols
\usepackage{amsmath}
\usepackage{amsbsy}   % allows bold mathematical symbols
\usepackage{amscd}
 \usepackage{a4wide}  % it is better to use the ``geometry'' package
\usepackage{array}    % 
\usepackage{fancyhdr} %  defines pagestyle fancy
\usepackage{epsfig}   % include graphics with epsfig
\usepackage{graphicx} % includegraphics
\usepackage{epstopdf}
\usepackage{wrapfig}
\usepackage{fancybox} % allows shadow-boxes
\usepackage{color}    % allows to use color in the text
%\usepackage{eepic}
\usepackage{flafter}  % places picture next to its reference
\usepackage{makeidx}  % make an index
%\usepackage{MnSymbol}  % 
%\usepackage{marvosym}  % 
\usepackage{textcase}
\usepackage{ulem} % defines strikeout \sout{}; underline \uline{}
                  % double underline \uuline{}; wave underline \uwave{}
                  % cross out \xout{}
%
%==========================================================================
%==  page layout  =========================================================
%==========================================================================
% eqnarray environment: reduce with of space in place of each ``&''
\setlength\arraycolsep{1.4pt}
\pagestyle{fancy}
%\renewcommand{\chaptermark}[1]{\markboth{\thechapter\ #1}{}}
\renewcommand{\chaptermark}[1]{\markboth{\MakeUppercase{\thechapter\ #1}}{}}
\fancyhf{} 
\fancyhead[LE]{\textsc{\thepage}\qquad\textsc{\leftmark}}
\fancyhead[RO]{\textsc{\leftmark}\qquad\textsc{\thepage}}
\renewcommand{\headrulewidth}{0.5pt}
\renewcommand{\footrulewidth}{0pt} 
\addtolength{\headheight}{2.5pt}
\fancypagestyle{plain}{\fancyhead{}
   \renewcommand{\headrulewidth}{0pt}
   \fancyfoot[CO]{\bfseries\thepage}}

% Line spacing -----------------------------------------------------------
\newlength{\defbaselineskip}
\setlength{\defbaselineskip}{\baselineskip}
\newcommand{\setlinespacing}[1]%
           {\setlength{\baselineskip}{#1 \defbaselineskip}}
\newcommand{\doublespacing}{\setlength{\baselineskip}%
                           {2.0 \defbaselineskip}}
\newcommand{\singlespacing}{\setlength{\baselineskip}{\defbaselineskip}}

% Absatz einr\"ucken ------------------------------------------------------
%\setlength{\parindent}{0pt}
\setlength{\parskip}{2pt}
% -------------------------------------------------------------------------
\ifthenelse{\boolean{german}}
  {\def\figurename{Abb.}}
  {\def\figurename{Fig.}}
%--------------------------------------------------------------------------
\renewcommand{\arraystretch}{1.15}  % skaliert den Zeilen abstand in der 
    % tabular und array umgebung
%
%==========================================================================
%==  boxes etc ============================================================
%==========================================================================
%== minipage in a shadowbox ===============================================
\newenvironment{myshadowminipage}[1]%
  {\par\noindent\begin{Sbox}\begin{minipage}{\linewidth}\vspace{0.1cm}\begin{center}\uppercase{#1}\end{center}}%
  {\vspace{0.1cm}\end{minipage}\end{Sbox}\shadowbox{\TheSbox}}
%
%== minipage in a framedbox ===============================================
\newenvironment{myframedminipage}%
  {\par\noindent\begin{Sbox}\begin{minipage}\linewidth\vspace{0.1cm}}%
  {\vspace{0.1cm}\end{minipage}\end{Sbox}\fbox{\TheSbox}}
%
\newcommand{\myshadowbox}[1]{\noindent\shadowbox{\parbox{\linewidth}{\smallskip #1\smallskip}}}
\newcommand{\myfbox}[1]{\noindent\fbox{\parbox{\linewidth}{\smallskip #1\smallskip}}\medskip}
%== minipage in a framedbox ===============================================
\newtheorem{defi}{Definition}[chapter]
\newenvironment{definition}[1]%
  {\par\noindent\begin{Sbox}\begin{minipage}{\linewidth}\vspace{0.1cm}\begin{defi}\uppercase{#1}\\\vspace{0.1cm}}%
  {\vspace{0.1cm}\end{defi}\end{minipage}\end{Sbox}\shadowbox{\TheSbox}}
%
%=========================================================================
% color used to point out information to the teacher
\definecolor{highlight}{rgb}{1.0,0.7,0.}
\newcommand{\Special}[1]{\textbf{\textcolor{highlight}{#1}}}
%=========================================================================
%  switch certain parts on and off. uses ifthen package
\newboolean{teacher}\setboolean{teacher}{false}
% this parameter can be changed in the manuscript again
\setboolean{teacher}{true} %private version if true!
\newcommand{\teacheronly}[1]{\ifthenelse{\boolean{teacher}}{#1\hfill\\ }}
\newcommand{\editor}[1]{\textcolor{blue}{\texttt{Editor: #1}}}
\newcommand{\MARK}[1]{\textcolor{blue}{#1}} 
\newcommand{\RED}[1]{\textcolor{red}{#1}} 
%
%==========================================================================
%==  define new symbols                                                 ===
%==========================================================================
% define \stat (stationary state) as an operator like \min
\DeclareMathOperator*{\stat}{stat}
\let\Vec=\mathbold   % cmbright.sty provides a bold/italic math alphabet
\let\Dot=\mathbold   % cmbright.sty provides a bold/italic math alphabet
\let\Ddot=\mathbold   % cmbright.sty provides a bold/italic math alphabet
%
\newcommand{\e}[1]{\mathrm{e}^{#1}}% exponential function
\renewcommand{\Re}{\mathrm{Re}}    % real part
\renewcommand{\Im}{\mathrm{Im}}    % imaginary part
\newcommand{\lagr}{\ell}           % Lagrange dichte
\newcommand{\Lagr}{\mathcal{L}}    % Lagrange Funktion
\newcommand{\erf}{{\rm erf}}       %
\newcommand{\atan}{{\rm atan}}     % arcus tangens
\newcommand{\mat}[1]{\bm{#1}}  % Matrix
\newcommand{\gmat}[1]{{\boldsymbol #1}}  % Matrix(symbol)
\newcommand{\defas}{\stackrel{\text{def}}{=}}  %  is defined as
\ifthenelse{\boolean{german}}
  {\newcommand{\rot}{{\rm\bf rot}}}    % curl
  {\newcommand{\rot}{{\rm\bf curl}}}   % curl
\newcommand{\sgn}{{\rm sgn}}       % sign
\ifthenelse{\boolean{german}}
   {\newcommand{\Tr}{\mathrm{Sp}}}      % trace
   {\newcommand{\Tr}{\mathrm{Tr}}}      % trace
\ifthenelse{\boolean{german}}
   {\newcommand{\grmn}[2]{\footnote{``#2'' hei{\ss}t in englisch ``#1''}}}
   {\newcommand{\grmn}[2]{\footnote{``#1'' translates as ``#2'' into German}}}
% define the equation reference
\ifthenelse{\boolean{german}}
   {\newcommand{\eq}[1]{\text{Gl.}~\ref{#1}}}
   {\newcommand{\eq}[1]{\text{Eq.}~\ref{#1}}}
% define a relation with an equation number ontop
\newcommand{\eqrel}[2]{\stackrel{\eq{#1}}{#2}}
\newcommand{\zero}{\varnothing}
%\newcommand{\ket}[1]{|#1\rangle} % contained in package braket
\newcommand{\sumint}{\int\hspace{-15pt}\sum}
\newcommand{\marker}[1]{\textcolor{blue}{\emph{#1}}}
\renewcommand*{\dot}[1]{\overset{\mbox{\large\bfseries .}}{#1}}
\renewcommand*{\ddot}[1]{\overset{\mbox{\large\bfseries\hspace{+0.1ex}.\hspace{-0.1ex}.}}{#1}}
%
%==========================================================================
%==                                                                     ===
%==========================================================================
% Prevent figures from appearing on a page by themselves
% from http://dcwww.camd.dtu.dk/~schiotz/comp/LatexTips/LatexTips.html
\renewcommand{\topfraction}{0.85}
\renewcommand{\textfraction}{0.1}
\renewcommand{\floatpagefraction}{0.75}
%
%==========================================================================
%==                                                                     ===
%==========================================================================
\makeindex    % make index. uses makeidx package.

%== allow links between documents ============================================
\usepackage{xr}
\usepackage{xr-hyper}
%==  hyperref package (must be last package)
\usepackage[colorlinks=true]{hyperref} %specify this as last package
\hypersetup{citecolor=blue}
\hypersetup{menucolor=magenta}
\hypersetup{urlcolor=blue}      % 
\hypersetup{filecolor=green}    % file links
\hypersetup{linkcolor=magenta}  %table of contents
\hypersetup{pdfauthor={Peter E. Bl\"ochl}}
\hypersetup{pdfdisplaydoctitle=true}
\externaldocument[phisx1-]{/Users/ptpb/Tree/PhiSX/ClassicalMechanics/Book/cm-gm}
\externaldocument[phisx2-]{/Users/ptpb/Tree/PhiSX/Electrodynamics/Book/el-gm}
\externaldocument[phisx3-]{/Users/ptpb/Tree/PhiSX/QuantumMechanics/Book/qm}
\externaldocument[phisx4-]{/Users/ptpb/Tree/PhiSX/StatisticalMechanics/Book/sm}
\externaldocument[phisxqm2-]{/Users/ptpb/Tree/PhiSX/QuantumMechanicsII/Book/qm2}
\externaldocument[phisxsm2-]{/Users/ptpb/Tree/PhiSX/StatisticalMechanicsII/Book/sm2}
\externaldocument[phisxcb-]{/Users/ptpb/Tree/PhiSX/Chemicalbond/Book/cb}
% Example: Figure~PhiSX:Quantum
% Mechanics-\ref{phisx3-fig:doubleslitwave} on page
% \pageref{phisx3-fig:doubleslitwave}


\hypersetup{pdftitle=paw_brillouin}
\newcommand{\petertt}[1]{\textcolor{red}{\texttt{#1}}}
\begin{document}
\begin{titlepage}
\begin{center}
\vspace*{3.5cm}
{\huge \textbf{The CLASSICAL object of the CP-PAW code}}\\
\vspace{0.5cm}
{\large Peter E. Bl\"ochl}
\vspace{0.5cm} 
\end{center}

\vfill
\begin{center}
Copyright Peter E. Bl\"ochl; Sept.2, 2013-\today\\
{\small
Institute of Theoretical Physics;
Clausthal University of Technology;\\ 
D-38678 Clausthal Zellerfeld; Germany;\\
http://www.pt.tu-clausthal.de/atp/}
\end{center}
\end{titlepage}
\noindent            
\tableofcontents
%====================================================================
\chapter{Purpose and theoretical background}
%====================================================================
The CLASSICAL object implements a molecular dynamics tool for a
general interatomic forcefield. In the classical molecular dynamics the
total energy is expressed in terms of the atomic structure. 

the classical object can also be used independent of the main paw-code
using the front-end described in appendix~\ref{sec:frontend} on
p.~\pageref{sec:frontend}.

The object separates the evaluation of total energies and forces from
the parameterized form of the force field by placing the potentials
onto interpolation grids, from which value and derivatives are
recovered by a spline interpolation.

Available force-fields:
\begin{itemize}
\item The default force field is the Universal force field
(UFF).\cite{rappe92_jacs114_10024}. 
%
\item The water model TIP4P by Joergensen et al.\cite{jorgensen83_jcp79_926}
is currently in the process of implementation.
\end{itemize}

%====================================================================
\chapter{Methods}
%====================================================================


%====================================================================
\section{Force-field terms}
%====================================================================
The force fields consists of long-ranged and short-ranged total energy
contributions. 

The short-ranged contributions are defined by a
pre-determined network of bonds and contain
\begin{itemize}
\item bond terms
\item angle terms
\item torsions
\item inversion terms
\end{itemize}

The long-ranged forces are calculated from a neighborlist that is
regularly updated. It contains
\begin{itemize}
\item Coulomb terms
\item Lennard-Jones terms
\end{itemize}
An important aspect of long-ranged contributions are \textbf{nonbonded
  exclusions}\index{exclusions !nonbonded}\index{nonbonded
  exclusions}: for atoms that are connected by short-ranged forces the
long-ranged contributions are suppressed.


%====================================================================
\subsection{Non-bonded exclusions}
%====================================================================
Long-ranged forces are typically suppressed for atoms connected by
bond or angle terms. This is the choice followed by
UFF\cite{rappe92_jacs114_10024} (see their section II.G).

These exclusions are evaluated in \verb|CLASSICAL_EXCLUSIONS| and
stored as array \verb|MD\%EXCLUSIONS|. \verb|MD%EXCLUSIONS| is an
array of pairs of two integers. The first integer encodes
\footnote{The encoding and decoding can be performed with the routines
  CLASSICAL\_ENCODEEXCLUSION and CLASSICAL\_DECODEEXCLUSION. The mapping to the integer $I$ is
\begin{eqnarray}
I=1+(t_1+M)
+\Bigl[1+2M\Bigr](t_2+M)
+\Bigl[(1+2M)^2\Bigr](t_3+M)
+\Bigl[(1+2M)^3\Bigr]R_1
+\Bigl[(1+2M)^3N\Bigr] R_2
\end{eqnarray}
where $(t_1,t_2,t_3)$ are the integer lattice translations of the
second atom, $(R_1,R_2)$ are the two atom indices in the pair. $M$ is
the largest lattice translation considered in the neighbor list and
$N$ is the number of atoms in the unit cell.  The pair is ordered such
that the first atom index is equal or larger than the second one,
i.e. $R_1\ge R_2$. This choice makes the exclusion unique and it
ensures that the largest index $I$ is as small as possible.
} 
the atom pair as (1st
atom index, 2nd atom index, 3-d array of integer displacements by the
lattice translation vectors of the second atom).

The second integer specifies the \textbf{exclusion
  type}\index{exclusion type}, because not all exclusions are treated
the same way.
\begin{center}
\begin{tabular}{|c|l|l|}
\hline
type & explanation & used by \\
\hline
0 & 1,2 (bond) and 1,3 (angle) exclusions & UFF,AMBER\\
1 & 1-4 exclusions                        & AMBER \\
2 & complete exclusions                   & TIP4P \\
\hline
\end{tabular}
\end{center}

The exclusions are collected by scanning all bond, angle torsion terms
subsequently. Finally they are sorted according to increasing index,
and double counted exclusions are trimmed. Before a double counted
exclusion is removed, its exclusion type is compared with the current
one, in order to make sure that the ``dominant'' exclusion type is
given to the exclusion that survives. The last exclusion is given a
index -1 to signify to allow for a consistency check.

In order to allow for a rapid search on the exclusion list, an array
is formed that points for each atom to the 


In classical\_exclusions a list of exclusions is set up on the basis
of the list of short-ranged interactions (bonds, angles, torsions,
inversions etc.) This can be done once and for all for a given network
of short ranged interaction.

In classical\_neighbors the exclusion information is transferred into
the present neighborlist.

in classical\_ecoulomb the long ranged interactions are evaluated.


%====================================================================
\subsection{Coulomb terms}
%====================================================================
%====================================================================
\subsection{Lennard-Jones potential}
%====================================================================



%====================================================================
\section{UFF force field}
%====================================================================
The UFF force field rest on the following publications
\begin{itemize}
\item Original UFF paper\cite{rappe92_jacs114_10024}.
\item Charge equilibration\cite{rappe91_jpc95_3358}.
\item Atom sizes and bond lengths\cite{okeefe92_jacs113_3226}.
\item DREIDING\cite{mayo90_jpc90_8897}, a predecessor of UFF.
\item Application of UFF to organic molecules\cite{casewit92_jacs114_10035}
\item Application of UFF to main group compounds\cite{casewit92_jacs114_11046}
\item see also\cite{addicoat14_jctc10_880}
\end{itemize}
%====================================================================
\subsection{Bond stretch}
%====================================================================
The bond stretch term has the form\footnote{UFF also allows for a
  Morse form for the bond distance.
\begin{eqnarray}
E^{bond}(r)&=&D\Bigl[\e{-\sqrt{k^{bond}/D^{bond}}(r-r^{bond})}-1\Bigr)^2
\end{eqnarray}
where $D=70$~kcal/mol times the bond order.  We have chosen the for
of an harmonic oscillator, because it does not allow for bond
breaking.} (UFF-Eq.1a)
\begin{eqnarray}
E^{bond}(r)&=&\frac{1}{2} k^{bond}\left(r-r^{bond}\right)^2
\end{eqnarray}
where $r$ is the present bond distance. The two parameters are the
force constant $k^{bond}$ and the equilibrium bond distance
$r^{bond}$.

The equilibrium bond distance is determined by the covalent radii, the
bond order $n$ and the electronegativities
(UFF-Eqs.~2,3,4)\footnote{Attention! In the UFF paper is a sign error
  for the electronegativity term in UFF Eq.4. See also comments for
  UFF from MCCS TOWHEE.}
\begin{eqnarray}
r^{bond}=\Bigl(r^{cov}_i+r^{cov}_j\Bigr)
\underbrace{\Bigl(1-0.1332\ln(n)\Bigr)}_{\text{bond order correction}}
-
\underbrace{r^{cov}_ir^{cov}_j
\frac{\Bigl(\sqrt{\chi_i}-\sqrt{\chi_j}\Bigr)^2}
{\Bigl(\chi_ir^{cov}_i+\chi_jr^{cov}_j\Bigr)^2}}
_{\text{electronegativity correction}}
\end{eqnarray}


The force constant is calculated from effective charges $Z^*_i$ as
\begin{eqnarray}
k^{bond}_{i,j}&=&\frac{Z^*_iZ^*_j}{r^{bond}}\times 664.12\;\text{kcal/mol}
\end{eqnarray}

%====================================================================
\subsection{Bond-angle terms}
%====================================================================
For the angle forces, two cases are distinguished: 
\begin{itemize}
\item for linear (n=2), trigonal (n=3), square planar (n=4), 
octahedral environments (n=4) the following form is chosen (UFF-Eq.~10)
\begin{eqnarray}
E^{angle}(\theta)&=&k^{angle}\frac{1}{n^2}\Bigl(1-\cos(n\theta)\Bigr) 
\end{eqnarray}
\item for the general non-linear case with an ideal bond angle $\theta_0$
the more complex form UFF-Eq.11,12) is selected.
\begin{eqnarray}
E^{angle}(\theta)&=&k^{angle}
\frac{\Bigl(2\cos^2(\theta_0)+1\Bigr)
-4\cos(\theta_0)\cos(\theta)
+\cos(2\theta)}{4\sin^2(\theta_0)}
\end{eqnarray}
\end{itemize}


The force constants are given in UFF-Eq.13
\begin{eqnarray}
k^{angle}= \frac{Z^*_iZ^*_k}{r_{ik}^3}
\frac{3r_{ij}r_{jk}\sin^2(\theta_0)
-r_{ik}^2\cos(\theta_0)}{r_{ik}^2}
\times 664.12 \text{kcal/mol}
\end{eqnarray}

\textcolor{red}{Careful!!! this equation is completely messed up!!!!}

%====================================================================
\subsection{Lennard-Jones potential}
%====================================================================
The Lennard-Jones potential has the form 
\begin{eqnarray}
E_{vdw}(d)=D^{vdw}\biggl[\Bigl(\frac{r^{vdw}}{d}\Bigr)^{12}-2
\Bigl(\frac{r^{vdw}}{d}\Bigr)^{6}\Bigr)\biggr]
\end{eqnarray}
where $d$ is the actual distance between the two atoms The two
parameters is the van-der Waals radius $r^{vdw}_{i,j}$ and the van der
Waals energy $D^{vdw}_{i,j}$.  Thus implies that the Lennard-Jones
potential has a minimum at radius $r^{vdw}$ with depth $D^{vdw}$.

UFF defines Lennard Jones parameters for each atom type. For a pair,
the parameters are combined using a geometric rule for the bond energies
\begin{eqnarray}
D^{vdw}_{i,j}&=&\sqrt{D^{vdw}_{i}D^{vdw}_{j}}
\\
r^{vdw}_{i,j}&=&\frac{1}{2}\left(r^{vdw}_i+r^{vdw}_j\right)
\end{eqnarray}

For the distance the UFF seems to be unspecific regarding whether the
parameters are combined with the arithmetic mean or as the geometic
mean.



%====================================================================
\section{Dummy atoms}
%====================================================================
Dummy atoms are fictitious atoms, which are introduced to allow to
attach force fields to sites whihc are not atomic positions.

The positions of the dummy atoms are a direct function of the
positions of real atoms.
\begin{eqnarray}
R_D=\sum_j w_j \vec{R}_j
\end{eqnarray}
This step is performed in routine \verb|CLASSICAL_DUMMY_POSITION|.

The forces on the physical atoms, which result from a force term
acting on the dummy atom, can be also directly calculated.
\begin{eqnarray}
F_i=-\vec{\nabla}_iE(\vec{R}_i,\vec{R}_D(\vec{R}_i)=
-\vec{\nabla}_iE(\vec{R}_i,\vec{R}_D)
-w_i\vec{\nabla}_DE(\vec{R}_i,\vec{R}_D)
\end{eqnarray}

The positions of the dummy atoms are determined just before the total
energy and forces are determioned. Right after calculating the forces,
the forces acting on the dummy atom are mapped onto the real atoms,
and the force on the dummy atom is set to zero.






\begin{tabular}{|l|l|}
\hline
\hline
Identifier & molecule \\
\hline
CPR   & cyclopentadienyl\\
CPR\_B & cyclopentadienyl\\
PIR    & olefin\\
CIR    & allyl\\
MTIP4P & water\\
\hline
\end{tabular}


\begin{itemize}
\item CPR,CPR\_B,PIR,CIR The position of the dummy atom is the
  geometrical center of all carbon atoms with type C\_2 and C\_R, to
  which it has a bond.
%
\item MTIP4P. The middle atom of the water molecule in the TIP4P model
  is given as weighted sum of the positions of the real atoms. The
  molecule is defined by the oxygen atom, the dummy atom is bonded to,
  and the two hydrogen atoms bonded to this oxygen atom.
\end{itemize}


%====================================================================
\section{TIP4P forcefield for water}
%====================================================================
The TIP4P forcefield\cite{jorgensen83_jcp79_926} for water uses a
rigid atomic structure, a lennard-jones potential between oxygen atoms
and charges on the hydrogen positions and a dummy atom. The parameters
are given in table~\ref{tab:tip4ppar}.

\begin{table}[htb!]
\begin{center}
\begin{tabular}{|l|c|}
\hline
\hline
\multicolumn{2}{|c|}{TIP4P parameters}\\
\hline
$d(OH)$ & 0.9572~\AA\\
$d(OM)$ & 0.1500~\AA\\
$\angle(HOH)$ & 104.52$^\circ$\\
$C_{12}$ & $0.6\times 10^6$~kcal/mol \AA$^{12}$\\
$C_{6}$ & $0.61\times 10^3$~kcal/mol \AA$^{6}$\\
$q(O)$ & $0$\\
$q(M)$ & $-1.04$~e \\
$q(H)$ & $+0.52$~e \\
\hline
\hline
\end{tabular}
\end{center}
\caption{\label{tab:tip4ppar} Parameters for the TIP4P model from
    Joergensen et al.\cite{jorgensen83_jcp79_926}.}
\end{table}

The TIP4P force field has the form
\begin{eqnarray}
E=\frac{1}{2}\sum_{m\neq n\in\text{molecules}} 
\biggl[&&
\frac{C_{12}}{|\vec{R}_{O,m}-\vec{R}_{O,n}|^{12}}
-\frac{C_{6}}{|\vec{R}_{O,m}-\vec{R}_{O,n}|^{6}}
\nonumber\\
&+&\sum_{i,j\in\{M,H_1,H_2}\frac{Q_iQ_j}
{4\pi\epsilon_0|\vec{R}_{i,m}-\vec{R}_{j,n}|}
\biggr]
\end{eqnarray}
The sum is performed over all pairs of molecules\footnote{The factor
  $\frac{1}{2}$ in front of the double sum avoids the double counting
  of pairs.}  With $\vec{R}_{i,n}$ we denote the atom $i$
($i\in\{O,M,H_1,H_2\}$)on the molecule $n$.  The Lennard-Jones
potential acts only between pairs of oxygen atoms at positions
$\vec{R}_{O,n}$. the electrostatic charges on the other hand are at
the two hydrogen positions and the dummy atoms denoted by M.

The atomic positions in a molecule are constained by three bond-length
constraints. 
\begin{eqnarray}
d(H_1H_2)=2 \sin\left(\frac{\angle(HOH)}{2}\right) d(OH)
\end{eqnarray}
The position of the dummy atom is expressed by the real atomic positions
\begin{eqnarray}
\vec{R}_M&=&w_O \vec{R}_O+w_h\left(\vec{R}_{H_1}+\vec{R}_{H_2}\right)
\nonumber\\
w_O&=&\frac{d(O,M)}{\cos\left(\frac{\angle(HOH)}{2}\right) d(OH)}
\\
w_H&=&\frac{1-w_O}{2}
\end{eqnarray}
%====================================================================
\section{Implementation}
%====================================================================
The TIP4P object is an object with multiple instances. Thus it must be
selected with an idenitifier and unselected with the id 'none'.

\begin{itemize}
\item TIP4P\$SELECT(ID): selects an existing instance or creates a new
  one. The instance is unselected with id='none'. The first time, any
  instance is selected also the constants of the TIP4P model are
  calculated and stored in the module.
%
\item TIP4P\$DUMMY\_POSITIONS(NAT,R): recalculates the position of the dummy
  atoms from the positions of the real atoms.
%
\item TIP4P\$DUMMY\_FORCES(NAT,F): maps the forces acting on the dummy atom
  onto the real atoms. The force on  the dummy atom is set to zero.
%
\item TIP4P\$CONSTRAINTS(NAT,MASS,R0,RP): enforces the constraints
  that keep the structure of each water molecule rigid.
\end{itemize}

The object sets up an array that holds the pointers to the four atoms
in a molecule. (After this step, the intramolecular bonds could be
removed from the bond array.) This is done with TIP4P\$COLLECT().  In
order to avoid that TIP4P\$COLLECT is executed several time, the
variable \verb|this\%nwater| is initialized to -1. TIP4P\$COLLECT is
executed only if \verb|this\%nwater| is negative. After execution it
is either zero or positive.

\subsection{Integration into the classical object}

I have added the atom types O\_TIP4P,M\_TIP4P, and H\_TIP4P to the UFF
atom types. By setting the effective charges to zero, I am avoiding
any intramolecular bond or angle terms. The van der Waals parameters
have been set to zero except for the oxygen atom. The van-der Waals
parameters of TIP4P are imported into the UFF table.

The atomic charges of the TIP4P water molecules are overwritten by the
TIP4P charges.

There are two operations that differ from the usual UFF operations,
namely the treatment of the dummy atom M and the constraints that
enforce the molecular structure.


\appendix
%====================================================================
\chapter{Frontend of the classical object}
\label{sec:frontend}
%====================================================================
This section contains a front-end of the classical object, which
allows to adress the classical object directly.


An example input file \verb|camno3.in| is given here
\begin{verbatim} 
!MDCNTL
  !GENERIC LUNIT=7.218756758 LONGRANGE=F TRACE=F !END
  !RDYN DT=5. FRIC=0. NSTEP=10000 !END
  !LATTICE t=1.0 0.0 0.0 0.0 1.0 0.0 0.0 0.0 1.0 !end
  !ATOM NAME='CA1'  R=0.0  0.0  0.0  TYPE='CA6+2' q= 2. !END
  !ATOM NAME='MN2'  R=0.5  0.5  0.5  TYPE='MN6+2' q= 4. !END
  !ATOM NAME='O_3'  R=0.5  0.51 0.0  TYPE='O_1'   q=-2. !END
  !ATOM NAME='O_4'  R=0.5  0.0  0.5  TYPE='O_1'   q=-2.  !END
  !ATOM NAME='O_5'  R=0.0  0.5  0.5  TYPE='O_1'   q=-2.  !END
  !BOND ATOM1='MN2' ATOM2='O_3:000' !END
  !BOND ATOM1='MN2' ATOM2='O_3:001' !END
  !BOND ATOM1='MN2' ATOM2='O_4:000' !END
  !BOND ATOM1='MN2' ATOM2='O_4:010' !END
  !BOND ATOM1='MN2' ATOM2='O_5:000' !END
  !BOND ATOM1='MN2' ATOM2='O_5:100' !END
!END
!EOB
\end{verbatim}

Here is the source code of the front-end.
\begin{verbatim} 
     PROGRAM MAIN
     USE LINKEDLIST_MODULE
     USE STRINGS_MODULE
     IMPLICIT NONE
     CHARACTER(128) :: ROOT
     CHARACTER(64)  :: ID
     REAL(8)        :: DELTA
     REAL(8)        :: ANNE
     REAL(8)        :: EPOT
     REAL(8)        :: EKIN
     REAL(8)        :: ETOT
     REAL(8)        :: PICO,SECOND
     INTEGER(4)     :: NITER=100
     INTEGER(4)     :: ITER
     INTEGER(4)     :: NFILO
     INTEGER(4)     :: NFIL
     CHARACTER(128) :: INFILE
     TYPE(LL_TYPE)  :: LL_IN
     LOGICAL(4)     :: TCHK
     INTEGER(4)     :: NARGS
!    *******************************************************************
     CALL CONSTANTS$GET('PICO',PICO)
     CALL CONSTANTS$GET('SECOND',SECOND)
     ID='MD'    
     CALL LIB$NARGS(NARGS)
     IF(NARGS.LT.1) THEN
       CALL ERROR$MSG('THE NAME OF THE ROOTFILE MUST BE GIVEN')
       CALL ERROR$STOP('MAIN')
     END IF
     CALL LIB$GETARG(1,ROOT)
     CALL FILEHANDLER$SETROOT(ROOT)
     CALL FILEHANDLER$SETFILE(+'PROT',.TRUE.,-'.PROT')
     CALL FILEHANDLER$SETSPECIFICATION(+'PROT','STATUS','UNKNOWN')
     CALL FILEHANDLER$SETSPECIFICATION(+'PROT','POSITION','REWIND')
     CALL FILEHANDLER$SETSPECIFICATION(+'PROT','ACTION','WRITE')
     CALL FILEHANDLER$SETSPECIFICATION(+'PROT','FORM','FORMATTED')
     CALL FILEHANDLER$SETFILE('IN',.TRUE.,-'.IN')
     CALL FILEHANDLER$SETSPECIFICATION('IN','STATUS','OLD')
     CALL FILEHANDLER$SETSPECIFICATION('IN','POSITION','REWIND')
     CALL FILEHANDLER$SETSPECIFICATION('IN','ACTION','READ')
     CALL FILEHANDLER$SETSPECIFICATION('IN','FORM','FORMATTED')
!
!    ===================================================================
!    ==  WRITE HEADER                                                 ==
!    ===================================================================
     CALL FILEHANDLER$UNIT(+'PROT',NFILO)
     WRITE(NFILO,FMT='(" ")')
     WRITE(NFILO,FMT='(72("*"))')
     WRITE(NFILO,FMT='(72("*"),T15," MD TEST PROGRAM ")')
     WRITE(NFILO,FMT='(72("*"))')
!
!    ===================================================================
!    ==  READ BUFFER OF INPUT FILE                                    ==
!    ===================================================================
     CALL LINKEDLIST$NEW(LL_IN)
     CALL FILEHANDLER$UNIT('IN',NFIL)
     CALL LINKEDLIST$READ(LL_IN,NFIL)
!    CALL LINKEDLIST$REPORT(LL_IN,NFILO)
!
!    ===================================================================
!    ==  SWITCH TRACE INFORMATION OFF OR ON                           ==
!    ===================================================================
     CALL LINKEDLIST$SELECT(LL_IN,'~')
     CALL LINKEDLIST$SELECT(LL_IN,'MDCNTL')
     CALL LINKEDLIST$SELECT(LL_IN,'GENERIC')
     CALL LINKEDLIST$EXISTD(LL_IN,'TRACE',1,TCHK)
     IF(TCHK) CALL LINKEDLIST$GET(LL_IN,'TRACE',1,TCHK)
     CALL TRACE$SETL4('ON',TCHK)
!
!    ===================================================================
!    ==  READ CONTROL DATA                                            ==
!    ===================================================================
     CALL LINKEDLIST$SELECT(LL_IN,'~')
     CALL LINKEDLIST$SELECT(LL_IN,'MDCNTL')
     CALL LINKEDLIST$SELECT(LL_IN,'RDYN')
!    __ TIMESTEP________________________________________________________
     DELTA=10.D0
     CALL LINKEDLIST$EXISTD(LL_IN,'DT',1,TCHK)
     IF(TCHK)CALL LINKEDLIST$GET(LL_IN,'DT',1,DELTA)
!    __FRICTION_________________________________________________________
     ANNE=0.D0
     CALL LINKEDLIST$EXISTD(LL_IN,'FRIC',1,TCHK)
     IF(TCHK)CALL LINKEDLIST$GET(LL_IN,'FRIC',1,ANNE)
!    __FRICTION_________________________________________________________
     NITER=100
     CALL LINKEDLIST$EXISTD(LL_IN,'NSTEP',1,TCHK)
     IF(TCHK)CALL LINKEDLIST$GET(LL_IN,'NSTEP',1,NITER)
!
!    ===================================================================
!    ==  INITIALIZE CLASSICAL OBJECT                                  ==
!    ===================================================================
     CALL SETUPCLASSICAL(LL_IN,ID)
     CALL FILEHANDLER$UNIT(+'PROT',NFILO)
     CALL CLASSICAL$SETI4('LOD',3)
     CALL CLASSICAL$REPORT(NFILO)
!
!    ===================================================================
!    ==  RUN CLASSICAL                                                ==
!    ===================================================================
     WRITE(NFILO,FMT='("#",A9,A10,3A15)')'ITER','T[PS]','ETOT','EKIN','EPOT'
     DO ITER=1,NITER
       CALL CLASSICAL$NEIGHBORS
       CALL CLASSICAL$ETOT(EPOT)
       CALL CLASSICAL$PROPAGATE(DELTA,ANNE)
       CALL CLASSICAL$EKIN(DELTA,EKIN)
       ETOT=EPOT+EKIN
       WRITE(NFILO,FMT='(I10,F10.3,3F15.5)')ITER &
    &                             ,DBLE(ITER-1)*DELTA/PICO/SECOND,ETOT,EKIN,EPOT
       CALL CLASSICAL$SWITCH
     ENDDO
!
!    ===================================================================
!    ==  REPORT AND CLOSE DOWN                                        ==
!    ===================================================================
     CALL CLASSICAL$REPORT(NFILO)
     WRITE(NFILO,FMT='()')
     WRITE(NFILO,FMT='(72("*"))')
     WRITE(NFILO,FMT='(72("*"),T15," MD TEST PROGRAM FINISHED ")')
     WRITE(NFILO,FMT='(72("*"))')
     STOP
     END

!    ...................................................................
     SUBROUTINE SETUPCLASSICAL(LL_IN_,ID)
     USE LINKEDLIST_MODULE
     USE PERIODICTABLE_MODULE
     IMPLICIT NONE
     CHARACTER(*) ,INTENT(IN)  :: ID
     TYPE(LL_TYPE),INTENT(IN)  :: LL_IN_
     TYPE(LL_TYPE)             :: LL_IN
     INTEGER(4)                :: NAT
     INTEGER(4)                :: NBOND
     REAL(8)      ,ALLOCATABLE :: R0(:,:)     !(3,NAT)
     REAL(8)      ,ALLOCATABLE :: RM(:,:)     !(3,NAT)
     REAL(8)      ,ALLOCATABLE :: QEL(:)      !(NAT)
     REAL(8)      ,ALLOCATABLE :: MASS(:)     !(NAT)
     CHARACTER(5) ,ALLOCATABLE :: TYPE(:)     !(NAT)
     CHARACTER(16),ALLOCATABLE :: NAME(:)     !(NAT)
     CHARACTER(16)             :: NAME1
     CHARACTER(16)             :: NAME2
     CHARACTER(32)             :: XNAME
     CHARACTER(2)              :: EL
     INTEGER(4)   ,ALLOCATABLE :: INDEX2(:,:) !(2,NBOND)
     REAL(8)      ,ALLOCATABLE :: BO(:)       !(NBOND)
     INTEGER(4)                :: IAT,IB
     LOGICAL(4)                :: TCHK
     REAL(8)                   :: PROTONMASS
     REAL(8)                   :: SVAR
     REAL(8)                   :: LUNIT
     REAL(8)                   :: RBAS(3,3)
     INTEGER(4)                :: IT(3)
     LOGICAL(4)                :: TLONGRANGE
!    *******************************************************************
     CALL CONSTANTS$GET('U',PROTONMASS)
     LL_IN=LL_IN_
     CALL LINKEDLIST$SELECT(LL_IN,'~')
     CALL LINKEDLIST$SELECT(LL_IN,'MDCNTL')
     CALL LINKEDLIST$SELECT(LL_IN,'GENERIC')
!    __READ LENGTH UNIT________________________________________________
     LUNIT=1
     CALL LINKEDLIST$EXISTD(LL_IN,'LUNIT',1,TCHK)
     IF(TCHK) CALL LINKEDLIST$GET(LL_IN,'LUNIT',1,LUNIT)
!
     TLONGRANGE=.FALSE.
     CALL LINKEDLIST$EXISTD(LL_IN,'LONGRANGE',1,TCHK)
     IF(TCHK) CALL LINKEDLIST$GET(LL_IN,'LONGRANGE',1,TLONGRANGE)

     CALL LINKEDLIST$SELECT(LL_IN,'~')
     CALL LINKEDLIST$SELECT(LL_IN,'MDCNTL')
     CALL LINKEDLIST$NLISTS(LL_IN,'ATOM',NAT)
     CALL LINKEDLIST$NLISTS(LL_IN,'BOND',NBOND)
!
!    ===================================================================
!    ==  ALLOCATE ARRAYS                                              ==
!    ===================================================================
     ALLOCATE(R0(3,NAT))
     ALLOCATE(RM(3,NAT))
     ALLOCATE(QEL(NAT))
     ALLOCATE(MASS(NAT))
     ALLOCATE(TYPE(NAT))
     ALLOCATE(NAME(NAT))
     ALLOCATE(INDEX2(5,NBOND))
     ALLOCATE(BO(NBOND))
!
!    ===================================================================
!    ==  READ DATA                                                    ==
!    ===================================================================
     CALL LINKEDLIST$EXISTL(LL_IN,'LATTICE',1,TCHK)
     IF(TCHK) THEN
       CALL LINKEDLIST$SELECT(LL_IN,'LATTICE',0)
       CALL LINKEDLIST$EXISTD(LL_IN,'T',1,TCHK)
       IF(.NOT.TCHK) THEN
         CALL ERROR$MSG('LATTICE:T NOT SPECIFIED')
         CALL ERROR$STOP('SETUPCLASSICAL')
       END IF
       CALL LINKEDLIST$GET(LL_IN,'T',1,RBAS)
       RBAS(:,:)=RBAS(:,:)*LUNIT
       CALL LINKEDLIST$SELECT(LL_IN,'..')
     ELSE
       RBAS(:,1)=(/1.0D0,0.D0,0.D0/)
       RBAS(:,2)=(/0.0D0,1.D0,0.D0/)
       RBAS(:,3)=(/0.0D0,0.D0,1.D0/)
     END IF
!
     DO IAT=1,NAT
       CALL LINKEDLIST$SELECT(LL_IN,'ATOM',IAT)
!      __READ NAME_______________________________________________________
       CALL LINKEDLIST$EXISTD(LL_IN,'NAME',1,TCHK)
       IF(.NOT.TCHK) THEN
         CALL ERROR$MSG('ATOM:NAME NOT SPECIFIED')
         CALL ERROR$STOP('SETUPCLASSICAL')
       END IF
       CALL LINKEDLIST$GET(LL_IN,'NAME',1,NAME(IAT))
!      __READ TYPE_______________________________________________________
       CALL LINKEDLIST$EXISTD(LL_IN,'TYPE',1,TCHK)
       IF(.NOT.TCHK) THEN
         CALL ERROR$MSG('ATOM:TYPE NOT SPECIFIED')
         CALL ERROR$STOP('SETUPCLASSICAL')
       END IF
       CALL LINKEDLIST$GET(LL_IN,'TYPE',1,TYPE(IAT))
!      __READ POSITION___________________________________________________
       CALL LINKEDLIST$EXISTD(LL_IN,'R',1,TCHK)
       IF(.NOT.TCHK) THEN
         CALL ERROR$MSG('ATOM:R NOT SPECIFIED')
         CALL ERROR$STOP('SETUPCLASSICAL')
       END IF
       CALL LINKEDLIST$GET(LL_IN,'R',1,R0(:,IAT))
       R0(:,IAT)=R0(:,IAT)*LUNIT
!      __READ MASS_______________________________________________________
       CALL LINKEDLIST$EXISTD(LL_IN,'M',1,TCHK)
       IF(TCHK) THEN
         CALL LINKEDLIST$GET(LL_IN,'M',1,MASS(IAT))
       ELSE
         EL=TYPE(IAT)
         CALL PERIODICTABLE$GET(TYPE(IAT)(1:2),'MASS',SVAR)
         MASS(IAT)=SVAR
       END IF
!
!      __READ CHARGE_______________________________________________________
       CALL LINKEDLIST$EXISTD(LL_IN,'Q',1,TCHK)
       IF(TCHK) THEN
         CALL LINKEDLIST$GET(LL_IN,'Q',1,QEL(IAT))
       ELSE
         QEL(IAT)=0.D0
       END IF

       CALL LINKEDLIST$SELECT(LL_IN,'..')
     ENDDO
     RM(:,:)=R0(:,:)
!
!    ===================================================================
!    ==  READ BONDS                                                   ==
!    ===================================================================
     DO IB=1,NBOND
       CALL LINKEDLIST$SELECT(LL_IN,'BOND',IB)
!      __READ FIRST ATOM________________________________________________
       CALL LINKEDLIST$EXISTD(LL_IN,'ATOM1',1,TCHK)
       IF(.NOT.TCHK) THEN
         CALL ERROR$MSG('BOND:ATOM1 NOT SPECIFIED')
         CALL ERROR$STOP('SETUPCLASSICAL')
       END IF
       CALL LINKEDLIST$GET(LL_IN,'ATOM1',1,NAME1)
       DO IAT=1,NAT
         IF(NAME1.EQ.NAME(IAT))INDEX2(1,IB)=IAT
       ENDDO
!      __READ SECOND ATOM_______________________________________________
       CALL LINKEDLIST$EXISTD(LL_IN,'ATOM2',1,TCHK)
       IF(.NOT.TCHK) THEN
         CALL ERROR$MSG('BOND:ATOM2 NOT SPECIFIED')
         CALL ERROR$STOP('SETUPCLASSICAL')
       END IF
       CALL LINKEDLIST$GET(LL_IN,'ATOM2',1,XNAME)
       CALL MD_RESOLVEEXTENDEDNAME(XNAME,NAME2,IT)
       DO IAT=1,NAT
         IF(NAME2.EQ.NAME(IAT))INDEX2(2,IB)=IAT
       ENDDO
       INDEX2(3:5,IB)=IT 

!      __READ BOND ORDER_________________________________________________
       CALL LINKEDLIST$EXISTD(LL_IN,'BO',1,TCHK)
       IF(TCHK) THEN
         CALL LINKEDLIST$GET(LL_IN,'BO',1,BO(IB))
       ELSE
         BO(IB)=1.D0
       END IF
!      _________________________________________________________________
       IF(INDEX2(1,IB).EQ.0) THEN
         CALL ERROR$MSG('BOND PARTNER ATOM1 NOT RECOGNIZED')
         CALL ERROR$CHVAL('ATOM1',NAME1)
         CALL ERROR$CHVAL('ATOM2',NAME2)
         CALL ERROR$STOP('SETUPCLASSICAL')
       END IF
       IF(INDEX2(2,IB).EQ.0) THEN
         CALL ERROR$MSG('BOND PARTNER ATOM2 NOT RECOGNIZED')
         CALL ERROR$CHVAL('ATOM1',NAME1)
         CALL ERROR$CHVAL('ATOM2',NAME2)
         CALL ERROR$STOP('SETUPCLASSICAL')
       END IF
       CALL LINKEDLIST$SELECT(LL_IN,'..')
     ENDDO
!
!    ===================================================================
!    ==  INITIALIZE CLASSICAL                                         ==
!    ===================================================================
     CALL CLASSICAL$SELECT(ID)
     CALL CLASSICAL$SETI4('NAT',NAT)
     CALL CLASSICAL$SETI4('NBOND',NBOND)
     CALL CLASSICAL$SETR8A('R(0)',3*NAT,R0)
     CALL CLASSICAL$SETR8A('CELL(0)',3*3,RBAS)
     CALL CLASSICAL$SETR8A('R(-)',3*NAT,RM)
     CALL CLASSICAL$SETCHA('TYPE',NAT,TYPE)
     CALL CLASSICAL$SETR8A('MASS',NAT,MASS)
     CALL CLASSICAL$SETR8A('QEL',NAT,QEL)
     CALL CLASSICAL$SETI4A('BOND',5*NBOND,INDEX2)
     CALL CLASSICAL$SETR8A('BONDORDER',NBOND,BO)
     CALL CLASSICAL$SETL4('LONGRANGE',TLONGRANGE)
!
!    ===================================================================
!    ==  DEALLOCATE ARRAYS                                            ==
!    ===================================================================
     DEALLOCATE(R0)
     DEALLOCATE(RM)
     DEALLOCATE(QEL)
     DEALLOCATE(MASS)
     DEALLOCATE(TYPE)
     DEALLOCATE(INDEX2)
     DEALLOCATE(BO)
     RETURN
     END
!
!     ...1.........2.........3.........4.........5.........6.........7.........8
      SUBROUTINE MD_RESOLVEEXTENDEDNAME(XNAME,NAME,IT)
!     **************************************************************************
!     **  RESOLVES THE EXTENDED ATOM NAME NOTATION, WHICH INCLUDES            **
!     **  A LATTICE TRANSLATION                                               **
!     **                                                                      **
!     **  THE EXTENDED NOTATION INCLUDES AN INTEGER LATTICE TRANSLATIONS      **
!     **  IN THE ATOM NAME FOLLOWING A COLON                                  **
!     **                                                                      **
!     **   'O_23:+1-1+1'  ATOM 'O_23' SHIFTED BY RBAS(:,1)-RBAS(:,2)+RBAS(:,3)**
!     **                                                                      **
!     **   THE '+'SIGNS ARE NOT REQUIRED.                                     **
!     **   ONLY SINGLE-DIGIT TRANSLATIONS ARE PERMITTED                       **
!     **                                                                      **
!     **   ORIGINALLY: STRCIN_RESOLVEEXTENDEDNAME                             **
!     **                                                                      **
!     **************************************************************************
      IMPLICIT NONE
      CHARACTER(*),INTENT(IN) :: XNAME  ! EXTENDED ATOM NAME
      CHARACTER(*),INTENT(OUT):: NAME   ! NON-EXTENDED ATOM NAME
      INTEGER(4)  ,INTENT(OUT):: IT(3)  ! INTEGER LATTICE TRANSLATIONS
      INTEGER(4)              :: ICOLON ! POSITION OF THE COLON IN XNAME
      INTEGER(4)              :: IPOS,IND,SGN
      INTEGER(4)              :: ICH    ! ASCII NUMBER OF THE SELECTED LETTER
!     **************************************************************************
      ICOLON=INDEX(XNAME,':')
!     == RETURN IF NO TRANSLATION VECTOR GIVEN =================================
      IF(ICOLON.EQ.0) THEN
        NAME=XNAME
        IT(:)=0
        RETURN
      END IF
!
!     ==========================================================================
!     == RESOLVE EXTENDED ATOM NAME                                           ==
!     ==========================================================================
      NAME=XNAME(:ICOLON-1)
      IPOS=ICOLON+1
      IND=0
      SGN=+1
      DO WHILE(IND.LT.3) 
        ICH=IACHAR(XNAME(IPOS:IPOS))
!       ==  IACHAR('+')=43; IACHAR('-')=45; IACHAR('0')=48; IACHAR('1')=49;...
        IF(ICH.GE.48.AND.ICH.LE.57) THEN
          IND=IND+1
          IT(IND)=SGN*(ICH-48)
          SGN=+1
        ELSE IF(ICH.EQ.43) THEN
          SGN=+1
        ELSE IF(ICH.EQ.45) THEN
          SGN=-1
        ELSE
          CALL ERROR$MSG('ILLEGAL CHARACTER IN EXTENDED ATOM NOTATION')  
          CALL ERROR$CHVAL('EXT. NAME ',XNAME)
          CALL ERROR$CHVAL('ILLEGAL CHARACTER ',XNAME(IPOS:IPOS))
          CALL ERROR$STOP('STRCIN_RESOLVEEXTENDEDNAME')
        END IF
        IPOS=IPOS+1
      ENDDO
      IF(XNAME(IPOS:).NE.' ') THEN
        CALL ERROR$MSG('LETTERS FOUND BEYOND END OF EXTENDED ATOM NOTATION')  
        CALL ERROR$CHVAL('EXT. NAME ',XNAME)
        CALL ERROR$CHVAL('ADDITIONAL LETTERS ',XNAME(IPOS:))
        CALL ERROR$STOP('STRCIN_RESOLVEEXTENDEDNAME')
      END IF
      RETURN
      END
\end{verbatim}

%====================================================================
\chapter{Errata on the original UFF paper.}
%====================================================================
These notes are taken from the UFF manual of the ADF Program system
Release 2014. (They are blanked out in the pdf version to avoid
copyright conflicts, when distributing the pdf.)


%% %==============================================================================
%% \subsubsection{Implementation of the Universal Force Field (UFF) in deMonNano}
%% %==============================================================================
%% As far as possible, UFF molecular mechanics forcefield in deMon
%% follows the published forcefield definition in
%% \cite{rappe92_jacs114_10024}.  In several cases, the definitions and
%% expressions in \cite{rappe92_jacs114_10024} are not consistent with
%% the published applications of the forcefield
%% \cite{rappe92_jacs114_10024,casewit92_jacs114_10035,casewit92_jacs114_11046}.
%% In those cases, an attept was made to correct the errors and
%% omissions, using information
%% from\cite{rappe91_jpc95_3358,okeefe92_jacs113_3226,mayo90_jpc90_8897}.
%% The following changed were made, compared to the published UFF
%% forcefield description (all equation and page numbers refer to
%% \cite{rappe92_jacs114_10024}. ).

%% 1. Sign error in Eq. 2 (equilibrium bond length) was corrected -
%% electronegativity correction must be negative!

%% 2. Equilibrium valence angle for O\_3\_z was corrected from 146.0 degree
%% to 145.45 degree.

%% 3. Bending periodicity (Eq. 10) for linear coordination was corrected
%% from 1 to 2.

%% 4. Sign errors were corrected in eqs. 13 and and unnumbered equation
%% for the beta parameter (between eqs. 13 and 14).

%% 5. The reference value of the UFF amide force constant, of 105.5
%% kcal/mol/rad**2 (p. 10028) is wrong. The results are consistent with
%% the force constant of 211.0 kcal/mol/rad**2.

%% 6. Equilibrium torsional angle for a bond between a group-6A atom
%% (oxygen ...) and an sp2 atom (90 degree) is wrong (p. 10028).  It
%% should be 0 degree.

%% 7. The conditional for the special-case sp2-sp3 torsion (p. 10029) is
%% wrong, and should be inverted - see \cite{mayo90_jpc90_8897}.

%% 8. The overall shape of the UFF torsional potential degenerates to a
%% Heavyside function when one of the bond angles becomes linear, leading
%% to failures in geometry optimization and force constant
%% evaluation. The UFF torsional term was augmented with a smooth masking
%% function, to avoid this. See "uff\_4centre.f90" for details.

%% 9. UFF inversion potential is not defined in
%% \cite{rappe92_jacs114_10024} for group 5A elements (from phosphorus
%% down). Taking the equilibrium inversion coordinate of 87 degree, and
%% the suitable expressions for the cosine weights (see
%% uff\_get\_inversion\_shape in "uff\_database.f90") appears to reproduce
%% published UFF structures and energetics.


\printindex
\clearpage
\bibliographystyle{unsrtnat} 
 \bibliography{../all}
\end{document}  
 
