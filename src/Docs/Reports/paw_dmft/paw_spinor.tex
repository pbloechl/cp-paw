\documentclass[11pt,a4paper]{report}
%%%%%%%%%%%%%%%%%%%%%%%%%%%%%%%%%%%%%%%%%%%%%%%%%%%%%%%%%%%%%%%%%%%%%
%%                                                                 %%
%%    Header file for the Phi-S-X Series                           %%
%%                                                                 %%
%%    german version header_gm.tex is derived from header.tex      %%
%%    by uncommenting the line ``\setboolean{german}{true}'' below %%
%%                                                                 %%
%%    Never edit the german version! all changes must be done      %%
%%    in the english version header.tex                            %%
%%                                                                 %%
%%%%%%%%%%%%%%%%%%%%%%%%%%%%%%%%%%%%%%%%%%%%%%%%%%%%%%%%%%%%%%%%%%%%%
%%%%%%%%%%%%%%%%%%%%%%%%%%%%%%%%%%%%%%%%%%%%%%%%%%%%%%%%%%%%%%%%%%%%%
%%                                                                 %%
%%    Header file for the Phi-S-X Series                           %%
%%                                                                 %%
%%    german version header_gm.tex is derived from header.tex      %%
%%    by uncommenting the line ``\setboolean{german}{true}'' below %%
%%                                                                 %%
%%    Never edit the german version! all changes must be done      %%
%%    in the english version header.tex                            %%
%%                                                                 %%
%%%%%%%%%%%%%%%%%%%%%%%%%%%%%%%%%%%%%%%%%%%%%%%%%%%%%%%%%%%%%%%%%%%%%
%====================================================================
%-- define flag for language adaptations
\usepackage{ifthen}   % allows to select only certain text
\provideboolean{german}
\setboolean{german}{false}
%\setboolean{german}{true}  % uncomment this line for german editions
%====================================================================
%
% Textschriftart: Computer modern Bright
% body:            CM-Bright 10pt
% section titles:  CM-Bright Bold
% formulas:        CM-Bright Math Oblique
%
\usepackage[standard-baselineskips]{cmbright}
\usepackage{cmbright}
\usepackage[T1]{fontenc}
\def\usedfonts{CM-Bright}
\usepackage{typearea}
%\typearea[current]{calc} % benutzt die aktuelle 
       % bindekorrektur (BCOR angabe als parameter in koma usepackage)
       % und berechnet satzspiegel neu
\typearea[current]{11} %fixed div value

\usepackage{textcomp} % special symbols
\usepackage{amsfonts} % special symols
                      % see ftp://ftp.ams.org/pub/tex/doc/amsfonts/amsfndoc.pdf
\usepackage{amssymb}  % CM-Bright provides the AMS symbols
\usepackage{exscale}  % allows to scale math expressions to big fonts, 
                      % e.g. \Huge
\usepackage{curves}
\usepackage{braket}
\usepackage{miller}     % miller indices
\usepackage{chemmacros} % http://www.mychemistry.eu/mychemistry/
\usepackage[numbers]{natbib}     % bibliography style
\usepackage{url}\urlstyle{tt}
\usepackage{float}
\usepackage{bm}       % provides the command \bm{} that makes bold math symbols
\usepackage{amsmath}
\usepackage{amsbsy}   % allows bold mathematical symbols
\usepackage{amscd}
 \usepackage{a4wide}  % it is better to use the ``geometry'' package
\usepackage{array}    % 
\usepackage{fancyhdr} %  defines pagestyle fancy
\usepackage{epsfig}   % include graphics with epsfig
\usepackage{graphicx} % includegraphics
\usepackage{epstopdf}
\usepackage{wrapfig}
\usepackage{fancybox} % allows shadow-boxes
\usepackage{color}    % allows to use color in the text
%\usepackage{eepic}
\usepackage{flafter}  % places picture next to its reference
\usepackage{makeidx}  % make an index
%\usepackage{MnSymbol}  % 
%\usepackage{marvosym}  % 
\usepackage{textcase}
\usepackage{ulem} % defines strikeout \sout{}; underline \uline{}
                  % double underline \uuline{}; wave underline \uwave{}
                  % cross out \xout{}
%
%==========================================================================
%==  page layout  =========================================================
%==========================================================================
% eqnarray environment: reduce with of space in place of each ``&''
\setlength\arraycolsep{1.4pt}
\pagestyle{fancy}
%\renewcommand{\chaptermark}[1]{\markboth{\thechapter\ #1}{}}
\renewcommand{\chaptermark}[1]{\markboth{\MakeUppercase{\thechapter\ #1}}{}}
\fancyhf{} 
\fancyhead[LE]{\textsc{\thepage}\qquad\textsc{\leftmark}}
\fancyhead[RO]{\textsc{\leftmark}\qquad\textsc{\thepage}}
\renewcommand{\headrulewidth}{0.5pt}
\renewcommand{\footrulewidth}{0pt} 
\addtolength{\headheight}{2.5pt}
\fancypagestyle{plain}{\fancyhead{}
   \renewcommand{\headrulewidth}{0pt}
   \fancyfoot[CO]{\bfseries\thepage}}

% Line spacing -----------------------------------------------------------
\newlength{\defbaselineskip}
\setlength{\defbaselineskip}{\baselineskip}
\newcommand{\setlinespacing}[1]%
           {\setlength{\baselineskip}{#1 \defbaselineskip}}
\newcommand{\doublespacing}{\setlength{\baselineskip}%
                           {2.0 \defbaselineskip}}
\newcommand{\singlespacing}{\setlength{\baselineskip}{\defbaselineskip}}

% Absatz einr\"ucken ------------------------------------------------------
%\setlength{\parindent}{0pt}
\setlength{\parskip}{2pt}
% -------------------------------------------------------------------------
\ifthenelse{\boolean{german}}
  {\def\figurename{Abb.}}
  {\def\figurename{Fig.}}
%--------------------------------------------------------------------------
\renewcommand{\arraystretch}{1.15}  % skaliert den Zeilen abstand in der 
    % tabular und array umgebung
%
%==========================================================================
%==  boxes etc ============================================================
%==========================================================================
%== minipage in a shadowbox ===============================================
\newenvironment{myshadowminipage}[1]%
  {\par\noindent\begin{Sbox}\begin{minipage}{\linewidth}\vspace{0.1cm}\begin{center}\uppercase{#1}\end{center}}%
  {\vspace{0.1cm}\end{minipage}\end{Sbox}\shadowbox{\TheSbox}}
%
%== minipage in a framedbox ===============================================
\newenvironment{myframedminipage}%
  {\par\noindent\begin{Sbox}\begin{minipage}\linewidth\vspace{0.1cm}}%
  {\vspace{0.1cm}\end{minipage}\end{Sbox}\fbox{\TheSbox}}
%
\newcommand{\myshadowbox}[1]{\noindent\shadowbox{\parbox{\linewidth}{\smallskip #1\smallskip}}}
\newcommand{\myfbox}[1]{\noindent\fbox{\parbox{\linewidth}{\smallskip #1\smallskip}}\medskip}
%== minipage in a framedbox ===============================================
\newtheorem{defi}{Definition}[chapter]
\newenvironment{definition}[1]%
  {\par\noindent\begin{Sbox}\begin{minipage}{\linewidth}\vspace{0.1cm}\begin{defi}\uppercase{#1}\\\vspace{0.1cm}}%
  {\vspace{0.1cm}\end{defi}\end{minipage}\end{Sbox}\shadowbox{\TheSbox}}
%
%=========================================================================
% color used to point out information to the teacher
\definecolor{highlight}{rgb}{1.0,0.7,0.}
\newcommand{\Special}[1]{\textbf{\textcolor{highlight}{#1}}}
%=========================================================================
%  switch certain parts on and off. uses ifthen package
\newboolean{teacher}\setboolean{teacher}{false}
% this parameter can be changed in the manuscript again
\setboolean{teacher}{true} %private version if true!
\newcommand{\teacheronly}[1]{\ifthenelse{\boolean{teacher}}{#1\hfill\\ }}
\newcommand{\editor}[1]{\textcolor{blue}{\texttt{Editor: #1}}}
\newcommand{\MARK}[1]{\textcolor{blue}{#1}} 
\newcommand{\RED}[1]{\textcolor{red}{#1}} 
%
%==========================================================================
%==  define new symbols                                                 ===
%==========================================================================
% define \stat (stationary state) as an operator like \min
\DeclareMathOperator*{\stat}{stat}
\let\Vec=\mathbold   % cmbright.sty provides a bold/italic math alphabet
\let\Dot=\mathbold   % cmbright.sty provides a bold/italic math alphabet
\let\Ddot=\mathbold   % cmbright.sty provides a bold/italic math alphabet
%
\newcommand{\e}[1]{\mathrm{e}^{#1}}% exponential function
\renewcommand{\Re}{\mathrm{Re}}    % real part
\renewcommand{\Im}{\mathrm{Im}}    % imaginary part
\newcommand{\lagr}{\ell}           % Lagrange dichte
\newcommand{\Lagr}{\mathcal{L}}    % Lagrange Funktion
\newcommand{\erf}{{\rm erf}}       %
\newcommand{\atan}{{\rm atan}}     % arcus tangens
\newcommand{\mat}[1]{\bm{#1}}  % Matrix
\newcommand{\gmat}[1]{{\boldsymbol #1}}  % Matrix(symbol)
\newcommand{\defas}{\stackrel{\text{def}}{=}}  %  is defined as
\ifthenelse{\boolean{german}}
  {\newcommand{\rot}{{\rm\bf rot}}}    % curl
  {\newcommand{\rot}{{\rm\bf curl}}}   % curl
\newcommand{\sgn}{{\rm sgn}}       % sign
\ifthenelse{\boolean{german}}
   {\newcommand{\Tr}{\mathrm{Sp}}}      % trace
   {\newcommand{\Tr}{\mathrm{Tr}}}      % trace
\ifthenelse{\boolean{german}}
   {\newcommand{\grmn}[2]{\footnote{``#2'' hei{\ss}t in englisch ``#1''}}}
   {\newcommand{\grmn}[2]{\footnote{``#1'' translates as ``#2'' into German}}}
% define the equation reference
\ifthenelse{\boolean{german}}
   {\newcommand{\eq}[1]{\text{Gl.}~\ref{#1}}}
   {\newcommand{\eq}[1]{\text{Eq.}~\ref{#1}}}
% define a relation with an equation number ontop
\newcommand{\eqrel}[2]{\stackrel{\eq{#1}}{#2}}
\newcommand{\zero}{\varnothing}
%\newcommand{\ket}[1]{|#1\rangle} % contained in package braket
\newcommand{\sumint}{\int\hspace{-15pt}\sum}
\newcommand{\marker}[1]{\textcolor{blue}{\emph{#1}}}
\renewcommand*{\dot}[1]{\overset{\mbox{\large\bfseries .}}{#1}}
\renewcommand*{\ddot}[1]{\overset{\mbox{\large\bfseries\hspace{+0.1ex}.\hspace{-0.1ex}.}}{#1}}
%
%==========================================================================
%==                                                                     ===
%==========================================================================
% Prevent figures from appearing on a page by themselves
% from http://dcwww.camd.dtu.dk/~schiotz/comp/LatexTips/LatexTips.html
\renewcommand{\topfraction}{0.85}
\renewcommand{\textfraction}{0.1}
\renewcommand{\floatpagefraction}{0.75}
%
%==========================================================================
%==                                                                     ===
%==========================================================================
\makeindex    % make index. uses makeidx package.

%== allow links between documents ============================================
\usepackage{xr}
\usepackage{xr-hyper}
%==  hyperref package (must be last package)
\usepackage[colorlinks=true]{hyperref} %specify this as last package
\hypersetup{citecolor=blue}
\hypersetup{menucolor=magenta}
\hypersetup{urlcolor=blue}      % 
\hypersetup{filecolor=green}    % file links
\hypersetup{linkcolor=magenta}  %table of contents
\hypersetup{pdfauthor={Peter E. Bl\"ochl}}
\hypersetup{pdfdisplaydoctitle=true}
\externaldocument[phisx1-]{/Users/ptpb/Tree/PhiSX/ClassicalMechanics/Book/cm-gm}
\externaldocument[phisx2-]{/Users/ptpb/Tree/PhiSX/Electrodynamics/Book/el-gm}
\externaldocument[phisx3-]{/Users/ptpb/Tree/PhiSX/QuantumMechanics/Book/qm}
\externaldocument[phisx4-]{/Users/ptpb/Tree/PhiSX/StatisticalMechanics/Book/sm}
\externaldocument[phisxqm2-]{/Users/ptpb/Tree/PhiSX/QuantumMechanicsII/Book/qm2}
\externaldocument[phisxsm2-]{/Users/ptpb/Tree/PhiSX/StatisticalMechanicsII/Book/sm2}
\externaldocument[phisxcb-]{/Users/ptpb/Tree/PhiSX/Chemicalbond/Book/cb}
% Example: Figure~PhiSX:Quantum
% Mechanics-\ref{phisx3-fig:doubleslitwave} on page
% \pageref{phisx3-fig:doubleslitwave}


\hypersetup{pdftitle=paw_brillouin}
\newcommand{\petertt}[1]{\textcolor{red}{\texttt{#1}}}
\begin{document}
\begin{titlepage}
\begin{center}
\vspace*{3.5cm}
{\huge \textbf{The SPINOR object of the CP-PAW code}}\\
\vspace{0.5cm}
{\large Peter E. Bl\"ochl}
\vspace{0.5cm} 
\end{center}

\vfill
\begin{center}
Copyright Peter E. Bl\"ochl; Sept.2, 2013-\today\\
{\small
Institute of Theoretical Physics;
Clausthal University of Technology;\\ 
D-38678 Clausthal Zellerfeld; Germany;\\
http://www.pt.tu-clausthal.de/atp/}
\end{center}
\end{titlepage}
\noindent            
\tableofcontents
%====================================================================
\newpage
%====================================================================
\chapter{Todo}
%====================================================================
%====================================================================
\section{Time-inversion symmetry}
%====================================================================
At the moment, we still describe density matrices as real objects in a
(total-density, magnetization-vector-density)-representation. This has
been used because it exploits time-inversion symmetry. The plan is to
change this to complex matrices, because the method can then be
generalized to non-equilibrium problems, such as calculations with a
fixed current density.


Regarding time-inversion symmetry there are still a few open
questions. Even after lifting the restriction to real density matrices
in a (total-density, magnetization-vector-density) representation, it
may be convenient to separate (large) equilibrium quantities from
(small) non-equilibrium quantities.

 The implications of time-inversion symmetry for
non-collinear calculations are unclear to me. Back-ground on
time-inversion symmetry can be found in P. B\"ochl, \textit{$\Phi$SX:
  Advanced topics of theoretical physics II, The electronic structure
  of matter}.
\begin{itemize}
\item Is a non-collinear calculation time-inversion symmetric? 
  In our
  calculations the magnetic field is replaced by a exchange
  correlation potential. The exchange correlation potential has the
  property that it is invariant with respect to a sign-change of the
  spin density. That is a sign-change of the spin induces a sign
  change of the spin-dependent potential. Whether this can be
  exploited for a non-collinear calculation is unclear.
%
\item can time-inversion symmetry be used to relate the wave functions
  with opposite wave vector with each other?
%
\item How does time-inversion symmetry enter in current-density
  functional theory?
\end{itemize}

%====================================================================
\chapter{SPINOR object}
%ATTENTION: This is derived from the section Working wity 
%Spin orbitals of the methods book
%====================================================================
%====================================================================
%\chapter{Working with spin orbitals}
\label{sec:spinorb}
%====================================================================
%====================================================================
\section{Spin orbitals}
\label{sec:spinorbitals}
%====================================================================
An electron is specified by a position and a spin. We combine position
and spin into a pseudo-fourdimensional vector
\begin{eqnarray}
\vec{x}=(\vec{r},\sigma)
\label{eq:spinorbdefx}
\end{eqnarray}
where $\sigma\in\{\uparrow,\downarrow\}$.

An electron wave function naturally obtains a two-component form
\begin{eqnarray}
\psi(\vec{x})=\psi(\vec{r},\sigma)=\langle\vec{r},\sigma|\psi\rangle
=\langle\vec{x}|\psi\rangle
\label{eq:spinorbdefpsi}
\end{eqnarray}

Similarly, we combine sum over spin indices and integration over
position into a quasi-fourdimensional integration
\begin{eqnarray}
\int d^4x\;=\sum_\sigma\int d^3r\;
\label{eq:spinorbdefint}
\end{eqnarray}

The identity operator has the form
\begin{eqnarray}
\hat{1}=\int d^4x\;|\vec{x}\rangle\langle\vec{x}|
\label{eq:spinorbunity}
\end{eqnarray}

%====================================================================
\section{Pauli matrices and observables}
%====================================================================
All hermitean matrices in the two-dimensional spinor space can be
represented as a superposition of the unit matrix and the three Pauli
matrices.  In other words, the Pauli matrices including the unit
element are a complete basis in the space of all complex $2\times2$
matrices. All hermitean matrices $2\times2$ matrices are a
superposition of Pauli matrices with \underline{real} coefficients.

\begin{eqnarray}
\mat{1}=\mat{\sigma}^{(0)}=\left(\begin{array}{rr}1&0\\0&1\end{array}\right)
&\qquad\text{and}\qquad&
\mat{\sigma}_x=\mat{\sigma}^{(1)}=\left(\begin{array}{cc}0&1\\1&0\end{array}\right)
\nonumber\\
\mat{\sigma}_y=\mat{\sigma}^{(2)}=\left(\begin{array}{rr}0&-i\\i&0\end{array}\right)
&\qquad\text{and}\qquad&
\mat{\sigma}_z=\mat{\sigma}^{(3)}=\left(\begin{array}{rr}1&0\\0&-1\end{array}\right)
\label{eq:defpaulimatrices}
\end{eqnarray}


The total density can be expressed by the unit matrix as
\begin{eqnarray*}
\rho(\vec{r})
&=&
-e^2 \sum_{\sigma,\sigma'}
\langle \psi|\vec{r},\sigma\rangle
\sigma^{(0)}_{\sigma,\sigma'}\langle\vec{r},\sigma'|\psi\rangle
=
-e^2 
\langle \psi|\Bigl[|\vec{r}\rangle\langle\vec{r}|\circ \hat{\sigma}^{(0)}\Bigr]
|\psi\rangle
\end{eqnarray*}
where $\hat{\sigma}^{(0)}$ is an operator in the two-dimensional
spinor state.  With the symbol ``$\circ$'' we denote the product or
two operators, where each operator acts in its own Hilbert space.


Similarly, we obtain the spin density in the form
\begin{eqnarray*}
S_j(\vec{r})
&=&
\frac{\hbar}{2}\sum_{\sigma,\sigma'}
\langle \psi|\vec{r},\sigma\rangle
\sigma^{(j)}_{\sigma,\sigma'}\langle\vec{r},\sigma'|\psi\rangle
=
\frac{\hbar}{2}\sum_{\sigma,\sigma'}
\langle \psi|\Bigl[|\vec{r}\rangle\langle\vec{r}|\circ\hat{\sigma}^{(j)}\Bigr]
|\psi\rangle
\end{eqnarray*}



%===================================================================
\subsubsection{Eigenvectors of Pauli matrices}
%===================================================================
The eigenvalue equation is for each $j\in\{x,y,z\}$
\begin{eqnarray}
\sigma^{(j)}\xi^{(+j)}=\xi^{(+j)}
\qquad\text{and}\qquad
\sigma^{(j)}\xi^{(-j)}=-\xi^{(-j)}
\end{eqnarray}
Thus the eigenvalues are $+1$ and $-1$.

The eigenvectors of the Pauli matrices
$\mat{\sigma}_x,\mat{\sigma}_y,\mat{\sigma}_z$ are
\begin{eqnarray}
\xi^{(\pm x)}=\frac{1}{\sqrt{2}}\left(\begin{array}{c}1\\\pm1\end{array}\right)
\qquad\text{and}\qquad
\xi^{(\pm y)}=\frac{1}{\sqrt{2}}\left(\begin{array}{c}1\\\pm i\end{array}\right)
\qquad\text{and}\qquad
\xi^{(\pm z)}=\frac{1}{2}\left(\begin{array}{c}1\pm1\\1\mp1\end{array}\right)
\label{eq:spineigenstatesxyz}
\end{eqnarray}
for the eigenvalues $\pm1$. 


More explicitely, 
\begin{eqnarray}
\xi^{(+ x)}=\frac{1}{\sqrt{2}}\left(\begin{array}{r}1\\1\end{array}\right)
\qquad\text{and}\qquad
\xi^{(+ y)}=\frac{1}{\sqrt{2}}\left(\begin{array}{r}1\\+ i\end{array}\right)
\qquad\text{and}\qquad
\xi^{(+ z)}=\left(\begin{array}{c}1\\0\end{array}\right)
\nonumber\\
\xi^{(- x)}=\frac{1}{\sqrt{2}}\left(\begin{array}{r}1\\-1\end{array}\right)
\qquad\text{and}\qquad
\xi^{(- y)}=\frac{1}{\sqrt{2}}\left(\begin{array}{r}1\\- i\end{array}\right)
\qquad\text{and}\qquad
\xi^{(- z)}=\left(\begin{array}{c}0\\1\end{array}\right)
\end{eqnarray}




%====================================================================
\section{Representation of a matrices in terms of Pauli matrices}
%====================================================================
%====================================================================
\subsubsection{Definition}
%====================================================================
Now we introduce a transformation between the two-dimensional matrix
representation and the total-spin-vector representation for matrices
\begin{myshadowminipage}{Transformation of matrices between 
$(\uparrow,\downarrow)$ and $(t,x,y,z)$ representation}
\begin{eqnarray}
\underbrace{\rho_{\alpha,\beta,\sigma,\sigma'}=\frac{1}{2}
\sum_{j=0}^3\bar{\rho}^{(j)}_{\alpha,\beta}\sigma^{(j)}_{\sigma,\sigma'}
}_{\text{back transform}}
\qquad\stackrel{Eqs.~\ref{eq:pauliorthonormality},\ref{eq:paulidyadicsum}}
{\Longleftrightarrow}\qquad
\underbrace{\bar{\rho}^{(j)}_{\alpha,\beta}\defas\sum_{\sigma,\sigma'} 
\rho_{\alpha,\beta,\sigma,\sigma'}\Bigr(\sigma^{(j)}_{\sigma,\sigma'}\Bigr)^*
}_{\text{forward transform}}
\label{eq:defdensitymatrixwithspindeptotandcart}
\end{eqnarray}
\end{myshadowminipage}


%====================================================================
\subsubsection{Conversion of a matrix from $\uparrow,\downarrow$ 
into $t,x,y,z$ representation and vice versa}
%====================================================================
We transform a matrix using
to \eq{eq:defdensitymatrixwithspindeptotandcart}, which yields for the
forward transformation
\begin{eqnarray}
\rho^{(0)}_{a,b}&=&\rho_{a,b,\uparrow,\uparrow}+\rho_{a,b,\downarrow,\downarrow}
\nonumber\\
\rho^{(x)}_{a,b}&=&\rho_{a,b,\downarrow,\uparrow}+\rho_{a,b,\uparrow,\downarrow}
\nonumber\\
\rho^{(y)}_{a,b}&=&-i
\Bigl(\rho_{a,b,\downarrow,\uparrow}-\rho_{a,b,\uparrow,\downarrow}\Bigr)
\nonumber\\
\rho^{(z)}_{a,b}&=&\rho_{a,b,\uparrow,\uparrow}-\rho_{a,b,\downarrow,\downarrow}
\label{eq:defdensitymatrixforwardexplicit}
\end{eqnarray}
and for the backward transformation
\begin{eqnarray}
\rho_{a,b,\uparrow,\uparrow}&=&\frac{1}{2}\Bigl(\rho^{(0)}_{a,b}+\rho^{(z)}_{a,b}\Bigr)
\nonumber\\
\rho_{a,b,\downarrow,\uparrow}&=&\frac{1}{2}\Bigl(\rho^{(x)}_{a,b}+i\rho^{(y)}_{a,b}\Bigr)
\nonumber\\
\rho_{a,b,\uparrow,\downarrow}&=&\frac{1}{2}\Bigl(\rho^{(x)}_{a,b}-i\rho^{(y)}_{a,b}\Bigr)
\nonumber\\
\rho_{a,b,\downarrow,\downarrow}&=&\frac{1}{2}\Bigl(\rho^{(0)}_{a,b}-\rho^{(z)}_{a,b}\Bigr)
\label{eq:defdensitymatrixbackexplicit}
\end{eqnarray}

In practice we distinguish the three cases, namely non spin-polarized,
collinear spin-polarized, and non-collinear.


The same in matrix form 
\begin{eqnarray}
\left(\begin{array}{c}
\rho_{ab}^{(0)}\\\rho_{ab}^{(x)}\\\rho_{ab}^{(y)}\\\rho_{ab}^{(z)}
\end{array}\right)
&=&
\left(\begin{array}{cccc}
1&0&0&1\\0&1&1&0\\0&i&-i&0\\1&0&0&-1
\end{array}\right)
\left(\begin{array}{c}
\rho_{ab,\uparrow,\uparrow}\\\rho_{ab,\downarrow,\uparrow}\\\rho_{ab,\downarrow,\uparrow}\\\rho_{ab,\downarrow,\downarrow}\end{array}\right)
\nonumber\\
\left(\begin{array}{c}
\rho_{ab,\uparrow,\uparrow}\\\rho_{ab,\downarrow,\uparrow}\\\rho_{ab,\downarrow,\uparrow}\\\rho_{ab,\downarrow,\downarrow}\end{array}\right)
&=&
\frac{1}{2}\left(\begin{array}{cccc}
1&0&0&1\\0&1&i&0\\0&1&-i&0\\1&0&0&-1
\end{array}\right)
\left(\begin{array}{c}
\rho_{ab}^{(0)}\\\rho_{ab}^{(x)}\\\rho_{ab}^{(y)}\\\rho_{ab}^{(z)}
\end{array}\right)
\end{eqnarray}



%===============================================================
\subsubsection{Motivation}
%===============================================================
The convention to apply the factor $1/2$ on the back transformation of
\eq{eq:defdensitymatrixwithspindeptotandcart} is motivated as follows:
For a collinear spin density in $z$-direction, the total density is
defined as $n_t=n_{\uparrow,\uparrow}+n_{\downarrow,\downarrow}$ and
the spin density is defined as
$n_s=n_{\uparrow,\uparrow}-n_{\downarrow,\downarrow}$. In that case,
the density matrix would have the form
\begin{eqnarray*}
n_{\sigma,\sigma'}
\eqrel{eq:defdensitymatrixwithspindeptotandcart}{=}
\frac{1}{2}n^{(0)}\sigma^{(0)}_{\sigma,\sigma'}
+\frac{1}{2}n^{(4)}\sigma^{(4)}_{\sigma,\sigma'}
=\left(\begin{array}{cc}\frac{1}{2}n^{(0)}+\frac{1}{2}n^{(4)}&0
\\0&\frac{1}{2}n^{(0)}-\frac{1}{2}n^{(4)}
\end{array}\right)
\end{eqnarray*}
which allows to identify $n^{(0)}=n_t$ with the total density and
$n^{(4)}=n_s$ with the spin density.


This yields
\begin{myshadowminipage}{Spin dependence of the density matrix}
\begin{eqnarray}
\rho(\vec{x},\vec{x'})
&=&\sum_{\alpha,\beta}\rho_{\alpha,\beta,\sigma,\sigma'}
\bar{\chi}_{\alpha}(\vec{r})\bar{\chi}_{\beta}(\vec{r'})
\\
&=&\frac{1}{2}\sum_{j=0}^3\sum_{\alpha,\beta}\rho_{\alpha,\beta}^{(j)}\sigma^{(j)}_{\sigma,\sigma'}
\bar{\chi}_{\alpha}(\vec{r})\bar{\chi}_{\beta}(\vec{r'})
\label{eq:denmatspindepcart}
\end{eqnarray}
where the orbitals $|\bar{\chi}_\alpha\rangle$ are pure spatial
orbitals without a spin contribution.
\end{myshadowminipage}



%====================================================================
\subsection{Identity in a spinor representation}
%====================================================================
The identity matrix in up-down representation\footnote{Here each
  matrix element is itself considered a matrix in the space of spatial
  orbitals.} is
\begin{eqnarray}
\left(\begin{array}{cc}
\mat{\rho}_{\uparrow,\uparrow}&\mat{\rho}_{\uparrow,\downarrow}\\
\mat{\rho}_{\downarrow,\uparrow}&\mat{\rho}_{\downarrow,\downarrow}\\
\end{array}\right)
=
\left(\begin{array}{cc}
\mat{1}_{\uparrow,\uparrow}&\mat{0}_{\uparrow,\downarrow}\\
\mat{0}_{\downarrow,\uparrow}&\mat{1}_{\downarrow,\downarrow}\\
\end{array}\right)
\end{eqnarray}

After conversion into the spinor representation we obtain
\begin{eqnarray}
\mat{1}^{(0)}=2\cdot\mat{1};\qquad
\mat{1}^{(1)}=\mat{0};\qquad
\mat{1}^{(2)}=\mat{0};\qquad
\mat{1}^{(3)}=\mat{0}
\end{eqnarray}

%====================================================================
\subsection{Hermitean conjugate  in a spinor representation}
%====================================================================
The hermitean conjugate of a matrix $\mat{\rho}$ in up-down
representation is
\begin{eqnarray}
\left(\begin{array}{cc}
\left(\mat{\rho}^\dagger\right)_{\uparrow,\uparrow}
&\left(\mat{\rho}^\dagger\right)_{\uparrow,\downarrow}\\
\left(\mat{\rho}^\dagger\right)_{\downarrow,\uparrow}
&\left(\mat{\rho}^\dagger\right)_{\downarrow,\downarrow}\\
\end{array}\right)
=
\left(\begin{array}{cc}
 \left(\mat{\rho}_{\uparrow,\uparrow}    \right)^\dagger
&\left(\mat{\rho}_{\downarrow,\uparrow}  \right)^\dagger\\
 \left(\mat{\rho}_{\uparrow,\downarrow}  \right)^\dagger
&\left(\mat{\rho}_{\downarrow,\downarrow}\right)^\dagger\\
\end{array}\right)
\end{eqnarray}

After conversion into the spinor representation we obtain
\begin{eqnarray}
\left(\mat{\rho}^\dagger\right)^{(0)}=\left(\mat{\rho}^{(0)}\right)^\dagger
;\qquad
\left(\mat{\rho}^\dagger\right)^{(x)}=\left(\mat{\rho}^{(x)}\right)^\dagger
;\qquad
\left(\mat{\rho}^\dagger\right)^{(y)}=\left(\mat{\rho}^{(y)}\right)^\dagger
;\qquad
\left(\mat{\rho}^\dagger\right)^{(z)}=\left(\mat{\rho}^{(z)}\right)^\dagger
\nonumber\\
\end{eqnarray}

%====================================================================
\subsection{Multiplication of matrices in a spinor representation}
%====================================================================
Consider two matrices in spinor representation
\begin{eqnarray*}
f_{a,b,\sigma,\sigma'}=\frac{1}{2}\sum_{j=0}^3
\mat{f}^{(j)}_{a,b}\sigma^{(j)}_{\sigma,\sigma'}
\qquad\text{and}\qquad
g_{a,b,\sigma,\sigma'}=\frac{1}{2}\sum_{j=0}^3
\mat{g}^{(j)}_{a,b}\sigma^{(j)}_{\sigma,\sigma'}
\end{eqnarray*}

We wish to perform a matrix multiplication
\begin{eqnarray*}
\sum_{c,\sigma''}
f_{a,c,\sigma,\sigma''}g_{c,b,\sigma'',\sigma'}
&=&
\sum_{c,\sigma''}
\biggl(\frac{1}{2}\sum_{j=0}^3
f^{(i)}_{a,c}\sigma^{(i)}_{\sigma,\sigma''}\biggr)
\biggl(\frac{1}{2}\sum_{j=0}^3
g^{(j)}_{c,b}\sigma^{(j)}_{\sigma'',\sigma'}\biggr)
\nonumber\\
&=&
\frac{1}{4}\sum_{i,j=0}^3
\Bigl(\sum_{c}
f^{(i)}_{a,c}g^{(j)}_{c,b}\Bigr)
\Bigl(\sum_{\sigma''}
\sigma^{(i)}_{\sigma,\sigma''}
\sigma^{(j)}_{\sigma'',\sigma'}\Bigr)
\nonumber\\
&\eqrel{eq:spinorproductformula}{=}&
\frac{1}{2}
\biggl[\frac{1}{2}\sum_{j=0}^3
\Bigl(\mat{f}^{(j)}\mat{g}^{(j)}\Bigr)_{a,b}\biggr]
\sigma^{(0)}_{\sigma,\sigma'}
\nonumber\\
&&+\frac{1}{2}\sum_{k=1}^3
\biggl\lbrace\frac{1}{2}\Bigl(\mat{f}^{(0)}\mat{g}^{(k)}\Bigr)_{a,b}
+\frac{1}{2}\Bigl(\mat{f}^{(k)}\mat{g}^{(0)}\Bigr)_{a,b}
\nonumber\\
&&\hspace{2cm}
+\frac{i}{2}\sum_{i,j=1}^3\epsilon_{i,j,k}\Bigl(\mat{f}^{(i)}\mat{g}^{(j)}\Bigr)_{a,b}
\biggr\rbrace
\sigma^{(k)}_{\sigma,\sigma'}
\end{eqnarray*}

Thus, if we denote the multiplication as defined above with the symbol
$\square$, we obtain
\begin{eqnarray}
\Bigl(\mat{f}\square\mat{g}\Bigr)^{(0)}_{a,b}
&=&\frac{1}{2}\sum_{j=0}^3\Bigl(\mat{f}^{(j)}\mat{g}^{(j)}\Bigr)_{a,b}
\\
\Bigl(\mat{f}\square\mat{g}\Bigr)^{(k)}_{a,b}
&=&\frac{1}{2}
\Bigl(\mat{f}^{(0)}\mat{g}^{(k)}\Bigr)_{a,b}
+\frac{1}{2}\Bigl(\mat{f}^{(k)}\mat{g}^{(0)}\Bigr)_{a,b}
+\frac{i}{2}\sum_{i,j=1}^3\epsilon_{i,j,k}
\Bigl(\mat{f}^{(i)}\mat{g}^{(j)}\Bigr)_{a,b}
\qquad\text{for $j>0$}
\end{eqnarray}

This expression requires 16 matrix multiplication in the
$a,b,c,\ldots$ space, just as if the operations would be done in the
$\uparrow.,\downarrow$ representation.

%====================================================================
\subsection{Inversion of a matrix in a spinor representation}
%====================================================================
The inversion is done by first bringing the matrix into the
$\uparrow,\downarrow$ representation
using \eq{eq:defdensitymatrixwithspindeptotandcart}.

The problem can then be formulated as a matrix inversion in the
(orbital/spin) space
\begin{eqnarray}
\left(\begin{array}{cc}
\mat{A}_{11}&\mat{A}_{12}\\\mat{A}_{21}&\mat{A}_{22}
\end{array}\right)
\left(\begin{array}{cc}
\mat{B}_{11}&\mat{B}_{12}\\\mat{B}_{21}&\mat{B}_{22}
\end{array}\right)
=
\left(\begin{array}{cc}
\mat{1}&\mat{0}\\\mat{0}&\mat{1}
\end{array}\right)
\end{eqnarray}

In components, we obtain
\begin{eqnarray}
\mat{A}_{11}\mat{B}_{11}+\mat{A}_{12}\mat{B}_{21}&=&\mat{1}
\nonumber\\
\mat{A}_{11}\mat{B}_{12}+\mat{A}_{12}\mat{B}_{22}&=&\mat{0}
\nonumber\\
\mat{A}_{21}\mat{B}_{11}+\mat{A}_{22}\mat{B}_{21}&=&\mat{0}
\nonumber\\
\mat{A}_{21}\mat{B}_{12}+\mat{A}_{22}\mat{B}_{22}&=&\mat{1}
\end{eqnarray}
which leads to
\begin{eqnarray}
\mat{B}_{12}&=&\underbrace{-\mat{A}_{11}^{-1}\mat{A}_{12}}_{a_{12}}\mat{B}_{22}
\nonumber\\
\mat{B}_{21}&=&\underbrace{-\mat{A}_{22}^{-1}\mat{A}_{21}}_{a_{21}}\mat{B}_{11}
\nonumber\\
\mat{B}_{11}&=&\Bigl(\mat{A}_{11}
-\mat{A}_{12}\underbrace{\mat{A}_{22}^{-1}\mat{A}_{21}}_{-a_{21}}\Bigr)^{-1}
\nonumber\\
\mat{B}_{22}&=&\Bigl(\mat{A}_{22}
-\mat{A}_{21}\underbrace{\mat{A}_{11}^{-1}\mat{A}_{12}}_{a_{12}}\Bigr)^{-1}
\nonumber
\end{eqnarray}

The operations are done in the following order
\begin{eqnarray*}
\mat{C}_{11}&=&\mat{A}_{11}^{-1}
\nonumber\\
\mat{C}_{12}&=&-\mat{C}_{11}\mat{A}_{12}=-\mat{A}_{11}^{-1}\mat{A}_{12}
\nonumber\\
\mat{C}_{22}&=&\mat{A}_{22}+\mat{A}_{21}\mat{C}_{12}
=\mat{A}_{22}-\mat{A}_{21}\mat{A}_{11}^{-1}\mat{A}_{12}
\nonumber\\
\mat{B}_{22}&=&\mat{C}_{22}^{-1}
=\Bigl(\mat{A}_{22}-\mat{A}_{21}\mat{A}_{11}^{-1}\mat{A}_{12}\Bigr)^{-1}
\nonumber\\
\mat{B}_{12}&=&\mat{C}_{12}\mat{B}_{22}
\nonumber\\
\mat{C}_{22}&=&\mat{A}_{22}^{-1}
\nonumber\\
\mat{C}_{21}&=&-\mat{C}_{22}\mat{A}_{21}=-\mat{A}_{22}^{-1}\mat{A}_{21}
\nonumber\\
\mat{C}_{11}&=&\mat{A}_{11}+\mat{A}_{12}\mat{C}_{21}
=\mat{A}_{11}-\mat{A}_{12}\mat{A}_{22}^{-1}\mat{A}_{21}
\nonumber\\
\mat{B}_{11}&=&\mat{C}_{11}^{-1}
=\Bigl(\mat{A}_{11}-\mat{A}_{12}\mat{A}_{22}^{-1}\mat{A}_{21}\Bigr)^{-1}
\nonumber\\
\mat{B}_{21}&=&\mat{C}_{21}\mat{B}_{11}
\nonumber\\
\end{eqnarray*}
The matrix $\mat{A}_{11}^{-1}$ can be overlayed with $\mat{B}_{11}$,
$\mat{A}_{22}^{-1}$ can be overlayed with $\mat{B}_{22}$,
$\mat{C}_{12}$ can be overlayed with $\mat{B}_{12}$,
and $\mat{C}_{21}$ can be overlayed with $\mat{B}_{21}$.


The operations can be done with 4 matrix inversions and 6 matrix
multiplications for a general non-collinear problem. (Operations that
scale better than the cube behavior are ignored.)


It may be interesting to analyze the scaling behavior.  Consider that
the computational effort for an inversion of a matrix with dimension
$n$ is $an^3$. The effort for a matrix multiplication shall be $bn^3$.
Thus the scaling will be
\begin{eqnarray}
a(2n)^3=4a^3+6bn^3+cn^2\qquad\rightarrow\qquad a=\frac{3}{2}b+\frac{c}{4n}
\end{eqnarray}
This implies that a matrix inversion takes about 1.5 times the
computational effort of a matrix multiplication.

%====================================================================
\subsection{Hermitean Matrices}
%====================================================================
A matrix that is hermitean in spin-up-down representation, that is
\begin{eqnarray}
\rho_{a,b,\sigma,\sigma'}=\rho^*_{b,a,\sigma',\sigma}
\Leftrightarrow
\rho^{(j)}_{a,b}=\left(\rho^{(j)}_{b,a}\right)^*
\end{eqnarray}
has hermitean matrices in total-spin representation, and vice versa



This is derived in the following. First we show that 
\begin{eqnarray}
\rho^{(j)}_{a,b}=\left(\rho^{(j)}_{b,a}\right)^*
\label{eq:hermiteanspinorcomponents}
\end{eqnarray}
can be obtained from the hermitean property in spin space, namely
\begin{eqnarray}
\rho_{a,b,\sigma,\sigma'}=\rho^*_{b,a,\sigma',\sigma}
\label{eq:spinspinhermitean}
\end{eqnarray}

This is shown as follows
\begin{eqnarray}
\rho^{(j)}_{a,b}
&\eqrel{eq:defdensitymatrixwithspindeptotandcart}{=}&
\sum_{\sigma\sigma'}\rho_{a,b,\sigma,\sigma'}
\left(\sigma^{(j)}_{\sigma,\sigma'}\right)^*
\eqrel{eq:spinspinhermitean}{=}
\sum_{\sigma\sigma'}\rho^*_{b,a,\sigma',\sigma}
\left(\sigma^{(j)}_{\sigma,\sigma'}\right)^*
\stackrel{\mat{\sigma}^{(j)}=\mat{\sigma}^{(j),\dagger}}{=}
\sum_{\sigma\sigma'}\rho^*_{b,a,\sigma',\sigma}
\left(\sigma^{(j)}_{\sigma',\sigma}\right)
\nonumber\\
&\eqrel{eq:defdensitymatrixwithspindeptotandcart}{=}&
\left(\rho^{(j)}_{b,a}\right)^*
\end{eqnarray}

Now we derive the opposite direction:
\begin{eqnarray}
\rho_{a,b,\sigma,\sigma'}
&\eqrel{eq:defdensitymatrixwithspindeptotandcart}{=}&
\frac{1}{2}\sum_{j=0}^3\rho^{(j)}_{a,b}\sigma^{(j)}_{\sigma,\sigma'}
\eqrel{eq:hermiteanspinorcomponents}{=}
\frac{1}{2}\sum_{j=0}^3\Bigl(\rho^{(j)}_{b,a}\Bigr)^*\sigma^{(j)}_{\sigma,\sigma'}
\stackrel{\mat{\sigma}^{(j)}=\mat{\sigma}^{(j),\dagger}}{=}
\Bigl(\frac{1}{2}\sum_{j=0}^3\rho^{(j)}_{b,a}\sigma^{(j)}_{\sigma',\sigma}\Bigr)^*
\nonumber\\
&\eqrel{eq:defdensitymatrixwithspindeptotandcart}{=}&
\Bigl(\rho_{b,a,\sigma',\sigma}\Bigr)^*
\end{eqnarray}


%====================================================================
\section{Density matrices and spin orbitals with defined spin}
%====================================================================
Let us choose a basis set $\{|\chi_\alpha\rangle\}$ with states that
are product states of a spatial orbital $\bar{\chi}_\alpha(\vec{r})$
and a spin orbital $\xi_\alpha$, such as
\begin{eqnarray}
\chi_\alpha(\vec{x})=\bar{\chi}_\alpha(\vec{r})\xi_\alpha(\sigma)
\label{eq:chiasproductofspaceandspin}
\end{eqnarray}
Typically, the spin orbitals are eigenstates to $\mat{\sigma}_z$ so
that $\xi_\alpha(\sigma)=\delta_{\sigma,\sigma_\alpha}$ and
$\sigma_\alpha\in\{\uparrow,\downarrow\}=\{(1,0),(0,1)\}$.

In that case we can write the density matrix
\begin{eqnarray}
\rho(\vec{x},\vec{x'})
&=&\sum_{\alpha,\beta}\chi_{\alpha}(\vec{x})\rho_{\alpha,\beta}\chi^*_{\beta}(\vec{x'})
\nonumber\\
&=&\sum_{\alpha,\beta}\bar{\chi}_{\alpha}(\vec{r})
\underbrace{\xi_\alpha(\sigma)\rho_{\alpha,\beta}\xi^*_\beta(\sigma')}
_{\rho_{\alpha,\beta,\sigma,\sigma'}}
\bar{\chi}^*_{\beta}(\vec{r'})
\nonumber\\
&=&\sum_{\alpha,\beta}\bar{\chi}_{\alpha}(\vec{r})\rho_{\alpha,\beta,\sigma,\sigma'}
\bar{\chi}^*_{\beta}(\vec{r'})
\label{eq:densitymatrixxxprimebyssigmaprime}
\end{eqnarray}

Here we have defined the density matrix with explicit spin dependence
\begin{eqnarray}
\rho_{\alpha,\beta,\sigma,\sigma'}\defas\xi_\alpha(\sigma)\rho_{\alpha,\beta}\xi^*_\beta(\sigma')
\label{eq:defdensitymatrixwithspindep}
\end{eqnarray}

The density matrices
$\rho_{\sigma,\sigma'}=\xi(\sigma)\xi^*(\sigma')$ for the spin
eigenstates \eq{eq:spineigenstatesxyz}
are
\begin{eqnarray*}
\mat{\rho}(\pm x)=\frac{1}{2}\left(\begin{array}{cc}1&\pm1\\\pm1&1\end{array}\right)
\qquad\text{and}\qquad
\mat{\rho}(\pm y)=\frac{1}{2}\left(\begin{array}{cc}1&\mp i\\\pm i&1\end{array}\right)
\qquad\text{and}\qquad
\mat{\rho}(\pm z)=\frac{1}{2}\left(\begin{array}{cc}1\pm1& 0\\0&1\mp1\end{array}\right)
\end{eqnarray*}
They obey $\Tr[\mat{\rho}(\pm
  j)\mat{\sigma}^{(j')}]=\pm\delta_{j,j'}$ for $j,j'\in\{x,y,z\}$ and 
$\Tr[\mat{\rho}(\pm j)\mat{\sigma}^{(0)}]=1$.






%====================================================================
\section{Potentials and spin orbitals}
%====================================================================

%====================================================================
\subsubsection{General definition of a potential for a complex matrix quantity}
%====================================================================
\begin{myshadowminipage}{Potential as energy derivative}
The potential of a matrix quantity must be written in the following
form
\begin{eqnarray*}
V=\frac{dE}{dn^*}
\end{eqnarray*}
For Hermitean quantities, this yields 
\begin{eqnarray*}
V_{\alpha,\beta}=\frac{dE}{d\rho_{\alpha,\beta}^*}=\frac{dE}{d\rho_{\beta,\alpha}}
\end{eqnarray*}
\end{myshadowminipage}

This has the following reasons
\begin{itemize}
\item The trace formula comes from
\begin{eqnarray*}
dE=\Tr[\hat{V}d\hat{\rho}]
=\sum_{\alpha,\beta}
\langle\alpha|\hat{V}|\beta\rangle
\langle\beta|d\hat{\rho}|\alpha\rangle
=\sum_{\alpha,\beta}
V_{\alpha,\beta}d\rho_{\beta,\alpha}
\stackrel{\rho=\rho^\dagger}{=}\sum_{\alpha,\beta}
V_{\alpha,\beta}d\rho^*_{\alpha,\beta}
\qquad\Rightarrow\qquad
\frac{\partial E}{\partial\rho_{\alpha,\beta}^*}=V_{\alpha,\beta}
\end{eqnarray*}
%------------------------------------------------------------------------------
\item Another form
\begin{eqnarray*}
E&=&F[\underbrace{\sum_{\alpha,\beta}\chi_\alpha(\vec{x})\rho_{\alpha,\beta}\chi^*_\beta(\vec{x'})}
_{\rho(\vec{x},\vec{x'})}]
\\
V_{\alpha,\beta}&=&
\frac{\partial E}{\partial\rho_{\alpha,\beta}^*}
=\biggl(\frac{\partial E}{\partial\rho_{\alpha,\beta}}\biggr)^*
\eqrel{eq:densitymatrixxxprimebyssigmaprime}{=}\biggl(\int d^4x\int d^4x'\;
\frac{\partial E}{\partial\rho(\vec{x},\vec{x'})}
\chi_\alpha(\vec{x})\chi^*_\beta(\vec{x'})]\biggr)^*
\\
&=&\int d^4x\int d^4x'\;
\chi^*_\alpha(\vec{x})
\underbrace{\frac{\partial E}{\partial\rho^*(\vec{x},\vec{x'})}}_{v(\vec{x},\vec{x'})}
\chi_\beta(\vec{x})
=\int d^4x\int d^4x'\;\chi^*_\alpha(\vec{x})v(\vec{x},\vec{x'})\chi_\beta(\vec{x'})
\\
&=&\langle\chi_\alpha|\hat{V}|\chi_\beta\rangle
\end{eqnarray*}
%------------------------------------------------------------------------------
\item
Similarly, we obtain
\begin{eqnarray*}
E&=&F[\sum_n\langle\pi_\alpha|\psi_n\rangle f_n\langle\psi_n|\pi_\beta\rangle]
\\
\frac{\partial E}{\partial\langle\psi_n|}
&=&
\sum_{\alpha,\beta}|\pi_\beta\rangle \frac{\partial F}{\partial\rho_{\alpha,\beta}}
\langle\pi_\alpha|\psi_n\rangle f_n
=\sum_{\alpha,\beta}|\pi_\beta\rangle \frac{\partial F}{\partial\rho^*_{\beta,\alpha}}
\langle\pi_\alpha|\psi_n\rangle f_n
\\
&=&\sum_{\alpha,\beta}|\pi_\beta\rangle V_{\beta,\alpha}
\langle\pi_\alpha|\psi_n\rangle f_n
\end{eqnarray*}
\item with k-points: The density matrix is defined as
\begin{eqnarray*}
\rho_{\alpha,\beta}(\vec{t}_\beta)=\frac{1}{N_k}\sum_{\vec{k}} 
\langle\pi_\alpha|\psi_n(\vec{k})\rangle f_n(\vec{k})
\langle\psi_n(\vec{k})|\pi_\beta\rangle\e{i\vec{k}\vec{t}_\beta}
\end{eqnarray*}
The density matrix connects an orbital $|\chi_\alpha\rangle$ at
$\vec{R}_\alpha$ with an orbital $|\chi_\beta\rangle$ at
$\vec{R}_\beta+\vec{t}_\beta$.


\begin{eqnarray*}
\frac{dF}{d\langle\psi_n(\vec{k})|}
&=&
\sum_{\alpha,\beta,\vec{t}_\beta}
 \frac{dF}{d\rho_{\alpha,\beta}(\vec{t}_\beta)}
\frac{d}{d\langle\psi_n(\vec{k})|}
\left[\frac{1}{N_k}\sum_{\vec{k}}
\sum_n\langle\pi_\alpha|\psi_n(\vec{k})\rangle f_n(\vec{k})
\langle\psi_n(\vec{k})|\pi_\beta\rangle\e{i\vec{k}\vec{t}_\beta}\right]
\\
&=&
\frac{1}{N_k}\sum_{\vec{k}}\sum_{\alpha,\beta,\vec{t}_\beta} 
|\pi_\beta\rangle
\biggl[
\frac{dF}{d\rho_{\alpha,\beta}(\vec{t}_\beta)}\e{i\vec{k}\vec{t}_\beta}
\langle\pi_\alpha|\psi_n(\vec{k})\rangle f_n(\vec{k})
\\
&=&
\frac{1}{N_k}\sum_{\vec{k}}\sum_{\beta}
|\pi_\beta\rangle
\biggl[\sum_{\alpha,\vec{t}_\beta} 
\underbrace{\frac{dF}{d\rho_{\alpha,\beta}(\vec{t}_\beta)}}
_{=V_{\beta,\alpha(\vec{t}_\alpha)}}
\e{i\vec{k}\vec{t}_\beta}
\langle\pi_\alpha|\psi_n(\vec{k})\rangle\biggr] f_n(\vec{k})
\end{eqnarray*}
In the implementation, I am using the variable hamil sometimes as $V$
and sometimes as derivative of the functional. The two are hermitean
adjuncts of each other,
i.e. $V_{\alpha,\beta}(\vec{t})=V_{\beta,\alpha}^*(-\vec{t})$.

\end{itemize}

%====================================================================
\subsubsection{Spin potentials}
%====================================================================
Let us now return to the potentials obtained as derivative with
respect to the different forms of the density matrix.
\begin{eqnarray*}
\bar{V}^{(j)}_{\alpha,\beta}&\defas&
\frac{\partial E}{\partial \bar{\rho}^{(j)*}_{\alpha,\beta}}
\\
V_{\alpha,\beta,\sigma,\sigma'}&=&\frac{\partial E}{\partial\rho_{\alpha,\beta,\sigma,\sigma'}^*}
=\sum_j\frac{\partial E}{\partial \bar{\rho}^{(j)*}_{\alpha,\beta}}
\frac{\partial\bar{\rho}^{(j)*}_{\alpha,\beta}}{\partial\rho_{\alpha,\beta,\sigma,\sigma'}^*}\
=\sum_j\bar{V}_{\alpha,\beta}^{(j)}
\left(\frac{\partial\bar{\rho}^{(j)}_{\alpha,\beta}}{\partial\rho_{\alpha,\beta,\sigma,\sigma'}}\right)^*
\\
&\eqrel{eq:defdensitymatrixwithspindeptotandcart}{=}&
\sum_j\bar{V}_{\alpha,\beta}^{(j)}\biggl(\sigma^{(j)}_{\sigma,\sigma'}\biggr)^*
\end{eqnarray*}



\begin{eqnarray}
dE&=&
\sum_{a,b,\sigma,\sigma'}
\frac{\delta E}{\delta\rho_{a,b,\sigma,\sigma'}}
\delta\rho_{a,b,\sigma,\sigma'}
=\sum_{a,b,\sigma,\sigma'}V_{b,a,\sigma',\sigma}\delta\rho_{a,b,\sigma,\sigma'}
\nonumber\\
&\eqrel{eq:defdensitymatrixforwardexplicit}{=}&
\sum_{a,b}\biggl\lbrace
\underbrace{
\frac{{\delta}E}{{\delta}\rho_{a,b,\uparrow,\uparrow}}
}_{V_{b,a,\uparrow,\uparrow}}
\underbrace{
\frac{1}{2}\biggl({\delta}\rho^{(t)}_{a,b}+{\delta}\rho^{(z)}_{a,b}\biggr)
}_{\delta\rho_{a,b,\uparrow,\uparrow}}
+
\underbrace{
\frac{{\delta}E}{\delta\rho_{a,b,\downarrow,\uparrow}}
}_{V_{b,a,\uparrow,\downarrow}}
\underbrace{
\frac{1}{2}\biggl({\delta}\rho^{(x)}_{a,b}+i{\delta}\rho^{(y)}_{a,b}\biggr)
}_{\delta\rho_{a,b,\downarrow,\uparrow}}
\nonumber\\
&&\hspace{0.5cm}+
\underbrace{
\frac{{\delta}E}{{\delta}\rho_{a,b,\uparrow,\downarrow}}
}_{V_{b,a,\downarrow,\uparrow}}
\underbrace{
\frac{1}{2}\biggl({\delta}\rho^{(x)}_{a,b}-i{\delta}\rho^{(y)}_{a,b}\biggr)
}_{\delta\rho_{a,b,\uparrow,\downarrow}}
+
\underbrace{
\frac{{\delta}E}{{\delta}\rho_{a,b,\downarrow,\downarrow}}
}_{V_{b,a,\downarrow,\uparrow}}
\underbrace{
\frac{1}{2}\biggl({\delta}\rho^{(t)}_{a,b}-{\delta}\rho^{(z)}_{a,b}\biggr)
}_{\delta\rho_{a,b,\downarrow,\downarrow}}
\biggr\rbrace
\nonumber\\
&=&
\sum_{a,b}\biggl\lbrace
\underbrace{
\frac{1}{2}
\biggl(\frac{\delta{E}}{\delta\rho_{a,b,\uparrow,\uparrow}}
+\frac{\delta{E}}{\delta\rho_{a,b,\downarrow,\downarrow}}
\biggr)
}_{\bar{V}_{b,a}^{(t)}}
\delta\rho^{(t)}
+
\underbrace{
\frac{1}{2}
\biggl(\frac{{\delta}E}{{\delta}\rho_{a,b,\downarrow,\uparrow}}
+\frac{{\delta}E}{\delta\rho_{a,b,\uparrow,\downarrow}}\biggr)
}_{\bar{V}_{b,a}^{(x)}}
{\delta}\rho^{(x)}_{a,b}
\nonumber\\
&&\hspace{0.5cm}+
\underbrace{
\frac{i}{2}
\biggl(\frac{{\delta}E}{{\delta}\rho_{a,b,\downarrow,\uparrow}}
+\frac{{\delta}E}{{\delta}\rho_{a,b,\uparrow,\downarrow}}\biggr)
}_{\bar{V}_{b,a}^{(y)}}
{\delta}\rho^{(y)}_{a,b}
+
\underbrace{
\frac{1}{2}
\biggl(\frac{\delta{E}}{\delta\rho_{a,b,\uparrow,\uparrow}}
-\frac{\delta{E}}{\delta\rho_{a,b,\downarrow,\downarrow}}\biggr)
}_{\bar{V}_{b,a}^{(z)}}
{\delta}\rho^{(z)}
\biggr\rbrace
\nonumber\\
&=&\sum_{j=0}^3 \sum_{a,b}\bar{V}^{(j)}_{b,a}\;\delta\rho^{(j)}_{a,b}
\end{eqnarray}

Thus
\begin{eqnarray}
\bar{V}_{b,a}^{(t)}&=&\frac{1}{2}\Bigl(V_{b,a,\uparrow,\uparrow}
+V_{b,a,\downarrow,\downarrow}\Bigr)
\nonumber\\
\bar{V}_{b,a}^{(x)}&=&\frac{1}{2}\Bigl(v_{b,a,\uparrow,\downarrow}
+V_{b,a,\downarrow,\uparrow}\Bigr)
\nonumber\\
\nonumber\\
\bar{V}_{b,a}^{(y)}&=&\frac{i}{2}\Bigl(V_{b,a,\uparrow,\downarrow}
-V_{b,a,\downarrow,\uparrow}\Bigr)
\nonumber\\
\bar{V}_{b,a}^{(z)}&=&\frac{1}{2}\Bigl(V_{b,a,\uparrow,\uparrow}
-V_{b,a,\downarrow,\downarrow}\Bigr)
\end{eqnarray}

\begin{myshadowminipage}{Transformation from a total-spin 
to an up-down representation}
\begin{minipage}{0.45\linewidth}
\begin{eqnarray*}
  \rho^{(t)}&=&\rho_{\uparrow,\uparrow}+\rho_{\downarrow,\downarrow}\biggl.\biggr.
  \\
  \rho^{(x)}&=&\rho_{\uparrow,\downarrow}+\rho_{\downarrow,\uparrow}\biggl.\biggr.
  \\
  \rho^{(y)}&=&i(\rho_{\uparrow,\downarrow}-\rho_{\downarrow,\uparrow})\biggl.\biggr.
  \\
  \rho^{(z)}&=&\rho_{\uparrow,\uparrow}-\rho_{\downarrow,\downarrow}\biggl.\biggr.
\end{eqnarray*}
\end{minipage}
\begin{minipage}{0.45\linewidth}
\begin{eqnarray*}
\rho_{\uparrow,\uparrow}&=&\frac{1}{2}(\rho^{(t)}+\rho^{(z)})
\\
\rho_{\uparrow,\downarrow}&=&\frac{1}{2}(\rho^{(x)}-i\rho^{(y)})
\\
\rho_{\downarrow,\uparrow}&=&\frac{1}{2}(\rho^{(x)}+i\rho^{(y)})
\\
\rho_{\downarrow,\downarrow}&=&\frac{1}{2}(\rho^{(t)}-\rho^{(z)})
\end{eqnarray*}
\end{minipage}\\
For the potentials $v=\left(\frac{\partial E}{\partial\rho}\right)^*$ 
we obtain\\
\begin{minipage}{0.45\linewidth}
\begin{eqnarray*}
v^{(t)}&=&\frac{1}{2}\biggl(v_{\uparrow,\uparrow}+v_{\downarrow,\downarrow}\biggr)
\\
v^{(x)}&=&\frac{1}{2}\biggl(v_{\uparrow,\downarrow}+v_{\downarrow,\uparrow}\biggr)
\\
v^{(y)}&=&\frac{i}{2}\biggl(v_{\uparrow,\downarrow}-v_{\downarrow,\uparrow}\biggr)
\\
v^{(z)}&=&\frac{1}{2}\biggl(v_{\uparrow,\uparrow}-v_{\downarrow,\downarrow}\biggr)
\end{eqnarray*}
\end{minipage}
\begin{minipage}{0.45\linewidth}
\begin{eqnarray*}
v_{\uparrow,\uparrow}&=&v^{(t)}+v^{(z)}\biggl.\biggr.
\\
v_{\uparrow,\downarrow}&=&v^{(x)}-iv^{(y)}\biggl.\biggr.
\\
v_{\downarrow,\uparrow}&=&v^{(x)}+iv^{(y)}\biggl.\biggr.
\\
v_{\downarrow,\downarrow}&=&v^{(t)}-v^{(z)}\biggl.\biggr.
\end{eqnarray*}
\end{minipage}
\end{myshadowminipage}

These transformations are used in \verb|LDAPLUSU_edft|,
\verb|LDAPLUSU_SPINDENMAT|, \verb|LMTO_NTBODENMAT|,
\verb|LMTO_NTBODENMATDER|, \verb|WAVES_DENMAT|, \verb|WAVES_DENSITY|.

%====================================================================
\section{Description of Subroutines}
%====================================================================
\cite{bloechl94_prb50_17953}

We consider a Hilbert space of two-component spinor wave functions.  A
real-space-spin basis is $|\vec{r},\sigma\rangle$. Instead of the real
space position we may also use a set of orbitals
$|\alpha,\sigma\rangle$, which are spin eigenstates with the spatial
dependence defined by $\alpha$, that is
$\langle\vec{r},\sigma|\alpha,\sigma\rangle=
\langle\vec{r},\sigma'|\alpha,\sigma'\rangle$ and
$\langle\vec{r},\sigma|\alpha,\sigma'\rangle=0$ for
$\sigma\neq\sigma'$.


In this basisset a matrix element has the form
\begin{eqnarray*}
A_{\alpha,\beta,\sigma,\sigma'}&=&\langle\alpha,\sigma|\hat{A}|\beta,\sigma'\rangle
\nonumber\\
\hat{A}&=&\sum_{\alpha,\beta,\sigma,\sigma'}
|\alpha,\sigma\rangle A_{\alpha,\beta,\sigma,\sigma'} 
\langle\beta,\sigma'|
\end{eqnarray*}

An expectation value is obtained by
\begin{eqnarray}
\langle A\rangle
&=&\sum_nf_n\langle\psi_n|\hat{A}|\psi_n\rangle
\nonumber\\
&=&
\sum_{\alpha,\beta,\sigma,\sigma'}
\underbrace{
\Bigl(\sum_n
\langle\beta,\sigma'|\psi_n\rangle f_n\langle\psi_n|\alpha,\sigma\rangle\Bigr)
}_{\rho_{\beta,\alpha,\sigma',\sigma}}
A_{\alpha,\beta,\sigma,\sigma'} 
\nonumber\\
&=&
\sum_{\alpha,\beta,\sigma,\sigma'}
\rho_{\beta,\alpha,\sigma',\sigma}
A_{\alpha,\beta,\sigma,\sigma'} 
\end{eqnarray}

This defines the matrix elements of the density matrix as
\begin{eqnarray}
\rho_{\alpha,\beta,\sigma,\sigma'}&=&
\langle\beta,\sigma'|\psi_n\rangle f_n\langle\psi_n|\alpha,\sigma\rangle
\end{eqnarray}

%==================================================
\subsubsection{(t,x,y,z) representation}
%==================================================
Let us transform the matrix elements 
\begin{eqnarray}
A^{(j)}_{\alpha,\beta}=\sum_{\sigma,\sigma'} 
A_{\alpha,\beta,\sigma,\sigma'} \sigma^{(j)}_{\sigma',\sigma}
\nonumber\\
\rho^{(j)}_{\alpha,\beta}=\sum_{\sigma,\sigma'} 
\rho_{\alpha,\beta,\sigma,\sigma'} \sigma^{(j)}_{\sigma',\sigma}
\end{eqnarray}

The back transformation is correspondingly
\begin{eqnarray}
A_{\alpha,\beta,\sigma,\sigma'}
&=&\frac{1}{2}\sum_{j=0}^3A^{(j)}_{\alpha,\beta}
\sigma^{(j)}_{\sigma,\sigma'}
\nonumber\\
\rho_{\alpha,\beta,\sigma,\sigma'}
&=&\frac{1}{2}\sum_{j=0}^3\rho^{(j)}_{\alpha,\beta}
\sigma^{(j)}_{\sigma,\sigma'}
\end{eqnarray}

Proof:
\begin{eqnarray}
\frac{1}{2}\sum_{j=0}^3\rho^{(j)}_{\alpha,\beta}
\sigma^{(j)}_{\sigma,\sigma'}
&=&
\frac{1}{2}\sum_{j=0}^3
\underbrace{\Bigl(\sum_{\sigma'',\sigma'''} 
\rho_{\alpha,\beta,\sigma'',\sigma'''} \sigma^{(j)}_{\sigma''',\sigma''}
\Bigr)}_{\rho^{(j)}_{\alpha,\beta}}
\sigma^{(j)}_{\sigma,\sigma'}
\nonumber\\
&=&
\sum_{\sigma'',\sigma'''} 
\rho_{\alpha,\beta,\sigma'',\sigma'''} 
\underbrace{\Bigl(
\frac{1}{2}\sum_{j=0}^3
\sigma^{(j)}_{\sigma''',\sigma''}
(\sigma^{(j)}_{\sigma',\sigma})^*
\Bigr)}_{\delta_{\sigma''',\sigma'}\delta_{\sigma'',\sigma}}
\nonumber\\
&=&
\rho_{\alpha,\beta,\sigma,\sigma'} 
\end{eqnarray}

\subsubsection{Expectation value by trace}

Now we need the expression for the expectation value
\begin{eqnarray*}
\langle{A}\rangle
&=&\Tr\Bigl(\hat{\rho}\;\hat{A}\Bigr)
=
\sum_{\alpha,\beta,\sigma,\sigma'}
\rho_{\alpha,\beta,\sigma,\sigma'} A_{\beta,\alpha,\sigma',\sigma}
=\sum_{\alpha,\beta,\sigma,\sigma'}
\underbrace{\frac{1}{2}\sum_{j=0}^3 
\bar{\rho}^{(j)}_{\alpha,\beta}
\sigma^{(j)}_{\sigma,\sigma'}}
_{\rho_{\alpha\beta,\sigma,\sigma'}} A_{\beta,\alpha,\sigma',\sigma}
\nonumber\\
&=&\frac{1}{2}
\sum_{j=0}^3 
\sum_{\alpha,\beta}
\bar{\rho}^{(j)}_{\alpha,\beta}
\underbrace{
\Bigl(
\sum_{\sigma,\sigma'}
A_{\beta,\alpha,\sigma',\sigma}\sigma^{(j)}_{\sigma,\sigma'}\Bigr)}
_{A^{(j)}_{\beta,\alpha}}
=\frac{1}{2}
\sum_{j=0}^3 
\sum_{\alpha,\beta}
\bar{\rho}^{(j)}_{\alpha,\beta}A^{(j)}_{\beta,\alpha}
\end{eqnarray*}

%====================================================================
\subsubsection{Physical}
%====================================================================
Total density
\begin{eqnarray}
\rho_t=\rho^{(0)}&=&\rho_{\uparrow,\uparrow}+\rho_{\downarrow,\downarrow}
\\
\rho_x=\rho^{(1)}&=&\rho_{\uparrow,\downarrow}+\rho_{\downarrow,\uparrow}
\\
\rho_y=\rho^{(2)}&=&-i\Bigl(\rho_{\uparrow,\downarrow}-\rho_{\downarrow,\uparrow}\Bigr)
\\
\rho_z=\rho^{(3)}&=&\rho_{\uparrow,\uparrow}-\rho_{\downarrow,\downarrow}
\end{eqnarray}

An expectation value is
\begin{eqnarray}
A_{\uparrow,\uparrow}&=&\frac{1}{2}\Bigl(A^{(t)}+A^{(z)}\Bigr)
\nonumber\\
A_{\uparrow,\downarrow}&=&\frac{1}{2}\Bigl(A^{(x)}-iA^{(y)}\Bigr)
\nonumber\\
A_{\downarrow,\uparrow}&=&\frac{1}{2}\Bigl(A^{(x)}+iA^{(y)}\Bigr)
\nonumber\\
A_{\downarrow,\downarrow}&=&\frac{1}{2}\Bigl(A^{(t)}-A^{(z)}\Bigr)
\end{eqnarray}

\begin{eqnarray}
dE&=&
\sum_{\alpha,\beta,\sigma,\sigma'}
\frac{dE}{d\rho_{\beta,\alpha,\sigma',\sigma}}
d\rho_{\beta,\alpha,\sigma',\sigma}
\nonumber\\
&=&\sum_{\alpha,\beta,\sigma,\sigma'}
\sum_{j=0}^3\frac{dE}{d\rho^{(j)}_{\beta,\alpha}}
\frac{d\rho^{(j)}_{\beta,\alpha}}{d\rho_{\beta,\alpha,\sigma',\sigma}}
d\rho_{\beta,\alpha,\sigma',\sigma}
\nonumber\\
&=&\sum_{j=0}^3
\sum_{\alpha,\beta}
\frac{dE}{d\rho^{(j)}_{\beta,\alpha}}
\Bigl(
\sum_{\sigma,\sigma'}
\sigma^{(j)}_{\sigma,\sigma'}
d\rho_{\beta,\alpha,\sigma',\sigma}
\Bigr)
\nonumber\\
&=&\sum_{j=0}^3
\sum_{\alpha,\beta}
\Bigl(\frac{dE}{d\rho^{(j)}_{\beta,\alpha}}\Bigr)
d\rho^{(j)}_{\beta,\alpha}
\end{eqnarray}

There is an ambiguity becauae of the trace formula
\begin{eqnarray}
A=\frac{1}{2}\sum_{j=0}^3 \rho^{(j)} A^{(j)}
\\
dE=\sum_{j=0}^3 \frac{dE}{d\rho^{(j)}}d\rho^{(j)}
\end{eqnarray}

\begin{eqnarray}
v^{(j)}=2\frac{dE}{d\rho^{(j)}}
\end{eqnarray}
Using the transformation equation for expectation values
\begin{eqnarray}
v_{\sigma,\sigma'}
=\frac{1}{2}\sum_{j=0}^3 v^{(j)}\sigma^{(j)}_{\sigma,\sigma'}
=\sum_{j=0}^3 \frac{dE}{d\rho^{(j)}}
\sigma^{(j)}_{\sigma,\sigma'}
\end{eqnarray}

\begin{eqnarray}
v_{\uparrow,\uparrow}&=&\frac{1}{2}(v_t+v_z)
\nonumber\\
v_{\uparrow,\downarrow}&=&\frac{1}{2}(v_x-iv_y)
\nonumber\\
v_{\downarrow,\uparrow}&=&\frac{1}{2}(v_x+iv_y)
\nonumber\\
v_{\downarrow,\downarrow}&=&\frac{1}{2}(v_t-v_z)
\end{eqnarray}


%====================================================================
\subsection{\texttt{SPINOR\$CONVERT}}
%====================================================================
Converts a density matrix from the  (t,x,y,z) into the (uu,ud,du,dd)
representation. Converting a 


\begin{eqnarray}
A^{(j)}_{\alpha,\beta}&=&
\frac{1}{2}
\sum_{\sigma,\sigma'}
A_{\alpha,\beta,\sigma,\sigma'}\sigma^{(j)}_{\sigma',\sigma}
\nonumber\\
\rho^{(j)}_{\alpha,\beta}&=&
\sum_{\sigma,\sigma'}
\rho_{\alpha,\beta,\sigma,\sigma'}\sigma^{(j)}_{\sigma',\sigma}
\end{eqnarray}


\begin{itemize}
\item \verb|ID='FWRD'|: (TOUPDN=.false.) transforms the density matrix from 

(uu,ud,du,dd)$\rightarrow$(t,x,y,z)
\item \verb|ID='BACK'|: (TOUPDN=.true.) (t,x,y,z)$\rightarrow$ (uu,ud,du,dd)
\end{itemize}


\appendix
%====================================================================
\chapter{Vector representation of Pauli matrices}
%====================================================================
Pauli matrices can be represented as vectors in four dimensions.  
\begin{eqnarray}
\mat{\sigma}^{(j)}\hat{=}\vec{\sigma}^{(j)}:=
\left(\begin{array}{c}\sigma^{(j)}_{11}\\\sigma^{(j)}_{12}\\\sigma^{(j)}_{21}\\\sigma^{(j)}_{22}\end{array}\right)
\label{eq:vectorrepresentationpaulimatrices}
\end{eqnarray}

The usefullness of this representation is that the scalar project of
two such vectors can be related to the trace of the corresponding
Pauli matrices
\begin{eqnarray}
\Bigl(\vec{\sigma}^{(j)}\Bigr)^*\cdot\vec{\sigma}^{(j')}
=\sum_{\sigma,\sigma'}\Bigl(\sigma^{(j)}_{\sigma,\sigma'}\Bigr)^*
\sigma^{(j')}_{\sigma,\sigma'}
=\sum_{\sigma,\sigma'}\sigma^{(j)}_{\sigma',\sigma}
\sigma^{(j')}_{\sigma,\sigma'}
=\Tr\Bigl[\mat{\sigma}^{(j)}
\mat{\sigma}^{(j')}\Bigr]
\label{eq:vectorrepresentationpaulimatricesscalarproduct}
\end{eqnarray}
We have exploited that a complex conjugation of the Pauli matrices is
identical to a transposition, which follows directly from their
being hermitean.


The vector representation of the  Pauli matrices is
\begin{eqnarray*}
\mat{\sigma}^{(0)}\hat{=}
\left(\begin{array}{c}1\\0\\0\\1\end{array}\right)
\qquad
\mat{\sigma}^{(1)}\hat{=}
\left(\begin{array}{c}0\\1\\1\\0\end{array}\right)
\qquad
\mat{\sigma}^{(2)}\hat{=}
\left(\begin{array}{r}0\\-i\\i\\0\end{array}\right)
\qquad
\mat{\sigma}^{(3)}\hat{=}
\left(\begin{array}{r}1\\0\\0\\-1\end{array}\right)
\end{eqnarray*}
These vectors have length $\sqrt{2}$ and they are mutually orthogonal
to each other in the sense
\begin{eqnarray}
\frac{1}{2}\Bigl(\vec{\sigma}^{(j)}\Bigr)^*\cdot\vec{\sigma}^{(j')}=\delta_{j,j'}
\label{eq:pauliorthonormality1}
\end{eqnarray}

The orthonormality \eq{eq:pauliorthonormality1} of these vectors
together with the expression
\eq{eq:vectorrepresentationpaulimatricesscalarproduct} for their
scalar product establishes\\
\myshadowbox{
\begin{eqnarray}
\frac{1}{2}\Tr\biggl[\mat{\sigma}^{(j)}\mat{\sigma}^{(j')}\biggr]=\delta_{j,j'}
\label{eq:pauliorthonormality}
\end{eqnarray}
}

The expression for the scalar products can be generalized to dyadic
products in the vector representation.
Let us consider the Product
\begin{eqnarray}
\sum_{\sigma,\sigma',\bar{\sigma},\bar{\sigma}'} 
A_{\sigma,\sigma'} \sigma^{(j)}_{\sigma,\sigma'}
\left(\sigma^{(j')}_{\bar{\sigma},\bar{\sigma}'} \right)^*
B_{\bar{\sigma},\bar{\sigma}'} 
&=&\Bigl[\vec{A}\cdot\vec{\sigma}^{(j)}\Bigr]
\Bigl[\Bigl(\vec{\sigma}^{(j')}\Bigr)^*\cdot\vec{B}\Bigr]
=\vec{A}\Bigl[\Bigl(\vec{\sigma}^{(j)}\Bigr)^*\otimes\vec{\sigma}^{(j')}\Bigr]\vec{B}
\nonumber\\
\Rightarrow\qquad\sigma^{(j)}_{\sigma,\sigma'}
\left(\sigma^{(j')}_{\bar{\sigma},\bar{\sigma}'} \right)^*
&=&\Bigl[\Bigl(\vec{\sigma}^{(j)}\Bigr)^*\otimes\vec{\sigma}^{(j')}\Bigr]
_{\sigma,\sigma';\bar{\sigma},\bar{\sigma}'}
\label{eq:paulivectorrepresentationmeaningouterproduct}
\end{eqnarray}


The sum of the outer products of the Pauli matrices in the vector
representation \eq{eq:vectorrepresentationpaulimatrices} gives the
identiy matrix.
\begin{eqnarray}
\frac{1}{2}\sum_{j=0}^3\vec{\sigma}^{(j)}\otimes\Bigl(\vec{\sigma}^{(j)}\Bigr)^*
=\left(\begin{array}{cccc}
1&0&0&0\\
0&1&0&0\\
0&0&1&0\\
0&0&0&1
\end{array}\right)
\label{eq:paulidyadicsum1}
\end{eqnarray}

Together with \eq{eq:paulivectorrepresentationmeaningouterproduct},
tha above result \eq{eq:paulidyadicsum1} provides the second important
relation
\myshadowbox{
\begin{eqnarray}
\frac{1}{2}\sum_j\sigma^{(j)}_{\sigma,\sigma'}
\Bigl(\sigma^{(j)}_{\bar{\sigma},\bar{\sigma}'}\Bigr)^*=\delta_{\sigma,\bar{\sigma}}
\delta_{\sigma',\bar{\sigma}'}
\label{eq:paulidyadicsum}
\end{eqnarray}
}


%===================================================================
\subsubsection{Product table of Pauli matrices}
%===================================================================
The product table of the Pauli matrices including the unit matrix as
element with $j=0$.
\begin{eqnarray}
\mat{\sigma}^{(i)}\mat{\sigma}^{(j)}
&=&\left(\begin{array}{cccc}
\mat{\sigma}^{(0)}& \mat{\sigma}^{(x)}  &\mat{\sigma}^{(y)}&\mat{\sigma}^{(z)}\\
\mat{\sigma}^{(x)}& \mat{\sigma}^{(0)} &i\mat{\sigma}^{(z)}&-i\mat{\sigma}^{(y)}\\
\mat{\sigma}^{(y)}&-i\mat{\sigma}^{(z)}&\mat{\sigma}^{(0)} &i\mat{\sigma}^{(x)}\\
\mat{\sigma}^{(z)}& i\mat{\sigma}^{(y)}&-i\mat{\sigma}^{(x)}&\mat{\sigma}^{(0)}\\
\end{array}\right)
\nonumber\\
&=&
\sum_k\Bigl(\delta_{i,j}\delta_{k,0}+
\delta_{i,0}\delta_{j,k}+\delta_{i,k}\delta_{j,0}
-2\delta_{i,0}\delta_{j,0}\delta_{k,0}
\\
&+&i(1-\delta_{i,0})(1-\delta_{j,0})(1-\delta_{k,0})\epsilon_{i,j,k}
\Bigr)\mat{\sigma}^{(k)}
\label{eq:producttablepaulimatrices}
\end{eqnarray}
Do not get confused, because $i$ is used as index and as $\sqrt{-1}$.

In the more intuitive notation with three-dimensional vectors we obtain
\begin{eqnarray}
\left(\begin{array}{c|c}
(\mat{\sigma}^{(0)} )^2
& \mat{\sigma}^{(0)}\vec{\mat{\sigma}} \\
\hline
\mat{\sigma}^{(0)}\vec{\mat{\sigma}} & \mat{\sigma}^{(i)}\mat{\sigma}^{(j)}
\end{array}\right)
=
\left(\begin{array}{c|c}
\mat{\sigma}^{(0)} & \vec{\mat{\sigma}} \\
\hline
\vec{\mat{\sigma}} & \delta_{i,j}\mat{\sigma}^{(0)}
+i\sum_{k=1,3}\epsilon_{i,j,k}\mat{\sigma}^{(k)}
\end{array}\right)
\end{eqnarray}

Thus we obtain
\begin{myshadowminipage}{}
\begin{eqnarray}
\sum_{i,j=0}^3 A_{i,j}\mat{\sigma}^{(i)}\mat{\sigma}^{(j)}
&=&\Bigl(\sum_{k=0}^3 A_{k,k}\Bigr)\mat{\sigma}^{(0)}
+\sum_{j=1,3}\Bigl(A_{0,j}+A_{j,0}+i\sum_{n,m=1}^3\epsilon_{j,k,l}A_{k,l}\Bigr)
\mat{\sigma}^{(j)}
\label{eq:spinorproductformula}
\end{eqnarray}
Take care of the extent of the sum indices. Some run over $[0,1,2,3]$,
others over $[1,2,3]$ only.
\end{myshadowminipage}{}

\clearpage
\bibliographystyle{unsrtnat}
\bibliography{../all}
\end{document}  
