\documentclass[11pt,a4paper]{report}
%%%%%%%%%%%%%%%%%%%%%%%%%%%%%%%%%%%%%%%%%%%%%%%%%%%%%%%%%%%%%%%%%%%%%
%%                                                                 %%
%%    Header file for the Phi-S-X Series                           %%
%%                                                                 %%
%%    german version header_gm.tex is derived from header.tex      %%
%%    by uncommenting the line ``\setboolean{german}{true}'' below %%
%%                                                                 %%
%%    Never edit the german version! all changes must be done      %%
%%    in the english version header.tex                            %%
%%                                                                 %%
%%%%%%%%%%%%%%%%%%%%%%%%%%%%%%%%%%%%%%%%%%%%%%%%%%%%%%%%%%%%%%%%%%%%%
%%%%%%%%%%%%%%%%%%%%%%%%%%%%%%%%%%%%%%%%%%%%%%%%%%%%%%%%%%%%%%%%%%%%%
%%                                                                 %%
%%    Header file for the Phi-S-X Series                           %%
%%                                                                 %%
%%    german version header_gm.tex is derived from header.tex      %%
%%    by uncommenting the line ``\setboolean{german}{true}'' below %%
%%                                                                 %%
%%    Never edit the german version! all changes must be done      %%
%%    in the english version header.tex                            %%
%%                                                                 %%
%%%%%%%%%%%%%%%%%%%%%%%%%%%%%%%%%%%%%%%%%%%%%%%%%%%%%%%%%%%%%%%%%%%%%
%====================================================================
%-- define flag for language adaptations
\usepackage{ifthen}   % allows to select only certain text
\provideboolean{german}
\setboolean{german}{false}
%\setboolean{german}{true}  % uncomment this line for german editions
%====================================================================
%
% Textschriftart: Computer modern Bright
% body:            CM-Bright 10pt
% section titles:  CM-Bright Bold
% formulas:        CM-Bright Math Oblique
%
\usepackage[standard-baselineskips]{cmbright}
\usepackage{cmbright}
\usepackage[T1]{fontenc}
\def\usedfonts{CM-Bright}
\usepackage{typearea}
%\typearea[current]{calc} % benutzt die aktuelle 
       % bindekorrektur (BCOR angabe als parameter in koma usepackage)
       % und berechnet satzspiegel neu
\typearea[current]{11} %fixed div value

\usepackage{textcomp} % special symbols
\usepackage{amsfonts} % special symols
                      % see ftp://ftp.ams.org/pub/tex/doc/amsfonts/amsfndoc.pdf
\usepackage{amssymb}  % CM-Bright provides the AMS symbols
\usepackage{exscale}  % allows to scale math expressions to big fonts, 
                      % e.g. \Huge
\usepackage{curves}
\usepackage{braket}
\usepackage{miller}     % miller indices
\usepackage{chemmacros} % http://www.mychemistry.eu/mychemistry/
\usepackage[numbers]{natbib}     % bibliography style
\usepackage{url}\urlstyle{tt}
\usepackage{float}
\usepackage{bm}       % provides the command \bm{} that makes bold math symbols
\usepackage{amsmath}
\usepackage{amsbsy}   % allows bold mathematical symbols
\usepackage{amscd}
 \usepackage{a4wide}  % it is better to use the ``geometry'' package
\usepackage{array}    % 
\usepackage{fancyhdr} %  defines pagestyle fancy
\usepackage{epsfig}   % include graphics with epsfig
\usepackage{graphicx} % includegraphics
\usepackage{epstopdf}
\usepackage{wrapfig}
\usepackage{fancybox} % allows shadow-boxes
\usepackage{color}    % allows to use color in the text
%\usepackage{eepic}
\usepackage{flafter}  % places picture next to its reference
\usepackage{makeidx}  % make an index
%\usepackage{MnSymbol}  % 
%\usepackage{marvosym}  % 
\usepackage{textcase}
\usepackage{ulem} % defines strikeout \sout{}; underline \uline{}
                  % double underline \uuline{}; wave underline \uwave{}
                  % cross out \xout{}
%
%==========================================================================
%==  page layout  =========================================================
%==========================================================================
% eqnarray environment: reduce with of space in place of each ``&''
\setlength\arraycolsep{1.4pt}
\pagestyle{fancy}
%\renewcommand{\chaptermark}[1]{\markboth{\thechapter\ #1}{}}
\renewcommand{\chaptermark}[1]{\markboth{\MakeUppercase{\thechapter\ #1}}{}}
\fancyhf{} 
\fancyhead[LE]{\textsc{\thepage}\qquad\textsc{\leftmark}}
\fancyhead[RO]{\textsc{\leftmark}\qquad\textsc{\thepage}}
\renewcommand{\headrulewidth}{0.5pt}
\renewcommand{\footrulewidth}{0pt} 
\addtolength{\headheight}{2.5pt}
\fancypagestyle{plain}{\fancyhead{}
   \renewcommand{\headrulewidth}{0pt}
   \fancyfoot[CO]{\bfseries\thepage}}

% Line spacing -----------------------------------------------------------
\newlength{\defbaselineskip}
\setlength{\defbaselineskip}{\baselineskip}
\newcommand{\setlinespacing}[1]%
           {\setlength{\baselineskip}{#1 \defbaselineskip}}
\newcommand{\doublespacing}{\setlength{\baselineskip}%
                           {2.0 \defbaselineskip}}
\newcommand{\singlespacing}{\setlength{\baselineskip}{\defbaselineskip}}

% Absatz einr\"ucken ------------------------------------------------------
%\setlength{\parindent}{0pt}
\setlength{\parskip}{2pt}
% -------------------------------------------------------------------------
\ifthenelse{\boolean{german}}
  {\def\figurename{Abb.}}
  {\def\figurename{Fig.}}
%--------------------------------------------------------------------------
\renewcommand{\arraystretch}{1.15}  % skaliert den Zeilen abstand in der 
    % tabular und array umgebung
%
%==========================================================================
%==  boxes etc ============================================================
%==========================================================================
%== minipage in a shadowbox ===============================================
\newenvironment{myshadowminipage}[1]%
  {\par\noindent\begin{Sbox}\begin{minipage}{\linewidth}\vspace{0.1cm}\begin{center}\uppercase{#1}\end{center}}%
  {\vspace{0.1cm}\end{minipage}\end{Sbox}\shadowbox{\TheSbox}}
%
%== minipage in a framedbox ===============================================
\newenvironment{myframedminipage}%
  {\par\noindent\begin{Sbox}\begin{minipage}\linewidth\vspace{0.1cm}}%
  {\vspace{0.1cm}\end{minipage}\end{Sbox}\fbox{\TheSbox}}
%
\newcommand{\myshadowbox}[1]{\noindent\shadowbox{\parbox{\linewidth}{\smallskip #1\smallskip}}}
\newcommand{\myfbox}[1]{\noindent\fbox{\parbox{\linewidth}{\smallskip #1\smallskip}}\medskip}
%== minipage in a framedbox ===============================================
\newtheorem{defi}{Definition}[chapter]
\newenvironment{definition}[1]%
  {\par\noindent\begin{Sbox}\begin{minipage}{\linewidth}\vspace{0.1cm}\begin{defi}\uppercase{#1}\\\vspace{0.1cm}}%
  {\vspace{0.1cm}\end{defi}\end{minipage}\end{Sbox}\shadowbox{\TheSbox}}
%
%=========================================================================
% color used to point out information to the teacher
\definecolor{highlight}{rgb}{1.0,0.7,0.}
\newcommand{\Special}[1]{\textbf{\textcolor{highlight}{#1}}}
%=========================================================================
%  switch certain parts on and off. uses ifthen package
\newboolean{teacher}\setboolean{teacher}{false}
% this parameter can be changed in the manuscript again
\setboolean{teacher}{true} %private version if true!
\newcommand{\teacheronly}[1]{\ifthenelse{\boolean{teacher}}{#1\hfill\\ }}
\newcommand{\editor}[1]{\textcolor{blue}{\texttt{Editor: #1}}}
\newcommand{\MARK}[1]{\textcolor{blue}{#1}} 
\newcommand{\RED}[1]{\textcolor{red}{#1}} 
%
%==========================================================================
%==  define new symbols                                                 ===
%==========================================================================
% define \stat (stationary state) as an operator like \min
\DeclareMathOperator*{\stat}{stat}
\let\Vec=\mathbold   % cmbright.sty provides a bold/italic math alphabet
\let\Dot=\mathbold   % cmbright.sty provides a bold/italic math alphabet
\let\Ddot=\mathbold   % cmbright.sty provides a bold/italic math alphabet
%
\newcommand{\e}[1]{\mathrm{e}^{#1}}% exponential function
\renewcommand{\Re}{\mathrm{Re}}    % real part
\renewcommand{\Im}{\mathrm{Im}}    % imaginary part
\newcommand{\lagr}{\ell}           % Lagrange dichte
\newcommand{\Lagr}{\mathcal{L}}    % Lagrange Funktion
\newcommand{\erf}{{\rm erf}}       %
\newcommand{\atan}{{\rm atan}}     % arcus tangens
\newcommand{\mat}[1]{\bm{#1}}  % Matrix
\newcommand{\gmat}[1]{{\boldsymbol #1}}  % Matrix(symbol)
\newcommand{\defas}{\stackrel{\text{def}}{=}}  %  is defined as
\ifthenelse{\boolean{german}}
  {\newcommand{\rot}{{\rm\bf rot}}}    % curl
  {\newcommand{\rot}{{\rm\bf curl}}}   % curl
\newcommand{\sgn}{{\rm sgn}}       % sign
\ifthenelse{\boolean{german}}
   {\newcommand{\Tr}{\mathrm{Sp}}}      % trace
   {\newcommand{\Tr}{\mathrm{Tr}}}      % trace
\ifthenelse{\boolean{german}}
   {\newcommand{\grmn}[2]{\footnote{``#2'' hei{\ss}t in englisch ``#1''}}}
   {\newcommand{\grmn}[2]{\footnote{``#1'' translates as ``#2'' into German}}}
% define the equation reference
\ifthenelse{\boolean{german}}
   {\newcommand{\eq}[1]{\text{Gl.}~\ref{#1}}}
   {\newcommand{\eq}[1]{\text{Eq.}~\ref{#1}}}
% define a relation with an equation number ontop
\newcommand{\eqrel}[2]{\stackrel{\eq{#1}}{#2}}
\newcommand{\zero}{\varnothing}
%\newcommand{\ket}[1]{|#1\rangle} % contained in package braket
\newcommand{\sumint}{\int\hspace{-15pt}\sum}
\newcommand{\marker}[1]{\textcolor{blue}{\emph{#1}}}
\renewcommand*{\dot}[1]{\overset{\mbox{\large\bfseries .}}{#1}}
\renewcommand*{\ddot}[1]{\overset{\mbox{\large\bfseries\hspace{+0.1ex}.\hspace{-0.1ex}.}}{#1}}
%
%==========================================================================
%==                                                                     ===
%==========================================================================
% Prevent figures from appearing on a page by themselves
% from http://dcwww.camd.dtu.dk/~schiotz/comp/LatexTips/LatexTips.html
\renewcommand{\topfraction}{0.85}
\renewcommand{\textfraction}{0.1}
\renewcommand{\floatpagefraction}{0.75}
%
%==========================================================================
%==                                                                     ===
%==========================================================================
\makeindex    % make index. uses makeidx package.

%== allow links between documents ============================================
\usepackage{xr}
\usepackage{xr-hyper}
%==  hyperref package (must be last package)
\usepackage[colorlinks=true]{hyperref} %specify this as last package
\hypersetup{citecolor=blue}
\hypersetup{menucolor=magenta}
\hypersetup{urlcolor=blue}      % 
\hypersetup{filecolor=green}    % file links
\hypersetup{linkcolor=magenta}  %table of contents
\hypersetup{pdfauthor={Peter E. Bl\"ochl}}
\hypersetup{pdfdisplaydoctitle=true}
\externaldocument[phisx1-]{/Users/ptpb/Tree/PhiSX/ClassicalMechanics/Book/cm-gm}
\externaldocument[phisx2-]{/Users/ptpb/Tree/PhiSX/Electrodynamics/Book/el-gm}
\externaldocument[phisx3-]{/Users/ptpb/Tree/PhiSX/QuantumMechanics/Book/qm}
\externaldocument[phisx4-]{/Users/ptpb/Tree/PhiSX/StatisticalMechanics/Book/sm}
\externaldocument[phisxqm2-]{/Users/ptpb/Tree/PhiSX/QuantumMechanicsII/Book/qm2}
\externaldocument[phisxsm2-]{/Users/ptpb/Tree/PhiSX/StatisticalMechanicsII/Book/sm2}
\externaldocument[phisxcb-]{/Users/ptpb/Tree/PhiSX/Chemicalbond/Book/cb}
% Example: Figure~PhiSX:Quantum
% Mechanics-\ref{phisx3-fig:doubleslitwave} on page
% \pageref{phisx3-fig:doubleslitwave}


\hypersetup{pdftitle=paw_brillouin}
\begin{document}
\begin{titlepage}
\begin{center}
\vspace*{3.5cm}
{\huge \textbf{The DOS tool of the CP-PAW code}}\\
\vspace{0.5cm}
{\large Peter E. Bl\"ochl}
\vspace{0.5cm} 
\end{center}

\vfill
\begin{center}
Copyright Peter E. Bl\"ochl; Sept.2, 2013-\today\\
{\small
Institute of Theoretical Physics;
Clausthal University of Technology;\\ 
D-38678 Clausthal Zellerfeld; Germany;\\
http://www.pt.tu-clausthal.de/atp/}
\end{center}
\end{titlepage}
\noindent            
\tableofcontents
%====================================================================
\chapter{Background}
%====================================================================
%============================================================================
\subsubsection{Expectation values and matrix elements}
%============================================================================
The expectation value of a one-particle operator $\hat{A}$ can be evaluated
as Brillouin zone integral.
\begin{eqnarray}
\langle{A}\rangle
=\sum_n\frac{1}{V_G}\int d^3k\; f_{\beta,\mu}(\epsilon_n(\vec{k}))
A_{n}(\vec{k})
\end{eqnarray}
where $f_{\beta,\mu}(\epsilon)$ is the Fermi distribution, $V_G$ is
the volume of the reciprocal unit cell and
\begin{eqnarray}
A_{n}(\vec{k})=\langle\psi_n(\vec{k})|\hat{A}|\psi_n(\vec{k})\rangle
\end{eqnarray}
are the matrix elements of the operator $\hat{A}$

%============================================================================
\subsubsection{Density of states}
%============================================================================
In order to understand how a material responds to external stimuli, it
is often useful to split off the Fermi distribution
\begin{eqnarray}
\langle{A}\rangle=\int d\epsilon\; D^{(A)}(\epsilon)
f_{\beta,\mu}(\epsilon)
\end{eqnarray}
wher
\begin{eqnarray}
D^{A}(\epsilon)=
\sum_n\frac{1}{V_G}\int d^3k\; A_{n}(\vec{k})\delta(\epsilon-\epsilon_n(\vec{k}))
\end{eqnarray}
is the A-weighted density of states.

%============================================================================
\subsubsection{Local orbitals}
%============================================================================
Let us introduce local orbitals $|\chi_{\alpha,\vec{t}}\rangle$, which
are characterized by an identifier $\alpha$, that is unique for each
orbital in a unit cell, and a lattice translation vector, that
identifies the relative lattice displacements.

A wave function is decomposed into local orbitals using the projector
functions $|\pi_{\alpha,\vec{t}}\rangle$, which obey the
bi-orthogonality condition
$\langle\pi_{\alpha,\vec{t}}|\chi_{\beta,\vec{t'}}\rangle=\delta_{\alpha,\beta}
\delta_{\vec{t},\vec{t'}}$, as
\begin{eqnarray}
|\psi_n(\vec{k})\rangle
&=&\sum_{\alpha}\sum_{\vec{t}}
|\chi_{\alpha,\vec{t}}\rangle
\langle\pi_{\alpha,\vec{t}}|\psi_n(\vec{k})\rangle
\nonumber\\
&=&\sum_{\alpha}\sum_{\vec{t}}
|\chi_{\alpha,\vec{t}}\rangle
\langle\pi_{\alpha,\vec{0}}|\psi_n(\vec{k})\rangle\e{i\vec{k}\vec{t}}
\end{eqnarray}

In a basis of local orbitals the matrix elements can be separated out
\begin{eqnarray}
A_{n}(\vec{k})=\langle\psi_n(\vec{k})|\pi_{\alpha,\vec{t}}\rangle
\langle\chi_{\alpha,\vec{t}}|\hat{A}|\chi_{\beta,\vec{t'}}\rangle
\langle\pi_{\beta,\vec{t'}}|\psi_n(\vec{k})\rangle
\end{eqnarray}

Now we can also abstract the density of states from the operator for
which the matrix elements shall be evaluated.
Thus we obtain the matrix element
\begin{eqnarray}
\langle{A}\rangle=
\sum_{\alpha,\beta}\langle\chi_{\alpha,\vec{t}}|\hat{A}|\chi_{\beta,\vec{t'}}\rangle
\left\lbrace\int d\epsilon\; D_{\beta,\vec{t'},\alpha,\vec{t}}(\epsilon)
f_{\beta,\mu}(\epsilon)\right\rbrace
\end{eqnarray}
where
\begin{eqnarray}
D_{\alpha,\vec{t},\beta,\vec{t'}}(\epsilon)
&=&
\sum_n\frac{1}{V_G}\int d^3k\; 
\langle\pi_{\alpha,\vec{t}}|\psi_n(\vec{k})\rangle
\;\delta(\epsilon-\epsilon_n(\vec{k}))\;
\langle\psi_n(\vec{k})|\pi_{\beta,\vec{t'}}\rangle
\nonumber\\
&=&
\sum_n\frac{1}{V_G}\int d^3k\; 
\langle\pi_{\alpha,\vec{0}}|\psi_n(\vec{k})\rangle
\;\delta(\epsilon-\epsilon_n(\vec{k}))\;
\langle\psi_n(\vec{k})|\pi_{\beta,\vec{0}}\rangle\e{i\vec{k}(\vec{t'}-\vec{t})}
\end{eqnarray}
is the density-of-states matrix.

%============================================================================
\subsubsection{Complex orbitals}
%============================================================================
Sometimes it is convenient to use complex orbitals
\begin{eqnarray}
|\Phi\rangle=\sum_{\alpha,\vec{t}}|\chi_{\alpha,\vec{t}}\rangle C_{\alpha,\vec{t}}
\end{eqnarray}
Then we can calculate the corresponding density of states
as
\begin{eqnarray}
D_{\Phi,\Psi}(\epsilon)
&=&\sum_{\alpha,\vec{t},\beta,\vec{t'}}
\Bigl(C^\Phi_{\alpha,\vec{t}}\Bigr)^*
D_{\alpha,\vec{t},\beta,\vec{t'}}(\epsilon)
C^\Psi_{\beta,\vec{t'}}
\nonumber\\
&=&
\sum_n\frac{1}{V_G}\int d^3k\; 
\biggl[\sum_\alpha \Bigl(C^\Phi_{\alpha,\vec{t}}\Bigr)^*
\langle\pi_{\alpha,\vec{0}}|\psi_n(\vec{k})\rangle
\e{-i\vec{k}\vec{t}}
\biggr]
\;\delta(\epsilon-\epsilon_n(\vec{k}))\;
\biggl[\sum_\alpha 
\Bigl(C^\Psi_{\beta,\vec{t'}}\Bigr)^*
\langle\pi_{\beta,\vec{0}}|\psi_n(\vec{k})\rangle
\e{-i\vec{k}\vec{t'}}
\biggr]^*
\nonumber\\
\end{eqnarray}
Therefore a general orbital is a characterized sequence of indices $\alpha$,
lattice translation vectors $\vec{t}$, and coefficients $C_{\alpha,\vec{t}}$.


%============================================================================
\subsubsection{Angular-momentum weights and COOP's}
%============================================================================



%====================================================================
\chapter{Routines}
%====================================================================
\cite{bloechl94_prb50_17953}
%====================================================================
\section{FLOW}
%====================================================================
\begin{itemize}
\item \verb|readcntl|: Collect data names from !dcntl!files!file. Attach
  files.
%
\item read pdos file:
%
\item \verb|readcntl$orbitals|: Build up a set or named orbitals, that
  may used to define weights or Coops. The data entry is
  !dcntl!orbital. The orbitals are stored as coefficient vectors in the
  orbitals\_module. Each orbital is normalized.
%
\item \verb|readcntl$sets|: construct matrix elements for all band
  states. Matrix elements are either crystal orbital overlap
  populations COOPS or WEIGHTS.
\begin{verbatim}
!dcntl
  !coop
     !orb1 ... !end
     !orb2 ... !end
  !end
  !weight type= spin= !end
  !weight 
    !orb ... !end
  !end
!end
\end{verbatim}
Type can be 'all', 'total', 'empty'

Coops and weights are both matrix elements of the
  type 
\begin{eqnarray}
A^{(\vec{a},\vec{b})}_{n}(\vec{k})
=\sum_{\alpha,\beta}
a^*_{\alpha}
\langle\pi_\alpha|\psi_n(\vec{k})\rangle
\langle\psi_n(\vec{k})|\pi_\beta\rangle
b_{\beta}
\end{eqnarray}
For weights the vectors $\vec{a},\vec{b}$ are identical, while for
COOP's they are different.

\textbf{Is the Bloch phase factor properly accounted for?}

\textbf{ are the projections scaled by the volume integral?}

An array \verb|SET(NB,NKPT,nspin,ISET)| is produced, which contains
the matrix elements for all states.


\item readcntl\$grid
\item readcntl\$output
\end{itemize}



%====================================================================
\section{READCNTL}
%====================================================================

%====================================================================
\section{Orbitals(new)}
%====================================================================
\begin{verbatim}
TYPE ORBITALCOEFFICIENT_TYPE
  INTEGER(4) :: IPRO
  INTEGER(4) :: IT(3)
  COMPLEX(8) :: C
END TYPE ORBITALCOEFFICIENT_TYPE
TYPE ORBItal_TYPE
  CHARACTER(32) :: NAME
  INTEGER(4)    :: NChi
  INTEGER(4)    :: NChix
  TYPE(ORBItalCOEFFICIENT_TYPE), POINTER :: CHI(:)
END TYPE ORBItal_TYPE
INTEGER(4)                :: NORBi=0      ! #(ORBITALS)
INTEGER(4)                :: NORBiX=0     ! MAX#(ORBITALS)
INTEGER(4)    ,PARAMETER  :: NORBiSTEP=10 ! STEP IN #(ORBITALS)
INTEGER(4)    ,PARAMETER  :: NchiSTEP=10 ! STEP IN #(coefficients)
TYPE(ORBItal_TYPE),allocatable :: ORBI(:)      ! orbital list
integer(4)                :: selectedorbital=0
\end{verbatim}


The orbitals object has the following functions:
\begin{verbatim}
SUBROUTINE ORBITALS$NEWORBITAL(NAME)
SUBROUTINE ORBITALS$SELECTORBITAL(NAME)
subroutine ORBITALS$ISELECTORBITAL(IORB)
SUBROUTINE ORBITALS$SETCHI(IPRO,IT,C)
SUBROUTINE ORBITALS$GETNCHI(NCHI)
SUBROUTINE ORBITALS$GETCHI(ICHI,IPRO,IT,C)
\end{verbatim}
The data structures are dynamical, so that one need not worry obout
the size of arrays. An orbital is a set of coefficients $C$ alongside
with an orbital index in the basi unit cell and a lattice translation
vector in relative coordinates. (Thus, the lattice translation vector
is a triple of integer numbers.)

A general orbital must be created by calling
\verb|ORBITALS$NEWORBITAL(NAME)|. The orbital can later be identified
by its name. The name 'none' is reserved and unselects all orbitals.

before one can do something to an orbital one has to select it either
by name or by number. This can be done using
either \verb|ORBITALS$SELECTORBITAL(NAME)| or
\verb|ORBITALS$ISELECTORBITAL(IORB)|.

Using the functions \verb|setchi| and \verb|getchi| contributions of
local orbitals can be added or retrieved.



%====================================================================
\section{Orbitals}
%====================================================================
The internal orbitals object maintains a list of orbitals. Each
orbital is characterized by an id and it is described by a coefficient
vector.

The object contains the following entities:
\begin{verbatim}
MODULE ORBITALS_MODULE
SUBROUTINE ORBITALS$SETORB(NAME_,LENG_,ORBITAL_)
SUBROUTINE ORBITALS$GETORB(NAME_,LENG_,ORBITAL_)
SUBROUTINE READONEORB(LL_CNTL,NAT,LMX,rbas,RPOS,NPRO,ORBITAL)
   SUBROUTINE RESOLVEATOM(ATOMEX,IAT,IT)
      SUBROUTINE RESOLVEEXTENDEDNAME(XNAME,NAME,IT)
   SUBROUTINE RESOLVEROTATION(DZ_,DX_,ROT)
   SUBROUTINE MAKEORBITAL(ATOM,LMXX,ORB,NPRO_,ORBITAL)
SUBROUTINE NORMALIZEORBITAL(NPRO_,ORBITAL)
\end{verbatim}


%====================================================================
\section{READONEORB}
%====================================================================
Readoneorb can read an orbital from the control file and returns a
coefficient vector. It can either simply pick a predefined orbital
from the list
\begin{verbatim}
    name= fac= 
\end{verbatim}
or it provide a new orbital with a new name
\begin{verbatim}
    name=  atom=  type= z= nnz= x= nnx= fac=
\end{verbatim}
\begin{itemize}
\item \verb|ATOM| uses the extended notation
\item The \verb|type|-identifier is case insensitive
\item z as vector or the atom nnz define the local z-axis: The
  local z axis points towards the atom specified by nnz.
\item x as vector or nnz defines the local x-axis: The local x axis is
  perpendicular to the local z axis, and it lies in the plane of the
  local z-axis and the specified x-vector. Note that the sign of the
  x-vector defines the orientation of the y-vector.
\item the allowed values for type are 's', 'px', 'py', 'pz', 'sp1',
  'sp2', 'sp3', 'dx2-y2', 'dxz', 'dyz', 'dxy', 'd3z2-r2'. The orbitals
  'sp1', 'sp2', 'sp3' all point towards the z-direction. The orbital
  is oriented according to the local coordinate
  system. \textbf{currently it is not possible to specify f-orbitals}
\item fac is a prefactor.
\end{itemize}

The input produces a local orbital vector, which is specific for a
given atom. In the routine makeorbital, this vector is placed into an
orbital vector for the entire system.



\appendix
%=================================================================
\section{}
%================================================================

\bibliographystyle{unsrtnat}
\bibliography{../all}
\end{document}  
