\documentclass[final,12pt]{article}    
%%%%%%%%%%%%%%%%%%%%%%%%%%%%%%%%%%%%%%%%%%%%%%%%%%%%%%%%%%%%%
%
%  include packages
%
\usepackage{times}
\usepackage{ifthen}
\usepackage{graphics}
\usepackage{color}
\usepackage{shadow}
%%%%%%%%%%%%%%%%%%%%%%%%%%%%%%%%%%%%%%%%%%%%%%%%%%%%%%%%%%%%%
%
%  if private=false certain descriptions are excluded using
%  the \ifthenelse expression
%
\newboolean{private}\setboolean{private}{true}
\newboolean{qmmm}\setboolean{qmmm}{true}
%%%%%%%%%%%%%%%%%%%%%%%%%%%%%%%%%%%%%%%%%%%%%%%%%%%%%%%%%%%%%
%
%  define commands
%
\newcommand{\block}[1]{\subsubsection[#1]{\shabox{\bf #1}}}
%
\newcommand{\brules}[1]{
\makebox[1in][l]{Rules:}\parbox[t]{110mm}{#1}\hfill\break\hfill}
%
\newcommand{\bdescr}[1]{
\makebox[1in][l]{Description:}\parbox[t]{110mm}{#1}\hfill\break}
%
%\newcommand{\key}[1]{\hfill\break
%\makebox[1in][l]{Keyword:}\parbox[t]{110mm}{{\bf #1}}\hfill\break}
%
\newcommand{\key}[1]{\hfill\break \makebox[1.5in][l]{\bf #1}\hfill\break}
%
%\newcommand{\vdescr}[1]{
%\makebox[1in][l]{}\parbox[t]{110mm}{#1}\hfill\break}
%
\newcommand{\vdescr}[1]{\makebox[1in][l]{}\parbox[t]{110mm}{#1}\hfill\break}
%
\newcommand{\vformat}[1]{
\makebox[1in][l]{}\parbox[t]{110mm}{\makebox[1in][l]{Type:}\parbox[t]{2.7in}{#1}}
\hfill\break}
%
\newcommand{\vrules}[1]{
\makebox[1in][l]{}\parbox[t]{110mm}{\makebox[1in][l]{Rules:}\parbox[t]{2.7in}{#1}}
\hfill\break}
%
\newcommand{\vdefault}[1]{
\makebox[1in][l]{}\parbox[t]{110mm}
{\makebox[1in][l]{Default:}\parbox[t]{2.7in}{#1}}
\hfill\break}
%
\newcommand{\mbax}[1]{#1}
%============================================================
\begin{document}          
%
%----------------------------------------------------------------------
\block{!CONTROL!ANALYSE!WAVE}
%----------------------------------------------------------------------
\brules{optional} \bdescr{Writes a wave function. The file
  created can then be processed by the paw\_wave tool to produce an
  input file for the IBM Dataexplorer.  }

\mbax{\key{TITLE}
\vdescr{title of the image. Currently it is not used other than in the printout.}
\vformat{character}
\vrules{optional}
\vdefault{none}}

\mbax{\key{FILE} 
\vdescr{full file name of the file to be produced,
    which can be converted into a input file for the IBM Dataexplorer
    using the paw\_wave tool.} 
\vformat{character} 
\vrules{mandatory}
\vdefault{none}}

\mbax{\key{DR} 
\vdescr{grid spacing for the output file. rounded to get an integer factor
relative to the real-space grid used in the calculation.} 
\vformat{real} 
\vrules{optional}
\vdefault{0.4}}

\mbax{\key{B}
\vdescr{band index}
\vformat{integer}
\vrules{mandatory}
\vdefault{none}}

\mbax{\key{K}
\vdescr{k-point index}
\vformat{integer}
\vrules{mandatory}
\vdefault{1}}

\mbax{\key{S}
\vdescr{spin index}
\vformat{integer}
\vrules{mandatory}
\vdefault{1}}

\mbax{\key{TIMAG}
\vdescr{use imaginary part}
\vformat{logical}
\vrules{optional}
\vdefault{F}}
%
%----------------------------------------------------------------------
\block{!CONTROL!ANALYSE!DENSITY}
%----------------------------------------------------------------------
\brules{optional} \bdescr{Writes a density. The file
  created can then be processed by the paw\_wave tool to produce an
  input file for the IBM Dataexplorer.  }

\mbax{\key{TITLE}
\vdescr{title of the image. Currently it is not used other than in the printout.}
\vformat{character}
\vrules{optional}
\vdefault{none}}

\mbax{\key{FILE} 
\vdescr{full file name of the file to be produced,
    which can be converted into a input file for the IBM Dataexplorer
    using the paw\_wave tool.} 
\vformat{character} 
\vrules{mandatory}
\vdefault{none}}

\mbax{\key{DR} 
\vdescr{grid spacing for the output file. rounded to get an integer factor
relative to the real-space grid used in the calculation.} 
\vformat{real} 
\vrules{optional}
\vdefault{0.4}}

\mbax{\key{TYPE}
\vdescr{can be 'TOTAL', 'SPIN', 'UP' or 'DOWN'. Determines the weights of the
  states in the density plots. 'TOTAL' takes the actual occupations
  and k-point or uniform weights (depending on TOCC). 'SPIN' is like `TOTAL', but counts states for
  spin=2 negative. 'UP' and 'DOWN' give the spin up and down densities respectively.}
\vformat{character}
\vrules{optional}
\vdefault{'TOTAL'}}

\mbax{\key{TOCC} 
\vdescr{use acutal occupations} 
\vformat{logical} 
\vrules{optional}
\vdefault{T}}

\mbax{\key{TDIAG} 
\vdescr{use eigenstates in the subspace of the dynamic wave functions.} 
\vformat{logical} 
\vrules{optional}
\vdefault{T}}

\mbax{\key{TCORE} 
\vdescr{include core density.} 
\vformat{logical} 
\vrules{optional}
\vdefault{F}}

\mbax{\key{EMIN[EV]} 
\vdescr{lowest eigenenergy to be included.} 
\vformat{real} 
\vrules{optional}
\vdefault{-1$^{-10}$}}

\mbax{\key{EMAX[EV]} 
\vdescr{highest eigenenergy to be included.} 
\vformat{real} 
\vrules{optional}
\vdefault{1$^{-10}$}}

\mbax{\key{BMIN} 
\vdescr{lowest band to be included.} 
\vformat{integer} 
\vrules{optional}
\vdefault{1}}

\mbax{\key{BMAX} 
\vdescr{highest band to be included.} 
\vformat{integer} 
\vrules{optional}
\vdefault{10000000}}
%
%----------------------------------------------------------------------
\block{!CONTROL!ANALYSE!POTENTIAL}
%----------------------------------------------------------------------
\brules{optional} \bdescr{Writes the potential. The file
  created can then be processed by the paw\_wave tool to produce an
  input file for the IBM Dataexplorer.  }

\mbax{\key{TITLE}
\vdescr{title of the image. Currently it is not used other than in the printout.}
\vformat{character}
\vrules{optional}
\vdefault{none}}

\mbax{\key{FILE} 
\vdescr{full file name of the file to be produced,
    which can be converted into a input file for the IBM Dataexplorer
    using the paw\_wave tool.} 
\vformat{character} 
\vrules{mandatory}
\vdefault{none}}

\mbax{\key{DR} 
\vdescr{grid spacing for the output file. rounded to get an integer factor
relative to the real-space grid used in the calculation.} 
\vformat{real} 
\vrules{optional}
\vdefault{0.4}}

\end{document}
\bye
%====================================================================
